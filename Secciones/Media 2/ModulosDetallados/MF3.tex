\subsubsection{MF3. Módulo modificador de la acústica}
Para realizar la modificación de la acústica en el estudio se debe de cambiar la superficie que interactúa con las ondas de sonido generadas, para poder realizar el cambio de las superficies debemos de girar los prismas triangulares de nuestro trabajo terminal, para lo cual debemos de trasmitir dicho movimiento desde un motor hasta dichos primas. Existen varias formas de transmitir dichos movimiento, en la Tabla \ref{tab:ComTransmision} se pueden 
%-----------------------------tabla-----------------------------
\begin{center}
\footnotesize
    \begin{longtable}[!htb]{| m{5em} | m{12em} | m{12em}| m{12em}|}
    \hline
    \textbf{Tipo}& \textbf{Descripción} & \textbf{Ventajas} & \textbf{Desventajas}\\
    \hline\hline
%---------------------------
    Bandas y poleas& Utilizar poleas y bandas dentadas para poder realizar el movimiento giratorio de los primas triangulares &
    \begin{itemize}
        \item Ruido mínimo
        \item Vibraciones mínimas
        \item Movimiento preciso con una sincronización exacta
        \item Resistencia a la abrasión, al óxido, productos químicos y contaminantes.
    \end{itemize}
    & 
    \begin{itemize}
        \item Se necesitan bandas de largas dimensiones
        \item Mayor costo
        \item Ideal para transferir a una potencia relativamente baja
        \item La potencia de transferencia está a una distancia relativamente menor en comparación con otras bandas de transmisión
    \end{itemize}\\
    \hline
%------------------------
    Tornillo sin fin y corona & Utilizar un tornillo sin fin que mueva coronas, las cuales están acopladas a los primas triangulares &
    \begin{itemize}
        \item Elevada capacidad de carga
        \item Ruido mínimo
        \item Movimiento preciso con una sincronización exacta
        \item Compacto
    \end{itemize}
    & 
    \begin{itemize}
        \item Bajo rendimiento en las etapas de reducción
        \item Requiere mantenimiento periódico debido a que sufren desgaste por fricción
        \item Costo de mantenimiento elevado si se llegara a requerir reparaciones.
    \end{itemize}\\
    \hline
%------------------------    
    Movimiento individual& Colocar un motor por cada uno de los prismas triangulares para controlarlos individualmente& 
    \begin{itemize}
        \item Control personalizado a cada prisma sin la necesidad de ningún otro mecanismo 
        \item Movimiento preciso con una sincronización exacta
    \end{itemize}
    & 
    \begin{itemize}
        \item Elevado costo debido a que se necesitan comprar una gran cantidad de motores
        \item Mayor consumo energético comparado con las otras opciones.
    \end{itemize}\\
    \hline
    \caption{Ventajas y desventajas de los distintos tipos de transmisi\'on de movimiento}
    \label{tab:ComTransmision}
    \end{longtable}
\end{center}

Después de observar las ventajas y desventajas de cada uno de los métodos de transmisión de movimiento nos decidimos por la trasmisión por medio de un tornillo sin fin y corona debido a que se tiene un buen control de la posición de los prismas que portan los paneles, tiene una gran capacidad de carga, además de que el ruido producido es bajo. Al elegir este tipo de transmisión surgieron problemas en cuanto las dimensiones del tornillo sin fin, ya que si optabamos por utilizar un solo tornillo sin fin para mover todos los primas este debia de ser de un largo bastante considerable, y realizar la fabricación de este sería una tarea difícil, por lo que en la Tabla \ref{tab:ComTornilloSF} se comparan las opciones que se tienen para realizar la transmisión de tornillo sin fin y corona.
%-----------------------------tabla-----------------------------
\begin{center}
\footnotesize
    \begin{longtable}[!htb]{| m{5em} | m{12em} | m{12em}| m{12em}|}
    \hline
    \textbf{Tipo}& \textbf{Descripción} & \textbf{Ventajas} & \textbf{Desventajas}\\
    \hline\hline
%---------------------------
    Bandas y poleas& Utilizar poleas y bandas dentadas para poder realizar el movimiento giratorio de los primas triangulares &
    \begin{itemize}
        \item Ruido mínimo
        \item Vibraciones mínimas
        \item Movimiento preciso con una sincronización exacta
        \item Resistencia a la abrasión, al óxido, productos químicos y contaminantes.
    \end{itemize}
    & 
    \begin{itemize}
        \item Se necesitan bandas de largas dimensiones
        \item Mayor costo
        \item Ideal para transferir a una potencia relativamente baja
        \item La potencia de transferencia está a una distancia relativamente menor en comparación con otras bandas de transmisión
    \end{itemize}\\
    \hline
%------------------------
    Tornillo sin fin y corona & Utilizar un tornillo sin fin que mueva coronas, las cuales están acopladas a los primas triangulares &
    \begin{itemize}
        \item Elevada capacidad de carga
        \item Ruido mínimo
        \item Movimiento preciso con una sincronización exacta
        \item Compacto
    \end{itemize}
    & 
    \begin{itemize}
        \item Bajo rendimiento en las etapas de reducción
        \item Requiere mantenimiento periódico debido a que sufren desgaste por fricción
        \item Costo de mantenimiento elevado si se llegara a requerir reparaciones.
    \end{itemize}\\
    \hline
%------------------------    
    Movimiento individual& Colocar un motor por cada uno de los prismas triangulares para controlarlos individualmente& 
    \begin{itemize}
        \item Control personalizado a cada prima sin la necesidad de ningún otro mecanismo 
        \item Movimiento preciso con una sincronización exacta
    \end{itemize}
    & 
    \begin{itemize}
        \item Elevado costo debido a que se necesitan comprar una gran cantidad de motores
        \item Mayor consumo energético comparado con las otras opciones.
    \end{itemize}\\
    \hline
    \caption{Ventajas y desventajas relacionadas a la utilización de un tornillo sin fin único o varios segmentos de tornillo sin fin}
    \label{tab:ComTornilloSF}
    \end{longtable}
\end{center}
% \section{Capitulo III}
\addcontentsline{toc}{section}{Apéndices}
\section*{Apéndices}
\appendix
% \subsection{Anexos}
\section{Códigos y scripts}


\subsection{Codigo del programa Absorptions}
\lstinputlisting[frame=single,numbers=left, style=Matlab-editor, basicstyle=\footnotesize]{Codigos/Absorptions.m}


\subsection{Código para realizar el trazado de los rayos y calcular la espuesta al impulso}
\lstinputlisting[frame=single,numbers=left, style=Matlab-editor, basicstyle=\footnotesize]{Codigos/RayTracingImpulseResponse.m}

\subsection{Código para medir la respuesta al impulso}
\lstinputlisting[frame=single,numbers=left, style=Matlab-editor, basicstyle=\footnotesize]{Codigos/ImpulseResponseMeasurer.m}

\subsection{Código de la función para calcular los parámetros acústicos}
\lstinputlisting[frame=single,numbers=left, style=Matlab-editor, basicstyle=\footnotesize]{Codigos/CalcAcousticParams.m}

\subsection{Código de la función para obtener la absorción del cuarto}
\lstinputlisting[frame=single,numbers=left, style=Matlab-editor, basicstyle=\footnotesize]{Codigos/GetAbsorption.m}

\subsection{Código de la función para obtener la coordenada en donde choco el rayo de sonido}
\lstinputlisting[frame=single,numbers=left, style=Matlab-editor, basicstyle=\footnotesize]{Codigos/getImpactWall.m}

\subsection{Código de la función que realiza la interpolación entre EDT, T20, T30, y T60, y un vector de tiempo}
\lstinputlisting[frame=single,numbers=left, style=Matlab-editor, basicstyle=\footnotesize]{Codigos/GetLinearFits.m}

\subsection{Código de la función que nos proporciona el vector normal a la pared en la que el rayo chocó}
\lstinputlisting[frame=single,numbers=left, style=Matlab-editor, basicstyle=\footnotesize]{Codigos/getWallNormalVector.m}

\subsection{Código para realizar el trazado de los rayos y calcular la espuesta al impulso}
\lstinputlisting[frame=single,numbers=left, style=Matlab-editor, basicstyle=\footnotesize]{Codigos/RayTracingImpulseResponse.m}

\subsection{Código de la función que genera un sonido que simula el disparo de una pistola}
\lstinputlisting[frame=single,numbers=left, style=Matlab-editor, basicstyle=\footnotesize]{Codigos/Gunshot.m}

\subsection{Código de la clase Panel}
\lstinputlisting[frame=single,numbers=left, style=Matlab-editor, basicstyle=\footnotesize]{Codigos/Panel.m}

\subsection{Código de la función que gráfica el cuarto de trabajo}
\lstinputlisting[frame=single,numbers=left, style=Matlab-editor, basicstyle=\footnotesize]{Codigos/plotRoom.m}

\subsection{Código de la función que gráfica las superficies del cuarto}
\lstinputlisting[frame=single,numbers=left, style=Matlab-editor, basicstyle=\footnotesize]{Codigos/plotSurfaces.m}

\subsection{Código de RandSampleSphere}
\lstinputlisting[frame=single,numbers=left, style=Matlab-editor, basicstyle=\footnotesize]{Codigos/RandSampleSphere.m}

\subsection{Código de SimuleteRIR}
\lstinputlisting[frame=single,numbers=left, style=Matlab-editor, basicstyle=\footnotesize]{Codigos/SimulateRIR.m}

\subsection{Código Sweep}
\lstinputlisting[frame=single,numbers=left, style=Matlab-editor, basicstyle=\footnotesize]{Codigos/Sweep.m}

%\begin{lstlisting}[language=Python, caption=Estres residual anillos 01-MatLab, label={cod:MatLab}]
%\end{lstlisting}

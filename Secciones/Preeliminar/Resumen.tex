\section*{Resumen}
Para lograr una acústica de adecuada calidad en los estudios de audio, se suele recurrir al uso de paneles acústicos, los cuales modifican las ondas sonoras de un cuarto que no estaba inicialmente diseñado para con propósito de grabar audio, u cualquier otra función en donde la calidad sonora sea de importancia. Sin embargo, y a pesar de que el campo de la acústica está profundamente estudiado, la instalación de paneles acústicos sigue siendo un proceso mayormente empírico y que además está limitado, lo anterior debido a que diferentes instrumentos requieren una acústica diferente, y el acercamiento de los paneles es buscar una acústica “aceptable” para un amplio rango de instrumentos. Este proyecto tiene como objetivo el diseño e implementación de un sistema que automatice el acondicionamiento acústico en un estudio de audio por medio del movimiento de paneles acústicos. Para este fin, nuestra propuesta consiste en un sistema de paneles acústicos móviles, que cambien la superficie que está en contacto directo con las ondas generadas en el recinto y que cambien la respuesta de este. El sistema será inteligente, ya que ajustará la acústica del cuarto en función del instrumento que se desee tocar en el recinto. El sistema permitirá, además de tener una acústica adecuada para cada instrumento, poder realizar acondicionamientos acústicos de manera automática.
\\
\\
\textbf{Palabras clave:} paneles acústicos, estudio de audio, acondicionamiento acústico, acondicionamiento automático, acústica inteligente, recinto.
\addcontentsline{toc}{section}{Justificación}
\section*{Justificación}
Ethan Winer advierte en su libro “The Audio Expert” que por más que se inviertan miles o millones de dólares en los equipos más precisos y de alta calidad, con un pobre tratamiento acústico del cuarto en que estos operen, no serán capaces de alcanzar su máximo potencial. “Es difícil escuchar lo que estás haciendo, lo que dificulta crear un buen sonido” afirma el autor sobre los cuartos sin un tratamiento acústico adecuado \cite{Winer2017}.
\\
Al momento, los esfuerzos en la industria para lograr la modificación de variables acústicas se enfocan en mejorar los dispositivos y software existentes de procesamiento de audio \cite{Weir2012}. El sistema que proponemos se enfoca en presentar una solución fuera de la oferta actual y permitir que los ingenieros de audio tengan una alternativa que les permita lograr una acústica óptima y dinámica en sus salas de grabación, lo cual permitirá tratar la acústica del recinto como un instrumento más a la hora de hacer sus funciones diarias \cite{AudioHunt}, esta implementación de un nuevo elemento para el tratamiento acústico pronostica una mejora en la calidad de los resultados ligada a una experiencia de trabajo mejorada, al no tener que destinar tiempo en procesos actualmente automatizables por medio de un sistema mecatrónico.
\\
R. Israel Quintero Tiscareño, ingeniero en audio de Espiral Estéreo, nos regala su perspectiva sobre las ventajas que presenta un sistema como este y el impacto que tendría en su trabajo: \textit{“Las ventajas que veo en un sistema de tratamiento acústico dinámico son tan variadas cómo lo pueda ser el mismo sistema, debido a que aparte de ser móvil, puede tener dos o más caras en su movilidad y textura, debido a que la reflexión de cada cara puede tener materiales y por obvias razones, densidades y factores de absorción distintas, con esto sería muy enriquecedor poder adaptar o ajustar un solo espacio con la posibilidad de emular distintas salas de grabación y ejecución... para así ecualizar con base en la o las aplicaciones deseadas! Este sistema tendría un gran impacto en el negocio de los estudios de grabación, debido a que en Latinoamérica no existen o son contados los lugares que cuentan con un sistema parecido... pero sin duda alguna no igual al que se propone”.}
\\
La búsqueda de un sistema que pueda modificar de manera automática la acústica de un cuarto requiere de las diferentes habilidades del ingeniero mecatrónico. Se requiere de un sistema mecánico que posicione los paneles dentro del cuarto, para influenciar su acústica. Además, se necesita un sistema de control, que ayude en la precisión y repetibilidad de los movimientos de estos paneles para poder afirmar que se tiene una acústica controlable. El sistema debe contar con una interfaz computacional que le permita al usuario monitorizarlo y controlarlo. Por último, contiene una parte electrónica, la cual dirigirá la parte mecánica en función de las señales recibidas por la interfaz de usuario y procesadas por el sistema de control.
\\
Un sistema como el que se espera obtener forma parte de los esfuerzos de diferentes equipos de investigación para dar solución a este problema. Es un concepto que recién comienza a desarrollarse y el diseño e implementación de sistemas para este propósito, desde diferentes enfoques, ayudara con el desarrollo de soluciones cada vez más complejas, precisas y confiables.
\\
El sistema hará uso de todas las áreas de la mecatrónica, ya que conllevará la creación de una interfaz de usuario que deberá comunicarse efectivamente con el sistema, al mismo tiempo debe mostrar una gran cantidad de datos al usuario de manera digerible. Además, implicará el diseño y fabricación de los diferentes paneles acústicos en conjunto con un sistema modular que deberá moverlos por el cuarto de manera precisa, haciendo uso del control inteligente. Necesitará de un dispositivo electrónico embebido que sirva como puente entre la interfaz de usuario y el sistema, y deberá permitir que el usuario solo deba preocuparse por enchufarlo y conectarlo a su computadora. Aunado al conocimiento necesario de todas las aéreas de la mecatrónica, será necesario un conocimiento de los algoritmos de aprendizaje de máquina más allá del cubierto por plan de estudios, debido a la complejidad que conlleva la creación de una relación entre las posiciones de los paneles y la acústica del cuarto. Por último, será necesario un entendimiento profundo en temas de acústica, desde la concepción del proyecto hasta las pruebas finales, debido a que gran parte de la complejidad del diseño e implementación recae en esta área externa. Por todo lo anterior, la complejidad del proyecto demanda de 4 integrantes para su desarrollo, los integrantes del equipo se especializarán en un área sin perder atención en las demás áreas, pero sí fungirán como responsables directos. Aarón Barbosa estará a cargo del control, Alberto Camarena de la programación, Teddy Muñoz de la mecánica y Daniel Sánchez de la electrónica.
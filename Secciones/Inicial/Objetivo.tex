\addcontentsline{toc}{section}{Objetivos}
\section*{Objetivos}

\addcontentsline{toc}{subsection}{Objetivo general}
\subsection*{Objetivo general}

Diseñar e implementar un sistema que automatice el acondicionamiento acústico de un estudio de audio por medio del movimiento de paneles acústicos.
\addcontentsline{toc}{subsection}{Objetivos específicos}
\subsection*{Objetivos específicos}

\addcontentsline{toc}{subsubsection}{Objetivos de Trabajo Terminal I}
\subsubsection*{Objetivos de Trabajo Terminal I}
\begin{itemize}
    \item Diseñar un módulo de medición de la respuesta a un tren de pulsos de un estudio de audio a distintas frecuencias para caracterizar su acústica.
    \item Calcular el tiempo de reverberación, claridad, sound strength esperada, relación de energía lateral, relación de bajos, relación de agudos e inteligibilidad del habla; a partir de la respuesta a un tren de pulsos.
    \item Verificar la existencia de ondas estacionarias en el estudio de audio mediante el uso de las dimensiones del estudio de audio para su tratamiento. 
    \item Diseñar un módulo de modificación de la acústica mediante el movimiento de paneles acústicos de absorción, reflexión y difusión, en el techo y paredes, con control inteligente de posición para acondicionar la acústica de un estudio de audio a diferentes instrumentos.
    \item Creación de relaciones entre las posiciones de los paneles y la acústica del cuarto mediante algoritmos de aprendizaje de máquina para el control preciso de la acústica mediante el control de posición de los paneles.
    \item Validar los valores de los parámetros acústicos mediante simulaciones para comprobar los algoritmos de relación entre posición y acústica.
    \item Validar la modificación de la acústica mediante simulaciones para comprobar el correcto tratamiento acústico.
\end{itemize}

\addcontentsline{toc}{subsubsection}{Objetivos de Trabajo Terminal II}
\subsubsection*{Objetivos de Trabajo Terminal II}
\begin{itemize}
    \item Manufacturar e implementar el sistema energético, verificar su funcionamiento.
    \item Implementar un módulo de medición de la respuesta a un tren de pulsos del estudio de audio a distintas frecuencias para caracterizas su acústica.
    \item Implementar un módulo de modificación de la acústica mediante el movimiento de paneles acústicos de absorción, reflexión y difusión, en el techo y paredes, con control inteligente de posición para acondicionar la acústica de un estudio de audio a diferentes instrumentos.
    \item Implementar una plataforma de control y monitoreo del sistema para el usuario mediante software, que muestre los parámetros acústicos calculados, mediciones, posiciones de los paneles y toda la información asociada.
    \item Validar el sistema mecatrónico en el ambiente de prueba mediante pruebas de acondicionamiento para distintos instrumentos para la resolución de errores y comprobación del correcto funcionamiento.
\end{itemize}
\hfill \break

\addcontentsline{toc}{section}{Planteamiento del problema}
\section*{Planteamiento del problema}
Para modificar el ambiente acústico de un espacio cerrado, se utilizan múltiples técnicas y herramientas que modifican la reverberación, difracción, reflexión, resonancias y otros efectos acústicos. Algunos de estos métodos son el diseño de la sala, el aislamiento con el ambiente, la ventilación, el posicionamiento de los altavoces, etc. \cite{E-Home} dentro de los cuales, el tratamiento de las paredes es el artilugio más común que encontramos en los estudios de audio. Desde que entramos en estos espacios podemos ver que las paredes están cubiertas por una diversidad de materiales y formas irregulares, cuya función es alterar la reflexión del sonido y evitar la formación de ondas estacionarias \cite{Albano}. Los elementos de control acústicos que se emplean para este tratamiento son paneles de absorción, paneles difusores y trampas de graves \cite{Ervine}.
\\
El procedimiento para realizar la correcta gestión acústica de un espacio con la técnica de tratamiento en paredes inicia con el análisis de la sala, con esto se detectan los problemas acústicos que puede llegar a tener el recinto y el siguiente paso es diseñar las soluciones para estos problemas encontrados, lo que involucra la instalación de los elementos de control acústicos antes mencionados en puntos estratégicos. Por último, se vuelven a hacer mediciones y se ajustan las posiciones con base en los parámetros que se desea obtener \cite{Irwin}. 
\\
Esto supone un caso ideal, pero en la realidad, el problema es que debido al conocimiento especializado que se necesita tener sobre los cálculos acústicos y a la inaccesibilidad del instrumental con el que se realizan las mediciones, el proceso termina haciéndose a prueba y error con los elementos de control acústico en diversas posiciones hasta llegar a un resultado que se considere aceptable, lejos de ser óptimo \cite{E-Home}. Otro problema que se observa para este tratamiento es que la acústica tratada mediante paneles es una acústica estática que se diseña según el propósito del espacio. Para poder modificar las variables acústicas habría que regresar al paso anterior y volver a hacer el estudio por cada modificación que se desea hacer o función que se desea realizar en el espacio. 
\\
Para resolver estos problemas actualmente se utilizan equipos de procesamiento de señal como ecualizadores, compresores, reverbs, delays, filtros, moduladores, entre otros, en conjunto con software especializado, ya que es necesario tener la posibilidad de alterar las características del sonido y no siempre se cuenta con el espacio ideal para esto \cite{Pack}. Aunque el uso de los aparatos y el software puede ser útil, presentan las desventajas de introducir puntos de falla y limitaciones \cite{E-Home}. Como son la pérdida de calidad en el sonido o generar distorsiones según la calidad de los componentes, sumado a esto, agregan una capa de complejidad al proceso, ya que para su operación se requiere capacitación para poder operar, de la misma manera que para el manejo del software.
\\
En resumen, el tratamiento acústico por paneles que se genera en la mayoría de los casos es un proceso empírico, el cual está lleno de puntos de falla provocados por la complejidad misma del proceso, aunado a esto, incluso cuando se hace un tratamiento idea su condición es estática, por lo que la más pequeña desviación del rango de operación para el que fue diseñado el cuarto, termina nuevamente en un tratamiento acústico deficiente. El problema es la falta de un sistema automático que controle las variables acústicas sin importar la disposición o diseño del cuarto.
\\
Un sistema que sirva para dar solución a este problema solo es posible gracias a la sinergia de la mecánica, para las piezas físicas que se moverán, la electrónica, para todas las señales que se tiene medir, el control, que permitirá la regulación del comportamiento y la computación para una interfaz moderna y fácilmente operable por el usuario.
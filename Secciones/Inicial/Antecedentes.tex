\addcontentsline{toc}{section}{Antecedentes}
\section*{Antecedentes}

%------------------------------------------------------------------------

\begin{center}
\scriptsize
\begin{longtable}[!htb]{ | m{2em} | m{10em} | m{10em} | m{10em} | m{5em} | m{8em} |m{2em} |}
\hline
\textbf{No.} & \textbf{Nombre} & \textbf{Descripción} & \textbf{Características} & \textbf{País} & \textbf{Instituto} & \textbf{Ref} \\
\hline
 1. & Panel
Acústico Variable Rotatorio. (P.A.V.R.) & Diseño y construcción de un prototipo de panel acústico variable rotatorio de 360° con control autómatico.(P.A.V.R) & 
\begin{itemize}
    \item Difusión, reflexión o absorción del sonido.
    \item Tarjeta de desarrollo Arduino Uno como para la implementación del control.
    \item Medidas 113cmx113cm.
\end{itemize}
& Colombia & Universidad de San Buenaventura & \cite{O.J.Mantilla}\\
\hline
 2. & Acoustic Enclosure with Intelligent Controllable Noise Insulation & Recinto acústico inteligente para las máquinas que siempre están trabajando bajo cargas dinámicas. & 
 \begin{itemize}
    \item Aislamiento del sonido con diferentes anchos de banda.
    \item El sistema de control del aislamiento consiste en transductor, PLC y servo motor.
\end{itemize} 
& China  & Qingdao Technological University  & \cite{L.Sen}\\
\hline
 3. & Investigation of structural response and noise reduction of an acoustical enclosure using SEA method &  Modelo SEA mejorado que incluye la respuesta no resonante y el coeficiente de transmisión más preciso de paneles finitos, tomando en cuenta la comparación de los resultados medidos y la mejor concordancia entre la predicción de la respuesta estructural prevista y la reducción de ruido de un recinto acústico. & 
\begin{itemize}
    \item Recinto acústico de 200 $m^3$.
    \item Recinto acústico con muchos muebles y artículos disipadores de sonido.
    \item Altoparlante generador de energía acústica a altas frecuencias.
    \item Altoparlante generador de adecuada potencia de sonido a bajas y medias frecuencias.
\end{itemize} 
& Australia, China & University of westerns Australia; Northwestern Polytechnical University & \cite{Y.Lei2012}\\
\hline
  4. & Soundspheres: Resonant Chamber & El “Soundspheres” es un proyecto acústico que conecta las interrelaciones entre material, forma espacial y sonido. Su diseño se centra en tres tipos de capas: duras, estáticas e inflexibles; físicamente manipulables; y electroacústico. Su objetivo es desarrollar un sistema interior envolvente cinético e interactivo destinado a transformar el entorno acústico a través de la dinámica espacial, materiales y tecnologías electroacústicas.  & 
\begin{itemize}
    \item Geometrías de superficie dinámicas
    \item Actuación y respuesta variables
    \item Modular
\end{itemize} 
& Estados Unidos de América & University of Michigan & \cite{Thun2012} \\
\hline
\caption{Tabla de antecedentes}
\label{tab:Antecedentes}
\end{longtable}
\end{center}

%------------------------------------------------------------------------
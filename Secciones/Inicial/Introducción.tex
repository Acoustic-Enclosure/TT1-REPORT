\addcontentsline{toc}{section}{Introducción}
\section*{Introducción}
La grabación de sonidos puede ser un proceso complejo, pero desde que la humanidad comenzó a hacerlo a finales del siglo XIX con la invención del fonógrafo por Thomas Edison \cite{Library2016} no hemos parado de esforzarnos en hacerla cada vez mejor. Ser capaces de capturar el sonido, significa conservar y reproducir a placer momentos únicos, significativos y emocionantes, podemos, por ejemplo, revivir actuaciones en vivo, recordar la cálida voz de nuestros seres queridos, descubrir música, transportarnos al bullicio de la ciudad o al calmado rozar del viento bailando entre los árboles y conectar con los demás al compartir nuestras experiencias del mundo. Esta actividad relativamente moderna se convirtió rápidamente en una nueva forma de memoria en la vida de nuestra especie.
\\
Hoy en día, la grabación de sonido es una práctica común y se utiliza en una amplia variedad de contextos, desde la producción musical pasando por la grabación de discursos y entrevistas hasta fines de toda índole como la política, el entretenimiento, la educación, la comunicación, el arte, etc. En cada una de estas áreas el fenómeno de la grabación permitió una mayor experimentación y por ende innovación. 
\\
Para grabar el sonido se captan las vibraciones invisibles del aire y se almacenan en algún medio, para reproducir el sonido podemos simplemente decir que es el proceso inverso. Existen dos clases principales de tecnologías de grabación, la analógica y la digital. Para la grabación digital es necesario un micrófono, nuestro instrumento de captación de ondas sonoras que a su vez las convierte en señales eléctricas, una interfaz de audio, necesaria para convertir las señales analógicas del micrófono a señales digitales almacenables en la mayoría de los dispositivos electrónicos, y un dispositivo de grabación como la computadora. Para la grabación digital es necesario un micrófono, nuestro instrumento de captación de ondas sonoras que a su vez las convierte en señales eléctricas, una interfaz de audio, necesaria para convertir las señales analógicas del micrófono a señales digitales almacenables en la mayoría de los dispositivos electrónicos, y un dispositivo de grabación como la computadora.
\\
La obtención de una grabación de alta calidad es un proceso complejo que depende de múltiples factores, como lo son la selección del equipo, la elección del espacio, toman en cuanta incluso la habilidad técnica de los ingenieros de grabación. Los profesionales del sonido, conocidos como ingenieros de audio \cite{Sheffieldav}, son los responsables de asegurar la calidad del audio en grabaciones y actuaciones en vivo. Los ingenieros de audio deben de tener conocimientos en el uso del software y el equipo a emplear, además de habilidades artísticas como la composición y empleo de instrumentos musicales.
\\
Dependiendo del propósito de la grabación, el equipo y los recursos que se tengan disponibles hay varias opciones de lugares en los que se puede grabar audio. La más profesional y con resultados de calidad superior dado a su medio controlado es en un estudio de grabación, nombre engañoso ya que no solo se realizan grabaciones en estos lugares, pero son comúnmente conocidos por este nombre. Este ambiente de grabación ofrece varias ventajas de entre las cuales destaca la calidad del sonido obtenido; ya que los estudios de audio están diseñados para ofrecer una acústica controlada y cuentan con equipos avanzados de grabación, reproducción, mezcla, edición, preproducción, producción y postproducción. Los estudios de grabación impactan, por tanto, a las industrias de la música, el cine y la televisión, la publicidad, los videojuegos, la radio y toda aquella en la que se desee hacer uso de servicios especializados en audio.
\\
Al inicio los estudios de grabación eran instalaciones muy sencillas que se limitaban a aislar el ruido del exterior, pero no eran comúnmente empleados, lo común en la música, por nombrar un ejemplo, era hacer la grabación en la misma sala de conciertos o teatros, sin embargo, los ingenieros en audio no tardaron en darse cuenta de que estas grabaciones no eran ideales debido a la acústica de los espacios en los que se realizaban, y también debido a la imposibilidad de corregir errores o ajustar la mezcla de sonido en postproducción. Hasta que con los avances tecnológicos y la grabación digital estos espacios comenzaron a popularizarse. En resumen, los estudios de grabación se comenzaron a usar como una forma de mejorar la calidad de las grabaciones de música y otros sonidos, y han evolucionado a lo largo del tiempo para convertirse en lugares altamente especializados y sofisticados que ofrecen la mejor calidad posible en la grabación de sonido.
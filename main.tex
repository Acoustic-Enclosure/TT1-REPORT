\documentclass[letterpaper,12pt,oneside]{article}
\usepackage[top=1in,left=1in,right=1in,bottom=1in]{geometry}
\usepackage{graphicx,textcomp} % Required for inserting images
\usepackage{amssymb,amsmath} %Paquetes matemáticos de la American Mathematical Society
%----------------------------------------------------
\usepackage{titlesec}
\titleclass{\subsubsubsection}{straight}[\subsection]
\newcounter{subsubsubsection}[subsubsection]
\renewcommand\thesubsubsubsection{\thesubsubsection.\arabic{subsubsubsection}}
\renewcommand\theparagraph{\thesubsubsubsection.\arabic{paragraph}} % optional; useful if paragraphs are to be numbered
\titleformat{\subsubsubsection}
{\normalfont\normalsize\bfseries}{\thesubsubsubsection}{1em}{}
\titlespacing*{\subsubsubsection}
{0pt}{3.25ex plus 1ex minus .2ex}{1.5ex plus .2ex}
\makeatletter
\renewcommand\paragraph{\@startsection{paragraph}{5}{\z@}%
{3.25ex \@plus1ex \@minus.2ex}%
{-1em}%
{\normalfont\normalsize\bfseries}}
\renewcommand\subparagraph{\@startsection{subparagraph}{6}{\parindent}%
{3.25ex \@plus1ex \@minus .2ex}%
{-1em}%
{\normalfont\normalsize\bfseries}}
\def\toclevel@subsubsubsection{4}
\def\toclevel@paragraph{5}
\def\toclevel@paragraph{6}
\def\l@subsubsubsection{\@dottedtocline{4}{7em}{4em}}
\def\l@paragraph{\@dottedtocline{5}{10em}{5em}}
\def\l@subparagraph{\@dottedtocline{6}{14em}{6em}}
\makeatother
\setcounter{secnumdepth}{4}
\setcounter{tocdepth}{4}
%----------------------------------------------------
\usepackage[english,spanish,es-tabla]{babel}
\usepackage[utf8]{inputenc}
%----------------------------------------------------
\usepackage{tabto}
\usepackage{enumerate}
\usepackage{array}
\usepackage{soul}
\usepackage{booktabs}
\usepackage{hyperref}
\usepackage{subcaption}
\usepackage{gensymb}
% \usepackage{breqn} Este es el del problema
\usepackage{longtable}
\usepackage{enumitem}

% -----------AGREGAR MÁS SUBSECCIONES----------------
\setcounter{tocdepth}{5}
%--------------BIBLIOGRAFÍA--------------------------
\usepackage[backend=biber,style=ieee,sorting=none]{biblatex}
\addbibresource{Referencias/predoc.bib}
%----------------------------------------------------
\usepackage{xcolor,colortbl}
\usepackage{csquotes}
\usepackage{multirow}
\usepackage{multicol}
\usepackage{pdflscape} % Para cambiar a modo paisaje
\usepackage{lipsum} % Para generar texto de relleno
\usepackage{placeins}
%----------------------------------------------------
\usepackage{fancyhdr}
\usepackage{listings}
\usepackage{matlab-prettifier}

\definecolor{codegreen}{rgb}{0,0.6,0}
\definecolor{codegray}{rgb}{0.5,0.5,0.5}
\definecolor{codepurple}{rgb}{0.58,0,0.82}
\definecolor{backcolour}{rgb}{0.95,0.95,0.92}

\lstdefinestyle{mystyle}{
    backgroundcolor=\color{backcolour},   
    commentstyle=\color{codegreen},
    keywordstyle=\color{magenta},
    numberstyle=\tiny\color{codegray},
    stringstyle=\color{codepurple},
    basicstyle=\ttfamily\footnotesize,
    breakatwhitespace=false,         
    breaklines=true,                 
    captionpos=b,                    
    keepspaces=true,                 
    numbers=left,                    
    numbersep=5pt,                  
    showspaces=false,                
    showstringspaces=false,
    showtabs=false,                  
    tabsize=2
}
%-------------------------------------------------------
\lstset{style=mystyle}
\renewcommand{\lstlistingname}{Código}% Listing -> Algorithm
\renewcommand{\lstlistlistingname}{List of \lstlistingname s}% List of Listings -> List of Algorithms
%-------------------------------------------------------
\pagestyle{fancy}
\fancyhead[L]{\footnotesize UPIITA}
\fancyhead[R]{\footnotesize IPN}
\fancyfoot[R]{\footnotesize Trabajo Terminal I}
\fancyfoot[C]{\thepage}
\fancyfoot[L]{\footnotesize Ing. Mecatrónica}
\renewcommand{\footrulewidth}{0.4pt}
%-------------------------------------------------------

\begin{document}

%-------------Preeliminar-------------
\begin{titlepage}
  \thispagestyle{empty}
  \begin{minipage}[c][0.17\textheight][c]{0.25\textwidth}
    \begin{center}
      \includegraphics[ height=4cm]{imagenes/ipn.png}
    \end{center}
  \end{minipage}
  \begin{minipage}[c][0.195\textheight][t]{0.75\textwidth}
    \begin{center}
      \vspace{0.3cm}
             {\textsc{\large INSTITUTO POLITÉCNICO NACIONAL} }\\[0.5cm]
             \vspace{0.3cm}
                    {\color{purple}\hrule height2.5pt}
                    \vspace{.2cm}
                           {\color{purple}\hrule height1pt}
                           \vspace{.8cm}
                           \textsc{unidad profesional interdisciplinaria en ingeniería y tecnologías avanzadas}\\[0.6cm] %
    \end{center}
  \end{minipage}
  \begin{minipage}[c][0.81\textheight][t]{0.25\textwidth}
    \vspace*{5mm}
    \begin{center}
      \hskip2.0mm
             {\color{red}\vrule width1pt height13cm }
             \vspace{5mm}
             \hskip2pt
                 {\color{red}\vrule width2.5pt height13cm}
                 \hskip2mm
                     {\color{red}\vrule width1pt height13cm} \\
                     \vspace{5mm}
                     \includegraphics[height=3.0cm]{imagenes/upiita.png}
    \end{center}
  \end{minipage}
  \begin{minipage}[c][0.81\textheight][t]{0.75\textwidth}
    \begin{center}
      \vspace{0.6cm}
      
        
      {
       % {\large\scshape Tercer reporte mensual  }\\[0.2cm]
        {\large\scshape Trabajo Terminal I }
      }\\[0.4cm]

      \vspace{0.8cm}            

      \textsc{\LARGE Sistema automatizado para el acondicionamiento acústico de un estudio de audio por medio del movimiento de paneles acústicos }\\[1cm]
      \textsc{\large que para obtener el t\'itulo de:}\\[0.3cm]
      \textsc{\large Ingeniero en Mecatrónica}\\[0.6cm]
      
      {\textsc{\large presentan:}}\\[0.3cm]
      \textsc{\large {Barbosa Mercado José Aarón}}\\
      \textsc{\large {Camarena Rodríguez Alberto }}\\
      \textsc{\large {Muñoz Ceballos Teddy Xavier}}\\
      \textsc{\large {Sánchez Trujillo Daniel}}\\[0.3cm]  

      \vspace{0.5cm}

      {\large\scshape 
        {Asesores:}\\[0.3cm] {Ing. Erick López Alarcón\\ 
         Dr. Alberto Luviano Juárez \\ Dr. Rafael Trovamala Landa}}\\[.2in]

      \vspace{0.5cm}

      \large{Estados Unidos Mexicanos\\ 
        Ciudad de México\\
        2024}
    \end{center}
  \end{minipage}
\end{titlepage}
%---------------------------------
\thispagestyle{empty}

\textcolor[rgb]{1.00,1.00,1.00}{palabra} % Pinta "palabra" de blanco para lograr anexar paginas en blanco

\begin{titlepage}
  \thispagestyle{empty}
  \begin{minipage}[c][0.17\textheight][c]{0.25\textwidth}
    \begin{center}
      \includegraphics[ height=4cm]{imagenes/ipn.png}
    \end{center}
  \end{minipage}
  \begin{minipage}[c][0.195\textheight][t]{0.75\textwidth}
    \begin{center}
      \vspace{0.3cm}
             {\textsc{\large INSTITUTO POLITÉCNICO NACIONAL} }\\[0.5cm]
             \vspace{0.3cm}
                    {\color{purple}\hrule height2.5pt}
                    \vspace{.2cm}
                           {\color{purple}\hrule height1pt}
                           \vspace{.8cm}
                           \textsc{unidad profesional interdisciplinaria en ingeniería y tecnologías avanzadas}\\[0.2cm] %
    \end{center}
  \end{minipage}
  \begin{minipage}[c][0.81\textheight][t]{0.25\textwidth}
    \vspace*{5mm}
    \begin{center}
      \hskip2.0mm
             {\color{red}\vrule width1pt height13cm }
             \vspace{5mm}
             \hskip2pt
                 {\color{red}\vrule width2.5pt height13cm}
                 \hskip2mm
                     {\color{red}\vrule width1pt height13cm} \\
                     \vspace{5mm}
                     \includegraphics[height=3.0cm]{imagenes/upiita.png}
    \end{center}
  \end{minipage}
  \begin{minipage}[c][0.81\textheight][t]{0.75\textwidth}
    \begin{center}
      \vspace{0.6cm}

      {
       % {\large\scshape Tercer reporte mensual  }\\[0.2cm]
        {\large\scshape Trabajo Terminal II }
      }\\[0.4cm]

      \vspace{0.4cm}            

      \textsc{\LARGE diseño de estructuras de prueba de sistemas micro electro-mecánicos para la caracterización del esfuerzo residual }\\[.8cm]
      \textsc{\large Presentan:}\\[0.4cm]
      
      %Modificaciones para firmas
      \begin{multicols}{2}
       \rule{50mm}{0.1mm}
       \textsc{ Rosendo Valdés Hernández}\\ %Columna 1
       \rule{50mm}{0.1mm}
       \textsc{ Luis Mario Trejo Hinojosa }\\[0.3cm]  %Columna 2
      \end{multicols}
     
      \vspace{0.5cm}
        {\large\scshape 
      
        {Asesores:}\\[0.3cm] 
        
        \begin{multicols}{2}
            \rule{50mm}{0.1mm}
            \textsc{ \normalsize DR. Héctor Báez Medina}\\[0.1cm] %Columna 1
            \rule{50mm}{0.1mm}
            \textsc{ \normalsize M.en C. Luis Alejandro Barranco Juárez}\\  %Columna 2
        \end{multicols}
        
            \rule{50mm}{0.1mm}\\
            \textsc{ \normalsize Dr. Aarón Israel Díaz Cano}
        }
      \vspace{0.5cm}
      
      \begin{multicols}{2}
            \textbf{Presidente del jurado}\\[0.8cm]
            \rule{50mm}{0.1mm}
            \textsc{ \normalsize \quad Dr. Brahim El Filali}\\[0.2cm] %Columna 1
            
            \textbf{Profesor titular}\\[0.8cm]
            \rule{50mm}{0.1mm}
            \textsc{ \normalsize Dr.en C. Juan Luis Mata Machuca}\\  %Columna 2
      \end{multicols}
      
    \end{center}
  \end{minipage}
\end{titlepage}
%---------------------------------
\thispagestyle{empty}

\textcolor[rgb]{1.00,1.00,1.00}{palabra} % Pinta "palabra" de blanco para lograr anexar paginas en blanco

\tableofcontents
% \addtocontents{toc}{~\hfill\textbf{Página}\par}
\addtocontents{toc}{~\hfill\textbf{Página}\par}
\renewcommand{\listfigurename}{Figuras}
\renewcommand{\listtablename}{Tablas}
\listoffigures
\listoftables
\section*{Resumen}
Para lograr una acústica de adecuada calidad en los estudios de audio, se suele recurrir al uso de paneles acústicos, los cuales modifican las ondas sonoras de un cuarto que no estaba inicialmente diseñado para con propósito de grabar audio, u cualquier otra función en donde la calidad sonora sea de importancia. Sin embargo, y a pesar de que el campo de la acústica está profundamente estudiado, la instalación de paneles acústicos sigue siendo un proceso mayormente empírico y que además está limitado, lo anterior debido a que diferentes instrumentos requieren una acústica diferente, y el acercamiento de los paneles es buscar una acústica “aceptable” para un amplio rango de instrumentos. Este proyecto tiene como objetivo el diseño e implementación de un sistema que automatice el acondicionamiento acústico en un estudio de audio por medio del movimiento de paneles acústicos. Para este fin, nuestra propuesta consiste en un sistema de paneles acústicos móviles, que cambien la superficie que está en contacto directo con las ondas generadas en el recinto y que cambien la respuesta de este. El sistema será inteligente, ya que ajustará la acústica del cuarto en función del instrumento que se desee tocar en el recinto. El sistema permitirá, además de tener una acústica adecuada para cada instrumento, poder realizar acondicionamientos acústicos de manera automática.
\\
\\
\textbf{Palabras clave:} paneles acústicos, estudio de audio, acondicionamiento acústico, acondicionamiento automático, acústica inteligente, recinto.
\section*{Abstract}
To achieve adequate quality acoustics in audio studios, it is often resorted to the use of acoustic panels, which modify the sound waves of a room that was not initially designed for the purpose of recording audio, or any other function where sound quality is important. However, even though the field of acoustics is deeply studied, the installation of acoustic panels is still a largely empirical and limited process, because different instruments require different acoustics, and the approach of the panels is to seek ``acceptable'' acoustics for a wide range of instruments. This project aims to design and implement a system that automates the acoustic conditioning in enclosures through the movement of sound modification panels. For this purpose, our proposal consists of a system of moving acoustic panels, which change the surface that is in direct contact with the waves generated in the audio studio and change the response of the enclosure. The system will be intelligent, as it will adjust the acoustics of the room depending on the instrument to be played in it. The system will allow, in addition to having a suitable acoustic for each instrument, to be able to perform acoustic conditioning automatically.
\\
\\
\textbf{Key words:} acoustic panels, enclosure, acoustic treatment, automatic treatment, intelligent acoustics, audio studio.

%---------------Inicial---------------
\addcontentsline{toc}{section}{Introducción}
\section*{Introducción}
La grabación de sonidos puede ser un proceso complejo, pero desde que la humanidad comenzó a hacerlo a finales del siglo XIX con la invención del fonógrafo por Thomas Edison \cite{Library2016} no hemos parado de esforzarnos en hacerla cada vez mejor. Ser capaces de capturar el sonido, significa conservar y reproducir a placer momentos únicos, significativos y emocionantes, podemos, por ejemplo, revivir actuaciones en vivo, recordar la cálida voz de nuestros seres queridos, descubrir música, transportarnos al bullicio de la ciudad o al calmado rozar del viento bailando entre los árboles y conectar con los demás al compartir nuestras experiencias del mundo. Esta actividad relativamente moderna se convirtió rápidamente en una nueva forma de memoria en la vida de nuestra especie.
\\
Hoy en día, la grabación de sonido es una práctica común y se utiliza en una amplia variedad de contextos, desde la producción musical pasando por la grabación de discursos y entrevistas hasta fines de toda índole como la política, el entretenimiento, la educación, la comunicación, el arte, etc. En cada una de estas áreas el fenómeno de la grabación permitió una mayor experimentación y por ende innovación. 
\\
Para grabar el sonido se captan las vibraciones invisibles del aire y se almacenan en algún medio, para reproducir el sonido podemos simplemente decir que es el proceso inverso. Existen dos clases principales de tecnologías de grabación, la analógica y la digital. Para la grabación digital es necesario un micrófono, nuestro instrumento de captación de ondas sonoras que a su vez las convierte en señales eléctricas, una interfaz de audio, necesaria para convertir las señales analógicas del micrófono a señales digitales almacenables en la mayoría de los dispositivos electrónicos, y un dispositivo de grabación como la computadora. Para la grabación digital es necesario un micrófono, nuestro instrumento de captación de ondas sonoras que a su vez las convierte en señales eléctricas, una interfaz de audio, necesaria para convertir las señales analógicas del micrófono a señales digitales almacenables en la mayoría de los dispositivos electrónicos, y un dispositivo de grabación como la computadora.
\\
La obtención de una grabación de alta calidad es un proceso complejo que depende de múltiples factores, como lo son la selección del equipo, la elección del espacio, toman en cuanta incluso la habilidad técnica de los ingenieros de grabación. Los profesionales del sonido, conocidos como ingenieros de audio \cite{Sheffieldav}, son los responsables de asegurar la calidad del audio en grabaciones y actuaciones en vivo. Los ingenieros de audio deben de tener conocimientos en el uso del software y el equipo a emplear, además de habilidades artísticas como la composición y empleo de instrumentos musicales.
\\
Dependiendo del propósito de la grabación, el equipo y los recursos que se tengan disponibles hay varias opciones de lugares en los que se puede grabar audio. La más profesional y con resultados de calidad superior dado a su medio controlado es en un estudio de grabación, nombre engañoso ya que no solo se realizan grabaciones en estos lugares, pero son comúnmente conocidos por este nombre. Este ambiente de grabación ofrece varias ventajas de entre las cuales destaca la calidad del sonido obtenido; ya que los estudios de audio están diseñados para ofrecer una acústica controlada y cuentan con equipos avanzados de grabación, reproducción, mezcla, edición, preproducción, producción y postproducción. Los estudios de grabación impactan, por tanto, a las industrias de la música, el cine y la televisión, la publicidad, los videojuegos, la radio y toda aquella en la que se desee hacer uso de servicios especializados en audio.
\\
Al inicio los estudios de grabación eran instalaciones muy sencillas que se limitaban a aislar el ruido del exterior, pero no eran comúnmente empleados, lo común en la música, por nombrar un ejemplo, era hacer la grabación en la misma sala de conciertos o teatros, sin embargo, los ingenieros en audio no tardaron en darse cuenta de que estas grabaciones no eran ideales debido a la acústica de los espacios en los que se realizaban, y también debido a la imposibilidad de corregir errores o ajustar la mezcla de sonido en postproducción. Hasta que con los avances tecnológicos y la grabación digital estos espacios comenzaron a popularizarse. En resumen, los estudios de grabación se comenzaron a usar como una forma de mejorar la calidad de las grabaciones de música y otros sonidos, y han evolucionado a lo largo del tiempo para convertirse en lugares altamente especializados y sofisticados que ofrecen la mejor calidad posible en la grabación de sonido.
\addcontentsline{toc}{section}{Planteamiento del problema}
\section*{Planteamiento del problema}
Para modificar el ambiente acústico de un espacio cerrado, se utilizan múltiples técnicas y herramientas que modifican la reverberación, difracción, reflexión, resonancias y otros efectos acústicos. Algunos de estos métodos son el diseño de la sala, el aislamiento con el ambiente, la ventilación, el posicionamiento de los altavoces, etc. \cite{E-Home} dentro de los cuales, el tratamiento de las paredes es el artilugio más común que encontramos en los estudios de audio. Desde que entramos en estos espacios podemos ver que las paredes están cubiertas por una diversidad de materiales y formas irregulares, cuya función es alterar la reflexión del sonido y evitar la formación de ondas estacionarias \cite{Albano}. Los elementos de control acústicos que se emplean para este tratamiento son paneles de absorción, paneles difusores y trampas de graves \cite{Ervine}.
\\
El procedimiento para realizar la correcta gestión acústica de un espacio con la técnica de tratamiento en paredes inicia con el análisis de la sala, con esto se detectan los problemas acústicos que puede llegar a tener el recinto y el siguiente paso es diseñar las soluciones para estos problemas encontrados, lo que involucra la instalación de los elementos de control acústicos antes mencionados en puntos estratégicos. Por último, se vuelven a hacer mediciones y se ajustan las posiciones con base en los parámetros que se desea obtener \cite{Irwin}. 
\\
Esto supone un caso ideal, pero en la realidad, el problema es que debido al conocimiento especializado que se necesita tener sobre los cálculos acústicos y a la inaccesibilidad del instrumental con el que se realizan las mediciones, el proceso termina haciéndose a prueba y error con los elementos de control acústico en diversas posiciones hasta llegar a un resultado que se considere aceptable, lejos de ser óptimo \cite{E-Home}. Otro problema que se observa para este tratamiento es que la acústica tratada mediante paneles es una acústica estática que se diseña según el propósito del espacio. Para poder modificar las variables acústicas habría que regresar al paso anterior y volver a hacer el estudio por cada modificación que se desea hacer o función que se desea realizar en el espacio. 
\\
Para resolver estos problemas actualmente se utilizan equipos de procesamiento de señal como ecualizadores, compresores, reverbs, delays, filtros, moduladores, entre otros, en conjunto con software especializado, ya que es necesario tener la posibilidad de alterar las características del sonido y no siempre se cuenta con el espacio ideal para esto \cite{Pack}. Aunque el uso de los aparatos y el software puede ser útil, presentan las desventajas de introducir puntos de falla y limitaciones \cite{E-Home}. Como son la pérdida de calidad en el sonido o generar distorsiones según la calidad de los componentes, sumado a esto, agregan una capa de complejidad al proceso, ya que para su operación se requiere capacitación para poder operar, de la misma manera que para el manejo del software.
\\
En resumen, el tratamiento acústico por paneles que se genera en la mayoría de los casos es un proceso empírico, el cual está lleno de puntos de falla provocados por la complejidad misma del proceso, aunado a esto, incluso cuando se hace un tratamiento idea su condición es estática, por lo que la más pequeña desviación del rango de operación para el que fue diseñado el cuarto, termina nuevamente en un tratamiento acústico deficiente. El problema es la falta de un sistema automático que controle las variables acústicas sin importar la disposición o diseño del cuarto.
\\
Un sistema que sirva para dar solución a este problema solo es posible gracias a la sinergia de la mecánica, para las piezas físicas que se moverán, la electrónica, para todas las señales que se tiene medir, el control, que permitirá la regulación del comportamiento y la computación para una interfaz moderna y fácilmente operable por el usuario.
\addcontentsline{toc}{section}{Justificación}
\section*{Justificación}
Ethan Winer advierte en su libro “The Audio Expert” que por más que se inviertan miles o millones de dólares en los equipos más precisos y de alta calidad, con un pobre tratamiento acústico del cuarto en que estos operen, no serán capaces de alcanzar su máximo potencial. “Es difícil escuchar lo que estás haciendo, lo que dificulta crear un buen sonido” afirma el autor sobre los cuartos sin un tratamiento acústico adecuado \cite{Winer2017}.
\\
Al momento, los esfuerzos en la industria para lograr la modificación de variables acústicas se enfocan en mejorar los dispositivos y software existentes de procesamiento de audio \cite{Weir2012}. El sistema que proponemos se enfoca en presentar una solución fuera de la oferta actual y permitir que los ingenieros de audio tengan una alternativa que les permita lograr una acústica óptima y dinámica en sus salas de grabación, lo cual permitirá tratar la acústica del recinto como un instrumento más a la hora de hacer sus funciones diarias \cite{AudioHunt}, esta implementación de un nuevo elemento para el tratamiento acústico pronostica una mejora en la calidad de los resultados ligada a una experiencia de trabajo mejorada, al no tener que destinar tiempo en procesos actualmente automatizables por medio de un sistema mecatrónico.
\\
R. Israel Quintero Tiscareño, ingeniero en audio de Espiral Estéreo, nos regala su perspectiva sobre las ventajas que presenta un sistema como este y el impacto que tendría en su trabajo: \textit{“Las ventajas que veo en un sistema de tratamiento acústico dinámico son tan variadas cómo lo pueda ser el mismo sistema, debido a que aparte de ser móvil, puede tener dos o más caras en su movilidad y textura, debido a que la reflexión de cada cara puede tener materiales y por obvias razones, densidades y factores de absorción distintas, con esto sería muy enriquecedor poder adaptar o ajustar un solo espacio con la posibilidad de emular distintas salas de grabación y ejecución... para así ecualizar con base en la o las aplicaciones deseadas! Este sistema tendría un gran impacto en el negocio de los estudios de grabación, debido a que en Latinoamérica no existen o son contados los lugares que cuentan con un sistema parecido... pero sin duda alguna no igual al que se propone”.}
\\
La búsqueda de un sistema que pueda modificar de manera automática la acústica de un cuarto requiere de las diferentes habilidades del ingeniero mecatrónico. Se requiere de un sistema mecánico que posicione los paneles dentro del cuarto, para influenciar su acústica. Además, se necesita un sistema de control, que ayude en la precisión y repetibilidad de los movimientos de estos paneles para poder afirmar que se tiene una acústica controlable. El sistema debe contar con una interfaz computacional que le permita al usuario monitorizarlo y controlarlo. Por último, contiene una parte electrónica, la cual dirigirá la parte mecánica en función de las señales recibidas por la interfaz de usuario y procesadas por el sistema de control.
\\
Un sistema como el que se espera obtener forma parte de los esfuerzos de diferentes equipos de investigación para dar solución a este problema. Es un concepto que recién comienza a desarrollarse y el diseño e implementación de sistemas para este propósito, desde diferentes enfoques, ayudara con el desarrollo de soluciones cada vez más complejas, precisas y confiables.
\\
El sistema hará uso de todas las áreas de la mecatrónica, ya que conllevará la creación de una interfaz de usuario que deberá comunicarse efectivamente con el sistema, al mismo tiempo debe mostrar una gran cantidad de datos al usuario de manera digerible. Además, implicará el diseño y fabricación de los diferentes paneles acústicos en conjunto con un sistema modular que deberá moverlos por el cuarto de manera precisa, haciendo uso del control inteligente. Necesitará de un dispositivo electrónico embebido que sirva como puente entre la interfaz de usuario y el sistema, y deberá permitir que el usuario solo deba preocuparse por enchufarlo y conectarlo a su computadora. Aunado al conocimiento necesario de todas las aéreas de la mecatrónica, será necesario un conocimiento de los algoritmos de aprendizaje de máquina más allá del cubierto por plan de estudios, debido a la complejidad que conlleva la creación de una relación entre las posiciones de los paneles y la acústica del cuarto. Por último, será necesario un entendimiento profundo en temas de acústica, desde la concepción del proyecto hasta las pruebas finales, debido a que gran parte de la complejidad del diseño e implementación recae en esta área externa. Por todo lo anterior, la complejidad del proyecto demanda de 4 integrantes para su desarrollo, los integrantes del equipo se especializarán en un área sin perder atención en las demás áreas, pero sí fungirán como responsables directos. Aarón Barbosa estará a cargo del control, Alberto Camarena de la programación, Teddy Muñoz de la mecánica y Daniel Sánchez de la electrónica.
\addcontentsline{toc}{section}{Objetivos}
\section*{Objetivos}

\addcontentsline{toc}{subsection}{Objetivo general}
\subsection*{Objetivo general}

Diseñar e implementar un sistema que automatice el acondicionamiento acústico de un estudio de audio por medio del movimiento de paneles acústicos.
\addcontentsline{toc}{subsection}{Objetivos específicos}
\subsection*{Objetivos específicos}

\addcontentsline{toc}{subsubsection}{Objetivos de Trabajo Terminal I}
\subsubsection*{Objetivos de Trabajo Terminal I}
\begin{itemize}
    \item Diseñar un módulo de medición de la respuesta a un tren de pulsos de un estudio de audio a distintas frecuencias para caracterizar su acústica.
    \item Calcular el tiempo de reverberación, claridad, sound strength esperada, relación de energía lateral, relación de bajos, relación de agudos e inteligibilidad del habla; a partir de la respuesta a un tren de pulsos.
    \item Verificar la existencia de ondas estacionarias en el estudio de audio mediante el uso de las dimensiones del estudio de audio para su tratamiento. 
    \item Diseñar un módulo de modificación de la acústica mediante el movimiento de paneles acústicos de absorción, reflexión y difusión, en el techo y paredes, con control inteligente de posición para acondicionar la acústica de un estudio de audio a diferentes instrumentos.
    \item Creación de relaciones entre las posiciones de los paneles y la acústica del cuarto mediante algoritmos de aprendizaje de máquina para el control preciso de la acústica mediante el control de posición de los paneles.
    \item Validar los valores de los parámetros acústicos mediante simulaciones para comprobar los algoritmos de relación entre posición y acústica.
    \item Validar la modificación de la acústica mediante simulaciones para comprobar el correcto tratamiento acústico.
\end{itemize}

\addcontentsline{toc}{subsubsection}{Objetivos de Trabajo Terminal II}
\subsubsection*{Objetivos de Trabajo Terminal II}
\begin{itemize}
    \item Manufacturar e implementar el sistema energético, verificar su funcionamiento.
    \item Implementar un módulo de medición de la respuesta a un tren de pulsos del estudio de audio a distintas frecuencias para caracterizas su acústica.
    \item Implementar un módulo de modificación de la acústica mediante el movimiento de paneles acústicos de absorción, reflexión y difusión, en el techo y paredes, con control inteligente de posición para acondicionar la acústica de un estudio de audio a diferentes instrumentos.
    \item Implementar una plataforma de control y monitoreo del sistema para el usuario mediante software, que muestre los parámetros acústicos calculados, mediciones, posiciones de los paneles y toda la información asociada.
    \item Validar el sistema mecatrónico en el ambiente de prueba mediante pruebas de acondicionamiento para distintos instrumentos para la resolución de errores y comprobación del correcto funcionamiento.
\end{itemize}
\hfill \break

\addcontentsline{toc}{section}{Antecedentes}
\section*{Antecedentes}

%------------------------------------------------------------------------

\begin{center}
\scriptsize
\begin{longtable}[!htb]{ | m{2em} | m{10em} | m{10em} | m{10em} | m{5em} | m{8em} |m{2em} |}
\hline
\textbf{No.} & \textbf{Nombre} & \textbf{Descripción} & \textbf{Características} & \textbf{País} & \textbf{Instituto} & \textbf{Ref} \\
\hline
 1. & Panel
Acústico Variable Rotatorio. (P.A.V.R.) & Diseño y construcción de un prototipo de panel acústico variable rotatorio de 360° con control autómatico.(P.A.V.R) & 
\begin{itemize}
    \item Difusión, reflexión o absorción del sonido.
    \item Tarjeta de desarrollo Arduino Uno como para la implementación del control.
    \item Medidas 113cmx113cm.
\end{itemize}
& Colombia & Universidad de San Buenaventura & \cite{O.J.Mantilla}\\
\hline
 2. & Acoustic Enclosure with Intelligent Controllable Noise Insulation & Recinto acústico inteligente para las máquinas que siempre están trabajando bajo cargas dinámicas. & 
 \begin{itemize}
    \item Aislamiento del sonido con diferentes anchos de banda.
    \item El sistema de control del aislamiento consiste en transductor, PLC y servo motor.
\end{itemize} 
& China  & Qingdao Technological University  & \cite{L.Sen}\\
\hline
 3. & Investigation of structural response and noise reduction of an acoustical enclosure using SEA method &  Modelo SEA mejorado que incluye la respuesta no resonante y el coeficiente de transmisión más preciso de paneles finitos, tomando en cuenta la comparación de los resultados medidos y la mejor concordancia entre la predicción de la respuesta estructural prevista y la reducción de ruido de un recinto acústico. & 
\begin{itemize}
    \item Recinto acústico de 200 $m^3$.
    \item Recinto acústico con muchos muebles y artículos disipadores de sonido.
    \item Altoparlante generador de energía acústica a altas frecuencias.
    \item Altoparlante generador de adecuada potencia de sonido a bajas y medias frecuencias.
\end{itemize} 
& Australia, China & University of westerns Australia; Northwestern Polytechnical University & \cite{Y.Lei2012}\\
\hline
  4. & Soundspheres: Resonant Chamber & El “Soundspheres” es un proyecto acústico que conecta las interrelaciones entre material, forma espacial y sonido. Su diseño se centra en tres tipos de capas: duras, estáticas e inflexibles; físicamente manipulables; y electroacústico. Su objetivo es desarrollar un sistema interior envolvente cinético e interactivo destinado a transformar el entorno acústico a través de la dinámica espacial, materiales y tecnologías electroacústicas.  & 
\begin{itemize}
    \item Geometrías de superficie dinámicas
    \item Actuación y respuesta variables
    \item Modular
\end{itemize} 
& Estados Unidos de América & University of Michigan & \cite{Thun2012} \\
\hline
\caption{Tabla de antecedentes}
\label{tab:Antecedentes}
\end{longtable}
\end{center}

%------------------------------------------------------------------------

%---------------Media 1---------------
\section{Marco de Referencia}
\subsection{Marco Teórico}

\subsubsection{Sonido.} El sonido se puede definir de formas muy diversas. De todas ellas, las más habituales 
son las siguientes:

\begin{itemize}
    \item 
    Vibración mecánica que se propaga a través de un medio (habitualmente el aire), y que es 
    capaz de producir una sensación auditiva. 
    \item
    Sensación auditiva producida por una vibración de carácter mecánico que se propaga a través de un medio \cite{Isbert1998}.   
\end{itemize}

\subsubsection{Generación y propagación del sonido.} El elemento generador del sonido se denomina 
fuente sonora (tambor, cuerda de un violín, cuerdas vocales, etc.). La generación del sonido 
tiene lugar cuando dicha fuente entra en vibración. Dicha vibración es transmitida a las 
partículas de aire adyacentes a la misma que, a su vez, la transmiten a nuevas partículas 
contiguas.

Las partículas no se desplazan con la perturbación, sino que simplemente oscilan alrededor 
de su posición de equilibrio. La manera en que la perturbación se traslada de un lugar a otro 
se denomina propagación de la onda sonora. La oscilación de las partículas tiene lugar en la 
misma dirección que la de propagación de la onda \cite{Isbert1998}.

Ahora nos preguntamos qué tan rápido se aleja la onda de la fuente. La respuesta es que el 
sonido se propaga con una velocidad c que en el aire a 23ºC vale
\begin{equation}
c = 345 \quad\text{m/s,}
\end{equation}
o bien 
\begin{equation}
c = 1242 \quad\text{km/h,}
\end{equation}
Esta velocidad varía algo con la temperatura (un \textbf{0.17} $\% / ^{\circ} C$),  por eso en diversos textos pueden encontrarse valores ligeramente diferentes \cite{Miyara2004}. 

\subsubsection{Presión sonora.} La manera más habitual de expresar cuantitativamente la magnitud de un campo sonoro es mediante la presión sonora, o fuerza que ejercen las partículas de aire por unidad de superficie \cite{Isbert1998}.

Si nos ubicamos en una posición fija, veremos que la presión atmosférica aumenta y disminuye periódicamente, conforme pasan por el lugar las sucesivas perturbaciones. Dado que nos referiremos bastante seguido a valores de presión, conviene aclarar que la unidad adoptada internacionalmente para la presión es el Pascal, abreviada Pa. Expresada en esta unidad, la presión atmosférica es del orden de 100,000 Pa. Los aumentos y las disminuciones de presión debidas a las ondas sonoras son realmente muy pequeños comparados con este valor de presión atmosférica. Los sonidos más intensos que se perciben como tales implican un aumento de unos 20 Pa. La presión sonora es lo que se debe agregar a la presión atmosférica en reposo para obtener el valor real de presión atmosférica. Las presiones sonoras audibles varían entre 0,00002 Pa y 20 Pa. El valor más pequeño, también expresado como 20 µPa se denomina umbral auditivo \cite{Miyara2004}. 

\subsubsection{Tren de pulsos.} Es una variante de la onda cuadrada en el cual el tiempo de permanencia en cada uno de los dos niveles no es el mismo. Se suele especificar un porcentaje que corresponde a la porción del periodo en el nivel alto \cite{Miyara2004}.

\subsubsection{Acústica.} Es la disciplina que se ocupa para estudiar el sonido en sus diversos aspectos. Se puede dividir en una gran cantidad de sub disciplinas \cite{Miyara2004}, algunas de las cuales son la acústica física, la psico acústica, la acústica musical y en la cual nos centraremos nosotros, es la acústica arquitectónica. 

\subsubsection{Acústica Arquitectónica.} Estudia los fenómenos vinculados con una propagación adecuada, fiel y funcional del sonido en un recinto. Las habitaciones o salas dedicadas a una aplicación determinada deben tener cualidades acústicas adecuadas para dicha aplicación. Por cualidades acústicas de un recinto entendemos una serie de propiedades relacionadas con el comportamiento del sonido en el recinto, entre las cuales se encuentran las reflexiones tempranas, la reverberación, la existencia o no de ecos y resonancias, la cobertura sonora de las fuentes, etc. \cite{Miyara2004}.

\subsubsection{Ecos.} El fenómeno más sencillo que tiene lugar en un ambiente con superficies reflectoras del sonido es el eco, consistente en una única reflexión que retorna al punto donde se encuentra la fuente unos 100 ms (o más) después de emitido el sonido. Se produce después de un tiempo t relacionado con la distancia d a la superficie más próxima por la expresión

\begin{equation}
t = \frac{2d}{c},
\end{equation}

donde c es la velocidad del sonido. El factor 2 se debe a que el sonido recorre de ida y de vuelta la distancia entre la fuente sonora y la superficie \cite{Miyara2004}.

\subsubsection{Reflexiones tempranas.} Cuando la fuente sonora está rodeada por varias superficies (piso, paredes, techo) un oyente recibirá el sonido directo, y además el sonido reflejado en cada pared. Las primeras reflexiones recibidas, que se encuentran bastante separadas en el tiempo, se denominan reflexiones tempranas \cite{Miyara2004}.

\subsubsection{Absorción sonora.} Las superficies de un recinto reflejan solo parcialmente el sonido que incide sobre ellas; el resto es absorbido. Según el tipo de material o recubrimiento de una pared, ésta podrá absorber más o menos el sonido, lo cual lleva a definir el coeficiente de absorción sonora, abreviado con la letra griega $\alpha$ (alfa), como el cociente entre la energía absorbida y la energía incidente \cite{Miyara2004}:

\begin{equation}
a = \frac{E_{absorbida}}{E_{incidente}}
\end{equation}

El coeficiente de absorción tiene una gran importancia para el comportamiento acústico de un ambiente, y por esa razón se han medido y tabulado los coeficientes de absorción para varios materiales y objetos \cite{Miyara2004}.

%----------------------Tabla 2----------------------

\begin{center}
\footnotesize
    \begin{longtable}[!htb]{| m{22em} | m{2.5em} | m{2.5em} | m{2.5em} | m{2.5em} | m{2.5em} |m{2.5em} |}
    \hline
    \multirow{2}{*}{\textbf{Materiales}} & \multicolumn{6}{c|}{\textbf{Coeficiente de absorción $\alpha$ a la frecuencia}}\\
    \cline{2-7}
    & \textbf{125} & \textbf{250} & \textbf{500} & \textbf{1000} & \textbf{2000} & \textbf{4000} \\
    \hline
    Hormig\'on sin pintar & 0.01 & 0.01 & 0.02 & 0.02 & 0.04 & 0.04\\
    \hline
    Hormig\'on pintado & 0.01 & 0.01 & 0.01 & 0.02 & 0.02 & 0.02\\
    \hline
    Ladrillo visto sin pintar & 0.02 & 0.02 & 0.03 & 0.04 & 0.05 & 0.05\\
    \hline
    Ladrillo visto pintado & 0.01 & 0.01 & 0.02 & 0.02 & 0.02 & 0.02\\
    \hline
    Revoque de cal y arena & 0.04 & 0.05 & 0.06 & 0.08 & 0.04 & 0.06\\
    \hline
    Placa de yeso (Durlock) 12mm a 10cm & 0.29 & 0.10 & 0.05 & 0.04 & 0.07 & 0.09\\
    \hline
    Yeso sobre metal desplegado & 0.04 & 0.04 & 0.04 & 0.06 & 0.06 & 0.03\\
    \hline
    M\'armol o azulejo & 0.01 & 0.01 & 0.01 & 0.01 & 0.02 & 0.02\\
    \hline
    Madera en paneles (a 5cm de la pared) & 0.30 & 0.25 & 0.20 & 0.17 & 0.15 & 0.10\\
    \hline
    Madera aglomerada en panel & 0.47 & 0.52 & 0.50 & 0.55 & 0.58 & 0.63\\
    \hline
    Parquet & 0.04 & 0.04 & 0.07 & 0.06 & 0.06 & 0.07\\
    \hline
    Parquet sobre asfalto & 0.05 & 0.03 & 0.06 & 0.09 & 0.10 & 0.22\\
    \hline
    Parquet sobre listones & 0.20 & 0.15 & 0.12 & 0.10 & 0.10 & 0.07\\
    \hline
    Alfombra de goma 0.5cm & 0.04 & 0.04 & 0.08 & 0.12 & 0.03 & 0.10\\
    \hline
    Alfombra de lana 1.2$kg/m^2$ & 0.10 & 0.16 & 0.11 & 0.30 & 0.50 & 0.47\\
    \hline
    Alfombra de lana 2.3$kg/m^2$ & 0.17 & 0.18 & 0.21 & 0.50 & 0.63 & 0.83\\
    \hline 
    Cortina 338$g/m^2$ & 0.03 & 0.04 & 0.11 & 0.14 & 0.24 & 0.35\\
    \hline
    Cortina 338$g/m^2$ fruncida al 50\% & 0.07 & 0.31 & 0.49 & 0.75 & 0.70 & 0.60\\
    \hline
    Espuma de poliuretano (Fonac) 35mm & 0.11 & 0.14 & 0.36 & 0.82 & 0.90 & 0.97\\
    \hline
    Espuma de poliuretano (Fonac) 50mm & 0.15 & 0.25 & 0.50 & 0.94 & 0.92 & 0.99\\
    \hline
    Espuma de poliuretano (Fonac) 75mm & 0.17 & 0.44 & 0.99 & 1.03 & 1.00 & 1.03\\
    \hline
    Espuma de poliuretano (Sonex) 35mm & 0.06 & 0.20 & 0.45 & 0.71 & 0.95 & 0.89\\
    \hline
    Espuma de poliuretano (Sonex) 50mm & 0.07 & 0.32 & 0.72 & 0.88 & 0.97 & 1.01\\
    \hline
    Espuma de poliuretano (Sonex) 75mm & 0.13 & 0.53 & 0.90 & 1.07 & 1.07 & 1.00\\
    \hline
    Lana de vidrio (fieltro 14$kg/m^3$) 25mm & 0.15 & 0.25 & 0.40 & 0.50 & 0.65 & 0.70\\
    \hline
    Lana de vidrio (fieltro 14$kg/m^3$) 50mm & 0.25 & 0.45 & 0.70 & 0.80 & 0.85 & 0.85\\
    \hline
    Lana de vidrio (panel 14$kg/m^3$) 25mm & 0.20 & 0.40 & 0.80 & 0.90 & 1.00 & 1.00\\
    \hline
    Lana de vidrio (panel 14$kg/m^3$) 50mm & 0.30 & 0.75 & 1.00 & 1.00 & 1.00 & 1.00\\
    \hline
    Ventana abierta & 1.00 & 1.00 & 1.00 & 1.00 & 1.00 & 1.00\\
    \hline
    Vidrio & 0.03 & 0.02 & 0.02 & 0.01 & 0.07 & 0.04\\
    \hline
    Panel cielorraso Spanacustic (Manville) 19mm & - & 0.80 & 0.71 & 0.86 & 0.68 & -\\
    \hline
    Panel cielorraso Acustidom (Manville) 4mm & - & 0.72 & 0.61 & 0.68 & 0.79 & -\\
    \hline
    Panel cielorraso Prismatic (Manville) 19mm & - & 0.70 & 0.61 & 0.70 & 0.78 & -\\
    \hline
    Panel cielorraso Profil (Manville) 19mm & - & 0.72 & 0.62 & 0.69 & 0.78 & -\\
    \hline
    Panel cielorraso fisurado Auratone (USG) $5/8"$ & 0.34 & 0.36 & 0.71 & 0.85 & 0.68 & 0.64\\
    \hline
    Panel cielorraso fisurado Cortega (AWI) $5/8"$ & 0.31 & 0.32 & 0.51 & 0.72 & 0.74 & 0.77\\
    \hline
    Asiento de madera (0.8 $m^2/asiento$) & 0.01 & 0.02 & 0.03 & 0.04 & 0.06 & 0.08\\
    \hline
    Asiento tapizado grueso (0.8 $m^2/asiento$) & 0.44 & 0.44 & 0.44 & 0.44 & 0.44 & 0.44\\
    \hline
    Personas en asiento de madera (0.8 $m^2/persona$) & 0.34 & 0.39 & 0.44 & 0.54 & 0.56 & 0.56\\
    \hline
    Personas en asiento tapizado (0.8 $m^2/persona$) & 0.53 & 0.51 & 0.51 & 0.56 & 0.56 & 0.59\\
    \hline
    Personas de pie (0.8 $m^2/persona$) & 0.25 & 0.44 & 0.59 & 0.56 & 0.62 & 0.50\\
    \hline
    \caption{Coeficientes de absorción de diversos materiales en función de la frecuencia (según varias fuentes) \cite{Miyara2004}.}
    \label{tab:CoefMateriales}
    \end{longtable}
\end{center}
%----------------------Tabla 2----------------------

En la tabla 2 se dan los valores de $\alpha$ para varios materiales típicos de construcción, objetos y personas. Se proporcionan para varias frecuencias, ya que $\alpha$ depende bastante de la frecuencia \cite{Miyara2004}.
%tabla 2. Coficientes de absorcion.

\subsubsection{Tiempos de reverberación.} Después del periodo de las reflexiones tempranas, comienzan a aparecer las reflexiones de las reflexiones, y las reflexiones de las reflexiones de las reflexiones, y así sucesivamente, dando origen a una situación muy compleja en la cual las reflexiones se densifican cada vez más. Esta permanencia del sonido aún después de interrumpida la fuente se denomina reverberación. Para medir cuánto demora este proceso de extinción del sonido se introduce el concepto de tiempo de reverberación, T, técnicamente definido como el tiempo que demora el sonido en bajar 60 dB por debajo de su nivel inicial \cite{Miyara2004}.
\\
La propiedad anterior se puede expresar por medio de una fórmula, denominada fórmula de Sabine, en honor al físico norteamericano que la obtuvo a principios de este siglo. Según dicha fórmula el tiempo de reverberación T puede calcularse como:
\begin{equation}
T = 0.161\frac{V}{\alpha S}
\end{equation}
donde V es el volumen de la habitación en $m^3$, S es el área de su superficie interior total en $m^2$, y $\alpha$ es el coeficiente de absorción sonora [16].

\subsubsection{Tiempos de reverberación óptimo.} Varias investigaciones realizadas evaluando las acústicas de las mejores salas del mundo (según la opinión de las audiencias o usuarios y de expertos) han revelado que para cada finalidad existe un tiempo de reverberación óptimo, que aumenta al aumentar el volumen en $m^3$ de la sala.

En la ilustración 1 se muestra el resultado de uno de estos estudios. Debe aclararse que no hay coincidencia entre los resultados presentados por diversos investigadores, aunque cualitativamente son similares.

%Ilustracion 1 
\begin{figure}[!htb]
    \centering
    \includegraphics[width=0.7\textwidth]{imagenes/1.jpg}
    \caption{\footnotesize Tiempo de reverberación óptimo en función del volumen de una sala (según L. L. Beranek). (a) Estudios de radiodifusión para voz. (b) Salas de conferencias. (c) Estudios de radiodifusión para música. (d) Salas de conciertos. (e) Iglesias \cite{Miyara2004}.}
    \label{fig:TiempoRevOpt}
\end{figure}
\FloatBarrier

\subsubsection{Resonancias.} Las resonancias o modos normales de vibración suceden como consecuencia de las reflexiones sucesivas en paredes opuestas. Una onda estacionaria es una onda que va y viene una y otra vez entre dos paredes, por lo que, si la distancia entre las dos paredes es L, la longitud de tal onda es 2L y por consiguiente deberá cumplir que

\begin{equation}
2\cdot L = \frac{c}{f}
\end{equation}

Donde c es la velocidad del sonido, y f la frecuencia del sonido resultante. De aquí se puede obtener la frecuencia, que resulta ser

\begin{equation}
f = \frac{c}{2\cdot L}
\end{equation}

Para las frecuencias de resonancia el tiempo de reverberación es mucho más prolongado, por lo cual dicha nota se prolongará más que las otras. Esto se considera un defecto acústico importante. Entre las posibles soluciones, están: a) evitar las superficies paralelas, que favorecen las resonancias, b) agregar absorción acústica que reduzca el tiempo de reverberación, c) ecualizar el sistema de sonido de modo de atenuar las frecuencias próximas a la resonancia o resaltar las otras frecuencias \cite{Miyara2004}.

\subsubsection{Claridad.} La claridad describe el grado en que cada detalle de las actuaciones puede ser percibido en lugar de que todo se difumine por los componentes de sonido reverberantes que llegan más tarde. Por lo tanto, la claridad es en gran medida una propiedad complementaria a la reverberancia.

Cuando las reflexiones se retrasan no más de 50-80 ms en relación con el sonido directo, el oído integrará estas contribuciones y el sonido directo, lo que significa que percibimos principalmente el efecto como si el sonido claro y original se hubiera amplificado en relación con la energía reverberante posterior. Por lo tanto, se ha encontrado que un parámetro objetivo que compara la relación entre la energía en la respuesta al impulso antes y después de 80 ms es un descriptor razonablemente bueno de claridad.

\begin{equation}
C = 10\log_{10}\left[{\frac{\int_0^{80ms} h^2(t) \mathrm{d}t}{\int_{80ms}^{\infty} h^2(t) \mathrm{d}t}}\right]
\end{equation}

Cuanto mayor sea el valor de C, más dominará el sonido inicial y mayor será la impresión de claridad \cite{Rossing2007}.

\subsubsection{Potencia sonora} La influencia de la sala en la sonoridad percibida es otro aspecto importante de la acústica de la sala. Una medida relevante de esta propiedad es simplemente la diferencia en dB entre el nivel de una fuente de sonido continua y calibrada medido en la habitación y el nivel que genera la misma fuente a 10 m de distancia en un entorno anecoico. Esta medida objetiva, denominada fuerza (relativa) G, también puede obtenerse a partir de registros de respuesta al impulso a partir de la relación entre la energía total de la respuesta al impulso y la energía del sonido directo, y esta última se registra a una distancia fija (10 m) de la fuente de sonido impulsivo:

\begin{equation}
G = 10\log_{10}\left[{\frac{\int_0^{\infty} h^2(t) \mathrm{d}t}{\int_0^{t_{dir}} h^2_{10m}(t) \mathrm{d}t}}\right]
\end{equation}

En este caso, el límite superior de integración en el denominador $t_{dir}$ debe limitarse a la duración del impulso de sonido directo (que en la práctica dependerá de la anchura de banda seleccionada). Se puede utilizar una distancia diferente de 10 m, si también se aplica una corrección para la atenuación de la distancia. 

El valor esperado de G según la teoría de campos difusos se convierte en una función de T, así como del volumen de la habitación, V \cite{Rossing2007}:

\begin{equation}
G_{exp} = 10\log_{10}\left({\frac{T}{V}}\right) + 45 dB
\end{equation}

\subsubsection{Medidas de amplitud} La amplitud es la sensación de que el sonido llega desde muchas direcciones diferentes en contraste con una impresión monofónica de todo el sonido que llega al oyente a través de una abertura estrecha. Ahora está claro que hay dos aspectos de la amplitud, los cuales son atractivos, especialmente cuando se escucha música:

Anchura aparente de la fuente (ASW): la impresión de que la imagen sonora es más amplia que la extensión visual y física de la(s) fuente(s) en el escenario. Envolvente del oyente (LEV): la impresión de estar dentro y rodeado por el campo sonoro reverberante de la habitación.

Se ha encontrado que ambos aspectos dependen de la dirección de incidencia de las 
reflexiones de respuesta al impulso. Cuando una mayor parte de la energía de reflexión temprana (hasta unos 80 ms) llega desde direcciones laterales (desde los lados), el ASW aumenta. Cuando el nivel de las reflexiones laterales tardías es alto, se produce un fuerte LEV.

Los componentes laterales de la energía de respuesta al impulso se pueden grabar utilizando un micrófono en forma de ocho con el eje sensible mantenido horizontal y perpendicular a la dirección hacia la fuente de sonido (de modo que la fuente se encuentre en el plano sordo del micrófono). Para la medición de la fracción de energía lateral (LEF), la parte inicial (hasta 80 ms) de esta energía sonora lateral se compara con la energía del sonido directo más todas las reflexiones tempranas captadas por un micrófono omnidireccional ordinario

\begin{equation}
LEF = \frac{\int_{t=5ms}^{t=80ms} h^2_1(t) \mathrm{d}t}{\int_{t=0ms}^{t=80ms} h^2(t) \mathrm{d}t}
\end{equation}

donde $h_1$ es la presión de respuesta al impulso registrada con un micrófono en forma de ocho, mientras que h se captura a través del micrófono omnidireccional (habitual). Es principalmente la energía a frecuencias bajas y medias la que contribuye a la amplitud. En consecuencia, LEF se promedia normalmente en las cuatro octavas 125-1000 Hz. Cuanto mayor sea el valor de LEF, más amplio será el ASW. En un campo completamente difuso, LEF sería constante con 
un valor de 0,33, que es superior al que normalmente se encuentra en las salas reales. La diferencia subjetiva para LEF es de aproximadamente el 5\%. El aspecto ASW de la amplitud no solo depende de la relación entre el sonido lateral temprano y el sonido temprano total; pero también en el nivel total del sonido. Cuanto mayor sea el valor G (y más fuerte sea la fuente de sonido), más amplia será la imagen acústica de la fuente. Sin embargo, en el momento de escribir este artículo, no existe una forma sólida de incorporar la influencia del nivel en la medida objetiva. La envolvente del oyente parece estar determinada principalmente por la distribución espacial y el nivel de las reflexiones tardías (que llegan después de 80 ms) \cite{Rossing2007}.

\subsubsection{Parámetros relacionados con el timbre o el color tonal} El timbre describe el grado en que la habitación influye en el equilibrio de frecuencias entre las frecuencias altas, medias y bajas, es decir, si el sonido es áspero, brillante, hueco, cálido o cualquier otro adjetivo que se use para describir el color tonal. Tradicionalmente, se ha utilizado un gráfico de la variación de frecuencia de T (por 1/1 o 1/3 de octava) para indicar esta cualidad; pero se ha sugerido un conveniente parámetro de un solo número destinado a medir la calidez de la sala: la relación de graves (BR) dada por:

\begin{equation}
BR = \frac{T_{125Hz}+T_{250Hz}}{T_{500Hz}+T_{1000Hz}}
\end{equation}

Del mismo modo, una relación de agudos (TR) se puede formar como:

\begin{equation}
BR = \frac{T_{2000Hz}+T_{4000Hz}}{T_{500Hz}+T_{1000Hz}}
\end{equation}

Sin embargo, en algunas salas, se experimenta una falta de sonido de graves a pesar de los altos valores de T en las frecuencias bajas. Por lo tanto, EDT o tal vez G versus frecuencia sería un parámetro mejor, e intuitivamente más lógico, para la medición del timbre. Del mismo modo, BR o TR podrían basarse en valores G en lugar de T \cite{Rossing2007}.

\subsubsection{Inteligibilidad del habla} Todos los parámetros objetivos mencionados anteriormente (excepto el parámetro básico T), son principalmente relevantes en auditorios más grandes destinados a la interpretación de música. En los auditorios utilizados para el habla, como las salas de conferencias o los teatros, la influencia de la acústica en la inteligibilidad es un problema importante. Actualmente, la forma más común de evaluar objetivamente la inteligibilidad del habla en las salas es mediante la medición del índice de transmisión del habla ST \cite{Rossing2007}.

%Imagen 2
\begin{figure}[!htb]
    \centering
    \includegraphics[width=0.9\textwidth]{imagenes/2.jpg}
    \caption{\footnotesize Ilustración de la teoría y el principio en la medición de STI o RASTI. La escala para la evolución de los valores RASTI se muestra en la parte inferior \cite{Rossing2007}.}
    \label{fig:IntelHabla}
\end{figure}
\FloatBarrier

%\addcontentsline{toc}{section}{Propuesta de soluci\'on}
%\section*{Propuesta de soluci\'on}
%\addcontentsline{toc}{subsection}{Definición de la metodología}
\subsection{Marco Precedimental}

\subsubsection{Definici\'on de la metodolog\'ia mecatr\'onica}

La mecatrónica es una ingeniería interdisciplinaria y, por tanto, es necesario buscar 
metodologías de diseño que nos permita especificar las necesidades del sistema, planear actividades, dividir tareas y validar resultados.
\\
En \cite{Geusemeir2002} se muestra la metodología mecatrónica ‘VDI 2206”. En ella se sigue un modelo 
procedimental flexible basado en tres elementos principales:
\begin{itemize}
    \item El ciclo general de resolución de problemas a pequeña escala (micro-level)
    \item El modelo en forma de V en la gran escala (macro-level)
    \item Módulo de procesos predefinidos para pasos de operación que se repiten durante el diseño de los sistemas mecatrónicos.
\end{itemize}

\subsubsection{Resolución de problemas en el micro-level}
La resolución de problemas en el micro-level comienza de dos maneras: analizando la situación actual y analizando la relación de la condición actual con la condición meta. Lo anterior nos lleva a una síntesis y análisis que permite generar, rechazar y elegir soluciones. El análisis de la solución y la evaluación nos puede llevar a replantear la meta o a volver a analizar la situación. Si el resultado es satisfactorio nos sirve como herramienta de aprendizaje y se planean las acciones futuras. \\
El flujo de esta metodología de resolución de problemas se puede ver en la figura \ref{fig:DiagramaMicroLevel}:
%---------------------------------Imagen 3---------------------------------
\begin{figure}[!htb]
    \centering
    \includegraphics[width=0.6\textwidth]{imagenes/3.jpg}
    \caption{Diagrama de flujo para la resolución de problemas en el micro-level.}
    \label{fig:DiagramaMicroLevel}
\end{figure}
\FloatBarrier
%---------------------------------Imagen 3---------------------------------

\subsubsection{Modelo en V en el macro-level}
En modelo en V, en donde la primera parte es un procedimiento de arriba hacia abajo y la segunda parte es de abajo hacia arriba, nos permitirá separar las tareas a gran escala y realizar validaciones constantes de los sistemas.\\
El modelo V busca partir de los requerimientos y realizar un diseño del sistema conceptual, el cual es multidominio y describe las características esenciales del producto. A continuación, hace diseños específicos para cada dominio, los cuales integra en una etapa posterior y analiza sus interrelaciones. A lo largo de este proceso se debe hacer una verificación y validación constante de que el diseño cumpla con los requerimientos y el diseño conceptual del producto, además de apoyarse en el modelado y la simulación por computadora para hacer estas validaciones. En caso de cumplirse con esto se cuenta con un producto que cumple con los requisitos.\\
El modelo V es un modelo iterativo, es decir, es poco probable que el primer producto al que se llegue después de la integración sea totalmente funcional o incluso el óptimo. Por lo anterior, se deben llevar a cabo múltiples ciclos de diseño y validación, para que el producto alcance un mayor grado de madurez. Cada etapa tiene un “contraparte” a la que se puede volver en caso de que no se cumpla alguna verificación, esto lleva a mejorar el diseño y analizar múltiples opciones de solución.
%---------------------------------Imagen 4---------------------------------
\begin{figure}[!htb]
    \centering
    \includegraphics[width=0.5\textwidth]{imagenes/4.jpg}
    \caption{\footnotesize Metodología por seguir.}
    \label{Fig:MetodologiaV}
\end{figure}
\FloatBarrier
%---------------------------------Imagen 4---------------------------------

\subsubsection{Procesos predefinidos}
En cada una de las etapas de diseño, se tienen procesos predefinidos que se repiten regularmente durante estas etapas. Cada uno de estos procesos están mejor definidos en \cite{Pahl1996} y se utilizarán en el desarrollo de este proyecto para la generación de soluciones.

%---------------Media 2---------------
\section{Diseño del sistema}
\subsection{Diseño conceptual}
\subsubsection{Necesidades y requerimientos}
La metodología VDI 2206 emplea los requerimientos como entrada para comenzar el proceso, por tanto, estos deben definirse con anterioridad. Para este propósito se utiliza la ingeniería de requerimientos.
\\
La ingeniería de requerimientos consiste en definir primero las necesidades funcionales y no funcionales y transformarlas en requerimientos que son simplemente propiedades medibles.

%-----------------------------tabla 3-----------------------------
\begin{center}
\footnotesize
    \begin{longtable}[!htb]{| m{3em} | m{30em} | m{6em}|}
    \hline
    \textbf{ID}& \textbf{Necesidades} & \textbf{Clasificación}\\
    \hline \hline
    N1 & Medición de la acústica & Funcional\\
    \hline
    N2 & Movimiento de paneles & Funcional\\
    \hline
    N3 & Modificación de la acústica & Funcional\\
    \hline
    N4 & Repetibilidad en acondicionamiento acústico & Funcional\\
    \hline
    N5 & Fiabilidad en acondicionamiento acústico & Funcional\\
    \hline
    N6 & Ingreso de datos por medio de interfaz de usuario & Funcional\\
    \hline
    N7 & Control local de la posición de los paneles & Funcional\\
    \hline
    N8 & Instalación posible en paredes y techo & Funcional\\
    \hline
    N9 & Hacer acondicionamiento a partir de tipo de instrumento a grabar & Funcional\\
    \hline
    N10 & Condiciones de operación propias de un estudio de audio & No funcional\\
    \hline
    N11 & No requerimiento de conocimientos en acondicionamiento acústico para la operación & No funcional \\
    \hline
    N12 & Despliegue de información por medio de interfaz & No funcional\\
    \hline

    \caption{Necesidades del sistema automatizado para el acondicionamiento acústico.}
    \label{tab:Necesidades}
    \end{longtable}
\end{center}
%-----------------------------tabla 3-----------------------------

%-----------------------------tabla 4-----------------------------
\begin{center}
\footnotesize
    \begin{longtable}[!htb]{| m{3em} | m{15em} | m{10em}|}
    \hline
    \textbf{ID}& \textbf{Requerimiento} & \textbf{Valor}\\
    \hline\hline
    R1 & Función principal & Hacer el acondicionamiento acústico de un estudio de audio de manera automática para diferentes instrumentos\\
    \hline
    R2 & Ubicación & Cuarto destinado a estudio de audio en $<$19.509021,-99.096411$>$\\
    \hline
    R3 & Dimensiones del estudio de audio & 6m x 4m x 3.6m\\
    \hline
    R4 & Medición del tiempo de reverberación & Si\\
    \hline
    R5 & Medición de la claridad & Si\\
    \hline
    R6 & Medición de la fuerza del sonido & Si\\
    \hline
    R7 & Medición de la relación de energía lateral & Si\\
    \hline
    R8 & Medición de la relación de agudos & Si\\
    \hline
    R9 & Medición de la relación de bajos & Si\\
    \hline
    R10 & Medición de la inteligibilidad del habla & Si\\
    \hline
    R11 & Disminución de ondas estacionarias & Si \\
    \hline
    R12 & Modificación mínima en tiempo de reverberación & 5 ms\\
    \hline
    R13 & Número de parámetros acústicos calculados & 7\\
    \hline
    R14 & Error de acondicionamiento & $<5\%$\\
    \hline
    R15 & Diferencia entre acondicionamientos sucesivos & $<5\%$\\
    \hline
    R16 & Tensión de alimentación & 127 VAC\\
    \hline
    R17 & Corriente máxima & 30 A \\
    \hline
    R18 & Sistema de selección de instrumento & Si\\
    \hline
    R19 & Soporte para paredes & Si\\
    \hline
    R20 & Soporte de techo & Si \\
    \hline
    R21 & Información en interfaz para usuario & Si\\
    \hline

    \caption{Tabla de requerimientos.}
    \label{tab:Requerimientos}
    \end{longtable}
\end{center}

%-----------------------------tabla 4-----------------------------

Se realizó la validación de los requerimientos acorde con las necesidades planteadas. Para ello, se elaboró la matriz de trazabilidad mostrada en la Tabla “Matriz de trazabilidad necesidades-requerimientos”, la cual permite visualizar de manera gráfica cómo se relacionan, así como corroborar que los requerimientos satisfacen todas las necesidades.
%-----------------------------matriz de trazabilidad-----------------------------
\definecolor{gr}{gray}{.5}
\begin{center}
\footnotesize
    \begin{longtable}[!htb]{| m{2em} || m{2em} | m{2em}| m{2em}| m{2em}| m{2em}| m{2em}| m{2em}| m{2em}| m{2em}| m{2em}| m{2em}| m{2em}|}
    \hline
    &N1 &N2 &N3 &N4 &N5 &N6 &N7 &N8 &N9 &N10 &N11 &N12\\
    \hline\hline
    R1 & \cellcolor{gr}{} &\cellcolor{gr}{} &\cellcolor{gr}{} &\cellcolor{gr}{} &\cellcolor{gr}{} &\cellcolor{gr}{} &\cellcolor{gr}{} & \cellcolor{gr}{}& \cellcolor{gr}{}& & &  \\
    \hline
    R2 & & & & & & & & & &\cellcolor{gr}{} & &  \\
    \hline
    R3 & & & & & & & & & &\cellcolor{gr}{} & & \\
    \hline
    R4 & \cellcolor{gr}{}& & & & & & & & & & & \\
    \hline
    R5 &\cellcolor{gr}{} & & & & & & & & & & & \\
    \hline
    R6 &\cellcolor{gr}{} & & & & & & & & & & & \\
    \hline
    R7 & \cellcolor{gr}{}& & & & & & & & & & & \\
    \hline
    R8 & \cellcolor{gr}{}& & & & & & & & & & & \\
    \hline
    R9 & \cellcolor{gr}{}& & & & & & & & & & & \\
    \hline
    R10 &\cellcolor{gr}{} & & & & & & & & & & & \\
    \hline
    R11 & &\cellcolor{gr}{} & \cellcolor{gr}{}& & \cellcolor{gr}{}& &\cellcolor{gr}{} & & & & & \\
    \hline
    R12 & & \cellcolor{gr}{}&\cellcolor{gr}{} & & \cellcolor{gr}{}& & \cellcolor{gr}{}& & & & & \\
    \hline
    R13 &\cellcolor{gr}{} & & \cellcolor{gr}{}& & \cellcolor{gr}{}& & \cellcolor{gr}{}& & & & & \\
    \hline
    R14 & \cellcolor{gr}{}&\cellcolor{gr}{} &\cellcolor{gr}{} & & & \cellcolor{gr}{}& &\cellcolor{gr}{} & & & & \\
    \hline
    R15 & \cellcolor{gr}{}&\cellcolor{gr}{} &\cellcolor{gr}{} & \cellcolor{gr}{}&\cellcolor{gr}{} & & & & & & & \\
    \hline
    R16 & & & & & & & & & &\cellcolor{gr}{} & & \\
    \hline
    R17 & & & & & & & & & &\cellcolor{gr}{} & & \\
    \hline
    R18 & & & & & &\cellcolor{gr}{} & & & & &\cellcolor{gr}{} &\cellcolor{gr}{} \\
    \hline
    R19 & & & & & & &\cellcolor{gr}{} &\cellcolor{gr}{} & & \cellcolor{gr}{}& & \\
    \hline
    R20 & & & & & & &\cellcolor{gr}{} & \cellcolor{gr}{}& & \cellcolor{gr}{}& & \\
    \hline
    R21 & & & & & &\cellcolor{gr}{} & & & & & \cellcolor{gr}{}& \cellcolor{gr}{}\\
    \hline

    \caption{Matriz de trazabilidad necesidades-requerimiento.}
    \label{tab:MatTraza}
    \end{longtable}
\end{center}
%-----------------------------matriz de trazabilidad-----------------------------

\subsubsection{Arquitectura funcional}
Siguiendo con la metodología, se decidió definir las funciones y subfunciones que debe desarrollar el sistema, para así poder comenzar su diseño.
\\ 
Función principal: Hacer un acondicionamiento acústico a un cuarto en función de la acústica óptima para un cierto instrumento.
\\\\
\begin{enumerate}[{F}1.]
    \item Medición de la acústica actual
    \begin{enumerate}[{C}1.]
        \item El sistema debe generar un sonido en el estudio de audio.
        \item El sistema debe grabar la respuesta del estudio de audio al sonido generado.
        \item El sistema debe calcular el tiempo de reverberación a partir de la respuesta del estudio de audio.
        \item El sistema debe calcular la claridad a partir de la respuesta del estudio de audio.
        \item El sistema debe calcular la fuerza del sonido a partir de la respuesta del estudio de audio.
        \item El sistema debe calcular la relación de energía lateral a partir de la respuesta del estudio de audio.
        \item El sistema debe calcular la relación de agudos a partir de la respuesta del estudio de audio.
        \item El sistema debe calcular la relación de graves a partir de la respuesta del estudio de audio.
        \item El sistema debe calcular la inteligibilidad del habla a partir de la respuesta del estudio de audio.
        \item El sistema debe calcular los modos normales de vibración del estudio de audio.
    \end{enumerate}

    \item Modificación de la acústica
    \begin{enumerate}[{C}1.]
        \item El sistema debe conocer la diferencia entre la acústica actual y la acústica deseada.
        \item El sistema debe calcular la posición de los paneles que generan la acústica deseada.
        \item El sistema debe calcular la diferencia entre la posición actual de los paneles y la posición deseada.
        \item El sistema debe mover los paneles a la posición deseada.
        \item El sistema debe disminuir la generación de ondas estacionarias.
    \end{enumerate}

    \item Interacción con el usuario
    \begin{enumerate}[{C}1.]
        \item El sistema debe recibir el tipo de instrumento que se va a grabar.
        \item El sistema debe saber la acústica óptima para el instrumento que se va a grabar.
        \item El sistema debe mostrar al usuario los parámetros acústicos que definen la acústica actual.
        \item El sistema debe mostrar al usuario las frecuencias problemáticas del estudio de audio (que generan ondas estacionarias).
        \item El sistema debe dar control al usuario de cuándo hacer un acondicionamiento acústico.
        \item El sistema debe dar a elegir al usuario para que tipo de instrumento hacer el acondicionamiento acústico.
        \item El sistema debe mostrar la acústica ideal para dicho instrumento.
        \item El sistema debe mostrar la acústica después del acondicionamiento y compararla con la acústica ideal. 
    \end{enumerate}

    \item Gestión de energía
    \begin{enumerate}[{C}1.]
        \item El sistema tiene un bajo consumo energético cuando no se está moviendo.
        \item El sistema gestiona el movimiento de los paneles para evitar un uso excesivo de corriente.
        \item El sistema debe transformar y controlar la energía que obtiene de la red eléctrica para entregar la tensión de alimentación correcta a cada componente.
    \end{enumerate}
\end{enumerate}


\begin{enumerate}
%----------------1------------------
    \item Medición de la acústica actual
    \begin{enumerate}[{1.}1.]
        \item Generación de onda de sonido
        \begin{enumerate}[{1.1.}1.]
            \item Selección de ondas necesarias
            \item Gestión de secuencia de ondas
            \item Generación de secuencia de ondas
        \end{enumerate}
        
        \item Obtención de la respuesta
        \begin{enumerate}[{1.2.}1.]
            \item Sensado de la respuesta
        \end{enumerate}   

        \item Procesamiento de la respuesta
        \begin{enumerate}[{1.3.}1.]
            \item Cálculo de los parámetros acústicos
            \begin{enumerate}[{1.3.1.}1.]
                \item Cálculo del tiempo de reverberación
                \item Cálculo de la claridad
                \item Cálculo de la fuerza del sonido
                \item Cálculo de la relación de energía lateral
                \item Cálculo de la relación de bajos
                \item Cálculo de la relación de agudos
                \item Cálculo de la inteligibilidad del sonido
            \end{enumerate}
            
            \item Cálculo de modos normales de vibración
            \begin{enumerate}[{1.3.2.}1.]
                \item Obtención de dimensiones del estudio de audio
                \item Obtención de posición de la fuente
                \item Obtención de posición del receptor
            \end{enumerate}
        \end{enumerate}
    \end{enumerate}   
    
%----------------2------------------
    \item Modificación de la acústica
    \begin{enumerate}[{2.}1.]
        \item Cálculo de posiciones deseadas
        \item Compensación para disminución de modos normales de vibración
        
        \item Movimiento de paneles
        \begin{enumerate}[{2.3.}1.]
        
            \item Medición del error de posición
            \begin{enumerate}[{2.3.1.}1.]
                \item Medición de la posición actual
                \item Cálculo del error de posición
            \end{enumerate}

            \item Procesamiento por estrategia de control            
        \end{enumerate}
        
    \end{enumerate}
    
%----------------3------------------   
    \item Interacción con el usuario
    \begin{enumerate}[{3.}1.]
        \item Recepción de instrucciones de acondicionamiento
        \begin{enumerate}[{3.1.}1.]
            \item Selección de instrumento
            \begin{enumerate}[{3.1.1.}1.]
                \item Despliegue de acústica ideal
            \end{enumerate}
        \end{enumerate}

        \item Despliegue de acústica actual
        \begin{enumerate}[{3.2.}1.]
            \item Despliegue de las variables acústicas calculadas
            \item Despliegue de las frecuencias de vibración problemáticas

        \end{enumerate}

        \item Despliegue de la acústica posterior al acondicionamiento
        \begin{enumerate}[{3.3.}1.]
            \item Despliegue de las variables acústicas calculadas
            \item Despliegue de las ondas estacionarias
        \end{enumerate}

        \item Control de apagado y encendido
    \end{enumerate}
%----------------4------------------
    \item Gestión de energía
    \begin{enumerate}[{4.}1.]
    
        \item Conversión y regulación de tensión
        \begin{enumerate}[{4.1.}1.]
            \item Conversión de tensión
            \item Regulación de tensión
        \end{enumerate}

        \item Medición del consumo energético
        \begin{enumerate}[{4.2.}1.]
            \item Medición de la tensión de entrada
            \item Medición de la corriente de entrada
        \end{enumerate}

        \item Gestión del consumo energético
        \begin{enumerate}[{4.3.}1.]
            \item Disminución de la potencia proporcional al consumo
        \end{enumerate}
        
    \end{enumerate}
\end{enumerate}

%Imagenes del idef0
\begin{figure}[!htb]
    \centering
    \includegraphics[width=0.8\textwidth]{imagenes/5.jpg}
    \caption{\footnotesize Diagrama de las funciones del sistema para el tratamiento acústico.}
    \label{fig:DiagramaFunciones}
\end{figure}
\FloatBarrier
%---------------------Imagenes de los idefs----------------------------------
\begin{figure}[!htb]
    \centering
    \includegraphics[width=0.8\textwidth]{imagenes/6.jpg}
    \caption{\footnotesize IDEF0 del sistema para el acondicionamiento acústico.}
    \label{fig:IDEF0_Sistema}
\end{figure}
\FloatBarrier

\begin{figure}[!htb]
    \centering
    \includegraphics[width=0.8\textwidth]{imagenes/7.jpg}
    \caption{\footnotesize IDEF0 del Módulo de medición de la acústica actual.}
    \label{fig:IDEF0_M1}
\end{figure}
\FloatBarrier

\begin{figure}[!htb]
    \centering
    \includegraphics[width=0.8\textwidth]{imagenes/8.jpg}
    \caption{\footnotesize IDEF0 del Módulo de modificación de la acústica.}
    \label{fig:IDEF0_M2}
\end{figure}
\FloatBarrier

\begin{figure}[!htb]
    \centering
    \includegraphics[width=0.8\textwidth]{imagenes/9.jpg}
    \caption{\footnotesize IDEF0 del Módulo de interacción con el usuario.}
    \label{fig:IDEF0_M3}
\end{figure}
\FloatBarrier

\begin{figure}[!htb]
    \centering
    \includegraphics[width=0.8\textwidth]{imagenes/10.jpg}
    \caption{\footnotesize IDEF0 del Módulo de gestión de energía.}
    \label{fig:IDEF0_M4}
\end{figure}
\FloatBarrier
%---------------------Imagenes de los idefs----------------------------------
Se procedió a validar la relación entre los requerimientos y las funciones, para lo cual se elaboró la matriz de trazabilidad presentada en la Tabla “Matriz de trazabilidad requerimientos-funciones”.

%-------------Matriz de trazabilidad requerimientos-funciones--------------------
\definecolor{gr}{gray}{.5}
\begin{center}
\footnotesize
    \begin{longtable}[!htb]{| m{2em} || m{2em} | m{2em}| m{2em}| m{2em}| m{2em}| m{2em}| m{2em}| m{2em}| m{2em}| m{2em}| m{2em}| m{2em}| m{2em}|}
    \hline
    &F1.1 &F1.2 &F1.3 &F2.1 &F2.2 &F2.3 &F3.1 &F3.2 &F3.3 &F3.4 &F4.1 &F4.2 &F4.3\\
    \hline\hline
    R1 & & \cellcolor{gr}{}&\cellcolor{gr}{} &\cellcolor{gr}{} & \cellcolor{gr}{}&\cellcolor{gr}{}&\cellcolor{gr}{} & & & & & & \\
    \hline
    R2 &\cellcolor{gr}{} &\cellcolor{gr}{} & & &\cellcolor{gr}{} & & & & & & & &\\
    \hline
    R3 &\cellcolor{gr}{} & \cellcolor{gr}{}& & &\cellcolor{gr}{} & & & & & & & &\\
    \hline
    R4 &\cellcolor{gr}{} &\cellcolor{gr}{} &\cellcolor{gr}{} & & & & &\cellcolor{gr}{} &\cellcolor{gr}{} & & & &\\
    \hline
    R5 &\cellcolor{gr}{} &\cellcolor{gr}{} &\cellcolor{gr}{} & & & & &\cellcolor{gr}{} &\cellcolor{gr}{} & & & &\\
    \hline
    R6 &\cellcolor{gr}{} &\cellcolor{gr}{} &\cellcolor{gr}{} & & & & &\cellcolor{gr}{} &\cellcolor{gr}{} & & & &\\
    \hline
    R7 &\cellcolor{gr}{} &\cellcolor{gr}{} &\cellcolor{gr}{} & & & & &\cellcolor{gr}{} &\cellcolor{gr}{} & & & &\\
    \hline
    R8 &\cellcolor{gr}{} &\cellcolor{gr}{} &\cellcolor{gr}{} & & & & &\cellcolor{gr}{} &\cellcolor{gr}{} & & & &\\
    \hline
    R9 &\cellcolor{gr}{} &\cellcolor{gr}{} &\cellcolor{gr}{} & & & & &\cellcolor{gr}{} &\cellcolor{gr}{} & & & &\\
    \hline
    R10 &\cellcolor{gr}{} &\cellcolor{gr}{} &\cellcolor{gr}{} & & & & &\cellcolor{gr}{} &\cellcolor{gr}{} & & & &\\
    \hline
    R11 & & & &\cellcolor{gr}{} &\cellcolor{gr}{} & \cellcolor{gr}{}& & & & & & &\\
    \hline
    R12 & & & &\cellcolor{gr}{} &\cellcolor{gr}{} &\cellcolor{gr}{} & & & & & & &\\
    \hline
    R13 &\cellcolor{gr}{} &\cellcolor{gr}{} &\cellcolor{gr}{} & & & & &\cellcolor{gr}{} &\cellcolor{gr}{} & & & &\\
    \hline
    R14 & & & &\cellcolor{gr}{} &\cellcolor{gr}{} &\cellcolor{gr}{} & & & & & & &\\
    \hline
    R15 & & & &\cellcolor{gr}{} &\cellcolor{gr}{} &\cellcolor{gr}{} & & & & & & &\\
    \hline
    R16 & & & & & & & & & & & \cellcolor{gr}{}& \cellcolor{gr}{}&\cellcolor{gr}{}\\
    \hline
    R17 & & & & & & & & & & & \cellcolor{gr}{}& \cellcolor{gr}{}&\cellcolor{gr}{}\\
    \hline
    R18 & & & & & & &\cellcolor{gr}{} & & &\cellcolor{gr}{} & & &\\
    \hline
    R19 & & & &\cellcolor{gr}{} &\cellcolor{gr}{} &\cellcolor{gr}{} & & & & & & &\\
    \hline
    R20 & & & &\cellcolor{gr}{} &\cellcolor{gr}{} &\cellcolor{gr}{} & & & & & & &\\
    \hline
    R21 & & & & & & & &\cellcolor{gr}{} &\cellcolor{gr}{} & & & &\\
    \hline

    \caption{\footnotesize Matriz de trazabilidad requerimientos-funciones.}
    \label{tab:MatTraza_ReqFun}
    \end{longtable}
\end{center}

%-------------Matriz de trazabilidad requerimientos-funciones--------------------
\subsubsection{Arquitectura física}
Para dar forma a la parte física del proyecto se definen los módulos que darán integridad a la propuesta de solución. La arquitectura física del sistema es representada por la transformación de las funciones en módulos, que tomarán parte en los ensambles y componentes físicos. La arquitectura física está planteada para ser capaz de realizar las funciones descritas en la arquitectura funcional y cumplir con los requerimientos del sistema para lograr el funcionamiento deseado.

\begin{enumerate}[{MF}1]
    \item Módulo de procesamiento: Es un módulo cuya función principal es el procesamiento de las ondas de sonido, el cálculo de las posiciones de los paneles y el procesamiento de los comandos ingresados por el usuario. El módulo se define en submódulos, donde el submódulo \textbf{MF}$_{1.1}$ es el encargado de procesar la respuesta del estudio y realizar los cálculos de los parámetros acústicos y los modos normales de vibración. En el submódulo \textbf{MF}$_{1.2}$ se generan los cálculos de las posiciones deseadas de los paneles, así como el procesamiento de la estrategia de control para el movimiento de los mismos, contemplando la medición y el cálculo del error de posición, para que la localización de los paneles sea fiable. El submódulo \textbf{MF}$_{1.3}$ se encarga de procesar todas las instrucciones de acondicionamiento ingresadas por el usuario, así como de todos los parámetros acústicos que se despliegan en la interfaz de muestreo.

    \item Módulo de generación y medición de la acústica: Tiene la función de generar las ondas que excitan el estudio (\textbf{MF}$_{2.1}$) y medir la interacción de las ondas con el recinto (\textbf{MF}$_{2.2}$). El submódulo de la generación de las ondas de sonido funciona a partir de un algoritmo que genera una señal se excitación que es transmitida al estudio mediante una bocina, las ondas que son transmitidas se encargan de describir las características del estudio. Así, las señales que interactúan con el estudio posteriormente son recibidas por el submódulo de medición que consta de un micrófono que capta todas las señales que han interactuado con el recinto.

    \item Módulo modificador de la acústica: Este módulo tiene la función de generar el movimiento de los paneles acústicos considerando la compensación para disminuir los modos normales de vibración (\textbf{MF}$_{3.1}$). El movimiento se lleva a cabo mediante un mecanismo que permite su colocación en un punto donde se requiere modificar la superficie que entra en contacto con las ondas sonoras, este mecanismo es capaz de sensar el error de posición con el propósito de que el lugar en el que el módulo requiere la localización del panel sea específicamente la solicitada. 

    \item Módulo interfaz de usuario: Además de gestionar el encendido y apagado del sistema, el submódulo de interacción directa (\textbf{MF}$_{4.1}$) se encarga de recibir las instrucciones por parte del usuario para el acondicionamiento del estudio, es decir, la selección del instrumento. Al seleccionar el instrumento la consola muestra a través del submódulo de muestreo de datos (\textbf{MF}$_{4.2}$) la acústica ideal para dicho instrumento y posteriormente de que el módulo dos realiza el acondicionamiento se muestran los nuevos valores de las variables acústicas. De igual forma la interfaz muestra las variables acústicas antes de realizar el proceso de acondicionamiento y las frecuencias de vibración que son un problema para el sistema.

    \item Módulo gestor de energía: Se encarga de gestionar el suministro eléctrico requerido por cada uno de los módulos y submódulos. El submódulo de conversión y regulación de la tensión de entrada (\textbf{MF}$_{5.1}$) convierte y regula la señal de corriente alterna que es suministrada al sistema dependiendo de las características de alimentación de cada uno de los componentes del sistema. El módulo es capaz de gestionar el consumo energético del sistema y sus componentes, además de disminuir proporcionalmente la potencia en contra del consumo, lográndolo a través del submódulo de medición del consumo energético (\textbf{MF}$_{5.2}$), es decir, la medición de la tensión de entrada y la corriente que demanda el sistema.
\end{enumerate}

De igual manera se realizó la validación de los módulos y submódulos acorde a las funciones planteadas para el sistema, haciendo uso de la matriz de trazabilidad que se muestra a continuación.


%----------------------Matriz de trazabilidad--------------------------
\definecolor{gr}{gray}{.5}
\begin{center}
\footnotesize
    \begin{longtable}[!htb]{| m{3em} || m{3em} | m{3em}| m{3em}| m{3em}| m{3em}| m{3em}| m{3em}| m{3em}| m{3em}| m{3em}|}
    \hline
    & \multicolumn{3}{c|}{\textbf{MF}$_{1}$} & \multicolumn{2}{|c|}{\textbf{MF}$_{2}$} & \textbf{MF}$_{3}$ & \multicolumn{2}{|c|}{\textbf{MF}$_{4}$} & \multicolumn{2}{|c|}{\textbf{MF}$_{5}$}\\
    \cline{2-11}
    & \textbf{MF}$_{1.1}$ & \textbf{MF}$_{1.2}$ & \textbf{MF}$_{1.3}$ & \textbf{MF}$_{2.1}$ & \textbf{MF}$_{2.2}$ & \textbf{MF}$_{3.1}$ & \textbf{MF}$_{4.1}$ & \textbf{MF}$_{4.2}$ & \textbf{MF}$_{5.1}$ & \textbf{MF}$_{5.2}$ \\
    \hline\hline
    
    \textbf{F}$_{1.1}$ & & & & \cellcolor{gr}{} & & & & & & \\
    \hline
    \textbf{F}$_{1.2}$ & & & & & \cellcolor{gr}{} & & & & & \\
    \hline
    \textbf{F}$_{1.3}$ & \cellcolor{gr}{}& & & & & & & & & \\
    \hline
    \textbf{F}$_{2.1}$ & & \cellcolor{gr}{}& & & & & & & & \\
    \hline
    \textbf{F}$_{2.2}$ & & \cellcolor{gr}{}& & & & \cellcolor{gr}{} & & & & \\
    \hline
    \textbf{F}$_{2.3}$ & & & & & & \cellcolor{gr}{} & & & & \\
    \hline
    \textbf{F}$_{3.1}$ & & & \cellcolor{gr}{} & & & & \cellcolor{gr}{} & & & \\
    \hline
    \textbf{F}$_{3.2}$ & & & \cellcolor{gr}{} & & & & & \cellcolor{gr}{} & & \\
    \hline
    \textbf{F}$_{3.3}$ & & & \cellcolor{gr}{} & & & & & \cellcolor{gr}{} & & \\
    \hline
    \textbf{F}$_{3.4}$ & & & & & & & \cellcolor{gr}{} & & & \\
    \hline
    \textbf{F}$_{4.1}$ & & & & & & & & & \cellcolor{gr}{} & \\
    \hline
    \textbf{F}$_{4.2}$ & & & & & & & & & & \cellcolor{gr}{} \\
    \hline
    \textbf{F}$_{4.3}$ & & & & & & & & & & \cellcolor{gr}{} \\
    \hline
    
    \caption{\footnotesize Matriz de trazabilidad de arquitectura física}
    \label{tab:MatTrazArquiFis}
    \end{longtable}
\end{center}
%----------------------Matriz de trazabilidad--------------------------
\begin{figure}[!htb]
    \centering
    \includegraphics[width=1\textwidth]{imagenes/sistema para acondicionamiento.jpg}
    \caption{\footnotesize Diagrama del sistema para el acondicionamiento acústico}
    \label{fig:DiagramaSistema}
\end{figure}
\FloatBarrier

\subsubsection{Propuesta de solución}
Para desarrollar la propuesta de solución, primero se hizo uso de la matriz morfológica para observar las diferentes alternativas que se tienen para cada uno de los sistemas. La matriz morfológica toma los aspectos mas relevantes del diseño físico y las diferentes soluciones. Lo anterior nos permitirá generar diferentes conceptos solución, que sean combinaciones de las alternativas de solución para los diferentes sistemas.
Es necesario hacer una valoración de los conceptos solución, y para este propósito, es necesario contar con criterios que nos permitan separar los unos de los otros. De acuerdo a la metodología, se hará uso del Proceso Analítico Jerárquico (AHP por sus siglas en ingles).

\begin{itemize}
    \item $C_{r1}$ Velocidad de actuación
    \item $C_{r2}$ Peso
    \item $C_{r3}$ Complejidad del mecanismo
    \item $C_{r4}$ Cantidad de actuadores
    \item $C_{r5}$ Costo de la actuación
    \item $C_{r6}$ Resolución acústica
    \item $C_{r7}$ Repetibilidad
    \item $C_{r8}$ Consumo energético
    \item $C_{r9}$ Variedad de potencias consumidas
    \item $C_{r10}$ Facilidad del control de la actuación
    \item $C_{r11}$ Tamaño
    \item $C_{r12}$ Confiabilidad del sistema de actuación
    \item $C_{r13}$ Complejidad de manufactura del mecanismo de actuación
    \item $C_{r14}$ Error de posición 
    \item $C_{r15}$ Cantidad de hardware externo al equipo de computo personal
    \item $C_{r16}$ Relación entre señal y ruido
    \item $C_{r17}$ Sensibilidad del micrófono
    \item $C_{r18}$ Potencia de procesado de la tarjeta embebida
    \item $C_{r19}$ Potencia de procesado del equipo de computo
    \item $C_{r20}$ Simplicidad de la medición en el error de posición
    \item $C_{r21}$ Robustez necesaria del anclaje
    \item $C_{r22}$ Propensión a fallas estructurales
    \item $C_{r23}$ Propensión a fallas eléctricas
    \item $C_{r24}$ Facilidad para saltar entre disposiciones 
    \item $C_{r25}$ Modularidad
    \item $C_{r26}$ Costo de la electrónica embebida
    \item $C_{r27}$ Cantidad de paneles utilizados
\end{itemize}
La mayoría de los criterios enlistados anteriormente están íntimamente relacionados con la forma en la que el sistema mueve los paneles a lo largo del cuarto, por lo que la generación de conceptos solución, tendrán como eje principal, las diferentes formas de movimiento de los paneles.
Algunas de las alternativas de solución para sistemas específicos, son compatibles con diferentes propuestas de solución, por lo que se estudiaran aparte y se adjuntaran al concepto solución elegido en base a los criterios que si son propios del mismo.
\definecolor{gr_l}{gray}{.8}
%-----------------------------Matriz Morfológica-----------------------------
\begin{center}
\scriptsize
\centering
    \begin{longtable}[!htb]{|>{\centering\arraybackslash}m{3em} ||>{\centering\arraybackslash}m{8em} | >{\centering\arraybackslash}m{8em}| >{\centering\arraybackslash}m{8em}| >{\centering\arraybackslash}m{8em}|>{\centering\arraybackslash}m{8em}|}
    \hline
    & \textbf{Características} & \textbf{Alternativa 1} & \textbf{Alternativa 2}& \textbf{Alternativa 3}& \textbf{Alternativa 4}\\
    \hline\hline
    \textbf{$C_1$} & Forma de movimiento de los paneles & Rotacional en base prismática & Tipo pick and place & Construcción y movimiento de líneas & Paneles en configuración plegable\\
    \hline
    \textbf{$C_2$} & Método para la reflexión & Paneles de reflexión & Pared de fondo & - & -\\
    \hline
    \textbf{$C_3$} & Grabación en varias posiciones & No & Manual & Automático & Con múltiples micrófonos\\
    \hline
    \textbf{$C_4$} & Generación de ondas de sonido & Bocina de equipo de cómputo & Bocina externa & Bocina para circuitos & - \\
    \hline
    \textbf{$C_5$} & Dispositivo de grabación & Micrófono de equipo de cómputo & Micrófono externo & Micrófono para circuitos & - \\
    \hline
    \textbf{$C_6$} & Plataforma de cálculo para las posiciones deseadas & Equipo de cómputo & Microcontrolador embebido & FPGA & - \\
    \hline
    \textbf{$C_7$} & Plataforma de procesado del control & Equipo de cómputo & Microcontrolador embebido & FPGA & - \\
    \hline
    \textbf{$C_8$} & Plataforma para procesado de la acústica & Equipo de cómputo & Microcontrolador embebido & FPGA & - \\
    \hline
    \textbf{$C_9$} & Sensado de las posiciones de los paneles & Encoders & Sensores de contacto & Visión artificial & Sensor de flexión \\
    \hline
    \textbf{$C_{10}$} & Plataforma de visualización de información & Equipo de cómputo & Pantalla OLED & Pantalla LCD & - \\
    \hline
    \textbf{$C_{11}$} & Control de interfaz & Pantalla táctil & Botonera & Ratón y teclado & - \\
    \hline
    \textbf{$C_{12}$} & Encapsulado de sistemas embebidos & Si & No & - & - \\
    \hline
    \textbf{$C_{13}$} & Inclusión del sistema energético en el sistema embebido & Si & No & - & - \\
    \hline
    \textbf{$C_{14}$} & Anclaje al suelo & Si & No & - & - \\
    \hline
    \textbf{$C_{15}$} & Método de anclaje & Tornillos & Colgados & Adhesivos & Magnéticos \\
    \hline
    \textbf{$C_{16}$} & Método de comunicación & Serial & Wi-fi & Bluetooth & - \\
    \hline

    \caption{Matriz morfológica}
    \label{tab:Matriz morfológica}
    \end{longtable}
\end{center}
%-----------------------------Matriz Morfológica-----------------------------

En base a las diferentes alternativas, se crearon 4 conceptos de solución, cuya mayor diferencia es el método de movimiento para los paneles. Debido a que hay alternativas que no son dependientes del método de movimiento, la selección de estas se hará independientemente. 
%-----------------------------Conceptos solución-----------------------------
\begin{center}
\scriptsize
\centering
    \begin{longtable}[!htb]{|>{\centering\arraybackslash}m{3em} ||>{\centering\arraybackslash}m{8em} | >{\centering\arraybackslash}m{8em}| >{\centering\arraybackslash}m{8em}| >{\centering\arraybackslash}m{8em}|>{\centering\arraybackslash}m{8em}|}
    \hline
    & \textbf{Características} & \textbf{$CS_1$} & \textbf{$CS_2$}& \textbf{$CS_3$}& \textbf{$CS_4$}\\
    \hline\hline
    \textbf{$C_1$} & Forma de movimiento de los paneles & Rotacional en base prismática & Tipo pick and place & Construcción y movimiento de líneas & Paneles en configuración plegable\\
    \hline
    \textbf{$C_2$} & Tipo de actuación & Con motor y acopladores magnéticos & Con motor y actuador magnético & Con motores & Motores lineales\\
    \hline
    \textbf{$C_3$} & Sensado de las posiciones de los paneles & Encoders, sensores de contacto o visión artificial & Encoders o visión artificial & Encoders & Sensores de flexión\\
    \hline
    \textbf{$C_4$} & Posicionamiento de paneles & Anclados a la base prismática & Acoplamiento en pared & Acoplado a un cable & Acoplados entre si\\
    \hline
    \textbf{$C_5$} & Sistema de alimentación de paneles & No & Si & Si & No \\
    \hline

    \caption{Conceptos solución}
    \label{tab:ConceptosSolucion}
    \end{longtable}
\end{center}
%-----------------------------Conceptos solución---------------------------------
Primero se hará la comparación de los diferentes conceptos solución con base en cada uno de los criterios. Es importante notar que no todos los criterios se pueden aplicar para discernir entre los diferentes conceptos solución, por tanto, esos criterios se aplicaran a las alternativas que se analizaran individualmente.
%-------------------------CR1 Velocidad de actuación-----------------------------
\begin{table}[!htbp]
    \begin{minipage}[b]{0.5\linewidth}
        \scriptsize
        \centering
            \begin{tabular}{|>{\centering\arraybackslash}m{2em} ||>{\centering\arraybackslash}m{2em} | >{\centering\arraybackslash}m{2em}| >{\centering\arraybackslash}m{2em}| >{\centering\arraybackslash}m{2em}|}
            \hline
            & \textbf{$CS_1$} & \textbf{$CS_2$}& \textbf{$CS_3$}& \textbf{$CS_4$}\\
            \hline\hline
            \textbf{$CS_1$} & \cellcolor{gr_l}{1}  &  10  &    5   &   2    \\
            \textbf{$CS_2$} & 0.10 &  \cellcolor{gr_l}{1} &  0.50  &  0.20  \\
            \textbf{$CS_3$} & 0.20 &  2   &  \cellcolor{gr_l}{1}   &  0.40  \\
            \textbf{$CS_4$} & 0.50 &  5   &  2.50  &   \cellcolor{gr_l}{1}  \\ 
            \hline
        \end{tabular}
        \caption{Matriz de comparación de $Cr_1$}
        \label{tab:MComCr1}
    \end{minipage}
    \begin{minipage}[b]{0.5\linewidth}
        \scriptsize
        \centering
            \begin{tabular}{|>{\centering\arraybackslash}m{2em} ||>{\centering\arraybackslash}m{2em} | >{\centering\arraybackslash}m{2em}| >{\centering\arraybackslash}m{2em}| >{\centering\arraybackslash}m{2em}|>{\centering\arraybackslash}m{2em}|}
            \hline
            & \textbf{$CS_1$} & \textbf{$CS_2$}& \textbf{$CS_3$}& \textbf{$CS_4$}& \textbf{$V_{Cr_1}$}\\
            \hline\hline
            \textbf{$CS_1$} & 0.56 &  0.56  &   0.56   &  0.56  &  0.56   \\
            \textbf{$CS_2$} & 0.06 &  0.06  &   0.06   &  0.06  &  0.06   \\
            \textbf{$CS_3$} & 0.11 &  0.11  &   0.11   &  0.11  &  0.11   \\
            \textbf{$CS_4$} & 0.28 &  0.28  &   0.28   &  0.28  &  0.28   \\ 
            \hline
        \end{tabular}
        \caption{Matriz normalizada de $Cr_1$ y $V_{Cr_1}$}
        \label{tab:MNorm_Cr1}
    \end{minipage}
\end{table}
%-------------------------CR1 Velocidad de actuación-----------------------------

%-------------------------CR2 Peso-----------------------------
\begin{table}[!htbp]
    \begin{minipage}[b]{0.5\linewidth}
        \scriptsize
        \centering
            \begin{tabular}{|>{\centering\arraybackslash}m{2em} ||>{\centering\arraybackslash}m{2em} | >{\centering\arraybackslash}m{2em}| >{\centering\arraybackslash}m{2em}| >{\centering\arraybackslash}m{2em}|}
            \hline
            & \textbf{$CS_1$} & \textbf{$CS_2$}& \textbf{$CS_3$}& \textbf{$CS_4$}\\
            \hline\hline
            \textbf{$CS_1$} & \cellcolor{gr_l}{1} &  0.38  &  0.38   &   1   \\
            \textbf{$CS_2$} & 2.67 &  \cellcolor{gr_l}{1}  &   1     &  2.67  \\
            \textbf{$CS_3$} & 2.67 &  1      &  \cellcolor{gr_l}{1}   &  2.67  \\
            \textbf{$CS_4$} &   1  &  0.38   &  0.38  &   \cellcolor{gr_l}{1}  \\ 
            \hline
        \end{tabular}
        \caption{Matriz de comparación de $Cr_2$}
        \label{tab:MComCr2}
    \end{minipage}
    \begin{minipage}[b]{0.5\linewidth}
        \scriptsize
        \centering
            \begin{tabular}{|>{\centering\arraybackslash}m{2em} ||>{\centering\arraybackslash}m{2em} | >{\centering\arraybackslash}m{2em}| >{\centering\arraybackslash}m{2em}| >{\centering\arraybackslash}m{2em}|>{\centering\arraybackslash}m{2em}|}
            \hline
            & \textbf{$CS_1$} & \textbf{$CS_2$}& \textbf{$CS_3$}& \textbf{$CS_4$}& \textbf{$V_{Cr_2}$}\\
            \hline\hline
            \textbf{$CS_1$} & 0.14 &  0.14  &   0.14   &  0.14  &  0.14   \\
            \textbf{$CS_2$} & 0.36 &  0.36  &   0.36   &  0.36  &  0.36   \\
            \textbf{$CS_3$} & 0.36 &  0.36  &   0.36   &  0.36  &  0.36   \\
            \textbf{$CS_4$} & 0.14 &  0.14  &   0.14   &  0.14  &  0.14   \\ 
            \hline
        \end{tabular}
        \caption{Matriz normalizada de $Cr_2$ y $V_{Cr_2}$}
        \label{tab:MNorm_Cr1}
    \end{minipage}
\end{table}
%-------------------------CR2 Peso-----------------------------

%-------------------------CR3 Complejidad de mecanismo-----------------------------
\begin{table}[!htbp]
    \begin{minipage}[b]{0.5\linewidth}
        \scriptsize
        \centering
            \begin{tabular}{|>{\centering\arraybackslash}m{2em} ||>{\centering\arraybackslash}m{2em} | >{\centering\arraybackslash}m{2em}| >{\centering\arraybackslash}m{2em}| >{\centering\arraybackslash}m{2em}|}
            \hline
            & \textbf{$CS_1$} & \textbf{$CS_2$}& \textbf{$CS_3$}& \textbf{$CS_4$}\\
            \hline\hline
            \textbf{$CS_1$} & \cellcolor{gr_l}{1}  &  2.67  &    4   &   4   \\
            \textbf{$CS_2$} & 0.38 &  \cellcolor{gr_l}{1} &  1.50   &  1.50  \\
            \textbf{$CS_3$} & 0.25 &  0.67   &  \cellcolor{gr_l}{1}   &  1  \\
            \textbf{$CS_4$} & 0.25 &  0.67   &  1  &   \cellcolor{gr_l}{1}  \\ 
            \hline
        \end{tabular}
        \caption{Matriz de comparación de $Cr_3$}
        \label{tab:MComCr3}
    \end{minipage}
    \begin{minipage}[b]{0.5\linewidth}
        \scriptsize
        \centering
            \begin{tabular}{|>{\centering\arraybackslash}m{2em} ||>{\centering\arraybackslash}m{2em} | >{\centering\arraybackslash}m{2em}| >{\centering\arraybackslash}m{2em}| >{\centering\arraybackslash}m{2em}|>{\centering\arraybackslash}m{2em}|}
            \hline
            & \textbf{$CS_1$} & \textbf{$CS_2$}& \textbf{$CS_3$}& \textbf{$CS_4$}& \textbf{$V_{Cr_3}$}\\
            \hline\hline
            \textbf{$CS_1$} & 0.53 &  0.53  &   0.53   &  0.53  &  0.53   \\
            \textbf{$CS_2$} & 0.20 &  0.20  &   0.20   &  0.20  &  0.20   \\
            \textbf{$CS_3$} & 0.13 &  0.13  &   0.13   &  0.13  &  0.13   \\
            \textbf{$CS_4$} & 0.13 &  0.13  &   0.13   &  0.13  &  0.13   \\ 
            \hline
        \end{tabular}
        \caption{Matriz normalizada de $Cr_3$ y $V_{Cr_3}$}
        \label{tab:MNorm_Cr3}
    \end{minipage}
\end{table}
%-------------------------CR3 Complejidad de mecanismo-----------------------------

%-------------------------CR4 Cantidad de actuadores-----------------------------
\begin{table}[!htbp]
    \begin{minipage}[b]{0.5\linewidth}
        \scriptsize
        \centering
            \begin{tabular}{|>{\centering\arraybackslash}m{2em} ||>{\centering\arraybackslash}m{2em} | >{\centering\arraybackslash}m{2em}| >{\centering\arraybackslash}m{2em}| >{\centering\arraybackslash}m{2em}|}
            \hline
            & \textbf{$CS_1$} & \textbf{$CS_2$}& \textbf{$CS_3$}& \textbf{$CS_4$}\\
            \hline\hline
            \textbf{$CS_1$} & \cellcolor{gr_l}{1}  &  1.17  &    1.40   &   3.50   \\
            \textbf{$CS_2$} & 0.86 &  \cellcolor{gr_l}{1} &  1.20   &  3.00  \\
            \textbf{$CS_3$} & 0.71 &  0.83   &  \cellcolor{gr_l}{1}   &  2.50  \\
            \textbf{$CS_4$} & 0.29 &  0.33   &  0.40  &   \cellcolor{gr_l}{1}  \\ 
            \hline
        \end{tabular}
        \caption{Matriz de comparación de $Cr_4$}
        \label{tab:MComCr4}
    \end{minipage}
    \begin{minipage}[b]{0.5\linewidth}
        \scriptsize
        \centering
            \begin{tabular}{|>{\centering\arraybackslash}m{2em} ||>{\centering\arraybackslash}m{2em} | >{\centering\arraybackslash}m{2em}| >{\centering\arraybackslash}m{2em}| >{\centering\arraybackslash}m{2em}|>{\centering\arraybackslash}m{2em}|}
            \hline
            & \textbf{$CS_1$} & \textbf{$CS_2$}& \textbf{$CS_3$}& \textbf{$CS_4$}& \textbf{$V_{Cr_4}$}\\
            \hline\hline
            \textbf{$CS_1$} & 0.35 &  0.35  &   0.35   &  0.35  &  0.35   \\
            \textbf{$CS_2$} & 0.30 &  0.30  &   0.30   &  0.30  &  0.30   \\
            \textbf{$CS_3$} & 0.25 &  0.25  &   0.25   &  0.25  &  0.25   \\
            \textbf{$CS_4$} & 0.10 &  0.10  &   0.10   &  0.10  &  0.10   \\ 
            \hline
        \end{tabular}
        \caption{Matriz normalizada de $Cr_4$ y $V_{Cr_4}$}
        \label{tab:MNorm_Cr4}
    \end{minipage}
\end{table}
%-------------------------CR4 Cantidad de actuadores-----------------------------

%-------------------------CR5 Costo de la actuación-----------------------------
\begin{table}[!htbp]
    \begin{minipage}[b]{0.5\linewidth}
        \scriptsize
        \centering
            \begin{tabular}{|>{\centering\arraybackslash}m{2em} ||>{\centering\arraybackslash}m{2em} | >{\centering\arraybackslash}m{2em}| >{\centering\arraybackslash}m{2em}| >{\centering\arraybackslash}m{2em}|}
            \hline
            & \textbf{$CS_1$} & \textbf{$CS_2$}& \textbf{$CS_3$}& \textbf{$CS_4$}\\
            \hline\hline
            \textbf{$CS_1$} & \cellcolor{gr_l}{1}  &  0.89  &    1.14   &   4   \\
            \textbf{$CS_2$} & 1.13 &  \cellcolor{gr_l}{1} &  1.29   &  4.50  \\
            \textbf{$CS_3$} & 0.88 &  0.78   &  \cellcolor{gr_l}{1}   &  3.50  \\
            \textbf{$CS_4$} & 0.25 &  0.22   &  0.29  &   \cellcolor{gr_l}{1}  \\ 
            \hline
        \end{tabular}
        \caption{Matriz de comparación de $Cr_5$}
        \label{tab:MComCr5}
    \end{minipage}
    \begin{minipage}[b]{0.5\linewidth}
        \scriptsize
        \centering
            \begin{tabular}{|>{\centering\arraybackslash}m{2em} ||>{\centering\arraybackslash}m{2em} | >{\centering\arraybackslash}m{2em}| >{\centering\arraybackslash}m{2em}| >{\centering\arraybackslash}m{2em}|>{\centering\arraybackslash}m{2em}|}
            \hline
            & \textbf{$CS_1$} & \textbf{$CS_2$}& \textbf{$CS_3$}& \textbf{$CS_4$}& \textbf{$V_{Cr_5}$}\\
            \hline\hline
            \textbf{$CS_1$} & 0.31 &  0.31  &   0.31   &  0.31  &  0.31   \\
            \textbf{$CS_2$} & 0.35 &  0.35  &   0.35   &  0.35  &  0.35   \\
            \textbf{$CS_3$} & 0.27 &  0.27  &   0.27   &  0.27  &  0.27   \\
            \textbf{$CS_4$} & 0.08 &  0.08  &   0.08   &  0.08  &  0.08   \\ 
            \hline
        \end{tabular}
        \caption{Matriz normalizada de $Cr_5$ y $V_{Cr_5}$}
        \label{tab:MNorm_Cr5}
    \end{minipage}
\end{table}
%-------------------------CR5 Costo de la actuación-----------------------------

%-------------------------CR6 Resolución acústica-----------------------------
\begin{table}[!htbp]
    \begin{minipage}[b]{0.5\linewidth}
        \scriptsize
        \centering
            \begin{tabular}{|>{\centering\arraybackslash}m{2em} ||>{\centering\arraybackslash}m{2em} | >{\centering\arraybackslash}m{2em}| >{\centering\arraybackslash}m{2em}| >{\centering\arraybackslash}m{2em}|}
            \hline
            & \textbf{$CS_1$} & \textbf{$CS_2$}& \textbf{$CS_3$}& \textbf{$CS_4$}\\
            \hline\hline
            \textbf{$CS_1$} & \cellcolor{gr_l}{1}  &  0.90  &    0.90   &   2.25   \\
            \textbf{$CS_2$} & 1.11 &  \cellcolor{gr_l}{1} &  1   &  2.50  \\
            \textbf{$CS_3$} & 1.11 &  1   &  \cellcolor{gr_l}{1}   &  2.50  \\
            \textbf{$CS_4$} & 0.44 &  0.40   &  0.40  &   \cellcolor{gr_l}{1}  \\ 
            \hline
        \end{tabular}
        \caption{Matriz de comparación de $Cr_6$}
        \label{tab:MComCr6}
    \end{minipage}
    \begin{minipage}[b]{0.5\linewidth}
        \scriptsize
        \centering
            \begin{tabular}{|>{\centering\arraybackslash}m{2em} ||>{\centering\arraybackslash}m{2em} | >{\centering\arraybackslash}m{2em}| >{\centering\arraybackslash}m{2em}| >{\centering\arraybackslash}m{2em}|>{\centering\arraybackslash}m{2em}|}
            \hline
            & \textbf{$CS_1$} & \textbf{$CS_2$}& \textbf{$CS_3$}& \textbf{$CS_4$}& \textbf{$V_{Cr_6}$}\\
            \hline\hline
            \textbf{$CS_1$} & 0.27 &  0.27  &   0.27   &  0.27  &  0.27   \\
            \textbf{$CS_2$} & 0.30 &  0.30  &   0.30   &  0.30  &  0.30   \\
            \textbf{$CS_3$} & 0.30 &  0.30  &   0.30   &  0.30  &  0.30    \\
            \textbf{$CS_4$} & 0.12 &  0.12  &   0.12   &  0.12  &  0.12   \\ 
            \hline
        \end{tabular}
        \caption{Matriz normalizada de $Cr_6$ y $V_{Cr_6}$}
        \label{tab:MNorm_Cr6}
    \end{minipage}
\end{table}
%-------------------------CR6 Resolución acústica-----------------------------

%-------------------------CR7 Repetibilidad-----------------------------
\begin{table}[!htbp]
    \begin{minipage}[b]{0.5\linewidth}
        \scriptsize
        \centering
            \begin{tabular}{|>{\centering\arraybackslash}m{2em} ||>{\centering\arraybackslash}m{2em} | >{\centering\arraybackslash}m{2em}| >{\centering\arraybackslash}m{2em}| >{\centering\arraybackslash}m{2em}|}
            \hline
            & \textbf{$CS_1$} & \textbf{$CS_2$}& \textbf{$CS_3$}& \textbf{$CS_4$}\\
            \hline\hline
            \textbf{$CS_1$} & \cellcolor{gr_l}{1}  &  5  &    3.33   &   1.43   \\
            \textbf{$CS_2$} & 0.20 &  \cellcolor{gr_l}{1} &  0.67   &  0.29  \\
            \textbf{$CS_3$} & 0.30 &  1.50   &  \cellcolor{gr_l}{1}   &  0.43  \\
            \textbf{$CS_4$} & 0.70 &  3.50   &  2.33  &   \cellcolor{gr_l}{1}  \\ 
            \hline
        \end{tabular}
        \caption{Matriz de comparación de $Cr_7$}
        \label{tab:MComCr7}
    \end{minipage}
    \begin{minipage}[b]{0.5\linewidth}
        \scriptsize
        \centering
            \begin{tabular}{|>{\centering\arraybackslash}m{2em} ||>{\centering\arraybackslash}m{2em} | >{\centering\arraybackslash}m{2em}| >{\centering\arraybackslash}m{2em}| >{\centering\arraybackslash}m{2em}|>{\centering\arraybackslash}m{2em}|}
            \hline
            & \textbf{$CS_1$} & \textbf{$CS_2$}& \textbf{$CS_3$}& \textbf{$CS_4$}& \textbf{$V_{Cr_7}$}\\
            \hline\hline
            \textbf{$CS_1$} & 0.45 &  0.45  &   0.45   &  0.45  &  0.45   \\
            \textbf{$CS_2$} & 0.09 &  0.09  &   0.09   &  0.09  &  0.09   \\
            \textbf{$CS_3$} & 0.14 &  0.14  &   0.14   &  0.14  &  0.14    \\
            \textbf{$CS_4$} & 0.32 &  0.32  &   0.32   &  0.32  &  0.32   \\ 
            \hline
        \end{tabular}
        \caption{Matriz normalizada de $Cr_7$ y $V_{Cr_7}$}
        \label{tab:MNorm_Cr7}
    \end{minipage}
\end{table}
%-------------------------CR7 Repetibilidad-----------------------------

%-------------------------CR8 Consumo energético-----------------------------
\begin{table}[!htbp]
    \begin{minipage}[b]{0.5\linewidth}
        \scriptsize
        \centering
            \begin{tabular}{|>{\centering\arraybackslash}m{2em} ||>{\centering\arraybackslash}m{2em} | >{\centering\arraybackslash}m{2em}| >{\centering\arraybackslash}m{2em}| >{\centering\arraybackslash}m{2em}|}
            \hline
            & \textbf{$CS_1$} & \textbf{$CS_2$}& \textbf{$CS_3$}& \textbf{$CS_4$}\\
            \hline\hline
            \textbf{$CS_1$} & \cellcolor{gr_l}{1}  &  3  &    2.25   &   4.50   \\
            \textbf{$CS_2$} & 0.33 &  \cellcolor{gr_l}{1} &   0.75   &  1.50  \\
            \textbf{$CS_3$} & 0.44 &  1.33   &  \cellcolor{gr_l}{1}   &  2  \\
            \textbf{$CS_4$} & 0.22 &  0.67   &  0.50  &   \cellcolor{gr_l}{1}  \\ 
            \hline
        \end{tabular}
        \caption{Matriz de comparación de $Cr_8$}
        \label{tab:MComCr8}
    \end{minipage}
    \begin{minipage}[b]{0.5\linewidth}
        \scriptsize
        \centering
            \begin{tabular}{|>{\centering\arraybackslash}m{2em} ||>{\centering\arraybackslash}m{2em} | >{\centering\arraybackslash}m{2em}| >{\centering\arraybackslash}m{2em}| >{\centering\arraybackslash}m{2em}|>{\centering\arraybackslash}m{2em}|}
            \hline
            & \textbf{$CS_1$} & \textbf{$CS_2$}& \textbf{$CS_3$}& \textbf{$CS_4$}& \textbf{$V_{Cr_8}$}\\
            \hline\hline
            \textbf{$CS_1$} & 0.50 &  0.50  &   0.50   &  0.50  &  0.50   \\
            \textbf{$CS_2$} & 0.17 &  0.17  &   0.17   &  0.17  &  0.17   \\
            \textbf{$CS_3$} & 0.22 &  0.22  &   0.22   &  0.22  &  0.22    \\
            \textbf{$CS_4$} & 0.11 &  0.11  &   0.11   &  0.11  &  0.11   \\ 
            \hline
        \end{tabular}
        \caption{Matriz normalizada de $Cr_8$ y $V_{Cr_8}$}
        \label{tab:MNorm_Cr8}
    \end{minipage}
\end{table}
%-------------------------CR8 Consumo energético-----------------------------

%-------------------------CR9 Variedad de potencias consumidas-----------------------------
\begin{table}[!htbp]
    \begin{minipage}[b]{0.5\linewidth}
        \scriptsize
        \centering
            \begin{tabular}{|>{\centering\arraybackslash}m{2em} ||>{\centering\arraybackslash}m{2em} | >{\centering\arraybackslash}m{2em}| >{\centering\arraybackslash}m{2em}| >{\centering\arraybackslash}m{2em}|}
            \hline
            & \textbf{$CS_1$} & \textbf{$CS_2$}& \textbf{$CS_3$}& \textbf{$CS_4$}\\
            \hline\hline
            \textbf{$CS_1$} & \cellcolor{gr_l}{1}  &  1  &    0.88   &   0.70   \\
            \textbf{$CS_2$} & 1 &  \cellcolor{gr_l}{1} &      0.88   &      0.70  \\
            \textbf{$CS_3$} & 1.14 &  1.14   &  \cellcolor{gr_l}{1}   &  0.80  \\
            \textbf{$CS_4$} & 1.43 &  1.43   &  1.25  &   \cellcolor{gr_l}{1}  \\ 
            \hline
        \end{tabular}
        \caption{Matriz de comparación de $Cr_9$}
        \label{tab:MComCr9}
    \end{minipage}
    \begin{minipage}[b]{0.5\linewidth}
        \scriptsize
        \centering
            \begin{tabular}{|>{\centering\arraybackslash}m{2em} ||>{\centering\arraybackslash}m{2em} | >{\centering\arraybackslash}m{2em}| >{\centering\arraybackslash}m{2em}| >{\centering\arraybackslash}m{2em}|>{\centering\arraybackslash}m{2em}|}
            \hline
            & \textbf{$CS_1$} & \textbf{$CS_2$}& \textbf{$CS_3$}& \textbf{$CS_4$}& \textbf{$V_{Cr_9}$}\\
            \hline\hline
            \textbf{$CS_1$} & 0.22 &  0.22  &   0.22   &  0.22  &  0.22   \\
            \textbf{$CS_2$} & 0.22 &  0.22  &   0.22   &  0.22  &  0.22  \\
            \textbf{$CS_3$} & 0.25 &  0.25  &   0.25   &  0.25  &  0.25    \\
            \textbf{$CS_4$} & 0.31 &  0.31  &   0.31   &  0.31  &  0.31   \\ 
            \hline
        \end{tabular}
        \caption{Matriz normalizada de $Cr_9$ y $V_{Cr_9}$}
        \label{tab:MNorm_Cr9}
    \end{minipage}
\end{table}
%-------------------------CR9 Variedad de potencias consumidas-----------------------------

%----------------------CR10 Facilidad en el control de la actuación--------------------------
\begin{table}[!htbp]
    \begin{minipage}[b]{0.5\linewidth}
        \scriptsize
        \centering
            \begin{tabular}{|>{\centering\arraybackslash}m{2em} ||>{\centering\arraybackslash}m{2em} | >{\centering\arraybackslash}m{2em}| >{\centering\arraybackslash}m{2em}| >{\centering\arraybackslash}m{2em}|}
            \hline
            & \textbf{$CS_1$} & \textbf{$CS_2$}& \textbf{$CS_3$}& \textbf{$CS_4$}\\
            \hline\hline
            \textbf{$CS_1$} & \cellcolor{gr_l}{1}  &  9  &    4.50   &   1.29   \\
            \textbf{$CS_2$} & 0.11 &  \cellcolor{gr_l}{1} &   0.50   &   0.14  \\
            \textbf{$CS_3$} & 0.22 &  2   &  \cellcolor{gr_l}{1}   &  0.29  \\
            \textbf{$CS_4$} & 0.78 &  7   &  3.50  &   \cellcolor{gr_l}{1}  \\ 
            \hline
        \end{tabular}
        \caption{Matriz de comparación de $Cr_{10}$}
        \label{tab:MComCr10}
    \end{minipage}
    \begin{minipage}[b]{0.5\linewidth}
        \scriptsize
        \centering
            \begin{tabular}{|>{\centering\arraybackslash}m{2em} ||>{\centering\arraybackslash}m{2em} | >{\centering\arraybackslash}m{2em}| >{\centering\arraybackslash}m{2em}| >{\centering\arraybackslash}m{2em}|>{\centering\arraybackslash}m{2em}|}
            \hline
            & \textbf{$CS_1$} & \textbf{$CS_2$}& \textbf{$CS_3$}& \textbf{$CS_4$}& \textbf{$V_{Cr_{10}}$}\\
            \hline\hline
            \textbf{$CS_1$} & 0.47 &  0.47  &   0.47   &  0.47  &  0.47   \\
            \textbf{$CS_2$} & 0.05 &  0.05  &   0.05   &  0.05  &  0.05  \\
            \textbf{$CS_3$} & 0.11 &  0.11  &   0.11   &  0.11  &  0.11    \\
            \textbf{$CS_4$} & 0.37 &  0.37  &   0.37   &  0.37  &  0.37   \\ 
            \hline
        \end{tabular}
        \caption{Matriz normalizada de $Cr_{10}$ y $V_{Cr_{10}}$}
        \label{tab:MNorm_Cr10}
    \end{minipage}
\end{table}
%----------------------CR10 Facilidad en el control de la actuación--------------------------

%----------------------CR11 Tamaño--------------------------
\begin{table}[!htbp]
    \begin{minipage}[b]{0.5\linewidth}
        \scriptsize
        \centering
            \begin{tabular}{|>{\centering\arraybackslash}m{2em} ||>{\centering\arraybackslash}m{2em} | >{\centering\arraybackslash}m{2em}| >{\centering\arraybackslash}m{2em}| >{\centering\arraybackslash}m{2em}|}
            \hline
            & \textbf{$CS_1$} & \textbf{$CS_2$}& \textbf{$CS_3$}& \textbf{$CS_4$}\\
            \hline\hline
            \textbf{$CS_1$} & \cellcolor{gr_l}{1}  &  1.60  &    2   &   0.89   \\
            \textbf{$CS_2$} & 0.63 &  \cellcolor{gr_l}{1} &   1.25   &   0.56  \\
            \textbf{$CS_3$} & 0.50 &  0.80   &  \cellcolor{gr_l}{1}   &  0.44  \\
            \textbf{$CS_4$} & 1.13 &  1.80   &  2.25  &   \cellcolor{gr_l}{1}  \\ 
            \hline
        \end{tabular}
        \caption{Matriz de comparación de $Cr_{11}$}
        \label{tab:MComCr11}
    \end{minipage}
    \begin{minipage}[b]{0.5\linewidth}
        \scriptsize
        \centering
            \begin{tabular}{|>{\centering\arraybackslash}m{2em} ||>{\centering\arraybackslash}m{2em} | >{\centering\arraybackslash}m{2em}| >{\centering\arraybackslash}m{2em}| >{\centering\arraybackslash}m{2em}|>{\centering\arraybackslash}m{2em}|}
            \hline
            & \textbf{$CS_1$} & \textbf{$CS_2$}& \textbf{$CS_3$}& \textbf{$CS_4$}& \textbf{$V_{Cr_{11}}$}\\
            \hline\hline
            \textbf{$CS_1$} & 0.31 &  0.31  &   0.31   &  0.31  &  0.31   \\
            \textbf{$CS_2$} & 0.19 &  0.19  &   0.19   &  0.19  &  0.19  \\
            \textbf{$CS_3$} & 0.15 &  0.15  &   0.15   &  0.15  &  0.15    \\
            \textbf{$CS_4$} & 0.35 &  0.35  &   0.35   &  0.35  &  0.35   \\ 
            \hline
        \end{tabular}
        \caption{Matriz normalizada de $Cr_{11}$ y $V_{Cr_{11}}$}
        \label{tab:MNorm_Cr11}
    \end{minipage}
\end{table}
%----------------------CR11 Tamaño--------------------------

%----------------------CR12 Confiabilidad del sistema de actuación--------------------------
\begin{table}[!htbp]
    \begin{minipage}[b]{0.5\linewidth}
        \scriptsize
        \centering
            \begin{tabular}{|>{\centering\arraybackslash}m{2em} ||>{\centering\arraybackslash}m{2em} | >{\centering\arraybackslash}m{2em}| >{\centering\arraybackslash}m{2em}| >{\centering\arraybackslash}m{2em}|}
            \hline
            & \textbf{$CS_1$} & \textbf{$CS_2$}& \textbf{$CS_3$}& \textbf{$CS_4$}\\
            \hline\hline
            \textbf{$CS_1$} & \cellcolor{gr_l}{1}  &  5  &    5   &   2.50   \\
            \textbf{$CS_2$} & 0.20 &  \cellcolor{gr_l}{1} &   1   &   0.50  \\
            \textbf{$CS_3$} & 0.20 &  1   &  \cellcolor{gr_l}{1}   &  0.50  \\
            \textbf{$CS_4$} & 0.40 &  2   &  2  &   \cellcolor{gr_l}{1}  \\ 
            \hline
        \end{tabular}
        \caption{Matriz de comparación de $Cr_{12}$}
        \label{tab:MComCr12}
    \end{minipage}
    \begin{minipage}[b]{0.5\linewidth}
        \scriptsize
        \centering
            \begin{tabular}{|>{\centering\arraybackslash}m{2em} ||>{\centering\arraybackslash}m{2em} | >{\centering\arraybackslash}m{2em}| >{\centering\arraybackslash}m{2em}| >{\centering\arraybackslash}m{2em}|>{\centering\arraybackslash}m{2em}|}
            \hline
            & \textbf{$CS_1$} & \textbf{$CS_2$}& \textbf{$CS_3$}& \textbf{$CS_4$}& \textbf{$V_{Cr_{12}}$}\\
            \hline\hline
            \textbf{$CS_1$} & 0.56 &  0.56  &   0.56   &  0.56  &  0.56   \\
            \textbf{$CS_2$} & 0.11 &  0.11  &   0.11   &  0.11  &  0.11  \\
            \textbf{$CS_3$} & 0.11 &  0.11  &   0.11   &  0.11  &  0.11    \\
            \textbf{$CS_4$} & 0.22 &  0.22  &   0.22   &  0.22  &  0.22   \\ 
            \hline
        \end{tabular}
        \caption{Matriz normalizada de $Cr_{12}$ y $V_{Cr_{12}}$}
        \label{tab:MNorm_Cr12}
    \end{minipage}
\end{table}
%----------------------CR12 Confiabilidad del sistema de actuación--------------------------

%---------------CR13 Complejidad de manufactura del mecanismo de actuación------------------
\begin{table}[!htbp]
    \begin{minipage}[b]{0.5\linewidth}
        \scriptsize
        \centering
            \begin{tabular}{|>{\centering\arraybackslash}m{2em} ||>{\centering\arraybackslash}m{2em} | >{\centering\arraybackslash}m{2em}| >{\centering\arraybackslash}m{2em}| >{\centering\arraybackslash}m{2em}|}
            \hline
            & \textbf{$CS_1$} & \textbf{$CS_2$}& \textbf{$CS_3$}& \textbf{$CS_4$}\\
            \hline\hline
            \textbf{$CS_1$} & \cellcolor{gr_l}{1}  &  2  &    3   &   2   \\
            \textbf{$CS_2$} & 0.50 &  \cellcolor{gr_l}{1} &   1.50   &   1  \\
            \textbf{$CS_3$} & 0.33 &  0.67   &  \cellcolor{gr_l}{1}   &  0.67  \\
            \textbf{$CS_4$} & 0.50 &  1   &  1.50  &   \cellcolor{gr_l}{1}  \\ 
            \hline
        \end{tabular}
        \caption{Matriz de comparación de $Cr_{13}$}
        \label{tab:MComCr13}
    \end{minipage}
    \begin{minipage}[b]{0.5\linewidth}
        \scriptsize
        \centering
            \begin{tabular}{|>{\centering\arraybackslash}m{2em} ||>{\centering\arraybackslash}m{2em} | >{\centering\arraybackslash}m{2em}| >{\centering\arraybackslash}m{2em}| >{\centering\arraybackslash}m{2em}|>{\centering\arraybackslash}m{2em}|}
            \hline
            & \textbf{$CS_1$} & \textbf{$CS_2$}& \textbf{$CS_3$}& \textbf{$CS_4$}& \textbf{$V_{Cr_{13}}$}\\
            \hline\hline
            \textbf{$CS_1$} & 0.43 &  0.43  &   0.43   &  0.43  &  0.43   \\
            \textbf{$CS_2$} & 0.21 &  0.21  &   0.21   &  0.21  &  0.21  \\
            \textbf{$CS_3$} & 0.14 &  0.14  &   0.14   &  0.14  &  0.14    \\
            \textbf{$CS_4$} & 0.21 &  0.21  &   0.21   &  0.21  &  0.21   \\ 
            \hline
        \end{tabular}
        \caption{Matriz normalizada de $Cr_{13}$ y $V_{Cr_{13}}$}
        \label{tab:MNorm_Cr13}
    \end{minipage}
\end{table}
%---------------CR13 Complejidad de manufactura del mecanismo de actuación------------------

%---------------CR14 Error de posición------------------
\begin{table}[!htbp]
    \begin{minipage}[b]{0.5\linewidth}
        \scriptsize
        \centering
            \begin{tabular}{|>{\centering\arraybackslash}m{2em} ||>{\centering\arraybackslash}m{2em} | >{\centering\arraybackslash}m{2em}| >{\centering\arraybackslash}m{2em}| >{\centering\arraybackslash}m{2em}|}
            \hline
            & \textbf{$CS_1$} & \textbf{$CS_2$}& \textbf{$CS_3$}& \textbf{$CS_4$}\\
            \hline\hline
            \textbf{$CS_1$} & \cellcolor{gr_l}{1}  &  10  &    2   &   1.25   \\
            \textbf{$CS_2$} & 0.10 &  \cellcolor{gr_l}{1} &   0.20   &   0.13  \\
            \textbf{$CS_3$} & 0.50 &  5   &  \cellcolor{gr_l}{1}   &  0.63  \\
            \textbf{$CS_4$} & 0.80 &  8   &  1.60  &   \cellcolor{gr_l}{1}  \\ 
            \hline
        \end{tabular}
        \caption{Matriz de comparación de $Cr_{14}$}
        \label{tab:MComCr14}
    \end{minipage}
    \begin{minipage}[b]{0.5\linewidth}
        \scriptsize
        \centering
            \begin{tabular}{|>{\centering\arraybackslash}m{2em} ||>{\centering\arraybackslash}m{2em} | >{\centering\arraybackslash}m{2em}| >{\centering\arraybackslash}m{2em}| >{\centering\arraybackslash}m{2em}|>{\centering\arraybackslash}m{2em}|}
            \hline
            & \textbf{$CS_1$} & \textbf{$CS_2$}& \textbf{$CS_3$}& \textbf{$CS_4$}& \textbf{$V_{Cr_{14}}$}\\
            \hline\hline
            \textbf{$CS_1$} & 0.42 &  0.42  &   0.42   &  0.42  &  0.42   \\
            \textbf{$CS_2$} & 0.04 &  0.04  &   0.04   &  0.04  &  0.04  \\
            \textbf{$CS_3$} & 0.21 &  0.21  &   0.21   &  0.21  &  0.21    \\
            \textbf{$CS_4$} & 0.33 &  0.33  &   0.33   &  0.33  &  0.33   \\ 
            \hline
        \end{tabular}
        \caption{Matriz normalizada de $Cr_{14}$ y $V_{Cr_{14}}$}
        \label{tab:MNorm_Cr14}
    \end{minipage}
\end{table}
%---------------CR14 Error de posición------------------

%---------------CR21 Robustez necesaria del anclaje------------------
\begin{table}[!htbp]
    \begin{minipage}[b]{0.5\linewidth}
        \scriptsize
        \centering
            \begin{tabular}{|>{\centering\arraybackslash}m{2em} ||>{\centering\arraybackslash}m{2em} | >{\centering\arraybackslash}m{2em}| >{\centering\arraybackslash}m{2em}| >{\centering\arraybackslash}m{2em}|}
            \hline
            & \textbf{$CS_1$} & \textbf{$CS_2$}& \textbf{$CS_3$}& \textbf{$CS_4$}\\
            \hline\hline
            \textbf{$CS_1$} & \cellcolor{gr_l}{1}  &  2  &    2   &   1   \\
            \textbf{$CS_2$} & 0.50 &  \cellcolor{gr_l}{1} &   1   &   0.50  \\
            \textbf{$CS_3$} & 0.50 &  1   &  \cellcolor{gr_l}{1}   &  0.50  \\
            \textbf{$CS_4$} & 1 &  2   &  2  &   \cellcolor{gr_l}{1}  \\ 
            \hline
        \end{tabular}
        \caption{Matriz de comparación de $Cr_{21}$}
        \label{tab:MComCr21}
    \end{minipage}
    \begin{minipage}[b]{0.5\linewidth}
        \scriptsize
        \centering
            \begin{tabular}{|>{\centering\arraybackslash}m{2em} ||>{\centering\arraybackslash}m{2em} | >{\centering\arraybackslash}m{2em}| >{\centering\arraybackslash}m{2em}| >{\centering\arraybackslash}m{2em}|>{\centering\arraybackslash}m{2em}|}
            \hline
            & \textbf{$CS_1$} & \textbf{$CS_2$}& \textbf{$CS_3$}& \textbf{$CS_4$}& \textbf{$V_{Cr_{21}}$}\\
            \hline\hline
            \textbf{$CS_1$} & 0.33 &  0.33  &   0.33   &  0.33  &  0.33   \\
            \textbf{$CS_2$} & 0.17 &  0.17  &   0.17   &  0.17  &  0.17  \\
            \textbf{$CS_3$} & 0.17 &  0.17  &   0.17   &  0.17  &  0.17    \\
            \textbf{$CS_4$} & 0.33 &  0.33  &   0.33   &  0.33  &  0.33   \\ 
            \hline
        \end{tabular}
        \caption{Matriz normalizada de $Cr_{21}$ y $V_{Cr_{21}}$}
        \label{tab:MNorm_Cr21}
    \end{minipage}
\end{table}
%---------------CR21 Robustez necesaria del anclaje------------------

%---------------CR22 Propension a fallas estructurales------------------
\begin{table}[!htbp]
    \begin{minipage}[b]{0.5\linewidth}
        \scriptsize
        \centering
            \begin{tabular}{|>{\centering\arraybackslash}m{2em} ||>{\centering\arraybackslash}m{2em} | >{\centering\arraybackslash}m{2em}| >{\centering\arraybackslash}m{2em}| >{\centering\arraybackslash}m{2em}|}
            \hline
            & \textbf{$CS_1$} & \textbf{$CS_2$}& \textbf{$CS_3$}& \textbf{$CS_4$}\\
            \hline\hline
            \textbf{$CS_1$} & \cellcolor{gr_l}{1}  &  7  &    7   &   1   \\
            \textbf{$CS_2$} & 0.14 &  \cellcolor{gr_l}{1} &   1   &   0.14  \\
            \textbf{$CS_3$} & 0.14 &  1   &  \cellcolor{gr_l}{1}   &  0.14  \\
            \textbf{$CS_4$} & 1 &  7   &  7  &   \cellcolor{gr_l}{1}  \\ 
            \hline
        \end{tabular}
        \caption{Matriz de comparación de $Cr_{22}$}
        \label{tab:MComCr22}
    \end{minipage}
    \begin{minipage}[b]{0.5\linewidth}
        \scriptsize
        \centering
            \begin{tabular}{|>{\centering\arraybackslash}m{2em} ||>{\centering\arraybackslash}m{2em} | >{\centering\arraybackslash}m{2em}| >{\centering\arraybackslash}m{2em}| >{\centering\arraybackslash}m{2em}|>{\centering\arraybackslash}m{2em}|}
            \hline
            & \textbf{$CS_1$} & \textbf{$CS_2$}& \textbf{$CS_3$}& \textbf{$CS_4$}& \textbf{$V_{Cr_{22}}$}\\
            \hline\hline
            \textbf{$CS_1$} & 0.44 &  0.44  &   0.44   &  0.44  &  0.44   \\
            \textbf{$CS_2$} & 0.06 &  0.06  &   0.06   &  0.06  &  0.06  \\
            \textbf{$CS_3$} & 0.06 &  0.06  &   0.06   &  0.06  &  0.06    \\
            \textbf{$CS_4$} & 0.44 &  0.44  &   0.44   &  0.44  &  0.44   \\ 
            \hline
        \end{tabular}
        \caption{Matriz normalizada de $Cr_{22}$ y $V_{Cr_{22}}$}
        \label{tab:MNorm_Cr22}
    \end{minipage}
\end{table}
%---------------CR22 Propension a fallas estructurales------------------

%---------------CR24 Facilidad para saltar entre disposiciones------------------
\begin{table}[!htbp]
    \begin{minipage}[b]{0.5\linewidth}
        \scriptsize
        \centering
            \begin{tabular}{|>{\centering\arraybackslash}m{2em} ||>{\centering\arraybackslash}m{2em} | >{\centering\arraybackslash}m{2em}| >{\centering\arraybackslash}m{2em}| >{\centering\arraybackslash}m{2em}|}
            \hline
            & \textbf{$CS_1$} & \textbf{$CS_2$}& \textbf{$CS_3$}& \textbf{$CS_4$}\\
            \hline\hline
            \textbf{$CS_1$} & \cellcolor{gr_l}{1}  &  10  &    5   &   1.25   \\
            \textbf{$CS_2$} & 0.10 &  \cellcolor{gr_l}{1} &   0.50   &   0.13  \\
            \textbf{$CS_3$} & 0.20 &  2   &  \cellcolor{gr_l}{1}   &  0.25  \\
            \textbf{$CS_4$} & 0.80 &  8   &  4  &   \cellcolor{gr_l}{1}  \\ 
            \hline
        \end{tabular}
        \caption{Matriz de comparación de $Cr_{24}$}
        \label{tab:MComCr24}
    \end{minipage}
    \begin{minipage}[b]{0.5\linewidth}
        \scriptsize
        \centering
            \begin{tabular}{|>{\centering\arraybackslash}m{2em} ||>{\centering\arraybackslash}m{2em} | >{\centering\arraybackslash}m{2em}| >{\centering\arraybackslash}m{2em}| >{\centering\arraybackslash}m{2em}|>{\centering\arraybackslash}m{2em}|}
            \hline
            & \textbf{$CS_1$} & \textbf{$CS_2$}& \textbf{$CS_3$}& \textbf{$CS_4$}& \textbf{$V_{Cr_{24}}$}\\
            \hline\hline
            \textbf{$CS_1$} & 0.48 &  0.48  &   0.48   &  0.48  &  0.48   \\
            \textbf{$CS_2$} & 0.05 &  0.05  &   0.05   &  0.05  &  0.05  \\
            \textbf{$CS_3$} & 0.10 &  0.10  &   0.10   &  0.10  &  0.10    \\
            \textbf{$CS_4$} & 0.38 &  0.38  &   0.38   &  0.38  &  0.38   \\ 
            \hline
        \end{tabular}
        \caption{Matriz normalizada de $Cr_{24}$ y $V_{Cr_{24}}$}
        \label{tab:MNorm_Cr24}
    \end{minipage}
\end{table}
%---------------CR24 Facilidad para saltar entre disposiciones------------------

%-------------------CR25 Modularidad----------------------
\begin{table}[!htbp]
    \begin{minipage}[b]{0.5\linewidth}
        \scriptsize
        \centering
            \begin{tabular}{|>{\centering\arraybackslash}m{2em} ||>{\centering\arraybackslash}m{2em} | >{\centering\arraybackslash}m{2em}| >{\centering\arraybackslash}m{2em}| >{\centering\arraybackslash}m{2em}|}
            \hline
            & \textbf{$CS_1$} & \textbf{$CS_2$}& \textbf{$CS_3$}& \textbf{$CS_4$}\\
            \hline\hline
            \textbf{$CS_1$} & \cellcolor{gr_l}{1}  &  4  &    2.67   &   1.33   \\
            \textbf{$CS_2$} & 0.25 &  \cellcolor{gr_l}{1} &   0.67   &   0.33  \\
            \textbf{$CS_3$} & 0.38 &  1.50   &  \cellcolor{gr_l}{1}   &  0.50  \\
            \textbf{$CS_4$} & 0.75 &  3   &  2  &   \cellcolor{gr_l}{1}  \\ 
            \hline
        \end{tabular}
        \caption{Matriz de comparación de $Cr_{25}$}
        \label{tab:MComCr25}
    \end{minipage}
    \begin{minipage}[b]{0.5\linewidth}
        \scriptsize
        \centering
            \begin{tabular}{|>{\centering\arraybackslash}m{2em} ||>{\centering\arraybackslash}m{2em} | >{\centering\arraybackslash}m{2em}| >{\centering\arraybackslash}m{2em}| >{\centering\arraybackslash}m{2em}|>{\centering\arraybackslash}m{2em}|}
            \hline
            & \textbf{$CS_1$} & \textbf{$CS_2$}& \textbf{$CS_3$}& \textbf{$CS_4$}& \textbf{$V_{Cr_{25}}$}\\
            \hline\hline
            \textbf{$CS_1$} & 0.42 &  0.42  &   0.42   &  0.42  &  0.42   \\
            \textbf{$CS_2$} & 0.11 &  0.11  &   0.11   &  0.11  &  0.11  \\
            \textbf{$CS_3$} & 0.16 &  0.16  &   0.16   &  0.16  &  0.16    \\
            \textbf{$CS_4$} & 0.32 &  0.32  &   0.32   &  0.32  &  0.32   \\ 
            \hline
        \end{tabular}
        \caption{Matriz normalizada de $Cr_{25}$ y $V_{Cr_{25}}$}
        \label{tab:MNorm_Cr25}
    \end{minipage}
\end{table}
%-------------------CR25 Modularidad----------------------

%-------------------CR27 Cantidad de paneles utilizados----------------------
\begin{table}[!htbp]
    \begin{minipage}[b]{0.5\linewidth}
        \scriptsize
        \centering
            \begin{tabular}{|>{\centering\arraybackslash}m{2em} ||>{\centering\arraybackslash}m{2em} | >{\centering\arraybackslash}m{2em}| >{\centering\arraybackslash}m{2em}| >{\centering\arraybackslash}m{2em}|}
            \hline
            & \textbf{$CS_1$} & \textbf{$CS_2$}& \textbf{$CS_3$}& \textbf{$CS_4$}\\
            \hline\hline
            \textbf{$CS_1$} & \cellcolor{gr_l}{1}  &  0.57  &    0.57   &   1   \\
            \textbf{$CS_2$} & 1.75 &  \cellcolor{gr_l}{1} &   1   &   1.75  \\
            \textbf{$CS_3$} & 1.75 &  1   &  \cellcolor{gr_l}{1}   &  1.75  \\
            \textbf{$CS_4$} & 1  &   0.57   &  0.57  &   \cellcolor{gr_l}{1}  \\ 
            \hline
        \end{tabular}
        \caption{Matriz de comparación de $Cr_{27}$}
        \label{tab:MComCr27}
    \end{minipage}
    \begin{minipage}[b]{0.5\linewidth}
        \scriptsize
        \centering
            \begin{tabular}{|>{\centering\arraybackslash}m{2em} ||>{\centering\arraybackslash}m{2em} | >{\centering\arraybackslash}m{2em}| >{\centering\arraybackslash}m{2em}| >{\centering\arraybackslash}m{2em}|>{\centering\arraybackslash}m{2em}|}
            \hline
            & \textbf{$CS_1$} & \textbf{$CS_2$}& \textbf{$CS_3$}& \textbf{$CS_4$}& \textbf{$V_{Cr_{27}}$}\\
            \hline\hline
            \textbf{$CS_1$} & 0.18 &  0.18  &   0.18   &  0.18  &  0.18   \\
            \textbf{$CS_2$} & 0.32 &  0.32  &   0.32   &  0.32  &  0.32  \\
            \textbf{$CS_3$} & 0.32 &  0.32  &   0.32   &  0.32  &  0.32    \\
            \textbf{$CS_4$} & 0.18 &  0.18  &   0.18   &  0.18  &  0.18   \\ 
            \hline
        \end{tabular}
        \caption{Matriz normalizada de $Cr_{27}$ y $V_{Cr_{27}}$}
        \label{tab:MNorm_Cr27}
    \end{minipage}
\end{table}
%-------------------CR27 Cantidad de paneles utilizados----------------------
A continuación, y siguiendo la metodología, se deben comparar los criterios de modo que nos permitan obtener un vector de prioridad. Lo anterior es resultado de la diferencia en la importancia que se le asigna a cada criterio. 
%-------------------Matriz de comparacion de criterios----------------------
\begin{landscape}
    \begin{table}[!htb]
    \scriptsize
    \centering
        \begin{tabular}{|>{\centering\arraybackslash}m{2em} ||>{\centering\arraybackslash}m{2em} | >{\centering\arraybackslash}m{2em}| >{\centering\arraybackslash}m{2em}| >{\centering\arraybackslash}m{2em}|>{\centering\arraybackslash}m{2em}|>{\centering\arraybackslash}m{2em}|>{\centering\arraybackslash}m{2em}|>{\centering\arraybackslash}m{2em}|>{\centering\arraybackslash}m{2em}|>{\centering\arraybackslash}m{2em}|>{\centering\arraybackslash}m{2em}|>{\centering\arraybackslash}m{2em}|>{\centering\arraybackslash}m{2em}|>{\centering\arraybackslash}m{2em}|>{\centering\arraybackslash}m{2em}|>{\centering\arraybackslash}m{2em}|>{\centering\arraybackslash}m{2em}|>{\centering\arraybackslash}m{2em}|>{\centering\arraybackslash}m{2em}|} % Especifica las columnas aquí
            \hline
             &$Cr_{1}$ & $Cr_{2}$ & $Cr_{3}$ & $Cr_{4}$ & $Cr_{5}$ & $Cr_{6}$ & $Cr_{7}$ & $Cr_{8}$ & $Cr_{9}$ & $Cr_{10}$ & $Cr_{11}$ & $Cr_{12}$ & $Cr_{13}$ & $Cr_{14}$& $Cr_{21}$ & $Cr_{22}$ & $Cr_{24}$ & $Cr_{25}$ & $Cr_{27}$\\
            \hline
            \hline
            $Cr_{1}$ & \cellcolor{gr_l}{1} & 5 & 0.75 & 3 & 3 & 1 & 3 & 5 & 10 & 5 & 3 & 2 & 5 & 2 & 8 & 2 & 1& 5& 4 \\
            $Cr_{2}$ & 0.20 & \cellcolor{gr_l}{1} & 6.67 & 1.67 & 1.67 & 5 & 1.67 & 1 & 0.50 & 1 & 1.67 & 2.50 & 1 & 2.50 & 0.63 & 2.50 & 5 & 1 & 1.25\\
            $Cr_{3}$ & 1.33 & 0.15 & \cellcolor{gr_l}{1} & 0.25 & 0.25 & 0.75 & 0.25 & 0.15 & 0.08 & 0.15 & 0.25 & 0.38 & 0.15 & 0.38 & 0.09 & 0.38 & 0.75 & 0.15 & 0.19\\
            $Cr_{4}$ & 0.33 & 0.60 & 4 & \cellcolor{gr_l}{1} & 1 & 3 & 1 & 0.60 & 0.30 & 0.60 & 1 & 1.50 & 0.60 & 1.50 & 0.38 & 1.50 & 3 & 0.60 & 0.75\\
            $Cr_{5}$ & 0.33 & 0.60 & 4 & 1 & \cellcolor{gr_l}{1} & 3 & 1 & 0.60 & 0.30 & 0.60 & 1 & 1.50 & 0.60 & 1.50 & 0.38 & 1.50 & 3 & 0.60 & 0.75\\
            $Cr_{6}$ & 1 & 0.20 & 1.33 & 0.33 & 0.33 & \cellcolor{gr_l}{1} & 0.33 & 0.20 & 0.10 & 0.20 & 0.33 & 0.50 & 0.20 & 0.50 & 0.13 & 0.50 & 1 & 0.20 & 0.25\\
            $Cr_{7}$ & 0.33 & 0.60 & 4 & 1 & 1 & 3 & \cellcolor{gr_l}{1} & 0.60 & 0.30 & 0.60 & 1 & 1.50 & 0.60 & 1.50 & 0.38 & 1.50 & 3 & 0.60 & 0.75\\
            $Cr_{8}$ & 0.20 & 1 & 6.67 & 1.67 & 1.67 & 5 & 1.67 & \cellcolor{gr_l}{1} & 0.50 & 1 & 1.67 & 2.50 & 1 & 2.50 & 0.63 & 2.50 & 5 & 1 & 1.25\\
            $Cr_{9}$ & 0.10 & 2 & 13.33 & 3.33 & 3.33 & 10 & 3.33 & 2 & \cellcolor{gr_l}{1} & 2 & 3.33 & 5 & 2 & 5 & 1.25 & 5 & 10 & 2 & 2.50\\
            $Cr_{10}$ & 0.20 & 1 & 6.67 & 1.67 & 1.67 & 5 & 1.67 & 1 & 0.50 & \cellcolor{gr_l}{1} & 1.67 & 2.50 & 1 & 2.50 & 0.63 & 2.50 & 5 & 1 & 1.25\\
            $Cr_{11}$ & 0.33 & 0.60 & 4 & 1 & 1 & 3 & 1 & 0.60 & 0.30 & 0.60 & \cellcolor{gr_l}{1} & 1.50 & 0.60 & 1.50 & 0.38 & 1.50 & 3 & 0.60 & 0.75\\
            $Cr_{12}$ & 0.50 & 0.40 & 2.67 & 0.67 & 0.67 & 2 & 0.67 & 0.40 & 0.20 & 0.40 & 0.67 & \cellcolor{gr_l}{1} & 0.40 & 1 & 0.25 & 1 & 2 & 0.40 & 0.50\\
            $Cr_{13}$ & 0.20 & 1 & 6.67 & 1.67 & 1.67 & 5 & 1.67 & 1 & 0.50 & 1 & 1.67 & 2.50 & \cellcolor{gr_l}{1} & 2.50 & 0.63 & 2.50 & 5 & 1 & 1.25\\
            $Cr_{14}$ & 0.50 & 0.40 & 2.67 & 0.67 & 0.67 & 2 & 0.67 & 0.40 & 0.20 & 0.40 & 0.67 & 1 & 0.40 & \cellcolor{gr_l}{1} & 0.25 & 1 & 2 & 0.40 & 0.50\\
            $Cr_{21}$ & 0.13 & 1.60 & 10.67 & 2.67 & 2.67 & 8 & 2.67 & 1.60 & 0.80 & 1.60 & 2.67 & 4 & 1.60 & 4 & \cellcolor{gr_l}{1} & 4 & 8 & 1.60 & 2\\
            $Cr_{22}$ & 0.50 & 0.40 & 2.67 & 0.67 & 0.67 & 2 & 0.67 & 0.40 & 0.20 & 0.40 & 0.67 & 1 & 0.40 & 1 & 0.25 & \cellcolor{gr_l}{1} & 2 & 0.40 & 0.50\\
            $Cr_{24}$ & 1 & 0.20 & 1.33 & 0.33 & 0.33 & 1 & 0.33 & 0.20 & 0.10 & 0.20 & 0.33 & 0.50 & 0.20 & 0.50 & 0.13 & 0.50 & \cellcolor{gr_l}{1} & 0.20 & 0.25\\
            $Cr_{25}$ & 0.20 & 1 & 6.67 & 1.67 & 1.67 & 5 & 1.67 & 1 & 0.50 & 1 & 1.67 & 2.50 & 1 & 2.50 & 0.63 & 2.50 & 5 & \cellcolor{gr_l}{1} & 1.25\\
            $Cr_{27}$ & 0.25 & 0.80 & 5.33 & 1.33 & 1.33 & 4 & 1.33 & 0.80 & 0.40 & 0.80 & 1.33 & 2 & 0.80 & 2 & 0.50 & 2 & 4 & 0.80 & \cellcolor{gr_l}{1}\\
            % Añade más filas según sea necesario
            \hline
        \end{tabular}
        \caption{Matriz de comparación de criterios}
        \label{tab:MCompCrit}
    \end{table}
%--------------------------------------------------------------------------------------------
    \begin{table}[!htb]
    \scriptsize
    \centering
        \begin{tabular}{|>{\centering\arraybackslash}m{2em} ||>{\centering\arraybackslash}m{2em} | >{\centering\arraybackslash}m{2em}| >{\centering\arraybackslash}m{2em}| >{\centering\arraybackslash}m{2em}|>{\centering\arraybackslash}m{2em}|>{\centering\arraybackslash}m{2em}|>{\centering\arraybackslash}m{2em}|>{\centering\arraybackslash}m{2em}|>{\centering\arraybackslash}m{2em}|>{\centering\arraybackslash}m{2em}|>{\centering\arraybackslash}m{2em}|>{\centering\arraybackslash}m{2em}|>{\centering\arraybackslash}m{2em}|>{\centering\arraybackslash}m{2em}|>{\centering\arraybackslash}m{2em}|>{\centering\arraybackslash}m{2em}|>{\centering\arraybackslash}m{2em}|>{\centering\arraybackslash}m{2em}|>{\centering\arraybackslash}m{2em}|>{\centering\arraybackslash}m{2em}|} % Especifica las columnas aquí
            \hline
             &$Cr_{1}$ & $Cr_{2}$ & $Cr_{3}$ & $Cr_{4}$ & $Cr_{5}$ & $Cr_{6}$ & $Cr_{7}$ & $Cr_{8}$ & $Cr_{9}$ & $Cr_{10}$ & $Cr_{11}$ & $Cr_{12}$ & $Cr_{13}$ & $Cr_{14}$& $Cr_{21}$ & $Cr_{22}$ & $Cr_{24}$ & $Cr_{25}$ & $Cr_{27}$ & $V_{C_r}$\\
            \hline
            \hline
            $Cr_{1}$ & 0.116 & 0.270 & 0.008 & 0.117 & 0.117 & 0.015 & 0.117 & 0.270 & 0.596 & 0.270 & 0.117 & 0.056 & 0.270 & 0.056 & 0.486 & 0.056 & 0.015 & 0.270 & 0.191 & 0.133 \\
            $Cr_{2}$ & 0.023 & 0.054 & 0.073 & 0.065 & 0.065 & 0.073 & 0.065 & 0.054 & 0.030 & 0.054 & 0.065 & 0.070 & 0.054 & 0.070 & 0.038 & 0.070 & 0.073 & 0.054 & 0.060 & 0.064 \\
            $Cr_{3}$ & 0.154 & 0.008 & 0.011 & 0.010 & 0.010 & 0.011 & 0.010 & 0.008 & 0.004 & 0.008 & 0.010 & 0.010 & 0.008 & 0.010 & 0.006 & 0.010 & 0.011 & 0.008 & 0.009 & 0.010 \\
            $Cr_{4}$ & 0.039 & 0.032 & 0.044 & 0.039 & 0.039 & 0.044 & 0.039 & 0.032 & 0.018 & 0.032 & 0.039 & 0.042 & 0.032 & 0.042 & 0.023 & 0.042 & 0.044 & 0.032 & 0.036 & 0.038 \\
            $Cr_{5}$ & 0.039 & 0.032 & 0.044 & 0.039 & 0.039 & 0.044 & 0.039 & 0.032 & 0.018 & 0.032 & 0.039 & 0.042 & 0.032 & 0.042 & 0.023 & 0.042 & 0.044 & 0.032 & 0.036 & 0.038 \\
            $Cr_{6}$ & 0.116 & 0.011 & 0.015 & 0.013 & 0.013 & 0.015 & 0.013 & 0.011 & 0.006 & 0.011 & 0.013 & 0.014 & 0.011 & 0.014 & 0.008 & 0.014 & 0.015 & 0.011 & 0.012 & 0.013 \\
            $Cr_{7}$ & 0.039 & 0.032 & 0.044 & 0.039 & 0.039 & 0.044 & 0.039 & 0.032 & 0.018 & 0.032 & 0.039 & 0.042 & 0.032 & 0.042 & 0.023 & 0.042 & 0.044 & 0.032 & 0.036 & 0.038 \\
            $Cr_{8}$ & 0.023 & 0.054 & 0.073 & 0.065 & 0.065 & 0.073 & 0.065 & 0.054 & 0.030 & 0.054 & 0.065 & 0.070 & 0.054 & 0.070 & 0.038 & 0.070 & 0.073 & 0.054 & 0.060 & 0.064 \\
            $Cr_{9}$ & 0.012 & 0.108 & 0.146 & 0.130 & 0.130 & 0.145 & 0.130 & 0.108 & 0.060 & 0.108 & 0.130 & 0.139 & 0.108 & 0.139 & 0.076 & 0.139 & 0.145 & 0.108 & 0.119 & 0.128 \\
            $Cr_{10}$ & 0.023 & 0.054 & 0.073 & 0.065 & 0.065 & 0.073 & 0.065 & 0.054 & 0.030 & 0.054 & 0.065 & 0.070 & 0.054 & 0.070 & 0.038 & 0.070 & 0.073 & 0.054 & 0.060 & 0.064 \\
            $Cr_{11}$ & 0.039 & 0.032 & 0.044 & 0.039 & 0.039 & 0.044 & 0.039 & 0.032 & 0.018 & 0.032 & 0.039 & 0.042 & 0.032 & 0.042 & 0.023 & 0.042 & 0.044 & 0.032 & 0.036 & 0.038 \\
            $Cr_{12}$ & 0.058 & 0.022 & 0.029 & 0.026 & 0.026 & 0.029 & 0.026 & 0.022 & 0.012 & 0.022 & 0.026 & 0.028 & 0.022 & 0.028 & 0.015 & 0.028 & 0.029 & 0.022 & 0.024 & 0.026 \\
            $Cr_{13}$ & 0.023 & 0.054 & 0.073 & 0.065 & 0.065 & 0.073 & 0.065 & 0.054 & 0.030 & 0.054 & 0.065 & 0.070 & 0.054 & 0.070 & 0.038 & 0.070 & 0.073 & 0.054 & 0.060 & 0.064 \\
            $Cr_{14}$ & 0.058 & 0.022 & 0.029 & 0.026 & 0.026 & 0.029 & 0.026 & 0.022 & 0.012 & 0.022 & 0.026 & 0.028 & 0.022 & 0.028 & 0.015 & 0.028 & 0.029 & 0.022 & 0.024 & 0.026 \\
            $Cr_{21}$ & 0.014 & 0.086 & 0.117 & 0.104 & 0.104 & 0.116 & 0.104 & 0.086 & 0.048 & 0.086 & 0.104 & 0.111 & 0.086 & 0.111 & 0.061 & 0.111 & 0.116 & 0.086 & 0.096 & 0.102\\
            $Cr_{22}$ & 0.058 & 0.022 & 0.029 & 0.026 & 0.026 & 0.029 & 0.026 & 0.022 & 0.012 & 0.022 & 0.026 & 0.028 & 0.022 & 0.028 & 0.015 & 0.028 & 0.029 & 0.022 & 0.024 & 0.026 \\
            $Cr_{24}$ & 0.116 & 0.011 & 0.015 & 0.013 & 0.013 & 0.015 & 0.013 & 0.011 & 0.006 & 0.011 & 0.013 & 0.014 & 0.011 & 0.014 & 0.008 & 0.014 & 0.015 & 0.011 & 0.012 & 0.013 \\
            $Cr_{25}$ & 0.023 & 0.054 & 0.073 & 0.065 & 0.065 & 0.073 & 0.065 & 0.054 & 0.030 & 0.054 & 0.065 & 0.070 & 0.054 & 0.070 & 0.038 & 0.070 & 0.073 & 0.054 & 0.060 & 0.064 \\
            $Cr_{27}$ & 0.029 & 0.043 & 0.059 & 0.052 & 0.052 & 0.058 & 0.052 & 0.043 & 0.024 & 0.043 & 0.052 & 0.056 & 0.043 & 0.056 & 0.030 & 0.056 & 0.058 & 0.043 & 0.048 & 0.051 \\
            % Añade más filas según sea necesario
            \hline
        \end{tabular}
        \caption{Matriz normalizada y vector de prioridad para la comparación de criterios.}
        \label{tab:MNormCrit}
    \end{table}
\end{landscape}

%-------------------Matriz de comparacion de criterios----------------------


%-------------------AHP Independientes----------------------
Ahora, se analizan las alternativas independientes para la decisión usando la misma metodología.

%-------------------C3 Grabación en varias prosiciones----------------------
\begin{table}[!htbp]
    \begin{minipage}[b]{0.5\linewidth}
        \scriptsize
        \centering
            \begin{tabular}{|>{\centering\arraybackslash}m{2em} ||>{\centering\arraybackslash}m{2em} | >{\centering\arraybackslash}m{2em}| >{\centering\arraybackslash}m{2em}| >{\centering\arraybackslash}m{2em}|}
            \hline
            & \textbf{$Alt_1$} & \textbf{$Alt_2$}& \textbf{$Alt_3$}& \textbf{$Alt_4$}\\
            \hline\hline
            \textbf{$Alt_1$} & \cellcolor{gr_l}{1}  &  1  &    2.67   &   1.33   \\
            \textbf{$Alt_2$} & 1 &  \cellcolor{gr_l}{1} &   2.67   &   1.33  \\
            \textbf{$Alt_3$} & 0.38 &  0.38   &  \cellcolor{gr_l}{1}   &  0.50  \\
            \textbf{$Alt_4$} & 0.75  &   0.75   &  2  &   \cellcolor{gr_l}{1}  \\ 
            \hline
        \end{tabular}
        \caption{Matriz de comparación de $C_{3}$}
        \label{tab:MComC3}
    \end{minipage}
    \begin{minipage}[b]{0.5\linewidth}
        \scriptsize
        \centering
            \begin{tabular}{|>{\centering\arraybackslash}m{2em} ||>{\centering\arraybackslash}m{2em} | >{\centering\arraybackslash}m{2em}| >{\centering\arraybackslash}m{2em}| >{\centering\arraybackslash}m{2em}|>{\centering\arraybackslash}m{2em}|}
            \hline
            & \textbf{$Alt_1$} & \textbf{$Alt_2$}& \textbf{$Alt_3$}& \textbf{$Alt_4$}& \textbf{$V_{C_{3}}$}\\
            \hline\hline
            \textbf{$Alt_1$} & 0.32 &  0.32  &   0.32   &  0.32  & \cellcolor{gr_l}{0.32}   \\
            \textbf{$Alt_2$} & 0.32 &  0.32  &   0.32   &  0.32  &  0.32  \\
            \textbf{$Alt_3$} & 0.12 &  0.12  &   0.12   &  0.12  &  0.12    \\
            \textbf{$Alt_4$} & 0.24 &  0.24  &   0.24   &  0.24  &  0.24   \\ 
            \hline
        \end{tabular}
        \caption{Matriz normalizada de $C_{3}$ y $V_{C_{3}}$}
        \label{tab:MNorm_C3}
    \end{minipage}
\end{table}
%-------------------C3 Grabación en varias prosiciones----------------------

%-------------------C4 Generación de ondas de sonido----------------------
\begin{table}[!htbp]
    \begin{minipage}[b]{0.5\linewidth}
        \scriptsize
        \centering
            \begin{tabular}{|>{\centering\arraybackslash}m{2em} ||>{\centering\arraybackslash}m{2em} | >{\centering\arraybackslash}m{2em}| >{\centering\arraybackslash}m{2em}| >{\centering\arraybackslash}m{2em}|}
            \hline
            & \textbf{$Alt_1$} & \textbf{$Alt_2$}& \textbf{$Alt_3$}& \textbf{$Alt_4$}\\
            \hline\hline
            \textbf{$Alt_1$} & \cellcolor{gr_l}{1}&         0.50         &      1.67            &   -   \\
            \textbf{$Alt_2$} &          2         &  \cellcolor{gr_l}{1} &      3.33            &   -   \\
            \textbf{$Alt_3$} &          0.60      &         0.30         &  \cellcolor{gr_l}{1} &   -   \\
            \textbf{$Alt_4$} &          -         &          -           &       -              &   \cellcolor{gr_l}{-}  \\ 
            \hline
        \end{tabular}
        \caption{Matriz de comparación de $C_{4}$}
        \label{tab:MComC4}
    \end{minipage}
    \begin{minipage}[b]{0.5\linewidth}
        \scriptsize
        \centering
            \begin{tabular}{|>{\centering\arraybackslash}m{2em} ||>{\centering\arraybackslash}m{2em} | >{\centering\arraybackslash}m{2em}| >{\centering\arraybackslash}m{2em}| >{\centering\arraybackslash}m{2em}|>{\centering\arraybackslash}m{2em}|}
            \hline
            & \textbf{$Alt_1$} & \textbf{$Alt_2$}& \textbf{$Alt_3$}& \textbf{$Alt_4$}& \textbf{$V_{C_{4}}$}\\
            \hline\hline
            \textbf{$Alt_1$} & 0.28 &  0.28  &   0.28   &    -   &  0.28   \\
            \textbf{$Alt_2$} & 0.56 &  0.56  &   0.56   &    -   &  \cellcolor{gr_l}{0.56}  \\
            \textbf{$Alt_3$} & 0.17 &  0.17  &   0.17   &    -   &  0.17    \\
            \textbf{$Alt_4$} &   -  &   -    &    -     &    -   &    -   \\ 
            \hline
        \end{tabular}
        \caption{Matriz normalizada de $C_{4}$ y $V_{C_{4}}$}
        \label{tab:MNorm_C4}
    \end{minipage}
\end{table}
%-------------------C4 Generación de ondas de sonido----------------------

%-------------------C5 dispositivo de grabación----------------------
\begin{table}[!htbp]
    \begin{minipage}[b]{0.5\linewidth}
        \scriptsize
        \centering
            \begin{tabular}{|>{\centering\arraybackslash}m{2em} ||>{\centering\arraybackslash}m{2em} | >{\centering\arraybackslash}m{2em}| >{\centering\arraybackslash}m{2em}| >{\centering\arraybackslash}m{2em}|}
            \hline
            & \textbf{$Alt_1$} & \textbf{$Alt_2$}& \textbf{$Alt_3$}& \textbf{$Alt_4$}\\
            \hline\hline
            \textbf{$Alt_1$} & \cellcolor{gr_l}{1}&         0.80         &      2               &   -   \\
            \textbf{$Alt_2$} &          1.25      &  \cellcolor{gr_l}{1} &      2.50            &   -   \\
            \textbf{$Alt_3$} &          0.50      &         0.40         &  \cellcolor{gr_l}{1} &   -   \\
            \textbf{$Alt_4$} &          -         &          -           &       -              &   \cellcolor{gr_l}{-}  \\ 
            \hline
        \end{tabular}
        \caption{Matriz de comparación de $C_{5}$}
        \label{tab:MComC5}
    \end{minipage}
    \begin{minipage}[b]{0.5\linewidth}
        \scriptsize
        \centering
            \begin{tabular}{|>{\centering\arraybackslash}m{2em} ||>{\centering\arraybackslash}m{2em} | >{\centering\arraybackslash}m{2em}| >{\centering\arraybackslash}m{2em}| >{\centering\arraybackslash}m{2em}|>{\centering\arraybackslash}m{2em}|}
            \hline
            & \textbf{$Alt_1$} & \textbf{$Alt_2$}& \textbf{$Alt_3$}& \textbf{$Alt_4$}& \textbf{$V_{C_{5}}$}\\
            \hline\hline
            \textbf{$Alt_1$} & 0.36 &  0.36  &   0.36   &    -   &  0.36   \\
            \textbf{$Alt_2$} & 0.45 &  0.45  &   0.45   &    -   &  \cellcolor{gr_l}{0.45}  \\
            \textbf{$Alt_3$} & 0.18 &  0.18  &   0.18   &    -   &  0.18    \\
            \textbf{$Alt_4$} &   -  &   -    &    -     &    -   &    -   \\ 
            \hline
        \end{tabular}
        \caption{Matriz normalizada de $C_{5}$ y $V_{C_{5}}$}
        \label{tab:MNorm_C5}
    \end{minipage}
\end{table}
%-------------------C5 dispositivo de grabación----------------------

%-------------------C6 Plataforma de cálculo de las posiciones deseadas----------------------
\begin{table}[!htbp]
    \begin{minipage}[b]{0.5\linewidth}
        \scriptsize
        \centering
            \begin{tabular}{|>{\centering\arraybackslash}m{2em} ||>{\centering\arraybackslash}m{2em} | >{\centering\arraybackslash}m{2em}| >{\centering\arraybackslash}m{2em}| >{\centering\arraybackslash}m{2em}|}
            \hline
            & \textbf{$Alt_1$} & \textbf{$Alt_2$}& \textbf{$Alt_3$}& \textbf{$Alt_4$}\\
            \hline\hline
            \textbf{$Alt_1$} & \cellcolor{gr_l}{1}&         0.89         &      1.33            &   -   \\
            \textbf{$Alt_2$} &          1.13      &  \cellcolor{gr_l}{1} &      1.50            &   -   \\
            \textbf{$Alt_3$} &          0.75      &         0.67         &  \cellcolor{gr_l}{1} &   -   \\
            \textbf{$Alt_4$} &          -         &          -           &       -              &   \cellcolor{gr_l}{-}  \\ 
            \hline
        \end{tabular}
        \caption{Matriz de comparación de $C_{6}$}
        \label{tab:MComC6}
    \end{minipage}
    \begin{minipage}[b]{0.5\linewidth}
        \scriptsize
        \centering
            \begin{tabular}{|>{\centering\arraybackslash}m{2em} ||>{\centering\arraybackslash}m{2em} | >{\centering\arraybackslash}m{2em}| >{\centering\arraybackslash}m{2em}| >{\centering\arraybackslash}m{2em}|>{\centering\arraybackslash}m{2em}|}
            \hline
            & \textbf{$Alt_1$} & \textbf{$Alt_2$}& \textbf{$Alt_3$}& \textbf{$Alt_4$}& \textbf{$V_{C_{6}}$}\\
            \hline\hline
            \textbf{$Alt_1$} & 0.35 &  0.35  &   0.35   &    -   &  0.35   \\
            \textbf{$Alt_2$} & 0.39 &  0.39  &   0.39   &    -   &  \cellcolor{gr_l}{0.39}  \\
            \textbf{$Alt_3$} & 0.26 &  0.26  &   0.26   &    -   &  0.26    \\
            \textbf{$Alt_4$} &   -  &   -    &    -     &    -   &    -   \\ 
            \hline
        \end{tabular}
        \caption{Matriz normalizada de $C_{6}$ y $V_{C_{6}}$}
        \label{tab:MNorm_C6}
    \end{minipage}
\end{table}
%-------------------C6 Plataforma de cálculo de las posiciones deseadas----------------------

%-------------------C7 Plataforma del procesado del control----------------------
\begin{table}[!htbp]
    \begin{minipage}[b]{0.5\linewidth}
        \scriptsize
        \centering
            \begin{tabular}{|>{\centering\arraybackslash}m{2em} ||>{\centering\arraybackslash}m{2em} | >{\centering\arraybackslash}m{2em}| >{\centering\arraybackslash}m{2em}| >{\centering\arraybackslash}m{2em}|}
            \hline
            & \textbf{$Alt_1$} & \textbf{$Alt_2$}& \textbf{$Alt_3$}& \textbf{$Alt_4$}\\
            \hline\hline
            \textbf{$Alt_1$} & \cellcolor{gr_l}{1}&         0.89         &      1.33            &   -   \\
            \textbf{$Alt_2$} &          1.13      &  \cellcolor{gr_l}{1} &      1.50            &   -   \\
            \textbf{$Alt_3$} &          0.75      &         0.67         &  \cellcolor{gr_l}{1} &   -   \\
            \textbf{$Alt_4$} &          -         &          -           &       -              &   \cellcolor{gr_l}{-}  \\ 
            \hline
        \end{tabular}
        \caption{Matriz de comparación de $C_{7}$}
        \label{tab:MComC7}
    \end{minipage}
    \begin{minipage}[b]{0.5\linewidth}
        \scriptsize
        \centering
            \begin{tabular}{|>{\centering\arraybackslash}m{2em} ||>{\centering\arraybackslash}m{2em} | >{\centering\arraybackslash}m{2em}| >{\centering\arraybackslash}m{2em}| >{\centering\arraybackslash}m{2em}|>{\centering\arraybackslash}m{2em}|}
            \hline
            & \textbf{$Alt_1$} & \textbf{$Alt_2$}& \textbf{$Alt_3$}& \textbf{$Alt_4$}& \textbf{$V_{C_{7}}$}\\
            \hline\hline
            \textbf{$Alt_1$} & 0.35 &  0.35  &   0.35   &    -   &  0.35   \\
            \textbf{$Alt_2$} & 0.39 &  0.39  &   0.39   &    -   &  \cellcolor{gr_l}{0.39}  \\
            \textbf{$Alt_3$} & 0.26 &  0.26  &   0.26   &    -   &  0.26    \\
            \textbf{$Alt_4$} &   -  &   -    &    -     &    -   &    -   \\ 
            \hline
        \end{tabular}
        \caption{Matriz normalizada de $C_{7}$ y $V_{C_{7}}$}
        \label{tab:MNorm_C7}
    \end{minipage}
\end{table}
%-------------------C7 Plataforma del procesado del control----------------------

%-------------------C8 Plataforma de procesado de la acústica----------------------
\begin{table}[!htbp]
    \begin{minipage}[b]{0.5\linewidth}
        \scriptsize
        \centering
            \begin{tabular}{|>{\centering\arraybackslash}m{2em} ||>{\centering\arraybackslash}m{2em} | >{\centering\arraybackslash}m{2em}| >{\centering\arraybackslash}m{2em}| >{\centering\arraybackslash}m{2em}|}
            \hline
            & \textbf{$Alt_1$} & \textbf{$Alt_2$}& \textbf{$Alt_3$}& \textbf{$Alt_4$}\\
            \hline\hline
            \textbf{$Alt_1$} & \cellcolor{gr_l}{1}&         0.89         &      1.33            &   -   \\
            \textbf{$Alt_2$} &          1.13      &  \cellcolor{gr_l}{1} &      1.50            &   -   \\
            \textbf{$Alt_3$} &          0.75      &         0.67         &  \cellcolor{gr_l}{1} &   -   \\
            \textbf{$Alt_4$} &          -         &          -           &       -              &   \cellcolor{gr_l}{-}  \\ 
            \hline
        \end{tabular}
        \caption{Matriz de comparación de $C_{8}$}
        \label{tab:MComC8}
    \end{minipage}
    \begin{minipage}[b]{0.5\linewidth}
        \scriptsize
        \centering
            \begin{tabular}{|>{\centering\arraybackslash}m{2em} ||>{\centering\arraybackslash}m{2em} | >{\centering\arraybackslash}m{2em}| >{\centering\arraybackslash}m{2em}| >{\centering\arraybackslash}m{2em}|>{\centering\arraybackslash}m{2em}|}
            \hline
            & \textbf{$Alt_1$} & \textbf{$Alt_2$}& \textbf{$Alt_3$}& \textbf{$Alt_4$}& \textbf{$V_{C_{8}}$}\\
            \hline\hline
            \textbf{$Alt_1$} & 0.35 &  0.35  &   0.35   &    -   &  0.35   \\
            \textbf{$Alt_2$} & 0.39 &  0.39  &   0.39   &    -   &  \cellcolor{gr_l}{0.39}  \\
            \textbf{$Alt_3$} & 0.26 &  0.26  &   0.26   &    -   &  0.26    \\
            \textbf{$Alt_4$} &   -  &   -    &    -     &    -   &    -   \\ 
            \hline
        \end{tabular}
        \caption{Matriz normalizada de $C_{8}$ y $V_{C_{8}}$}
        \label{tab:MNorm_C8}
    \end{minipage}
\end{table}
%-------------------C8 Plataforma de procesado de la acústica----------------------

%-------------------C9 Sensado de las posiciones de los paneles----------------------
\begin{table}[!htbp]
    \begin{minipage}[b]{0.5\linewidth}
        \scriptsize
        \centering
            \begin{tabular}{|>{\centering\arraybackslash}m{2em} ||>{\centering\arraybackslash}m{2em} | >{\centering\arraybackslash}m{2em}| >{\centering\arraybackslash}m{2em}| >{\centering\arraybackslash}m{2em}|}
            \hline
            & \textbf{$Alt_1$} & \textbf{$Alt_2$}& \textbf{$Alt_3$}& \textbf{$Alt_4$}\\
            \hline\hline
            \textbf{$Alt_1$} & \cellcolor{gr_l}{1}&         3            &      4.50            &   -   \\
            \textbf{$Alt_2$} &          0.33      &  \cellcolor{gr_l}{1} &      1.50            &   -   \\
            \textbf{$Alt_3$} &          0.22      &         0.67         &  \cellcolor{gr_l}{1} &   -   \\
            \textbf{$Alt_4$} &          -         &          -           &       -              &   \cellcolor{gr_l}{-}  \\ 
            \hline
        \end{tabular}
        \caption{Matriz de comparación de $C_{9}$}
        \label{tab:MComC9}
    \end{minipage}
    \begin{minipage}[b]{0.5\linewidth}
        \scriptsize
        \centering
            \begin{tabular}{|>{\centering\arraybackslash}m{2em} ||>{\centering\arraybackslash}m{2em} | >{\centering\arraybackslash}m{2em}| >{\centering\arraybackslash}m{2em}| >{\centering\arraybackslash}m{2em}|>{\centering\arraybackslash}m{2em}|}
            \hline
            & \textbf{$Alt_1$} & \textbf{$Alt_2$}& \textbf{$Alt_3$}& \textbf{$Alt_4$}& \textbf{$V_{C_{9}}$}\\
            \hline\hline
            \textbf{$Alt_1$} & 0.64 &  0.64  &   0.64   &    -   &  \cellcolor{gr_l}{0.64}   \\
            \textbf{$Alt_2$} & 0.21 &  0.21  &   0.21   &    -   &  0.21  \\
            \textbf{$Alt_3$} & 0.14 &  0.14  &   0.14   &    -   &  0.14    \\
            \textbf{$Alt_4$} &   -  &   -    &    -     &    -   &    -   \\ 
            \hline
        \end{tabular}
        \caption{Matriz normalizada de $C_{9}$ y $V_{C_{9}}$}
        \label{tab:MNorm_C9}
    \end{minipage}
\end{table}
%-------------------C9 Sensado de las posiciones de los paneles----------------------

%-------------------C10 Plataforma de visualizacion de información----------------------
\begin{table}[!htbp]
    \begin{minipage}[b]{0.5\linewidth}
        \scriptsize
        \centering
            \begin{tabular}{|>{\centering\arraybackslash}m{2em} ||>{\centering\arraybackslash}m{2em} | >{\centering\arraybackslash}m{2em}| >{\centering\arraybackslash}m{2em}| >{\centering\arraybackslash}m{2em}|}
            \hline
            & \textbf{$Alt_1$} & \textbf{$Alt_2$}& \textbf{$Alt_3$}& \textbf{$Alt_4$}\\
            \hline\hline
            \textbf{$Alt_1$} & \cellcolor{gr_l}{1}&         1.25         &      1.43            &   -   \\
            \textbf{$Alt_2$} &          0.80      &  \cellcolor{gr_l}{1} &      1.14            &   -   \\
            \textbf{$Alt_3$} &          0.70      &         0.88         &  \cellcolor{gr_l}{1} &   -   \\
            \textbf{$Alt_4$} &          -         &          -           &       -              &   \cellcolor{gr_l}{-}  \\ 
            \hline
        \end{tabular}
        \caption{Matriz de comparación de $C_{10}$}
        \label{tab:MComC10}
    \end{minipage}
    \begin{minipage}[b]{0.5\linewidth}
        \scriptsize
        \centering
            \begin{tabular}{|>{\centering\arraybackslash}m{2em} ||>{\centering\arraybackslash}m{2em} | >{\centering\arraybackslash}m{2em}| >{\centering\arraybackslash}m{2em}| >{\centering\arraybackslash}m{2em}|>{\centering\arraybackslash}m{2em}|}
            \hline
            & \textbf{$Alt_1$} & \textbf{$Alt_2$}& \textbf{$Alt_3$}& \textbf{$Alt_4$}& \textbf{$V_{C_{10}}$}\\
            \hline\hline
            \textbf{$Alt_1$} & 0.40 &  0.40  &   0.40   &    -   &  \cellcolor{gr_l}{0.40}   \\
            \textbf{$Alt_2$} & 0.32 &  0.32  &   0.32   &    -   &  0.32  \\
            \textbf{$Alt_3$} & 0.28 &  0.28  &   0.28   &    -   &  0.28    \\
            \textbf{$Alt_4$} &   -  &   -    &    -     &    -   &    -   \\ 
            \hline
        \end{tabular}
        \caption{Matriz normalizada de $C_{10}$ y $V_{C_{10}}$}
        \label{tab:MNorm_C10}
    \end{minipage}
\end{table}
%-------------------C10 Plataforma de visualizacion de información----------------------

%-------------------C11 Control de interfaz----------------------
\begin{table}[!htbp]
    \begin{minipage}[b]{0.5\linewidth}
        \scriptsize
        \centering
            \begin{tabular}{|>{\centering\arraybackslash}m{2em} ||>{\centering\arraybackslash}m{2em} | >{\centering\arraybackslash}m{2em}| >{\centering\arraybackslash}m{2em}| >{\centering\arraybackslash}m{2em}|}
            \hline
            & \textbf{$Alt_1$} & \textbf{$Alt_2$}& \textbf{$Alt_3$}& \textbf{$Alt_4$}\\
            \hline\hline
            \textbf{$Alt_1$} & \cellcolor{gr_l}{1}&         1.14         &      0.80            &   -   \\
            \textbf{$Alt_2$} &          0.88      &  \cellcolor{gr_l}{1} &      0.70            &   -   \\
            \textbf{$Alt_3$} &          1.25      &         1.43         &  \cellcolor{gr_l}{1} &   -   \\
            \textbf{$Alt_4$} &          -         &          -           &       -              &   \cellcolor{gr_l}{-}  \\ 
            \hline
        \end{tabular}
        \caption{Matriz de comparación de $C_{11}$}
        \label{tab:MComC11}
    \end{minipage}
    \begin{minipage}[b]{0.5\linewidth}
        \scriptsize
        \centering
            \begin{tabular}{|>{\centering\arraybackslash}m{2em} ||>{\centering\arraybackslash}m{2em} | >{\centering\arraybackslash}m{2em}| >{\centering\arraybackslash}m{2em}| >{\centering\arraybackslash}m{2em}|>{\centering\arraybackslash}m{2em}|}
            \hline
            & \textbf{$Alt_1$} & \textbf{$Alt_2$}& \textbf{$Alt_3$}& \textbf{$Alt_4$}& \textbf{$V_{C_{11}}$}\\
            \hline\hline
            \textbf{$Alt_1$} & 0.32 &  0.32  &   0.32   &    -   &  0.32    \\
            \textbf{$Alt_2$} & 0.28 &  0.28  &   0.28   &    -   &  0.28   \\
            \textbf{$Alt_3$} & 0.40 &  0.40  &   0.40   &    -   &  \cellcolor{gr_l}{0.40}   \\
            \textbf{$Alt_4$} &   -  &   -    &    -     &    -   &    -   \\ 
            \hline
        \end{tabular}
        \caption{Matriz normalizada de $C_{11}$ y $V_{C_{11}}$}
        \label{tab:MNorm_C11}
    \end{minipage}
\end{table}
%-------------------C11 Control de interfaz----------------------

%-------------------C12 Encapsulado de sistemas embebidos----------------------
\begin{table}[!htbp]
    \begin{minipage}[b]{0.5\linewidth}
        \scriptsize
        \centering
            \begin{tabular}{|>{\centering\arraybackslash}m{2em} ||>{\centering\arraybackslash}m{2em} | >{\centering\arraybackslash}m{2em}| >{\centering\arraybackslash}m{2em}| >{\centering\arraybackslash}m{2em}|}
            \hline
            & \textbf{$Alt_1$} & \textbf{$Alt_2$}& \textbf{$Alt_3$}& \textbf{$Alt_4$}\\
            \hline\hline
            \textbf{$Alt_1$} & \cellcolor{gr_l}{1}&         5            &      -               &   -   \\
            \textbf{$Alt_2$} &          0.20      &  \cellcolor{gr_l}{1} &      -               &   -   \\
            \textbf{$Alt_3$} &          -         &         -            &  \cellcolor{gr_l}{-} &   -   \\
            \textbf{$Alt_4$} &          -         &          -           &       -              &   \cellcolor{gr_l}{-}  \\ 
            \hline
        \end{tabular}
        \caption{Matriz de comparación de $C_{12}$}
        \label{tab:MComC12}
    \end{minipage}
    \begin{minipage}[b]{0.5\linewidth}
        \scriptsize
        \centering
            \begin{tabular}{|>{\centering\arraybackslash}m{2em} ||>{\centering\arraybackslash}m{2em} | >{\centering\arraybackslash}m{2em}| >{\centering\arraybackslash}m{2em}| >{\centering\arraybackslash}m{2em}|>{\centering\arraybackslash}m{2em}|}
            \hline
            & \textbf{$Alt_1$} & \textbf{$Alt_2$}& \textbf{$Alt_3$}& \textbf{$Alt_4$}& \textbf{$V_{C_{12}}$}\\
            \hline\hline
            \textbf{$Alt_1$} & 0.83 &  0.83  &    -     &    -   &   \cellcolor{gr_l}{0.83}    \\
            \textbf{$Alt_2$} & 0.17 &  0.17  &   -      &    -   &  0.17   \\
            \textbf{$Alt_3$} &  -   &  -     &   -      &    -   &  -      \\
            \textbf{$Alt_4$} &   -  &   -    &    -     &    -   &    -   \\ 
            \hline
        \end{tabular}
        \caption{Matriz normalizada de $C_{12}$ y $V_{C_{12}}$}
        \label{tab:MNorm_C12}
    \end{minipage}
\end{table}
%-------------------C12 Encapsulado de sistemas embebidos----------------------

%-------------------C13 Inclusión del sistema energético en sistema embebido----------------------
\begin{table}[!htbp]
    \begin{minipage}[b]{0.5\linewidth}
        \scriptsize
        \centering
            \begin{tabular}{|>{\centering\arraybackslash}m{2em} ||>{\centering\arraybackslash}m{2em} | >{\centering\arraybackslash}m{2em}| >{\centering\arraybackslash}m{2em}| >{\centering\arraybackslash}m{2em}|}
            \hline
            & \textbf{$Alt_1$} & \textbf{$Alt_2$}& \textbf{$Alt_3$}& \textbf{$Alt_4$}\\
            \hline\hline
            \textbf{$Alt_1$} & \cellcolor{gr_l}{1}&         1.60         &      -               &   -   \\
            \textbf{$Alt_2$} &          0.63      &  \cellcolor{gr_l}{1} &      -               &   -   \\
            \textbf{$Alt_3$} &          -         &         -            &  \cellcolor{gr_l}{-} &   -   \\
            \textbf{$Alt_4$} &          -         &          -           &       -              &   \cellcolor{gr_l}{-}  \\ 
            \hline
        \end{tabular}
        \caption{Matriz de comparación de $C_{13}$}
        \label{tab:MComC13}
    \end{minipage}
    \begin{minipage}[b]{0.5\linewidth}
        \scriptsize
        \centering
            \begin{tabular}{|>{\centering\arraybackslash}m{2em} ||>{\centering\arraybackslash}m{2em} | >{\centering\arraybackslash}m{2em}| >{\centering\arraybackslash}m{2em}| >{\centering\arraybackslash}m{2em}|>{\centering\arraybackslash}m{2em}|}
            \hline
            & \textbf{$Alt_1$} & \textbf{$Alt_2$}& \textbf{$Alt_3$}& \textbf{$Alt_4$}& \textbf{$V_{C_{13}}$}\\
            \hline\hline
            \textbf{$Alt_1$} & 0.62 &  0.62  &    -     &    -   &   \cellcolor{gr_l}{0.62}    \\
            \textbf{$Alt_2$} & 0.38 &  0.38  &   -      &    -   &  0.38   \\
            \textbf{$Alt_3$} &  -   &  -     &   -      &    -   &  -      \\
            \textbf{$Alt_4$} &   -  &   -    &    -     &    -   &    -   \\ 
            \hline
        \end{tabular}
        \caption{Matriz normalizada de $C_{13}$ y $V_{C_{13}}$}
        \label{tab:MNorm_C13}
    \end{minipage}
\end{table}
%-------------------C13 Inclusión del sistema energético en sistema embebido----------------------

%-------------------C14 Anclaje al suelo----------------------
\begin{table}[!htbp]
    \begin{minipage}[b]{0.5\linewidth}
        \scriptsize
        \centering
            \begin{tabular}{|>{\centering\arraybackslash}m{2em} ||>{\centering\arraybackslash}m{2em} | >{\centering\arraybackslash}m{2em}| >{\centering\arraybackslash}m{2em}| >{\centering\arraybackslash}m{2em}|}
            \hline
            & \textbf{$Alt_1$} & \textbf{$Alt_2$}& \textbf{$Alt_3$}& \textbf{$Alt_4$}\\
            \hline\hline
            \textbf{$Alt_1$} & \cellcolor{gr_l}{1}&         1.50         &      -               &   -   \\
            \textbf{$Alt_2$} &          0.67      &  \cellcolor{gr_l}{1} &      -               &   -   \\
            \textbf{$Alt_3$} &          -         &         -            &  \cellcolor{gr_l}{-} &   -   \\
            \textbf{$Alt_4$} &          -         &          -           &       -              &   \cellcolor{gr_l}{-}  \\ 
            \hline
        \end{tabular}
        \caption{Matriz de comparación de $C_{14}$}
        \label{tab:MComC14}
    \end{minipage}
    \begin{minipage}[b]{0.5\linewidth}
        \scriptsize
        \centering
            \begin{tabular}{|>{\centering\arraybackslash}m{2em} ||>{\centering\arraybackslash}m{2em} | >{\centering\arraybackslash}m{2em}| >{\centering\arraybackslash}m{2em}| >{\centering\arraybackslash}m{2em}|>{\centering\arraybackslash}m{2em}|}
            \hline
            & \textbf{$Alt_1$} & \textbf{$Alt_2$}& \textbf{$Alt_3$}& \textbf{$Alt_4$}& \textbf{$V_{C_{14}}$}\\
            \hline\hline
            \textbf{$Alt_1$} & 0.60 &  0.60  &    -     &    -   &   \cellcolor{gr_l}{0.60}    \\
            \textbf{$Alt_2$} & 0.40 &  0.40  &   -      &    -   &  0.40   \\
            \textbf{$Alt_3$} &  -   &  -     &   -      &    -   &  -      \\
            \textbf{$Alt_4$} &   -  &   -    &    -     &    -   &    -   \\ 
            \hline
        \end{tabular}
        \caption{Matriz normalizada de $C_{14}$ y $V_{C_{14}}$}
        \label{tab:MNorm_C14}
    \end{minipage}
\end{table}
%-------------------C14 Anclaje al suelo----------------------

%-------------------C15 Método de anclaje----------------------
\begin{table}[!htbp]
    \begin{minipage}[b]{0.5\linewidth}
        \scriptsize
        \centering
            \begin{tabular}{|>{\centering\arraybackslash}m{2em} ||>{\centering\arraybackslash}m{2em} | >{\centering\arraybackslash}m{2em}| >{\centering\arraybackslash}m{2em}| >{\centering\arraybackslash}m{2em}|}
            \hline
            & \textbf{$Alt_1$} & \textbf{$Alt_2$}& \textbf{$Alt_3$}& \textbf{$Alt_4$}\\
            \hline\hline
            \textbf{$Alt_1$} & \cellcolor{gr_l}{1}  &  1.50  &    2.25   &   4.50  \\
            \textbf{$Alt_2$} & 0.67 &  \cellcolor{gr_l}{1} &   1.50   &   3  \\
            \textbf{$Alt_3$} & 0.44 &  0.67   &  \cellcolor{gr_l}{1}   &  2  \\
            \textbf{$Alt_4$} & 0.22  &   0.33   &  0.50  &   \cellcolor{gr_l}{1}  \\ 
            \hline
        \end{tabular}
        \caption{Matriz de comparación de $C_{15}$}
        \label{tab:MComC15}
    \end{minipage}
    \begin{minipage}[b]{0.5\linewidth}
        \scriptsize
        \centering
            \begin{tabular}{|>{\centering\arraybackslash}m{2em} ||>{\centering\arraybackslash}m{2em} | >{\centering\arraybackslash}m{2em}| >{\centering\arraybackslash}m{2em}| >{\centering\arraybackslash}m{2em}|>{\centering\arraybackslash}m{2em}|}
            \hline
            & \textbf{$Alt_1$} & \textbf{$Alt_2$}& \textbf{$Alt_3$}& \textbf{$Alt_4$}& \textbf{$V_{C_{15}}$}\\
            \hline\hline
            \textbf{$Alt_1$} & 0.43 &  0.43  &   0.43   &  0.43  & \cellcolor{gr_l}{0.43}   \\
            \textbf{$Alt_2$} & 0.29 &  0.29  &   0.29   &  0.29  &  0.29  \\
            \textbf{$Alt_3$} & 0.19 &  0.19  &   0.19   &  0.19  &  0.19    \\
            \textbf{$Alt_4$} & 0.10 &  0.10  &   0.10   &  0.10  &  0.10   \\ 
            \hline
        \end{tabular}
        \caption{Matriz normalizada de $C_{15}$ y $V_{C_{15}}$}
        \label{tab:MNorm_C15}
    \end{minipage}
\end{table}
%-------------------C15 Método de anclaje----------------------

%-------------------C16 Método de comunicación----------------------

\begin{table}[!htbp]
    \begin{minipage}[b]{0.5\linewidth}
        \scriptsize
        \centering
            \begin{tabular}{|>{\centering\arraybackslash}m{2em} ||>{\centering\arraybackslash}m{2em} | >{\centering\arraybackslash}m{2em}| >{\centering\arraybackslash}m{2em}| >{\centering\arraybackslash}m{2em}|}
            \hline
            & \textbf{$Alt_1$} & \textbf{$Alt_2$}& \textbf{$Alt_3$}& \textbf{$Alt_4$}\\
            \hline\hline
            \textbf{$Alt_1$} & \cellcolor{gr_l}{1}&         0.78         &      0.88            &   -   \\
            \textbf{$Alt_2$} &          1.29      &  \cellcolor{gr_l}{1} &      1.13            &   -   \\
            \textbf{$Alt_3$} &          1.14      &         0.89         &  \cellcolor{gr_l}{1} &   -   \\
            \textbf{$Alt_4$} &          -         &          -           &       -              &   \cellcolor{gr_l}{-}  \\ 
            \hline
        \end{tabular}
        \caption{Matriz de comparación de $C_{16}$}
        \label{tab:MComC16}
    \end{minipage}
    \begin{minipage}[b]{0.5\linewidth}
        \scriptsize
        \centering
            \begin{tabular}{|>{\centering\arraybackslash}m{2em} ||>{\centering\arraybackslash}m{2em} | >{\centering\arraybackslash}m{2em}| >{\centering\arraybackslash}m{2em}| >{\centering\arraybackslash}m{2em}|>{\centering\arraybackslash}m{2em}|}
            \hline
            & \textbf{$Alt_1$} & \textbf{$Alt_2$}& \textbf{$Alt_3$}& \textbf{$Alt_4$}& \textbf{$V_{C_{16}}$}\\
            \hline\hline
            \textbf{$Alt_1$} & 0.29 &  0.29  &   0.29   &    -   &  0.29    \\
            \textbf{$Alt_2$} & 0.38 &  0.38  &   0.38   &    -   &  \cellcolor{gr_l}{0.38} \\
            \textbf{$Alt_3$} & 0.33 &  0.33  &   0.33   &    -   &  0.33   \\
            \textbf{$Alt_4$} &   -  &   -    &    -     &    -   &    -   \\ 
            \hline
        \end{tabular}
        \caption{Matriz normalizada de $C_{16}$ y $V_{C_{16}}$}
        \label{tab:MNorm_C16}
    \end{minipage}
\end{table}
\FloatBarrier
%-------------------C16 Método de comunicación----------------------
Una vez que se cuenta con los valores del cumplimiento de cada alternativa de solución con respecto a cada criterio, se utiliza en conjunto con el vector de prioridad para obtener la puntuación de cada alternativa de solución. La alternativa elegida sera la de mayor valor.
%------------------------------Selección----------------------------------
\begin{center}
\footnotesize
\centering
    \begin{longtable}[!htb]{|>{\centering\arraybackslash}m{3em} ||>{\centering\arraybackslash}m{3em} | >{\centering\arraybackslash}m{3em}| >{\centering\arraybackslash}m{3em}| >{\centering\arraybackslash}m{3em}|>{\centering\arraybackslash}m{3em}|}
    \hline
    & \textbf{$CS_1$} & \textbf{$CS_2$}& \textbf{$CS_3$}& \textbf{$CS_4$} & \textbf{$V_{Cr}$}\\
    \hline\hline
    $V_{Cr_{1}}$ &  0.56 & 0.06 & 0.11 & 0.28 & 0.133\\
    \hline
    $V_{Cr_{2}}$ &  0.14 & 0.36 & 0.36 & 0.14 & 0.064\\
    \hline
    $V_{Cr_{3}}$ &  0.53 & 0.20 & 0.13 & 0.13 & 0.010\\
    \hline
    $V_{Cr_{4}}$ &  0.35 & 0.30 & 0.25 & 0.10 & 0.038\\
    \hline
    $V_{Cr_{5}}$ & 0.31 & 0.35 & 0.27 & 0.08 & 0.038\\
    \hline
    $V_{Cr_{6}}$ & 0.27 & 0.30 & 0.30 & 0.12 & 0.013\\
    \hline
    $V_{Cr_{7}}$ & 0.45 & 0.09 & 0.14 & 0.32 & 0.038\\
    \hline
    $V_{Cr_{8}}$ & 0.50 & 0.17 & 0.22 & 0.11 & 0.064\\
    \hline
    $V_{Cr_{9}}$ & 0.22 & 0.22 & 0.25 & 0.31 & 0.128\\
    \hline
    $V_{Cr_{10}}$ & 0.47 & 0.05 & 0.11 & 0.37 & 0.064\\
    \hline
    $V_{Cr_{11}}$ & 0.31 & 0.19 & 0.15 & 0.35 & 0.038\\
    \hline
    $V_{Cr_{12}}$ & 0.56 & 0.11 & 0.11 & 0.22 & 0.026\\
    \hline
    $V_{Cr_{13}}$ & 0.43 & 0.21 & 0.14 & 0.21 & 0.064\\
    \hline
    $V_{Cr_{14}}$ & 0.42 & 0.04 & 0.21 & 0.33 & 0.026\\
    \hline
    $V_{Cr_{21}}$ & 0.33 & 0.17 & 0.17 & 0.33 & 0.102\\
    \hline
    $V_{Cr_{22}}$ & 0.44 & 0.06 & 0.06 & 0.44 & 0.026\\
    \hline
    $V_{Cr_{24}}$ & 0.48 & 0.05 & 0.10 & 0.38 & 0.013\\
    \hline
    $V_{Cr_{25}}$ & 0.42 & 0.11 & 0.16 & 0.32 & 0.064\\
    \hline
    $V_{Cr_{27}}$ & 0.18 & 0.32 & 0.32 & 0.18 & 0.051\\
    \hline
    \textbf{Total} & \cellcolor{gr_l}{0.376} & 0.174 & 0.191 & 0.259 & 1.000\\
    \hline
    
    \caption{Matriz de decisión}
    \label{tab:MatrizDesicion}
    \end{longtable}
\end{center}
\FloatBarrier
%------------------------------Selección----------------------------------
Por ultimo, la siguiente tabla hace una recopilación de la alternativa de solución elegida en conjunto con las alternativas independientes que fueron elegidas. 
%------------------------------Elegido----------------------------------
\begin{center}
\footnotesize
\centering
    \begin{longtable}[!htbp]{|>{\centering\arraybackslash}m{20em} |>{\centering\arraybackslash}m{20em} |}
    \hline
    \textbf{Características} & \textbf{Concepto solución}\\
    \hline\hline
    Forma de movimiento de los paneles & Rotacional en base prismática \\
    \hline
    Tipo de actuación & Con motor y acopladores magnéticos \\
    \hline
    Método para la reflexión & Paneles reflectivos  \\
    \hline
    Sensado de las posiciones de los paneles & Encoders, sensores de contacto o visión artificial  \\
    \hline
    Posicionamiento de paneles & Anclados a la base prismática \\
    \hline
    Sistema de alimentación de paneles & No \\
    \hline
    Grabación en varias posiciones & No \\
    \hline
    Generación de ondas de sonido & Bocina externa \\
    \hline
    Dispositivo de grabación & Micrófono externo \\
    \hline
    Plataforma para el cálculo de las posiciones deseadas & Microcontrolador embebido \\
    \hline
    Plataforma de procesado del control &  Microcontrolador embebido\\
    \hline
    Plataforma de procesado de la acústica & Microcontrolador embebido \\
    \hline
    Sensado de las posiciones de los paneles & Encoders \\
    \hline
    Plataforma de visualización de información &  Equipo de cómputo\\
    \hline
    Control de interfaz & Ratón y teclado \\
    \hline
    Encapsulado de sistemas embebidos &  Si\\
    \hline
    Inclusión del sistema energético en sistema embebido & Si \\
    \hline
    Anclaje al suelo & Si \\
    \hline
    Método de anclaje &  Tornillos \\
    \hline
    Método de comunicación & Wi-fi \\
    \hline    
    \caption{Características del concepto solución elegido}
    \label{tab:ConceptoElegido}
    \end{longtable}
\end{center}
\FloatBarrier

Se desarrollo el diseño del concepto solución elegido en un software CAD, es a partir de esta propuesta inicial que se comenzaran a desarrollar los cálculos y validaciones para el diseño final.
\begin{figure}[!htb]
    \centering
    \includegraphics[width=1\textwidth]{imagenes/ConceptoSolucion.jpg}
    \caption{\footnotesize Concepto solución elegido}
    \label{fig:ConceptoSolucionElegido}
\end{figure}
\FloatBarrier
%------------------------------Elegido----------------------------------


%Imágenes del prototipo

%\definecolor{gr_l}{gray}{.8}
%-----------------------------Matriz Morfológica-----------------------------
\begin{center}
\scriptsize
\centering
    \begin{longtable}[!htb]{|>{\centering\arraybackslash}m{3em} ||>{\centering\arraybackslash}m{8em} | >{\centering\arraybackslash}m{8em}| >{\centering\arraybackslash}m{8em}| >{\centering\arraybackslash}m{8em}|>{\centering\arraybackslash}m{8em}|}
    \hline
    & \textbf{Características} & \textbf{Alternativa 1} & \textbf{Alternativa 2}& \textbf{Alternativa 3}& \textbf{Alternativa 4}\\
    \hline\hline
    \textbf{$C_1$} & Forma de movimiento de los paneles & Rotacional en base prismática & Tipo pick and place & Construcción y movimiento de líneas & Paneles en configuración plegable\\
    \hline
    \textbf{$C_2$} & Método para la reflexión & Paneles de reflexión & Pared de fondo & - & -\\
    \hline
    \textbf{$C_3$} & Grabación en varias posiciones & No & Manual & Automático & Con múltiples micrófonos\\
    \hline
    \textbf{$C_4$} & Generación de ondas de sonido & Bocina de equipo de cómputo & Bocina externa & Bocina para circuitos & - \\
    \hline
    \textbf{$C_5$} & Dispositivo de grabación & Micrófono de equipo de cómputo & Micrófono externo & Micrófono para circuitos & - \\
    \hline
    \textbf{$C_6$} & Plataforma de cálculo para las posiciones deseadas & Equipo de cómputo & Microcontrolador embebido & FPGA & - \\
    \hline
    \textbf{$C_7$} & Plataforma de procesado del control & Equipo de cómputo & Microcontrolador embebido & FPGA & - \\
    \hline
    \textbf{$C_8$} & Plataforma para procesado de la acústica & Equipo de cómputo & Microcontrolador embebido & FPGA & - \\
    \hline
    \textbf{$C_9$} & Sensado de las posiciones de los paneles & Encoders & Sensores de contacto & Visión artificial & Sensor de flexión \\
    \hline
    \textbf{$C_{10}$} & Plataforma de visualización de información & Equipo de cómputo & Pantalla OLED & Pantalla LCD & - \\
    \hline
    \textbf{$C_{11}$} & Control de interfaz & Pantalla táctil & Botonera & Ratón y teclado & - \\
    \hline
    \textbf{$C_{12}$} & Encapsulado de sistemas embebidos & Si & No & - & - \\
    \hline
    \textbf{$C_{13}$} & Inclusión del sistema energético en el sistema embebido & Si & No & - & - \\
    \hline
    \textbf{$C_{14}$} & Anclaje al suelo & Si & No & - & - \\
    \hline
    \textbf{$C_{15}$} & Método de anclaje & Tornillos & Colgados & Adhesivos & Magnéticos \\
    \hline
    \textbf{$C_{16}$} & Método de comunicación & Serial & Wi-fi & Bluetooth & - \\
    \hline

    \caption{Matriz morfológica}
    \label{tab:Matriz morfológica}
    \end{longtable}
\end{center}
%-----------------------------Matriz Morfológica-----------------------------

En base a las diferentes alternativas, se crearon 4 conceptos de solución, cuya mayor diferencia es el método de movimiento para los paneles. Debido a que hay alternativas que no son dependientes del método de movimiento, la selección de estas se hará independientemente. 
%-----------------------------Conceptos solución-----------------------------
\begin{center}
\scriptsize
\centering
    \begin{longtable}[!htb]{|>{\centering\arraybackslash}m{3em} ||>{\centering\arraybackslash}m{8em} | >{\centering\arraybackslash}m{8em}| >{\centering\arraybackslash}m{8em}| >{\centering\arraybackslash}m{8em}|>{\centering\arraybackslash}m{8em}|}
    \hline
    & \textbf{Características} & \textbf{$CS_1$} & \textbf{$CS_2$}& \textbf{$CS_3$}& \textbf{$CS_4$}\\
    \hline\hline
    \textbf{$C_1$} & Forma de movimiento de los paneles & Rotacional en base prismática & Tipo pick and place & Construcción y movimiento de líneas & Paneles en configuración plegable\\
    \hline
    \textbf{$C_2$} & Tipo de actuación & Con motor y acopladores magnéticos & Con motor y actuador magnético & Con motores & Motores lineales\\
    \hline
    \textbf{$C_3$} & Sensado de las posiciones de los paneles & Encoders, sensores de contacto o visión artificial & Encoders o visión artificial & Encoders & Sensores de flexión\\
    \hline
    \textbf{$C_4$} & Posicionamiento de paneles & Anclados a la base prismática & Acoplamiento en pared & Acoplado a un cable & Acoplados entre si\\
    \hline
    \textbf{$C_5$} & Sistema de alimentación de paneles & No & Si & Si & No \\
    \hline

    \caption{Conceptos solución}
    \label{tab:ConceptosSolucion}
    \end{longtable}
\end{center}
%-----------------------------Conceptos solución---------------------------------
Primero se hará la comparación de los diferentes conceptos solución con base en cada uno de los criterios. Es importante notar que no todos los criterios se pueden aplicar para discernir entre los diferentes conceptos solución, por tanto, esos criterios se aplicaran a las alternativas que se analizaran individualmente.
%-------------------------CR1 Velocidad de actuación-----------------------------
\begin{table}[!htbp]
    \begin{minipage}[b]{0.5\linewidth}
        \scriptsize
        \centering
            \begin{tabular}{|>{\centering\arraybackslash}m{2em} ||>{\centering\arraybackslash}m{2em} | >{\centering\arraybackslash}m{2em}| >{\centering\arraybackslash}m{2em}| >{\centering\arraybackslash}m{2em}|}
            \hline
            & \textbf{$CS_1$} & \textbf{$CS_2$}& \textbf{$CS_3$}& \textbf{$CS_4$}\\
            \hline\hline
            \textbf{$CS_1$} & \cellcolor{gr_l}{1}  &  10  &    5   &   2    \\
            \textbf{$CS_2$} & 0.10 &  \cellcolor{gr_l}{1} &  0.50  &  0.20  \\
            \textbf{$CS_3$} & 0.20 &  2   &  \cellcolor{gr_l}{1}   &  0.40  \\
            \textbf{$CS_4$} & 0.50 &  5   &  2.50  &   \cellcolor{gr_l}{1}  \\ 
            \hline
        \end{tabular}
        \caption{Matriz de comparación de $Cr_1$}
        \label{tab:MComCr1}
    \end{minipage}
    \begin{minipage}[b]{0.5\linewidth}
        \scriptsize
        \centering
            \begin{tabular}{|>{\centering\arraybackslash}m{2em} ||>{\centering\arraybackslash}m{2em} | >{\centering\arraybackslash}m{2em}| >{\centering\arraybackslash}m{2em}| >{\centering\arraybackslash}m{2em}|>{\centering\arraybackslash}m{2em}|}
            \hline
            & \textbf{$CS_1$} & \textbf{$CS_2$}& \textbf{$CS_3$}& \textbf{$CS_4$}& \textbf{$V_{Cr_1}$}\\
            \hline\hline
            \textbf{$CS_1$} & 0.56 &  0.56  &   0.56   &  0.56  &  0.56   \\
            \textbf{$CS_2$} & 0.06 &  0.06  &   0.06   &  0.06  &  0.06   \\
            \textbf{$CS_3$} & 0.11 &  0.11  &   0.11   &  0.11  &  0.11   \\
            \textbf{$CS_4$} & 0.28 &  0.28  &   0.28   &  0.28  &  0.28   \\ 
            \hline
        \end{tabular}
        \caption{Matriz normalizada de $Cr_1$ y $V_{Cr_1}$}
        \label{tab:MNorm_Cr1}
    \end{minipage}
\end{table}
%-------------------------CR1 Velocidad de actuación-----------------------------

%-------------------------CR2 Peso-----------------------------
\begin{table}[!htbp]
    \begin{minipage}[b]{0.5\linewidth}
        \scriptsize
        \centering
            \begin{tabular}{|>{\centering\arraybackslash}m{2em} ||>{\centering\arraybackslash}m{2em} | >{\centering\arraybackslash}m{2em}| >{\centering\arraybackslash}m{2em}| >{\centering\arraybackslash}m{2em}|}
            \hline
            & \textbf{$CS_1$} & \textbf{$CS_2$}& \textbf{$CS_3$}& \textbf{$CS_4$}\\
            \hline\hline
            \textbf{$CS_1$} & \cellcolor{gr_l}{1} &  0.38  &  0.38   &   1   \\
            \textbf{$CS_2$} & 2.67 &  \cellcolor{gr_l}{1}  &   1     &  2.67  \\
            \textbf{$CS_3$} & 2.67 &  1      &  \cellcolor{gr_l}{1}   &  2.67  \\
            \textbf{$CS_4$} &   1  &  0.38   &  0.38  &   \cellcolor{gr_l}{1}  \\ 
            \hline
        \end{tabular}
        \caption{Matriz de comparación de $Cr_2$}
        \label{tab:MComCr2}
    \end{minipage}
    \begin{minipage}[b]{0.5\linewidth}
        \scriptsize
        \centering
            \begin{tabular}{|>{\centering\arraybackslash}m{2em} ||>{\centering\arraybackslash}m{2em} | >{\centering\arraybackslash}m{2em}| >{\centering\arraybackslash}m{2em}| >{\centering\arraybackslash}m{2em}|>{\centering\arraybackslash}m{2em}|}
            \hline
            & \textbf{$CS_1$} & \textbf{$CS_2$}& \textbf{$CS_3$}& \textbf{$CS_4$}& \textbf{$V_{Cr_2}$}\\
            \hline\hline
            \textbf{$CS_1$} & 0.14 &  0.14  &   0.14   &  0.14  &  0.14   \\
            \textbf{$CS_2$} & 0.36 &  0.36  &   0.36   &  0.36  &  0.36   \\
            \textbf{$CS_3$} & 0.36 &  0.36  &   0.36   &  0.36  &  0.36   \\
            \textbf{$CS_4$} & 0.14 &  0.14  &   0.14   &  0.14  &  0.14   \\ 
            \hline
        \end{tabular}
        \caption{Matriz normalizada de $Cr_2$ y $V_{Cr_2}$}
        \label{tab:MNorm_Cr1}
    \end{minipage}
\end{table}
%-------------------------CR2 Peso-----------------------------

%-------------------------CR3 Complejidad de mecanismo-----------------------------
\begin{table}[!htbp]
    \begin{minipage}[b]{0.5\linewidth}
        \scriptsize
        \centering
            \begin{tabular}{|>{\centering\arraybackslash}m{2em} ||>{\centering\arraybackslash}m{2em} | >{\centering\arraybackslash}m{2em}| >{\centering\arraybackslash}m{2em}| >{\centering\arraybackslash}m{2em}|}
            \hline
            & \textbf{$CS_1$} & \textbf{$CS_2$}& \textbf{$CS_3$}& \textbf{$CS_4$}\\
            \hline\hline
            \textbf{$CS_1$} & \cellcolor{gr_l}{1}  &  2.67  &    4   &   4   \\
            \textbf{$CS_2$} & 0.38 &  \cellcolor{gr_l}{1} &  1.50   &  1.50  \\
            \textbf{$CS_3$} & 0.25 &  0.67   &  \cellcolor{gr_l}{1}   &  1  \\
            \textbf{$CS_4$} & 0.25 &  0.67   &  1  &   \cellcolor{gr_l}{1}  \\ 
            \hline
        \end{tabular}
        \caption{Matriz de comparación de $Cr_3$}
        \label{tab:MComCr3}
    \end{minipage}
    \begin{minipage}[b]{0.5\linewidth}
        \scriptsize
        \centering
            \begin{tabular}{|>{\centering\arraybackslash}m{2em} ||>{\centering\arraybackslash}m{2em} | >{\centering\arraybackslash}m{2em}| >{\centering\arraybackslash}m{2em}| >{\centering\arraybackslash}m{2em}|>{\centering\arraybackslash}m{2em}|}
            \hline
            & \textbf{$CS_1$} & \textbf{$CS_2$}& \textbf{$CS_3$}& \textbf{$CS_4$}& \textbf{$V_{Cr_3}$}\\
            \hline\hline
            \textbf{$CS_1$} & 0.53 &  0.53  &   0.53   &  0.53  &  0.53   \\
            \textbf{$CS_2$} & 0.20 &  0.20  &   0.20   &  0.20  &  0.20   \\
            \textbf{$CS_3$} & 0.13 &  0.13  &   0.13   &  0.13  &  0.13   \\
            \textbf{$CS_4$} & 0.13 &  0.13  &   0.13   &  0.13  &  0.13   \\ 
            \hline
        \end{tabular}
        \caption{Matriz normalizada de $Cr_3$ y $V_{Cr_3}$}
        \label{tab:MNorm_Cr3}
    \end{minipage}
\end{table}
%-------------------------CR3 Complejidad de mecanismo-----------------------------

%-------------------------CR4 Cantidad de actuadores-----------------------------
\begin{table}[!htbp]
    \begin{minipage}[b]{0.5\linewidth}
        \scriptsize
        \centering
            \begin{tabular}{|>{\centering\arraybackslash}m{2em} ||>{\centering\arraybackslash}m{2em} | >{\centering\arraybackslash}m{2em}| >{\centering\arraybackslash}m{2em}| >{\centering\arraybackslash}m{2em}|}
            \hline
            & \textbf{$CS_1$} & \textbf{$CS_2$}& \textbf{$CS_3$}& \textbf{$CS_4$}\\
            \hline\hline
            \textbf{$CS_1$} & \cellcolor{gr_l}{1}  &  1.17  &    1.40   &   3.50   \\
            \textbf{$CS_2$} & 0.86 &  \cellcolor{gr_l}{1} &  1.20   &  3.00  \\
            \textbf{$CS_3$} & 0.71 &  0.83   &  \cellcolor{gr_l}{1}   &  2.50  \\
            \textbf{$CS_4$} & 0.29 &  0.33   &  0.40  &   \cellcolor{gr_l}{1}  \\ 
            \hline
        \end{tabular}
        \caption{Matriz de comparación de $Cr_4$}
        \label{tab:MComCr4}
    \end{minipage}
    \begin{minipage}[b]{0.5\linewidth}
        \scriptsize
        \centering
            \begin{tabular}{|>{\centering\arraybackslash}m{2em} ||>{\centering\arraybackslash}m{2em} | >{\centering\arraybackslash}m{2em}| >{\centering\arraybackslash}m{2em}| >{\centering\arraybackslash}m{2em}|>{\centering\arraybackslash}m{2em}|}
            \hline
            & \textbf{$CS_1$} & \textbf{$CS_2$}& \textbf{$CS_3$}& \textbf{$CS_4$}& \textbf{$V_{Cr_4}$}\\
            \hline\hline
            \textbf{$CS_1$} & 0.35 &  0.35  &   0.35   &  0.35  &  0.35   \\
            \textbf{$CS_2$} & 0.30 &  0.30  &   0.30   &  0.30  &  0.30   \\
            \textbf{$CS_3$} & 0.25 &  0.25  &   0.25   &  0.25  &  0.25   \\
            \textbf{$CS_4$} & 0.10 &  0.10  &   0.10   &  0.10  &  0.10   \\ 
            \hline
        \end{tabular}
        \caption{Matriz normalizada de $Cr_4$ y $V_{Cr_4}$}
        \label{tab:MNorm_Cr4}
    \end{minipage}
\end{table}
%-------------------------CR4 Cantidad de actuadores-----------------------------

%-------------------------CR5 Costo de la actuación-----------------------------
\begin{table}[!htbp]
    \begin{minipage}[b]{0.5\linewidth}
        \scriptsize
        \centering
            \begin{tabular}{|>{\centering\arraybackslash}m{2em} ||>{\centering\arraybackslash}m{2em} | >{\centering\arraybackslash}m{2em}| >{\centering\arraybackslash}m{2em}| >{\centering\arraybackslash}m{2em}|}
            \hline
            & \textbf{$CS_1$} & \textbf{$CS_2$}& \textbf{$CS_3$}& \textbf{$CS_4$}\\
            \hline\hline
            \textbf{$CS_1$} & \cellcolor{gr_l}{1}  &  0.89  &    1.14   &   4   \\
            \textbf{$CS_2$} & 1.13 &  \cellcolor{gr_l}{1} &  1.29   &  4.50  \\
            \textbf{$CS_3$} & 0.88 &  0.78   &  \cellcolor{gr_l}{1}   &  3.50  \\
            \textbf{$CS_4$} & 0.25 &  0.22   &  0.29  &   \cellcolor{gr_l}{1}  \\ 
            \hline
        \end{tabular}
        \caption{Matriz de comparación de $Cr_5$}
        \label{tab:MComCr5}
    \end{minipage}
    \begin{minipage}[b]{0.5\linewidth}
        \scriptsize
        \centering
            \begin{tabular}{|>{\centering\arraybackslash}m{2em} ||>{\centering\arraybackslash}m{2em} | >{\centering\arraybackslash}m{2em}| >{\centering\arraybackslash}m{2em}| >{\centering\arraybackslash}m{2em}|>{\centering\arraybackslash}m{2em}|}
            \hline
            & \textbf{$CS_1$} & \textbf{$CS_2$}& \textbf{$CS_3$}& \textbf{$CS_4$}& \textbf{$V_{Cr_5}$}\\
            \hline\hline
            \textbf{$CS_1$} & 0.31 &  0.31  &   0.31   &  0.31  &  0.31   \\
            \textbf{$CS_2$} & 0.35 &  0.35  &   0.35   &  0.35  &  0.35   \\
            \textbf{$CS_3$} & 0.27 &  0.27  &   0.27   &  0.27  &  0.27   \\
            \textbf{$CS_4$} & 0.08 &  0.08  &   0.08   &  0.08  &  0.08   \\ 
            \hline
        \end{tabular}
        \caption{Matriz normalizada de $Cr_5$ y $V_{Cr_5}$}
        \label{tab:MNorm_Cr5}
    \end{minipage}
\end{table}
%-------------------------CR5 Costo de la actuación-----------------------------

%-------------------------CR6 Resolución acústica-----------------------------
\begin{table}[!htbp]
    \begin{minipage}[b]{0.5\linewidth}
        \scriptsize
        \centering
            \begin{tabular}{|>{\centering\arraybackslash}m{2em} ||>{\centering\arraybackslash}m{2em} | >{\centering\arraybackslash}m{2em}| >{\centering\arraybackslash}m{2em}| >{\centering\arraybackslash}m{2em}|}
            \hline
            & \textbf{$CS_1$} & \textbf{$CS_2$}& \textbf{$CS_3$}& \textbf{$CS_4$}\\
            \hline\hline
            \textbf{$CS_1$} & \cellcolor{gr_l}{1}  &  0.90  &    0.90   &   2.25   \\
            \textbf{$CS_2$} & 1.11 &  \cellcolor{gr_l}{1} &  1   &  2.50  \\
            \textbf{$CS_3$} & 1.11 &  1   &  \cellcolor{gr_l}{1}   &  2.50  \\
            \textbf{$CS_4$} & 0.44 &  0.40   &  0.40  &   \cellcolor{gr_l}{1}  \\ 
            \hline
        \end{tabular}
        \caption{Matriz de comparación de $Cr_6$}
        \label{tab:MComCr6}
    \end{minipage}
    \begin{minipage}[b]{0.5\linewidth}
        \scriptsize
        \centering
            \begin{tabular}{|>{\centering\arraybackslash}m{2em} ||>{\centering\arraybackslash}m{2em} | >{\centering\arraybackslash}m{2em}| >{\centering\arraybackslash}m{2em}| >{\centering\arraybackslash}m{2em}|>{\centering\arraybackslash}m{2em}|}
            \hline
            & \textbf{$CS_1$} & \textbf{$CS_2$}& \textbf{$CS_3$}& \textbf{$CS_4$}& \textbf{$V_{Cr_6}$}\\
            \hline\hline
            \textbf{$CS_1$} & 0.27 &  0.27  &   0.27   &  0.27  &  0.27   \\
            \textbf{$CS_2$} & 0.30 &  0.30  &   0.30   &  0.30  &  0.30   \\
            \textbf{$CS_3$} & 0.30 &  0.30  &   0.30   &  0.30  &  0.30    \\
            \textbf{$CS_4$} & 0.12 &  0.12  &   0.12   &  0.12  &  0.12   \\ 
            \hline
        \end{tabular}
        \caption{Matriz normalizada de $Cr_6$ y $V_{Cr_6}$}
        \label{tab:MNorm_Cr6}
    \end{minipage}
\end{table}
%-------------------------CR6 Resolución acústica-----------------------------

%-------------------------CR7 Repetibilidad-----------------------------
\begin{table}[!htbp]
    \begin{minipage}[b]{0.5\linewidth}
        \scriptsize
        \centering
            \begin{tabular}{|>{\centering\arraybackslash}m{2em} ||>{\centering\arraybackslash}m{2em} | >{\centering\arraybackslash}m{2em}| >{\centering\arraybackslash}m{2em}| >{\centering\arraybackslash}m{2em}|}
            \hline
            & \textbf{$CS_1$} & \textbf{$CS_2$}& \textbf{$CS_3$}& \textbf{$CS_4$}\\
            \hline\hline
            \textbf{$CS_1$} & \cellcolor{gr_l}{1}  &  5  &    3.33   &   1.43   \\
            \textbf{$CS_2$} & 0.20 &  \cellcolor{gr_l}{1} &  0.67   &  0.29  \\
            \textbf{$CS_3$} & 0.30 &  1.50   &  \cellcolor{gr_l}{1}   &  0.43  \\
            \textbf{$CS_4$} & 0.70 &  3.50   &  2.33  &   \cellcolor{gr_l}{1}  \\ 
            \hline
        \end{tabular}
        \caption{Matriz de comparación de $Cr_7$}
        \label{tab:MComCr7}
    \end{minipage}
    \begin{minipage}[b]{0.5\linewidth}
        \scriptsize
        \centering
            \begin{tabular}{|>{\centering\arraybackslash}m{2em} ||>{\centering\arraybackslash}m{2em} | >{\centering\arraybackslash}m{2em}| >{\centering\arraybackslash}m{2em}| >{\centering\arraybackslash}m{2em}|>{\centering\arraybackslash}m{2em}|}
            \hline
            & \textbf{$CS_1$} & \textbf{$CS_2$}& \textbf{$CS_3$}& \textbf{$CS_4$}& \textbf{$V_{Cr_7}$}\\
            \hline\hline
            \textbf{$CS_1$} & 0.45 &  0.45  &   0.45   &  0.45  &  0.45   \\
            \textbf{$CS_2$} & 0.09 &  0.09  &   0.09   &  0.09  &  0.09   \\
            \textbf{$CS_3$} & 0.14 &  0.14  &   0.14   &  0.14  &  0.14    \\
            \textbf{$CS_4$} & 0.32 &  0.32  &   0.32   &  0.32  &  0.32   \\ 
            \hline
        \end{tabular}
        \caption{Matriz normalizada de $Cr_7$ y $V_{Cr_7}$}
        \label{tab:MNorm_Cr7}
    \end{minipage}
\end{table}
%-------------------------CR7 Repetibilidad-----------------------------

%-------------------------CR8 Consumo energético-----------------------------
\begin{table}[!htbp]
    \begin{minipage}[b]{0.5\linewidth}
        \scriptsize
        \centering
            \begin{tabular}{|>{\centering\arraybackslash}m{2em} ||>{\centering\arraybackslash}m{2em} | >{\centering\arraybackslash}m{2em}| >{\centering\arraybackslash}m{2em}| >{\centering\arraybackslash}m{2em}|}
            \hline
            & \textbf{$CS_1$} & \textbf{$CS_2$}& \textbf{$CS_3$}& \textbf{$CS_4$}\\
            \hline\hline
            \textbf{$CS_1$} & \cellcolor{gr_l}{1}  &  3  &    2.25   &   4.50   \\
            \textbf{$CS_2$} & 0.33 &  \cellcolor{gr_l}{1} &   0.75   &  1.50  \\
            \textbf{$CS_3$} & 0.44 &  1.33   &  \cellcolor{gr_l}{1}   &  2  \\
            \textbf{$CS_4$} & 0.22 &  0.67   &  0.50  &   \cellcolor{gr_l}{1}  \\ 
            \hline
        \end{tabular}
        \caption{Matriz de comparación de $Cr_8$}
        \label{tab:MComCr8}
    \end{minipage}
    \begin{minipage}[b]{0.5\linewidth}
        \scriptsize
        \centering
            \begin{tabular}{|>{\centering\arraybackslash}m{2em} ||>{\centering\arraybackslash}m{2em} | >{\centering\arraybackslash}m{2em}| >{\centering\arraybackslash}m{2em}| >{\centering\arraybackslash}m{2em}|>{\centering\arraybackslash}m{2em}|}
            \hline
            & \textbf{$CS_1$} & \textbf{$CS_2$}& \textbf{$CS_3$}& \textbf{$CS_4$}& \textbf{$V_{Cr_8}$}\\
            \hline\hline
            \textbf{$CS_1$} & 0.50 &  0.50  &   0.50   &  0.50  &  0.50   \\
            \textbf{$CS_2$} & 0.17 &  0.17  &   0.17   &  0.17  &  0.17   \\
            \textbf{$CS_3$} & 0.22 &  0.22  &   0.22   &  0.22  &  0.22    \\
            \textbf{$CS_4$} & 0.11 &  0.11  &   0.11   &  0.11  &  0.11   \\ 
            \hline
        \end{tabular}
        \caption{Matriz normalizada de $Cr_8$ y $V_{Cr_8}$}
        \label{tab:MNorm_Cr8}
    \end{minipage}
\end{table}
%-------------------------CR8 Consumo energético-----------------------------

%-------------------------CR9 Variedad de potencias consumidas-----------------------------
\begin{table}[!htbp]
    \begin{minipage}[b]{0.5\linewidth}
        \scriptsize
        \centering
            \begin{tabular}{|>{\centering\arraybackslash}m{2em} ||>{\centering\arraybackslash}m{2em} | >{\centering\arraybackslash}m{2em}| >{\centering\arraybackslash}m{2em}| >{\centering\arraybackslash}m{2em}|}
            \hline
            & \textbf{$CS_1$} & \textbf{$CS_2$}& \textbf{$CS_3$}& \textbf{$CS_4$}\\
            \hline\hline
            \textbf{$CS_1$} & \cellcolor{gr_l}{1}  &  1  &    0.88   &   0.70   \\
            \textbf{$CS_2$} & 1 &  \cellcolor{gr_l}{1} &      0.88   &      0.70  \\
            \textbf{$CS_3$} & 1.14 &  1.14   &  \cellcolor{gr_l}{1}   &  0.80  \\
            \textbf{$CS_4$} & 1.43 &  1.43   &  1.25  &   \cellcolor{gr_l}{1}  \\ 
            \hline
        \end{tabular}
        \caption{Matriz de comparación de $Cr_9$}
        \label{tab:MComCr9}
    \end{minipage}
    \begin{minipage}[b]{0.5\linewidth}
        \scriptsize
        \centering
            \begin{tabular}{|>{\centering\arraybackslash}m{2em} ||>{\centering\arraybackslash}m{2em} | >{\centering\arraybackslash}m{2em}| >{\centering\arraybackslash}m{2em}| >{\centering\arraybackslash}m{2em}|>{\centering\arraybackslash}m{2em}|}
            \hline
            & \textbf{$CS_1$} & \textbf{$CS_2$}& \textbf{$CS_3$}& \textbf{$CS_4$}& \textbf{$V_{Cr_9}$}\\
            \hline\hline
            \textbf{$CS_1$} & 0.22 &  0.22  &   0.22   &  0.22  &  0.22   \\
            \textbf{$CS_2$} & 0.22 &  0.22  &   0.22   &  0.22  &  0.22  \\
            \textbf{$CS_3$} & 0.25 &  0.25  &   0.25   &  0.25  &  0.25    \\
            \textbf{$CS_4$} & 0.31 &  0.31  &   0.31   &  0.31  &  0.31   \\ 
            \hline
        \end{tabular}
        \caption{Matriz normalizada de $Cr_9$ y $V_{Cr_9}$}
        \label{tab:MNorm_Cr9}
    \end{minipage}
\end{table}
%-------------------------CR9 Variedad de potencias consumidas-----------------------------

%----------------------CR10 Facilidad en el control de la actuación--------------------------
\begin{table}[!htbp]
    \begin{minipage}[b]{0.5\linewidth}
        \scriptsize
        \centering
            \begin{tabular}{|>{\centering\arraybackslash}m{2em} ||>{\centering\arraybackslash}m{2em} | >{\centering\arraybackslash}m{2em}| >{\centering\arraybackslash}m{2em}| >{\centering\arraybackslash}m{2em}|}
            \hline
            & \textbf{$CS_1$} & \textbf{$CS_2$}& \textbf{$CS_3$}& \textbf{$CS_4$}\\
            \hline\hline
            \textbf{$CS_1$} & \cellcolor{gr_l}{1}  &  9  &    4.50   &   1.29   \\
            \textbf{$CS_2$} & 0.11 &  \cellcolor{gr_l}{1} &   0.50   &   0.14  \\
            \textbf{$CS_3$} & 0.22 &  2   &  \cellcolor{gr_l}{1}   &  0.29  \\
            \textbf{$CS_4$} & 0.78 &  7   &  3.50  &   \cellcolor{gr_l}{1}  \\ 
            \hline
        \end{tabular}
        \caption{Matriz de comparación de $Cr_{10}$}
        \label{tab:MComCr10}
    \end{minipage}
    \begin{minipage}[b]{0.5\linewidth}
        \scriptsize
        \centering
            \begin{tabular}{|>{\centering\arraybackslash}m{2em} ||>{\centering\arraybackslash}m{2em} | >{\centering\arraybackslash}m{2em}| >{\centering\arraybackslash}m{2em}| >{\centering\arraybackslash}m{2em}|>{\centering\arraybackslash}m{2em}|}
            \hline
            & \textbf{$CS_1$} & \textbf{$CS_2$}& \textbf{$CS_3$}& \textbf{$CS_4$}& \textbf{$V_{Cr_{10}}$}\\
            \hline\hline
            \textbf{$CS_1$} & 0.47 &  0.47  &   0.47   &  0.47  &  0.47   \\
            \textbf{$CS_2$} & 0.05 &  0.05  &   0.05   &  0.05  &  0.05  \\
            \textbf{$CS_3$} & 0.11 &  0.11  &   0.11   &  0.11  &  0.11    \\
            \textbf{$CS_4$} & 0.37 &  0.37  &   0.37   &  0.37  &  0.37   \\ 
            \hline
        \end{tabular}
        \caption{Matriz normalizada de $Cr_{10}$ y $V_{Cr_{10}}$}
        \label{tab:MNorm_Cr10}
    \end{minipage}
\end{table}
%----------------------CR10 Facilidad en el control de la actuación--------------------------

%----------------------CR11 Tamaño--------------------------
\begin{table}[!htbp]
    \begin{minipage}[b]{0.5\linewidth}
        \scriptsize
        \centering
            \begin{tabular}{|>{\centering\arraybackslash}m{2em} ||>{\centering\arraybackslash}m{2em} | >{\centering\arraybackslash}m{2em}| >{\centering\arraybackslash}m{2em}| >{\centering\arraybackslash}m{2em}|}
            \hline
            & \textbf{$CS_1$} & \textbf{$CS_2$}& \textbf{$CS_3$}& \textbf{$CS_4$}\\
            \hline\hline
            \textbf{$CS_1$} & \cellcolor{gr_l}{1}  &  1.60  &    2   &   0.89   \\
            \textbf{$CS_2$} & 0.63 &  \cellcolor{gr_l}{1} &   1.25   &   0.56  \\
            \textbf{$CS_3$} & 0.50 &  0.80   &  \cellcolor{gr_l}{1}   &  0.44  \\
            \textbf{$CS_4$} & 1.13 &  1.80   &  2.25  &   \cellcolor{gr_l}{1}  \\ 
            \hline
        \end{tabular}
        \caption{Matriz de comparación de $Cr_{11}$}
        \label{tab:MComCr11}
    \end{minipage}
    \begin{minipage}[b]{0.5\linewidth}
        \scriptsize
        \centering
            \begin{tabular}{|>{\centering\arraybackslash}m{2em} ||>{\centering\arraybackslash}m{2em} | >{\centering\arraybackslash}m{2em}| >{\centering\arraybackslash}m{2em}| >{\centering\arraybackslash}m{2em}|>{\centering\arraybackslash}m{2em}|}
            \hline
            & \textbf{$CS_1$} & \textbf{$CS_2$}& \textbf{$CS_3$}& \textbf{$CS_4$}& \textbf{$V_{Cr_{11}}$}\\
            \hline\hline
            \textbf{$CS_1$} & 0.31 &  0.31  &   0.31   &  0.31  &  0.31   \\
            \textbf{$CS_2$} & 0.19 &  0.19  &   0.19   &  0.19  &  0.19  \\
            \textbf{$CS_3$} & 0.15 &  0.15  &   0.15   &  0.15  &  0.15    \\
            \textbf{$CS_4$} & 0.35 &  0.35  &   0.35   &  0.35  &  0.35   \\ 
            \hline
        \end{tabular}
        \caption{Matriz normalizada de $Cr_{11}$ y $V_{Cr_{11}}$}
        \label{tab:MNorm_Cr11}
    \end{minipage}
\end{table}
%----------------------CR11 Tamaño--------------------------

%----------------------CR12 Confiabilidad del sistema de actuación--------------------------
\begin{table}[!htbp]
    \begin{minipage}[b]{0.5\linewidth}
        \scriptsize
        \centering
            \begin{tabular}{|>{\centering\arraybackslash}m{2em} ||>{\centering\arraybackslash}m{2em} | >{\centering\arraybackslash}m{2em}| >{\centering\arraybackslash}m{2em}| >{\centering\arraybackslash}m{2em}|}
            \hline
            & \textbf{$CS_1$} & \textbf{$CS_2$}& \textbf{$CS_3$}& \textbf{$CS_4$}\\
            \hline\hline
            \textbf{$CS_1$} & \cellcolor{gr_l}{1}  &  5  &    5   &   2.50   \\
            \textbf{$CS_2$} & 0.20 &  \cellcolor{gr_l}{1} &   1   &   0.50  \\
            \textbf{$CS_3$} & 0.20 &  1   &  \cellcolor{gr_l}{1}   &  0.50  \\
            \textbf{$CS_4$} & 0.40 &  2   &  2  &   \cellcolor{gr_l}{1}  \\ 
            \hline
        \end{tabular}
        \caption{Matriz de comparación de $Cr_{12}$}
        \label{tab:MComCr12}
    \end{minipage}
    \begin{minipage}[b]{0.5\linewidth}
        \scriptsize
        \centering
            \begin{tabular}{|>{\centering\arraybackslash}m{2em} ||>{\centering\arraybackslash}m{2em} | >{\centering\arraybackslash}m{2em}| >{\centering\arraybackslash}m{2em}| >{\centering\arraybackslash}m{2em}|>{\centering\arraybackslash}m{2em}|}
            \hline
            & \textbf{$CS_1$} & \textbf{$CS_2$}& \textbf{$CS_3$}& \textbf{$CS_4$}& \textbf{$V_{Cr_{12}}$}\\
            \hline\hline
            \textbf{$CS_1$} & 0.56 &  0.56  &   0.56   &  0.56  &  0.56   \\
            \textbf{$CS_2$} & 0.11 &  0.11  &   0.11   &  0.11  &  0.11  \\
            \textbf{$CS_3$} & 0.11 &  0.11  &   0.11   &  0.11  &  0.11    \\
            \textbf{$CS_4$} & 0.22 &  0.22  &   0.22   &  0.22  &  0.22   \\ 
            \hline
        \end{tabular}
        \caption{Matriz normalizada de $Cr_{12}$ y $V_{Cr_{12}}$}
        \label{tab:MNorm_Cr12}
    \end{minipage}
\end{table}
%----------------------CR12 Confiabilidad del sistema de actuación--------------------------

%---------------CR13 Complejidad de manufactura del mecanismo de actuación------------------
\begin{table}[!htbp]
    \begin{minipage}[b]{0.5\linewidth}
        \scriptsize
        \centering
            \begin{tabular}{|>{\centering\arraybackslash}m{2em} ||>{\centering\arraybackslash}m{2em} | >{\centering\arraybackslash}m{2em}| >{\centering\arraybackslash}m{2em}| >{\centering\arraybackslash}m{2em}|}
            \hline
            & \textbf{$CS_1$} & \textbf{$CS_2$}& \textbf{$CS_3$}& \textbf{$CS_4$}\\
            \hline\hline
            \textbf{$CS_1$} & \cellcolor{gr_l}{1}  &  2  &    3   &   2   \\
            \textbf{$CS_2$} & 0.50 &  \cellcolor{gr_l}{1} &   1.50   &   1  \\
            \textbf{$CS_3$} & 0.33 &  0.67   &  \cellcolor{gr_l}{1}   &  0.67  \\
            \textbf{$CS_4$} & 0.50 &  1   &  1.50  &   \cellcolor{gr_l}{1}  \\ 
            \hline
        \end{tabular}
        \caption{Matriz de comparación de $Cr_{13}$}
        \label{tab:MComCr13}
    \end{minipage}
    \begin{minipage}[b]{0.5\linewidth}
        \scriptsize
        \centering
            \begin{tabular}{|>{\centering\arraybackslash}m{2em} ||>{\centering\arraybackslash}m{2em} | >{\centering\arraybackslash}m{2em}| >{\centering\arraybackslash}m{2em}| >{\centering\arraybackslash}m{2em}|>{\centering\arraybackslash}m{2em}|}
            \hline
            & \textbf{$CS_1$} & \textbf{$CS_2$}& \textbf{$CS_3$}& \textbf{$CS_4$}& \textbf{$V_{Cr_{13}}$}\\
            \hline\hline
            \textbf{$CS_1$} & 0.43 &  0.43  &   0.43   &  0.43  &  0.43   \\
            \textbf{$CS_2$} & 0.21 &  0.21  &   0.21   &  0.21  &  0.21  \\
            \textbf{$CS_3$} & 0.14 &  0.14  &   0.14   &  0.14  &  0.14    \\
            \textbf{$CS_4$} & 0.21 &  0.21  &   0.21   &  0.21  &  0.21   \\ 
            \hline
        \end{tabular}
        \caption{Matriz normalizada de $Cr_{13}$ y $V_{Cr_{13}}$}
        \label{tab:MNorm_Cr13}
    \end{minipage}
\end{table}
%---------------CR13 Complejidad de manufactura del mecanismo de actuación------------------

%---------------CR14 Error de posición------------------
\begin{table}[!htbp]
    \begin{minipage}[b]{0.5\linewidth}
        \scriptsize
        \centering
            \begin{tabular}{|>{\centering\arraybackslash}m{2em} ||>{\centering\arraybackslash}m{2em} | >{\centering\arraybackslash}m{2em}| >{\centering\arraybackslash}m{2em}| >{\centering\arraybackslash}m{2em}|}
            \hline
            & \textbf{$CS_1$} & \textbf{$CS_2$}& \textbf{$CS_3$}& \textbf{$CS_4$}\\
            \hline\hline
            \textbf{$CS_1$} & \cellcolor{gr_l}{1}  &  10  &    2   &   1.25   \\
            \textbf{$CS_2$} & 0.10 &  \cellcolor{gr_l}{1} &   0.20   &   0.13  \\
            \textbf{$CS_3$} & 0.50 &  5   &  \cellcolor{gr_l}{1}   &  0.63  \\
            \textbf{$CS_4$} & 0.80 &  8   &  1.60  &   \cellcolor{gr_l}{1}  \\ 
            \hline
        \end{tabular}
        \caption{Matriz de comparación de $Cr_{14}$}
        \label{tab:MComCr14}
    \end{minipage}
    \begin{minipage}[b]{0.5\linewidth}
        \scriptsize
        \centering
            \begin{tabular}{|>{\centering\arraybackslash}m{2em} ||>{\centering\arraybackslash}m{2em} | >{\centering\arraybackslash}m{2em}| >{\centering\arraybackslash}m{2em}| >{\centering\arraybackslash}m{2em}|>{\centering\arraybackslash}m{2em}|}
            \hline
            & \textbf{$CS_1$} & \textbf{$CS_2$}& \textbf{$CS_3$}& \textbf{$CS_4$}& \textbf{$V_{Cr_{14}}$}\\
            \hline\hline
            \textbf{$CS_1$} & 0.42 &  0.42  &   0.42   &  0.42  &  0.42   \\
            \textbf{$CS_2$} & 0.04 &  0.04  &   0.04   &  0.04  &  0.04  \\
            \textbf{$CS_3$} & 0.21 &  0.21  &   0.21   &  0.21  &  0.21    \\
            \textbf{$CS_4$} & 0.33 &  0.33  &   0.33   &  0.33  &  0.33   \\ 
            \hline
        \end{tabular}
        \caption{Matriz normalizada de $Cr_{14}$ y $V_{Cr_{14}}$}
        \label{tab:MNorm_Cr14}
    \end{minipage}
\end{table}
%---------------CR14 Error de posición------------------

%---------------CR21 Robustez necesaria del anclaje------------------
\begin{table}[!htbp]
    \begin{minipage}[b]{0.5\linewidth}
        \scriptsize
        \centering
            \begin{tabular}{|>{\centering\arraybackslash}m{2em} ||>{\centering\arraybackslash}m{2em} | >{\centering\arraybackslash}m{2em}| >{\centering\arraybackslash}m{2em}| >{\centering\arraybackslash}m{2em}|}
            \hline
            & \textbf{$CS_1$} & \textbf{$CS_2$}& \textbf{$CS_3$}& \textbf{$CS_4$}\\
            \hline\hline
            \textbf{$CS_1$} & \cellcolor{gr_l}{1}  &  2  &    2   &   1   \\
            \textbf{$CS_2$} & 0.50 &  \cellcolor{gr_l}{1} &   1   &   0.50  \\
            \textbf{$CS_3$} & 0.50 &  1   &  \cellcolor{gr_l}{1}   &  0.50  \\
            \textbf{$CS_4$} & 1 &  2   &  2  &   \cellcolor{gr_l}{1}  \\ 
            \hline
        \end{tabular}
        \caption{Matriz de comparación de $Cr_{21}$}
        \label{tab:MComCr21}
    \end{minipage}
    \begin{minipage}[b]{0.5\linewidth}
        \scriptsize
        \centering
            \begin{tabular}{|>{\centering\arraybackslash}m{2em} ||>{\centering\arraybackslash}m{2em} | >{\centering\arraybackslash}m{2em}| >{\centering\arraybackslash}m{2em}| >{\centering\arraybackslash}m{2em}|>{\centering\arraybackslash}m{2em}|}
            \hline
            & \textbf{$CS_1$} & \textbf{$CS_2$}& \textbf{$CS_3$}& \textbf{$CS_4$}& \textbf{$V_{Cr_{21}}$}\\
            \hline\hline
            \textbf{$CS_1$} & 0.33 &  0.33  &   0.33   &  0.33  &  0.33   \\
            \textbf{$CS_2$} & 0.17 &  0.17  &   0.17   &  0.17  &  0.17  \\
            \textbf{$CS_3$} & 0.17 &  0.17  &   0.17   &  0.17  &  0.17    \\
            \textbf{$CS_4$} & 0.33 &  0.33  &   0.33   &  0.33  &  0.33   \\ 
            \hline
        \end{tabular}
        \caption{Matriz normalizada de $Cr_{21}$ y $V_{Cr_{21}}$}
        \label{tab:MNorm_Cr21}
    \end{minipage}
\end{table}
%---------------CR21 Robustez necesaria del anclaje------------------

%---------------CR22 Propension a fallas estructurales------------------
\begin{table}[!htbp]
    \begin{minipage}[b]{0.5\linewidth}
        \scriptsize
        \centering
            \begin{tabular}{|>{\centering\arraybackslash}m{2em} ||>{\centering\arraybackslash}m{2em} | >{\centering\arraybackslash}m{2em}| >{\centering\arraybackslash}m{2em}| >{\centering\arraybackslash}m{2em}|}
            \hline
            & \textbf{$CS_1$} & \textbf{$CS_2$}& \textbf{$CS_3$}& \textbf{$CS_4$}\\
            \hline\hline
            \textbf{$CS_1$} & \cellcolor{gr_l}{1}  &  7  &    7   &   1   \\
            \textbf{$CS_2$} & 0.14 &  \cellcolor{gr_l}{1} &   1   &   0.14  \\
            \textbf{$CS_3$} & 0.14 &  1   &  \cellcolor{gr_l}{1}   &  0.14  \\
            \textbf{$CS_4$} & 1 &  7   &  7  &   \cellcolor{gr_l}{1}  \\ 
            \hline
        \end{tabular}
        \caption{Matriz de comparación de $Cr_{22}$}
        \label{tab:MComCr22}
    \end{minipage}
    \begin{minipage}[b]{0.5\linewidth}
        \scriptsize
        \centering
            \begin{tabular}{|>{\centering\arraybackslash}m{2em} ||>{\centering\arraybackslash}m{2em} | >{\centering\arraybackslash}m{2em}| >{\centering\arraybackslash}m{2em}| >{\centering\arraybackslash}m{2em}|>{\centering\arraybackslash}m{2em}|}
            \hline
            & \textbf{$CS_1$} & \textbf{$CS_2$}& \textbf{$CS_3$}& \textbf{$CS_4$}& \textbf{$V_{Cr_{22}}$}\\
            \hline\hline
            \textbf{$CS_1$} & 0.44 &  0.44  &   0.44   &  0.44  &  0.44   \\
            \textbf{$CS_2$} & 0.06 &  0.06  &   0.06   &  0.06  &  0.06  \\
            \textbf{$CS_3$} & 0.06 &  0.06  &   0.06   &  0.06  &  0.06    \\
            \textbf{$CS_4$} & 0.44 &  0.44  &   0.44   &  0.44  &  0.44   \\ 
            \hline
        \end{tabular}
        \caption{Matriz normalizada de $Cr_{22}$ y $V_{Cr_{22}}$}
        \label{tab:MNorm_Cr22}
    \end{minipage}
\end{table}
%---------------CR22 Propension a fallas estructurales------------------

%---------------CR24 Facilidad para saltar entre disposiciones------------------
\begin{table}[!htbp]
    \begin{minipage}[b]{0.5\linewidth}
        \scriptsize
        \centering
            \begin{tabular}{|>{\centering\arraybackslash}m{2em} ||>{\centering\arraybackslash}m{2em} | >{\centering\arraybackslash}m{2em}| >{\centering\arraybackslash}m{2em}| >{\centering\arraybackslash}m{2em}|}
            \hline
            & \textbf{$CS_1$} & \textbf{$CS_2$}& \textbf{$CS_3$}& \textbf{$CS_4$}\\
            \hline\hline
            \textbf{$CS_1$} & \cellcolor{gr_l}{1}  &  10  &    5   &   1.25   \\
            \textbf{$CS_2$} & 0.10 &  \cellcolor{gr_l}{1} &   0.50   &   0.13  \\
            \textbf{$CS_3$} & 0.20 &  2   &  \cellcolor{gr_l}{1}   &  0.25  \\
            \textbf{$CS_4$} & 0.80 &  8   &  4  &   \cellcolor{gr_l}{1}  \\ 
            \hline
        \end{tabular}
        \caption{Matriz de comparación de $Cr_{24}$}
        \label{tab:MComCr24}
    \end{minipage}
    \begin{minipage}[b]{0.5\linewidth}
        \scriptsize
        \centering
            \begin{tabular}{|>{\centering\arraybackslash}m{2em} ||>{\centering\arraybackslash}m{2em} | >{\centering\arraybackslash}m{2em}| >{\centering\arraybackslash}m{2em}| >{\centering\arraybackslash}m{2em}|>{\centering\arraybackslash}m{2em}|}
            \hline
            & \textbf{$CS_1$} & \textbf{$CS_2$}& \textbf{$CS_3$}& \textbf{$CS_4$}& \textbf{$V_{Cr_{24}}$}\\
            \hline\hline
            \textbf{$CS_1$} & 0.48 &  0.48  &   0.48   &  0.48  &  0.48   \\
            \textbf{$CS_2$} & 0.05 &  0.05  &   0.05   &  0.05  &  0.05  \\
            \textbf{$CS_3$} & 0.10 &  0.10  &   0.10   &  0.10  &  0.10    \\
            \textbf{$CS_4$} & 0.38 &  0.38  &   0.38   &  0.38  &  0.38   \\ 
            \hline
        \end{tabular}
        \caption{Matriz normalizada de $Cr_{24}$ y $V_{Cr_{24}}$}
        \label{tab:MNorm_Cr24}
    \end{minipage}
\end{table}
%---------------CR24 Facilidad para saltar entre disposiciones------------------

%-------------------CR25 Modularidad----------------------
\begin{table}[!htbp]
    \begin{minipage}[b]{0.5\linewidth}
        \scriptsize
        \centering
            \begin{tabular}{|>{\centering\arraybackslash}m{2em} ||>{\centering\arraybackslash}m{2em} | >{\centering\arraybackslash}m{2em}| >{\centering\arraybackslash}m{2em}| >{\centering\arraybackslash}m{2em}|}
            \hline
            & \textbf{$CS_1$} & \textbf{$CS_2$}& \textbf{$CS_3$}& \textbf{$CS_4$}\\
            \hline\hline
            \textbf{$CS_1$} & \cellcolor{gr_l}{1}  &  4  &    2.67   &   1.33   \\
            \textbf{$CS_2$} & 0.25 &  \cellcolor{gr_l}{1} &   0.67   &   0.33  \\
            \textbf{$CS_3$} & 0.38 &  1.50   &  \cellcolor{gr_l}{1}   &  0.50  \\
            \textbf{$CS_4$} & 0.75 &  3   &  2  &   \cellcolor{gr_l}{1}  \\ 
            \hline
        \end{tabular}
        \caption{Matriz de comparación de $Cr_{25}$}
        \label{tab:MComCr25}
    \end{minipage}
    \begin{minipage}[b]{0.5\linewidth}
        \scriptsize
        \centering
            \begin{tabular}{|>{\centering\arraybackslash}m{2em} ||>{\centering\arraybackslash}m{2em} | >{\centering\arraybackslash}m{2em}| >{\centering\arraybackslash}m{2em}| >{\centering\arraybackslash}m{2em}|>{\centering\arraybackslash}m{2em}|}
            \hline
            & \textbf{$CS_1$} & \textbf{$CS_2$}& \textbf{$CS_3$}& \textbf{$CS_4$}& \textbf{$V_{Cr_{25}}$}\\
            \hline\hline
            \textbf{$CS_1$} & 0.42 &  0.42  &   0.42   &  0.42  &  0.42   \\
            \textbf{$CS_2$} & 0.11 &  0.11  &   0.11   &  0.11  &  0.11  \\
            \textbf{$CS_3$} & 0.16 &  0.16  &   0.16   &  0.16  &  0.16    \\
            \textbf{$CS_4$} & 0.32 &  0.32  &   0.32   &  0.32  &  0.32   \\ 
            \hline
        \end{tabular}
        \caption{Matriz normalizada de $Cr_{25}$ y $V_{Cr_{25}}$}
        \label{tab:MNorm_Cr25}
    \end{minipage}
\end{table}
%-------------------CR25 Modularidad----------------------

%-------------------CR27 Cantidad de paneles utilizados----------------------
\begin{table}[!htbp]
    \begin{minipage}[b]{0.5\linewidth}
        \scriptsize
        \centering
            \begin{tabular}{|>{\centering\arraybackslash}m{2em} ||>{\centering\arraybackslash}m{2em} | >{\centering\arraybackslash}m{2em}| >{\centering\arraybackslash}m{2em}| >{\centering\arraybackslash}m{2em}|}
            \hline
            & \textbf{$CS_1$} & \textbf{$CS_2$}& \textbf{$CS_3$}& \textbf{$CS_4$}\\
            \hline\hline
            \textbf{$CS_1$} & \cellcolor{gr_l}{1}  &  0.57  &    0.57   &   1   \\
            \textbf{$CS_2$} & 1.75 &  \cellcolor{gr_l}{1} &   1   &   1.75  \\
            \textbf{$CS_3$} & 1.75 &  1   &  \cellcolor{gr_l}{1}   &  1.75  \\
            \textbf{$CS_4$} & 1  &   0.57   &  0.57  &   \cellcolor{gr_l}{1}  \\ 
            \hline
        \end{tabular}
        \caption{Matriz de comparación de $Cr_{27}$}
        \label{tab:MComCr27}
    \end{minipage}
    \begin{minipage}[b]{0.5\linewidth}
        \scriptsize
        \centering
            \begin{tabular}{|>{\centering\arraybackslash}m{2em} ||>{\centering\arraybackslash}m{2em} | >{\centering\arraybackslash}m{2em}| >{\centering\arraybackslash}m{2em}| >{\centering\arraybackslash}m{2em}|>{\centering\arraybackslash}m{2em}|}
            \hline
            & \textbf{$CS_1$} & \textbf{$CS_2$}& \textbf{$CS_3$}& \textbf{$CS_4$}& \textbf{$V_{Cr_{27}}$}\\
            \hline\hline
            \textbf{$CS_1$} & 0.18 &  0.18  &   0.18   &  0.18  &  0.18   \\
            \textbf{$CS_2$} & 0.32 &  0.32  &   0.32   &  0.32  &  0.32  \\
            \textbf{$CS_3$} & 0.32 &  0.32  &   0.32   &  0.32  &  0.32    \\
            \textbf{$CS_4$} & 0.18 &  0.18  &   0.18   &  0.18  &  0.18   \\ 
            \hline
        \end{tabular}
        \caption{Matriz normalizada de $Cr_{27}$ y $V_{Cr_{27}}$}
        \label{tab:MNorm_Cr27}
    \end{minipage}
\end{table}
%-------------------CR27 Cantidad de paneles utilizados----------------------
A continuación, y siguiendo la metodología, se deben comparar los criterios de modo que nos permitan obtener un vector de prioridad. Lo anterior es resultado de la diferencia en la importancia que se le asigna a cada criterio. 
%-------------------Matriz de comparacion de criterios----------------------
\begin{landscape}
    \begin{table}[!htb]
    \scriptsize
    \centering
        \begin{tabular}{|>{\centering\arraybackslash}m{2em} ||>{\centering\arraybackslash}m{2em} | >{\centering\arraybackslash}m{2em}| >{\centering\arraybackslash}m{2em}| >{\centering\arraybackslash}m{2em}|>{\centering\arraybackslash}m{2em}|>{\centering\arraybackslash}m{2em}|>{\centering\arraybackslash}m{2em}|>{\centering\arraybackslash}m{2em}|>{\centering\arraybackslash}m{2em}|>{\centering\arraybackslash}m{2em}|>{\centering\arraybackslash}m{2em}|>{\centering\arraybackslash}m{2em}|>{\centering\arraybackslash}m{2em}|>{\centering\arraybackslash}m{2em}|>{\centering\arraybackslash}m{2em}|>{\centering\arraybackslash}m{2em}|>{\centering\arraybackslash}m{2em}|>{\centering\arraybackslash}m{2em}|>{\centering\arraybackslash}m{2em}|} % Especifica las columnas aquí
            \hline
             &$Cr_{1}$ & $Cr_{2}$ & $Cr_{3}$ & $Cr_{4}$ & $Cr_{5}$ & $Cr_{6}$ & $Cr_{7}$ & $Cr_{8}$ & $Cr_{9}$ & $Cr_{10}$ & $Cr_{11}$ & $Cr_{12}$ & $Cr_{13}$ & $Cr_{14}$& $Cr_{21}$ & $Cr_{22}$ & $Cr_{24}$ & $Cr_{25}$ & $Cr_{27}$\\
            \hline
            \hline
            $Cr_{1}$ & \cellcolor{gr_l}{1} & 5 & 0.75 & 3 & 3 & 1 & 3 & 5 & 10 & 5 & 3 & 2 & 5 & 2 & 8 & 2 & 1& 5& 4 \\
            $Cr_{2}$ & 0.20 & \cellcolor{gr_l}{1} & 6.67 & 1.67 & 1.67 & 5 & 1.67 & 1 & 0.50 & 1 & 1.67 & 2.50 & 1 & 2.50 & 0.63 & 2.50 & 5 & 1 & 1.25\\
            $Cr_{3}$ & 1.33 & 0.15 & \cellcolor{gr_l}{1} & 0.25 & 0.25 & 0.75 & 0.25 & 0.15 & 0.08 & 0.15 & 0.25 & 0.38 & 0.15 & 0.38 & 0.09 & 0.38 & 0.75 & 0.15 & 0.19\\
            $Cr_{4}$ & 0.33 & 0.60 & 4 & \cellcolor{gr_l}{1} & 1 & 3 & 1 & 0.60 & 0.30 & 0.60 & 1 & 1.50 & 0.60 & 1.50 & 0.38 & 1.50 & 3 & 0.60 & 0.75\\
            $Cr_{5}$ & 0.33 & 0.60 & 4 & 1 & \cellcolor{gr_l}{1} & 3 & 1 & 0.60 & 0.30 & 0.60 & 1 & 1.50 & 0.60 & 1.50 & 0.38 & 1.50 & 3 & 0.60 & 0.75\\
            $Cr_{6}$ & 1 & 0.20 & 1.33 & 0.33 & 0.33 & \cellcolor{gr_l}{1} & 0.33 & 0.20 & 0.10 & 0.20 & 0.33 & 0.50 & 0.20 & 0.50 & 0.13 & 0.50 & 1 & 0.20 & 0.25\\
            $Cr_{7}$ & 0.33 & 0.60 & 4 & 1 & 1 & 3 & \cellcolor{gr_l}{1} & 0.60 & 0.30 & 0.60 & 1 & 1.50 & 0.60 & 1.50 & 0.38 & 1.50 & 3 & 0.60 & 0.75\\
            $Cr_{8}$ & 0.20 & 1 & 6.67 & 1.67 & 1.67 & 5 & 1.67 & \cellcolor{gr_l}{1} & 0.50 & 1 & 1.67 & 2.50 & 1 & 2.50 & 0.63 & 2.50 & 5 & 1 & 1.25\\
            $Cr_{9}$ & 0.10 & 2 & 13.33 & 3.33 & 3.33 & 10 & 3.33 & 2 & \cellcolor{gr_l}{1} & 2 & 3.33 & 5 & 2 & 5 & 1.25 & 5 & 10 & 2 & 2.50\\
            $Cr_{10}$ & 0.20 & 1 & 6.67 & 1.67 & 1.67 & 5 & 1.67 & 1 & 0.50 & \cellcolor{gr_l}{1} & 1.67 & 2.50 & 1 & 2.50 & 0.63 & 2.50 & 5 & 1 & 1.25\\
            $Cr_{11}$ & 0.33 & 0.60 & 4 & 1 & 1 & 3 & 1 & 0.60 & 0.30 & 0.60 & \cellcolor{gr_l}{1} & 1.50 & 0.60 & 1.50 & 0.38 & 1.50 & 3 & 0.60 & 0.75\\
            $Cr_{12}$ & 0.50 & 0.40 & 2.67 & 0.67 & 0.67 & 2 & 0.67 & 0.40 & 0.20 & 0.40 & 0.67 & \cellcolor{gr_l}{1} & 0.40 & 1 & 0.25 & 1 & 2 & 0.40 & 0.50\\
            $Cr_{13}$ & 0.20 & 1 & 6.67 & 1.67 & 1.67 & 5 & 1.67 & 1 & 0.50 & 1 & 1.67 & 2.50 & \cellcolor{gr_l}{1} & 2.50 & 0.63 & 2.50 & 5 & 1 & 1.25\\
            $Cr_{14}$ & 0.50 & 0.40 & 2.67 & 0.67 & 0.67 & 2 & 0.67 & 0.40 & 0.20 & 0.40 & 0.67 & 1 & 0.40 & \cellcolor{gr_l}{1} & 0.25 & 1 & 2 & 0.40 & 0.50\\
            $Cr_{21}$ & 0.13 & 1.60 & 10.67 & 2.67 & 2.67 & 8 & 2.67 & 1.60 & 0.80 & 1.60 & 2.67 & 4 & 1.60 & 4 & \cellcolor{gr_l}{1} & 4 & 8 & 1.60 & 2\\
            $Cr_{22}$ & 0.50 & 0.40 & 2.67 & 0.67 & 0.67 & 2 & 0.67 & 0.40 & 0.20 & 0.40 & 0.67 & 1 & 0.40 & 1 & 0.25 & \cellcolor{gr_l}{1} & 2 & 0.40 & 0.50\\
            $Cr_{24}$ & 1 & 0.20 & 1.33 & 0.33 & 0.33 & 1 & 0.33 & 0.20 & 0.10 & 0.20 & 0.33 & 0.50 & 0.20 & 0.50 & 0.13 & 0.50 & \cellcolor{gr_l}{1} & 0.20 & 0.25\\
            $Cr_{25}$ & 0.20 & 1 & 6.67 & 1.67 & 1.67 & 5 & 1.67 & 1 & 0.50 & 1 & 1.67 & 2.50 & 1 & 2.50 & 0.63 & 2.50 & 5 & \cellcolor{gr_l}{1} & 1.25\\
            $Cr_{27}$ & 0.25 & 0.80 & 5.33 & 1.33 & 1.33 & 4 & 1.33 & 0.80 & 0.40 & 0.80 & 1.33 & 2 & 0.80 & 2 & 0.50 & 2 & 4 & 0.80 & \cellcolor{gr_l}{1}\\
            % Añade más filas según sea necesario
            \hline
        \end{tabular}
        \caption{Matriz de comparación de criterios}
        \label{tab:MCompCrit}
    \end{table}
%--------------------------------------------------------------------------------------------
    \begin{table}[!htb]
    \scriptsize
    \centering
        \begin{tabular}{|>{\centering\arraybackslash}m{2em} ||>{\centering\arraybackslash}m{2em} | >{\centering\arraybackslash}m{2em}| >{\centering\arraybackslash}m{2em}| >{\centering\arraybackslash}m{2em}|>{\centering\arraybackslash}m{2em}|>{\centering\arraybackslash}m{2em}|>{\centering\arraybackslash}m{2em}|>{\centering\arraybackslash}m{2em}|>{\centering\arraybackslash}m{2em}|>{\centering\arraybackslash}m{2em}|>{\centering\arraybackslash}m{2em}|>{\centering\arraybackslash}m{2em}|>{\centering\arraybackslash}m{2em}|>{\centering\arraybackslash}m{2em}|>{\centering\arraybackslash}m{2em}|>{\centering\arraybackslash}m{2em}|>{\centering\arraybackslash}m{2em}|>{\centering\arraybackslash}m{2em}|>{\centering\arraybackslash}m{2em}|>{\centering\arraybackslash}m{2em}|} % Especifica las columnas aquí
            \hline
             &$Cr_{1}$ & $Cr_{2}$ & $Cr_{3}$ & $Cr_{4}$ & $Cr_{5}$ & $Cr_{6}$ & $Cr_{7}$ & $Cr_{8}$ & $Cr_{9}$ & $Cr_{10}$ & $Cr_{11}$ & $Cr_{12}$ & $Cr_{13}$ & $Cr_{14}$& $Cr_{21}$ & $Cr_{22}$ & $Cr_{24}$ & $Cr_{25}$ & $Cr_{27}$ & $V_{C_r}$\\
            \hline
            \hline
            $Cr_{1}$ & 0.116 & 0.270 & 0.008 & 0.117 & 0.117 & 0.015 & 0.117 & 0.270 & 0.596 & 0.270 & 0.117 & 0.056 & 0.270 & 0.056 & 0.486 & 0.056 & 0.015 & 0.270 & 0.191 & 0.133 \\
            $Cr_{2}$ & 0.023 & 0.054 & 0.073 & 0.065 & 0.065 & 0.073 & 0.065 & 0.054 & 0.030 & 0.054 & 0.065 & 0.070 & 0.054 & 0.070 & 0.038 & 0.070 & 0.073 & 0.054 & 0.060 & 0.064 \\
            $Cr_{3}$ & 0.154 & 0.008 & 0.011 & 0.010 & 0.010 & 0.011 & 0.010 & 0.008 & 0.004 & 0.008 & 0.010 & 0.010 & 0.008 & 0.010 & 0.006 & 0.010 & 0.011 & 0.008 & 0.009 & 0.010 \\
            $Cr_{4}$ & 0.039 & 0.032 & 0.044 & 0.039 & 0.039 & 0.044 & 0.039 & 0.032 & 0.018 & 0.032 & 0.039 & 0.042 & 0.032 & 0.042 & 0.023 & 0.042 & 0.044 & 0.032 & 0.036 & 0.038 \\
            $Cr_{5}$ & 0.039 & 0.032 & 0.044 & 0.039 & 0.039 & 0.044 & 0.039 & 0.032 & 0.018 & 0.032 & 0.039 & 0.042 & 0.032 & 0.042 & 0.023 & 0.042 & 0.044 & 0.032 & 0.036 & 0.038 \\
            $Cr_{6}$ & 0.116 & 0.011 & 0.015 & 0.013 & 0.013 & 0.015 & 0.013 & 0.011 & 0.006 & 0.011 & 0.013 & 0.014 & 0.011 & 0.014 & 0.008 & 0.014 & 0.015 & 0.011 & 0.012 & 0.013 \\
            $Cr_{7}$ & 0.039 & 0.032 & 0.044 & 0.039 & 0.039 & 0.044 & 0.039 & 0.032 & 0.018 & 0.032 & 0.039 & 0.042 & 0.032 & 0.042 & 0.023 & 0.042 & 0.044 & 0.032 & 0.036 & 0.038 \\
            $Cr_{8}$ & 0.023 & 0.054 & 0.073 & 0.065 & 0.065 & 0.073 & 0.065 & 0.054 & 0.030 & 0.054 & 0.065 & 0.070 & 0.054 & 0.070 & 0.038 & 0.070 & 0.073 & 0.054 & 0.060 & 0.064 \\
            $Cr_{9}$ & 0.012 & 0.108 & 0.146 & 0.130 & 0.130 & 0.145 & 0.130 & 0.108 & 0.060 & 0.108 & 0.130 & 0.139 & 0.108 & 0.139 & 0.076 & 0.139 & 0.145 & 0.108 & 0.119 & 0.128 \\
            $Cr_{10}$ & 0.023 & 0.054 & 0.073 & 0.065 & 0.065 & 0.073 & 0.065 & 0.054 & 0.030 & 0.054 & 0.065 & 0.070 & 0.054 & 0.070 & 0.038 & 0.070 & 0.073 & 0.054 & 0.060 & 0.064 \\
            $Cr_{11}$ & 0.039 & 0.032 & 0.044 & 0.039 & 0.039 & 0.044 & 0.039 & 0.032 & 0.018 & 0.032 & 0.039 & 0.042 & 0.032 & 0.042 & 0.023 & 0.042 & 0.044 & 0.032 & 0.036 & 0.038 \\
            $Cr_{12}$ & 0.058 & 0.022 & 0.029 & 0.026 & 0.026 & 0.029 & 0.026 & 0.022 & 0.012 & 0.022 & 0.026 & 0.028 & 0.022 & 0.028 & 0.015 & 0.028 & 0.029 & 0.022 & 0.024 & 0.026 \\
            $Cr_{13}$ & 0.023 & 0.054 & 0.073 & 0.065 & 0.065 & 0.073 & 0.065 & 0.054 & 0.030 & 0.054 & 0.065 & 0.070 & 0.054 & 0.070 & 0.038 & 0.070 & 0.073 & 0.054 & 0.060 & 0.064 \\
            $Cr_{14}$ & 0.058 & 0.022 & 0.029 & 0.026 & 0.026 & 0.029 & 0.026 & 0.022 & 0.012 & 0.022 & 0.026 & 0.028 & 0.022 & 0.028 & 0.015 & 0.028 & 0.029 & 0.022 & 0.024 & 0.026 \\
            $Cr_{21}$ & 0.014 & 0.086 & 0.117 & 0.104 & 0.104 & 0.116 & 0.104 & 0.086 & 0.048 & 0.086 & 0.104 & 0.111 & 0.086 & 0.111 & 0.061 & 0.111 & 0.116 & 0.086 & 0.096 & 0.102\\
            $Cr_{22}$ & 0.058 & 0.022 & 0.029 & 0.026 & 0.026 & 0.029 & 0.026 & 0.022 & 0.012 & 0.022 & 0.026 & 0.028 & 0.022 & 0.028 & 0.015 & 0.028 & 0.029 & 0.022 & 0.024 & 0.026 \\
            $Cr_{24}$ & 0.116 & 0.011 & 0.015 & 0.013 & 0.013 & 0.015 & 0.013 & 0.011 & 0.006 & 0.011 & 0.013 & 0.014 & 0.011 & 0.014 & 0.008 & 0.014 & 0.015 & 0.011 & 0.012 & 0.013 \\
            $Cr_{25}$ & 0.023 & 0.054 & 0.073 & 0.065 & 0.065 & 0.073 & 0.065 & 0.054 & 0.030 & 0.054 & 0.065 & 0.070 & 0.054 & 0.070 & 0.038 & 0.070 & 0.073 & 0.054 & 0.060 & 0.064 \\
            $Cr_{27}$ & 0.029 & 0.043 & 0.059 & 0.052 & 0.052 & 0.058 & 0.052 & 0.043 & 0.024 & 0.043 & 0.052 & 0.056 & 0.043 & 0.056 & 0.030 & 0.056 & 0.058 & 0.043 & 0.048 & 0.051 \\
            % Añade más filas según sea necesario
            \hline
        \end{tabular}
        \caption{Matriz normalizada y vector de prioridad para la comparación de criterios.}
        \label{tab:MNormCrit}
    \end{table}
\end{landscape}

%-------------------Matriz de comparacion de criterios----------------------


%-------------------AHP Independientes----------------------
Ahora, se analizan las alternativas independientes para la decisión usando la misma metodología.

%-------------------C3 Grabación en varias prosiciones----------------------
\begin{table}[!htbp]
    \begin{minipage}[b]{0.5\linewidth}
        \scriptsize
        \centering
            \begin{tabular}{|>{\centering\arraybackslash}m{2em} ||>{\centering\arraybackslash}m{2em} | >{\centering\arraybackslash}m{2em}| >{\centering\arraybackslash}m{2em}| >{\centering\arraybackslash}m{2em}|}
            \hline
            & \textbf{$Alt_1$} & \textbf{$Alt_2$}& \textbf{$Alt_3$}& \textbf{$Alt_4$}\\
            \hline\hline
            \textbf{$Alt_1$} & \cellcolor{gr_l}{1}  &  1  &    2.67   &   1.33   \\
            \textbf{$Alt_2$} & 1 &  \cellcolor{gr_l}{1} &   2.67   &   1.33  \\
            \textbf{$Alt_3$} & 0.38 &  0.38   &  \cellcolor{gr_l}{1}   &  0.50  \\
            \textbf{$Alt_4$} & 0.75  &   0.75   &  2  &   \cellcolor{gr_l}{1}  \\ 
            \hline
        \end{tabular}
        \caption{Matriz de comparación de $C_{3}$}
        \label{tab:MComC3}
    \end{minipage}
    \begin{minipage}[b]{0.5\linewidth}
        \scriptsize
        \centering
            \begin{tabular}{|>{\centering\arraybackslash}m{2em} ||>{\centering\arraybackslash}m{2em} | >{\centering\arraybackslash}m{2em}| >{\centering\arraybackslash}m{2em}| >{\centering\arraybackslash}m{2em}|>{\centering\arraybackslash}m{2em}|}
            \hline
            & \textbf{$Alt_1$} & \textbf{$Alt_2$}& \textbf{$Alt_3$}& \textbf{$Alt_4$}& \textbf{$V_{C_{3}}$}\\
            \hline\hline
            \textbf{$Alt_1$} & 0.32 &  0.32  &   0.32   &  0.32  & \cellcolor{gr_l}{0.32}   \\
            \textbf{$Alt_2$} & 0.32 &  0.32  &   0.32   &  0.32  &  0.32  \\
            \textbf{$Alt_3$} & 0.12 &  0.12  &   0.12   &  0.12  &  0.12    \\
            \textbf{$Alt_4$} & 0.24 &  0.24  &   0.24   &  0.24  &  0.24   \\ 
            \hline
        \end{tabular}
        \caption{Matriz normalizada de $C_{3}$ y $V_{C_{3}}$}
        \label{tab:MNorm_C3}
    \end{minipage}
\end{table}
%-------------------C3 Grabación en varias prosiciones----------------------

%-------------------C4 Generación de ondas de sonido----------------------
\begin{table}[!htbp]
    \begin{minipage}[b]{0.5\linewidth}
        \scriptsize
        \centering
            \begin{tabular}{|>{\centering\arraybackslash}m{2em} ||>{\centering\arraybackslash}m{2em} | >{\centering\arraybackslash}m{2em}| >{\centering\arraybackslash}m{2em}| >{\centering\arraybackslash}m{2em}|}
            \hline
            & \textbf{$Alt_1$} & \textbf{$Alt_2$}& \textbf{$Alt_3$}& \textbf{$Alt_4$}\\
            \hline\hline
            \textbf{$Alt_1$} & \cellcolor{gr_l}{1}&         0.50         &      1.67            &   -   \\
            \textbf{$Alt_2$} &          2         &  \cellcolor{gr_l}{1} &      3.33            &   -   \\
            \textbf{$Alt_3$} &          0.60      &         0.30         &  \cellcolor{gr_l}{1} &   -   \\
            \textbf{$Alt_4$} &          -         &          -           &       -              &   \cellcolor{gr_l}{-}  \\ 
            \hline
        \end{tabular}
        \caption{Matriz de comparación de $C_{4}$}
        \label{tab:MComC4}
    \end{minipage}
    \begin{minipage}[b]{0.5\linewidth}
        \scriptsize
        \centering
            \begin{tabular}{|>{\centering\arraybackslash}m{2em} ||>{\centering\arraybackslash}m{2em} | >{\centering\arraybackslash}m{2em}| >{\centering\arraybackslash}m{2em}| >{\centering\arraybackslash}m{2em}|>{\centering\arraybackslash}m{2em}|}
            \hline
            & \textbf{$Alt_1$} & \textbf{$Alt_2$}& \textbf{$Alt_3$}& \textbf{$Alt_4$}& \textbf{$V_{C_{4}}$}\\
            \hline\hline
            \textbf{$Alt_1$} & 0.28 &  0.28  &   0.28   &    -   &  0.28   \\
            \textbf{$Alt_2$} & 0.56 &  0.56  &   0.56   &    -   &  \cellcolor{gr_l}{0.56}  \\
            \textbf{$Alt_3$} & 0.17 &  0.17  &   0.17   &    -   &  0.17    \\
            \textbf{$Alt_4$} &   -  &   -    &    -     &    -   &    -   \\ 
            \hline
        \end{tabular}
        \caption{Matriz normalizada de $C_{4}$ y $V_{C_{4}}$}
        \label{tab:MNorm_C4}
    \end{minipage}
\end{table}
%-------------------C4 Generación de ondas de sonido----------------------

%-------------------C5 dispositivo de grabación----------------------
\begin{table}[!htbp]
    \begin{minipage}[b]{0.5\linewidth}
        \scriptsize
        \centering
            \begin{tabular}{|>{\centering\arraybackslash}m{2em} ||>{\centering\arraybackslash}m{2em} | >{\centering\arraybackslash}m{2em}| >{\centering\arraybackslash}m{2em}| >{\centering\arraybackslash}m{2em}|}
            \hline
            & \textbf{$Alt_1$} & \textbf{$Alt_2$}& \textbf{$Alt_3$}& \textbf{$Alt_4$}\\
            \hline\hline
            \textbf{$Alt_1$} & \cellcolor{gr_l}{1}&         0.80         &      2               &   -   \\
            \textbf{$Alt_2$} &          1.25      &  \cellcolor{gr_l}{1} &      2.50            &   -   \\
            \textbf{$Alt_3$} &          0.50      &         0.40         &  \cellcolor{gr_l}{1} &   -   \\
            \textbf{$Alt_4$} &          -         &          -           &       -              &   \cellcolor{gr_l}{-}  \\ 
            \hline
        \end{tabular}
        \caption{Matriz de comparación de $C_{5}$}
        \label{tab:MComC5}
    \end{minipage}
    \begin{minipage}[b]{0.5\linewidth}
        \scriptsize
        \centering
            \begin{tabular}{|>{\centering\arraybackslash}m{2em} ||>{\centering\arraybackslash}m{2em} | >{\centering\arraybackslash}m{2em}| >{\centering\arraybackslash}m{2em}| >{\centering\arraybackslash}m{2em}|>{\centering\arraybackslash}m{2em}|}
            \hline
            & \textbf{$Alt_1$} & \textbf{$Alt_2$}& \textbf{$Alt_3$}& \textbf{$Alt_4$}& \textbf{$V_{C_{5}}$}\\
            \hline\hline
            \textbf{$Alt_1$} & 0.36 &  0.36  &   0.36   &    -   &  0.36   \\
            \textbf{$Alt_2$} & 0.45 &  0.45  &   0.45   &    -   &  \cellcolor{gr_l}{0.45}  \\
            \textbf{$Alt_3$} & 0.18 &  0.18  &   0.18   &    -   &  0.18    \\
            \textbf{$Alt_4$} &   -  &   -    &    -     &    -   &    -   \\ 
            \hline
        \end{tabular}
        \caption{Matriz normalizada de $C_{5}$ y $V_{C_{5}}$}
        \label{tab:MNorm_C5}
    \end{minipage}
\end{table}
%-------------------C5 dispositivo de grabación----------------------

%-------------------C6 Plataforma de cálculo de las posiciones deseadas----------------------
\begin{table}[!htbp]
    \begin{minipage}[b]{0.5\linewidth}
        \scriptsize
        \centering
            \begin{tabular}{|>{\centering\arraybackslash}m{2em} ||>{\centering\arraybackslash}m{2em} | >{\centering\arraybackslash}m{2em}| >{\centering\arraybackslash}m{2em}| >{\centering\arraybackslash}m{2em}|}
            \hline
            & \textbf{$Alt_1$} & \textbf{$Alt_2$}& \textbf{$Alt_3$}& \textbf{$Alt_4$}\\
            \hline\hline
            \textbf{$Alt_1$} & \cellcolor{gr_l}{1}&         0.89         &      1.33            &   -   \\
            \textbf{$Alt_2$} &          1.13      &  \cellcolor{gr_l}{1} &      1.50            &   -   \\
            \textbf{$Alt_3$} &          0.75      &         0.67         &  \cellcolor{gr_l}{1} &   -   \\
            \textbf{$Alt_4$} &          -         &          -           &       -              &   \cellcolor{gr_l}{-}  \\ 
            \hline
        \end{tabular}
        \caption{Matriz de comparación de $C_{6}$}
        \label{tab:MComC6}
    \end{minipage}
    \begin{minipage}[b]{0.5\linewidth}
        \scriptsize
        \centering
            \begin{tabular}{|>{\centering\arraybackslash}m{2em} ||>{\centering\arraybackslash}m{2em} | >{\centering\arraybackslash}m{2em}| >{\centering\arraybackslash}m{2em}| >{\centering\arraybackslash}m{2em}|>{\centering\arraybackslash}m{2em}|}
            \hline
            & \textbf{$Alt_1$} & \textbf{$Alt_2$}& \textbf{$Alt_3$}& \textbf{$Alt_4$}& \textbf{$V_{C_{6}}$}\\
            \hline\hline
            \textbf{$Alt_1$} & 0.35 &  0.35  &   0.35   &    -   &  0.35   \\
            \textbf{$Alt_2$} & 0.39 &  0.39  &   0.39   &    -   &  \cellcolor{gr_l}{0.39}  \\
            \textbf{$Alt_3$} & 0.26 &  0.26  &   0.26   &    -   &  0.26    \\
            \textbf{$Alt_4$} &   -  &   -    &    -     &    -   &    -   \\ 
            \hline
        \end{tabular}
        \caption{Matriz normalizada de $C_{6}$ y $V_{C_{6}}$}
        \label{tab:MNorm_C6}
    \end{minipage}
\end{table}
%-------------------C6 Plataforma de cálculo de las posiciones deseadas----------------------

%-------------------C7 Plataforma del procesado del control----------------------
\begin{table}[!htbp]
    \begin{minipage}[b]{0.5\linewidth}
        \scriptsize
        \centering
            \begin{tabular}{|>{\centering\arraybackslash}m{2em} ||>{\centering\arraybackslash}m{2em} | >{\centering\arraybackslash}m{2em}| >{\centering\arraybackslash}m{2em}| >{\centering\arraybackslash}m{2em}|}
            \hline
            & \textbf{$Alt_1$} & \textbf{$Alt_2$}& \textbf{$Alt_3$}& \textbf{$Alt_4$}\\
            \hline\hline
            \textbf{$Alt_1$} & \cellcolor{gr_l}{1}&         0.89         &      1.33            &   -   \\
            \textbf{$Alt_2$} &          1.13      &  \cellcolor{gr_l}{1} &      1.50            &   -   \\
            \textbf{$Alt_3$} &          0.75      &         0.67         &  \cellcolor{gr_l}{1} &   -   \\
            \textbf{$Alt_4$} &          -         &          -           &       -              &   \cellcolor{gr_l}{-}  \\ 
            \hline
        \end{tabular}
        \caption{Matriz de comparación de $C_{7}$}
        \label{tab:MComC7}
    \end{minipage}
    \begin{minipage}[b]{0.5\linewidth}
        \scriptsize
        \centering
            \begin{tabular}{|>{\centering\arraybackslash}m{2em} ||>{\centering\arraybackslash}m{2em} | >{\centering\arraybackslash}m{2em}| >{\centering\arraybackslash}m{2em}| >{\centering\arraybackslash}m{2em}|>{\centering\arraybackslash}m{2em}|}
            \hline
            & \textbf{$Alt_1$} & \textbf{$Alt_2$}& \textbf{$Alt_3$}& \textbf{$Alt_4$}& \textbf{$V_{C_{7}}$}\\
            \hline\hline
            \textbf{$Alt_1$} & 0.35 &  0.35  &   0.35   &    -   &  0.35   \\
            \textbf{$Alt_2$} & 0.39 &  0.39  &   0.39   &    -   &  \cellcolor{gr_l}{0.39}  \\
            \textbf{$Alt_3$} & 0.26 &  0.26  &   0.26   &    -   &  0.26    \\
            \textbf{$Alt_4$} &   -  &   -    &    -     &    -   &    -   \\ 
            \hline
        \end{tabular}
        \caption{Matriz normalizada de $C_{7}$ y $V_{C_{7}}$}
        \label{tab:MNorm_C7}
    \end{minipage}
\end{table}
%-------------------C7 Plataforma del procesado del control----------------------

%-------------------C8 Plataforma de procesado de la acústica----------------------
\begin{table}[!htbp]
    \begin{minipage}[b]{0.5\linewidth}
        \scriptsize
        \centering
            \begin{tabular}{|>{\centering\arraybackslash}m{2em} ||>{\centering\arraybackslash}m{2em} | >{\centering\arraybackslash}m{2em}| >{\centering\arraybackslash}m{2em}| >{\centering\arraybackslash}m{2em}|}
            \hline
            & \textbf{$Alt_1$} & \textbf{$Alt_2$}& \textbf{$Alt_3$}& \textbf{$Alt_4$}\\
            \hline\hline
            \textbf{$Alt_1$} & \cellcolor{gr_l}{1}&         0.89         &      1.33            &   -   \\
            \textbf{$Alt_2$} &          1.13      &  \cellcolor{gr_l}{1} &      1.50            &   -   \\
            \textbf{$Alt_3$} &          0.75      &         0.67         &  \cellcolor{gr_l}{1} &   -   \\
            \textbf{$Alt_4$} &          -         &          -           &       -              &   \cellcolor{gr_l}{-}  \\ 
            \hline
        \end{tabular}
        \caption{Matriz de comparación de $C_{8}$}
        \label{tab:MComC8}
    \end{minipage}
    \begin{minipage}[b]{0.5\linewidth}
        \scriptsize
        \centering
            \begin{tabular}{|>{\centering\arraybackslash}m{2em} ||>{\centering\arraybackslash}m{2em} | >{\centering\arraybackslash}m{2em}| >{\centering\arraybackslash}m{2em}| >{\centering\arraybackslash}m{2em}|>{\centering\arraybackslash}m{2em}|}
            \hline
            & \textbf{$Alt_1$} & \textbf{$Alt_2$}& \textbf{$Alt_3$}& \textbf{$Alt_4$}& \textbf{$V_{C_{8}}$}\\
            \hline\hline
            \textbf{$Alt_1$} & 0.35 &  0.35  &   0.35   &    -   &  0.35   \\
            \textbf{$Alt_2$} & 0.39 &  0.39  &   0.39   &    -   &  \cellcolor{gr_l}{0.39}  \\
            \textbf{$Alt_3$} & 0.26 &  0.26  &   0.26   &    -   &  0.26    \\
            \textbf{$Alt_4$} &   -  &   -    &    -     &    -   &    -   \\ 
            \hline
        \end{tabular}
        \caption{Matriz normalizada de $C_{8}$ y $V_{C_{8}}$}
        \label{tab:MNorm_C8}
    \end{minipage}
\end{table}
%-------------------C8 Plataforma de procesado de la acústica----------------------

%-------------------C9 Sensado de las posiciones de los paneles----------------------
\begin{table}[!htbp]
    \begin{minipage}[b]{0.5\linewidth}
        \scriptsize
        \centering
            \begin{tabular}{|>{\centering\arraybackslash}m{2em} ||>{\centering\arraybackslash}m{2em} | >{\centering\arraybackslash}m{2em}| >{\centering\arraybackslash}m{2em}| >{\centering\arraybackslash}m{2em}|}
            \hline
            & \textbf{$Alt_1$} & \textbf{$Alt_2$}& \textbf{$Alt_3$}& \textbf{$Alt_4$}\\
            \hline\hline
            \textbf{$Alt_1$} & \cellcolor{gr_l}{1}&         3            &      4.50            &   -   \\
            \textbf{$Alt_2$} &          0.33      &  \cellcolor{gr_l}{1} &      1.50            &   -   \\
            \textbf{$Alt_3$} &          0.22      &         0.67         &  \cellcolor{gr_l}{1} &   -   \\
            \textbf{$Alt_4$} &          -         &          -           &       -              &   \cellcolor{gr_l}{-}  \\ 
            \hline
        \end{tabular}
        \caption{Matriz de comparación de $C_{9}$}
        \label{tab:MComC9}
    \end{minipage}
    \begin{minipage}[b]{0.5\linewidth}
        \scriptsize
        \centering
            \begin{tabular}{|>{\centering\arraybackslash}m{2em} ||>{\centering\arraybackslash}m{2em} | >{\centering\arraybackslash}m{2em}| >{\centering\arraybackslash}m{2em}| >{\centering\arraybackslash}m{2em}|>{\centering\arraybackslash}m{2em}|}
            \hline
            & \textbf{$Alt_1$} & \textbf{$Alt_2$}& \textbf{$Alt_3$}& \textbf{$Alt_4$}& \textbf{$V_{C_{9}}$}\\
            \hline\hline
            \textbf{$Alt_1$} & 0.64 &  0.64  &   0.64   &    -   &  \cellcolor{gr_l}{0.64}   \\
            \textbf{$Alt_2$} & 0.21 &  0.21  &   0.21   &    -   &  0.21  \\
            \textbf{$Alt_3$} & 0.14 &  0.14  &   0.14   &    -   &  0.14    \\
            \textbf{$Alt_4$} &   -  &   -    &    -     &    -   &    -   \\ 
            \hline
        \end{tabular}
        \caption{Matriz normalizada de $C_{9}$ y $V_{C_{9}}$}
        \label{tab:MNorm_C9}
    \end{minipage}
\end{table}
%-------------------C9 Sensado de las posiciones de los paneles----------------------

%-------------------C10 Plataforma de visualizacion de información----------------------
\begin{table}[!htbp]
    \begin{minipage}[b]{0.5\linewidth}
        \scriptsize
        \centering
            \begin{tabular}{|>{\centering\arraybackslash}m{2em} ||>{\centering\arraybackslash}m{2em} | >{\centering\arraybackslash}m{2em}| >{\centering\arraybackslash}m{2em}| >{\centering\arraybackslash}m{2em}|}
            \hline
            & \textbf{$Alt_1$} & \textbf{$Alt_2$}& \textbf{$Alt_3$}& \textbf{$Alt_4$}\\
            \hline\hline
            \textbf{$Alt_1$} & \cellcolor{gr_l}{1}&         1.25         &      1.43            &   -   \\
            \textbf{$Alt_2$} &          0.80      &  \cellcolor{gr_l}{1} &      1.14            &   -   \\
            \textbf{$Alt_3$} &          0.70      &         0.88         &  \cellcolor{gr_l}{1} &   -   \\
            \textbf{$Alt_4$} &          -         &          -           &       -              &   \cellcolor{gr_l}{-}  \\ 
            \hline
        \end{tabular}
        \caption{Matriz de comparación de $C_{10}$}
        \label{tab:MComC10}
    \end{minipage}
    \begin{minipage}[b]{0.5\linewidth}
        \scriptsize
        \centering
            \begin{tabular}{|>{\centering\arraybackslash}m{2em} ||>{\centering\arraybackslash}m{2em} | >{\centering\arraybackslash}m{2em}| >{\centering\arraybackslash}m{2em}| >{\centering\arraybackslash}m{2em}|>{\centering\arraybackslash}m{2em}|}
            \hline
            & \textbf{$Alt_1$} & \textbf{$Alt_2$}& \textbf{$Alt_3$}& \textbf{$Alt_4$}& \textbf{$V_{C_{10}}$}\\
            \hline\hline
            \textbf{$Alt_1$} & 0.40 &  0.40  &   0.40   &    -   &  \cellcolor{gr_l}{0.40}   \\
            \textbf{$Alt_2$} & 0.32 &  0.32  &   0.32   &    -   &  0.32  \\
            \textbf{$Alt_3$} & 0.28 &  0.28  &   0.28   &    -   &  0.28    \\
            \textbf{$Alt_4$} &   -  &   -    &    -     &    -   &    -   \\ 
            \hline
        \end{tabular}
        \caption{Matriz normalizada de $C_{10}$ y $V_{C_{10}}$}
        \label{tab:MNorm_C10}
    \end{minipage}
\end{table}
%-------------------C10 Plataforma de visualizacion de información----------------------

%-------------------C11 Control de interfaz----------------------
\begin{table}[!htbp]
    \begin{minipage}[b]{0.5\linewidth}
        \scriptsize
        \centering
            \begin{tabular}{|>{\centering\arraybackslash}m{2em} ||>{\centering\arraybackslash}m{2em} | >{\centering\arraybackslash}m{2em}| >{\centering\arraybackslash}m{2em}| >{\centering\arraybackslash}m{2em}|}
            \hline
            & \textbf{$Alt_1$} & \textbf{$Alt_2$}& \textbf{$Alt_3$}& \textbf{$Alt_4$}\\
            \hline\hline
            \textbf{$Alt_1$} & \cellcolor{gr_l}{1}&         1.14         &      0.80            &   -   \\
            \textbf{$Alt_2$} &          0.88      &  \cellcolor{gr_l}{1} &      0.70            &   -   \\
            \textbf{$Alt_3$} &          1.25      &         1.43         &  \cellcolor{gr_l}{1} &   -   \\
            \textbf{$Alt_4$} &          -         &          -           &       -              &   \cellcolor{gr_l}{-}  \\ 
            \hline
        \end{tabular}
        \caption{Matriz de comparación de $C_{11}$}
        \label{tab:MComC11}
    \end{minipage}
    \begin{minipage}[b]{0.5\linewidth}
        \scriptsize
        \centering
            \begin{tabular}{|>{\centering\arraybackslash}m{2em} ||>{\centering\arraybackslash}m{2em} | >{\centering\arraybackslash}m{2em}| >{\centering\arraybackslash}m{2em}| >{\centering\arraybackslash}m{2em}|>{\centering\arraybackslash}m{2em}|}
            \hline
            & \textbf{$Alt_1$} & \textbf{$Alt_2$}& \textbf{$Alt_3$}& \textbf{$Alt_4$}& \textbf{$V_{C_{11}}$}\\
            \hline\hline
            \textbf{$Alt_1$} & 0.32 &  0.32  &   0.32   &    -   &  0.32    \\
            \textbf{$Alt_2$} & 0.28 &  0.28  &   0.28   &    -   &  0.28   \\
            \textbf{$Alt_3$} & 0.40 &  0.40  &   0.40   &    -   &  \cellcolor{gr_l}{0.40}   \\
            \textbf{$Alt_4$} &   -  &   -    &    -     &    -   &    -   \\ 
            \hline
        \end{tabular}
        \caption{Matriz normalizada de $C_{11}$ y $V_{C_{11}}$}
        \label{tab:MNorm_C11}
    \end{minipage}
\end{table}
%-------------------C11 Control de interfaz----------------------

%-------------------C12 Encapsulado de sistemas embebidos----------------------
\begin{table}[!htbp]
    \begin{minipage}[b]{0.5\linewidth}
        \scriptsize
        \centering
            \begin{tabular}{|>{\centering\arraybackslash}m{2em} ||>{\centering\arraybackslash}m{2em} | >{\centering\arraybackslash}m{2em}| >{\centering\arraybackslash}m{2em}| >{\centering\arraybackslash}m{2em}|}
            \hline
            & \textbf{$Alt_1$} & \textbf{$Alt_2$}& \textbf{$Alt_3$}& \textbf{$Alt_4$}\\
            \hline\hline
            \textbf{$Alt_1$} & \cellcolor{gr_l}{1}&         5            &      -               &   -   \\
            \textbf{$Alt_2$} &          0.20      &  \cellcolor{gr_l}{1} &      -               &   -   \\
            \textbf{$Alt_3$} &          -         &         -            &  \cellcolor{gr_l}{-} &   -   \\
            \textbf{$Alt_4$} &          -         &          -           &       -              &   \cellcolor{gr_l}{-}  \\ 
            \hline
        \end{tabular}
        \caption{Matriz de comparación de $C_{12}$}
        \label{tab:MComC12}
    \end{minipage}
    \begin{minipage}[b]{0.5\linewidth}
        \scriptsize
        \centering
            \begin{tabular}{|>{\centering\arraybackslash}m{2em} ||>{\centering\arraybackslash}m{2em} | >{\centering\arraybackslash}m{2em}| >{\centering\arraybackslash}m{2em}| >{\centering\arraybackslash}m{2em}|>{\centering\arraybackslash}m{2em}|}
            \hline
            & \textbf{$Alt_1$} & \textbf{$Alt_2$}& \textbf{$Alt_3$}& \textbf{$Alt_4$}& \textbf{$V_{C_{12}}$}\\
            \hline\hline
            \textbf{$Alt_1$} & 0.83 &  0.83  &    -     &    -   &   \cellcolor{gr_l}{0.83}    \\
            \textbf{$Alt_2$} & 0.17 &  0.17  &   -      &    -   &  0.17   \\
            \textbf{$Alt_3$} &  -   &  -     &   -      &    -   &  -      \\
            \textbf{$Alt_4$} &   -  &   -    &    -     &    -   &    -   \\ 
            \hline
        \end{tabular}
        \caption{Matriz normalizada de $C_{12}$ y $V_{C_{12}}$}
        \label{tab:MNorm_C12}
    \end{minipage}
\end{table}
%-------------------C12 Encapsulado de sistemas embebidos----------------------

%-------------------C13 Inclusión del sistema energético en sistema embebido----------------------
\begin{table}[!htbp]
    \begin{minipage}[b]{0.5\linewidth}
        \scriptsize
        \centering
            \begin{tabular}{|>{\centering\arraybackslash}m{2em} ||>{\centering\arraybackslash}m{2em} | >{\centering\arraybackslash}m{2em}| >{\centering\arraybackslash}m{2em}| >{\centering\arraybackslash}m{2em}|}
            \hline
            & \textbf{$Alt_1$} & \textbf{$Alt_2$}& \textbf{$Alt_3$}& \textbf{$Alt_4$}\\
            \hline\hline
            \textbf{$Alt_1$} & \cellcolor{gr_l}{1}&         1.60         &      -               &   -   \\
            \textbf{$Alt_2$} &          0.63      &  \cellcolor{gr_l}{1} &      -               &   -   \\
            \textbf{$Alt_3$} &          -         &         -            &  \cellcolor{gr_l}{-} &   -   \\
            \textbf{$Alt_4$} &          -         &          -           &       -              &   \cellcolor{gr_l}{-}  \\ 
            \hline
        \end{tabular}
        \caption{Matriz de comparación de $C_{13}$}
        \label{tab:MComC13}
    \end{minipage}
    \begin{minipage}[b]{0.5\linewidth}
        \scriptsize
        \centering
            \begin{tabular}{|>{\centering\arraybackslash}m{2em} ||>{\centering\arraybackslash}m{2em} | >{\centering\arraybackslash}m{2em}| >{\centering\arraybackslash}m{2em}| >{\centering\arraybackslash}m{2em}|>{\centering\arraybackslash}m{2em}|}
            \hline
            & \textbf{$Alt_1$} & \textbf{$Alt_2$}& \textbf{$Alt_3$}& \textbf{$Alt_4$}& \textbf{$V_{C_{13}}$}\\
            \hline\hline
            \textbf{$Alt_1$} & 0.62 &  0.62  &    -     &    -   &   \cellcolor{gr_l}{0.62}    \\
            \textbf{$Alt_2$} & 0.38 &  0.38  &   -      &    -   &  0.38   \\
            \textbf{$Alt_3$} &  -   &  -     &   -      &    -   &  -      \\
            \textbf{$Alt_4$} &   -  &   -    &    -     &    -   &    -   \\ 
            \hline
        \end{tabular}
        \caption{Matriz normalizada de $C_{13}$ y $V_{C_{13}}$}
        \label{tab:MNorm_C13}
    \end{minipage}
\end{table}
%-------------------C13 Inclusión del sistema energético en sistema embebido----------------------

%-------------------C14 Anclaje al suelo----------------------
\begin{table}[!htbp]
    \begin{minipage}[b]{0.5\linewidth}
        \scriptsize
        \centering
            \begin{tabular}{|>{\centering\arraybackslash}m{2em} ||>{\centering\arraybackslash}m{2em} | >{\centering\arraybackslash}m{2em}| >{\centering\arraybackslash}m{2em}| >{\centering\arraybackslash}m{2em}|}
            \hline
            & \textbf{$Alt_1$} & \textbf{$Alt_2$}& \textbf{$Alt_3$}& \textbf{$Alt_4$}\\
            \hline\hline
            \textbf{$Alt_1$} & \cellcolor{gr_l}{1}&         1.50         &      -               &   -   \\
            \textbf{$Alt_2$} &          0.67      &  \cellcolor{gr_l}{1} &      -               &   -   \\
            \textbf{$Alt_3$} &          -         &         -            &  \cellcolor{gr_l}{-} &   -   \\
            \textbf{$Alt_4$} &          -         &          -           &       -              &   \cellcolor{gr_l}{-}  \\ 
            \hline
        \end{tabular}
        \caption{Matriz de comparación de $C_{14}$}
        \label{tab:MComC14}
    \end{minipage}
    \begin{minipage}[b]{0.5\linewidth}
        \scriptsize
        \centering
            \begin{tabular}{|>{\centering\arraybackslash}m{2em} ||>{\centering\arraybackslash}m{2em} | >{\centering\arraybackslash}m{2em}| >{\centering\arraybackslash}m{2em}| >{\centering\arraybackslash}m{2em}|>{\centering\arraybackslash}m{2em}|}
            \hline
            & \textbf{$Alt_1$} & \textbf{$Alt_2$}& \textbf{$Alt_3$}& \textbf{$Alt_4$}& \textbf{$V_{C_{14}}$}\\
            \hline\hline
            \textbf{$Alt_1$} & 0.60 &  0.60  &    -     &    -   &   \cellcolor{gr_l}{0.60}    \\
            \textbf{$Alt_2$} & 0.40 &  0.40  &   -      &    -   &  0.40   \\
            \textbf{$Alt_3$} &  -   &  -     &   -      &    -   &  -      \\
            \textbf{$Alt_4$} &   -  &   -    &    -     &    -   &    -   \\ 
            \hline
        \end{tabular}
        \caption{Matriz normalizada de $C_{14}$ y $V_{C_{14}}$}
        \label{tab:MNorm_C14}
    \end{minipage}
\end{table}
%-------------------C14 Anclaje al suelo----------------------

%-------------------C15 Método de anclaje----------------------
\begin{table}[!htbp]
    \begin{minipage}[b]{0.5\linewidth}
        \scriptsize
        \centering
            \begin{tabular}{|>{\centering\arraybackslash}m{2em} ||>{\centering\arraybackslash}m{2em} | >{\centering\arraybackslash}m{2em}| >{\centering\arraybackslash}m{2em}| >{\centering\arraybackslash}m{2em}|}
            \hline
            & \textbf{$Alt_1$} & \textbf{$Alt_2$}& \textbf{$Alt_3$}& \textbf{$Alt_4$}\\
            \hline\hline
            \textbf{$Alt_1$} & \cellcolor{gr_l}{1}  &  1.50  &    2.25   &   4.50  \\
            \textbf{$Alt_2$} & 0.67 &  \cellcolor{gr_l}{1} &   1.50   &   3  \\
            \textbf{$Alt_3$} & 0.44 &  0.67   &  \cellcolor{gr_l}{1}   &  2  \\
            \textbf{$Alt_4$} & 0.22  &   0.33   &  0.50  &   \cellcolor{gr_l}{1}  \\ 
            \hline
        \end{tabular}
        \caption{Matriz de comparación de $C_{15}$}
        \label{tab:MComC15}
    \end{minipage}
    \begin{minipage}[b]{0.5\linewidth}
        \scriptsize
        \centering
            \begin{tabular}{|>{\centering\arraybackslash}m{2em} ||>{\centering\arraybackslash}m{2em} | >{\centering\arraybackslash}m{2em}| >{\centering\arraybackslash}m{2em}| >{\centering\arraybackslash}m{2em}|>{\centering\arraybackslash}m{2em}|}
            \hline
            & \textbf{$Alt_1$} & \textbf{$Alt_2$}& \textbf{$Alt_3$}& \textbf{$Alt_4$}& \textbf{$V_{C_{15}}$}\\
            \hline\hline
            \textbf{$Alt_1$} & 0.43 &  0.43  &   0.43   &  0.43  & \cellcolor{gr_l}{0.43}   \\
            \textbf{$Alt_2$} & 0.29 &  0.29  &   0.29   &  0.29  &  0.29  \\
            \textbf{$Alt_3$} & 0.19 &  0.19  &   0.19   &  0.19  &  0.19    \\
            \textbf{$Alt_4$} & 0.10 &  0.10  &   0.10   &  0.10  &  0.10   \\ 
            \hline
        \end{tabular}
        \caption{Matriz normalizada de $C_{15}$ y $V_{C_{15}}$}
        \label{tab:MNorm_C15}
    \end{minipage}
\end{table}
%-------------------C15 Método de anclaje----------------------

%-------------------C16 Método de comunicación----------------------

\begin{table}[!htbp]
    \begin{minipage}[b]{0.5\linewidth}
        \scriptsize
        \centering
            \begin{tabular}{|>{\centering\arraybackslash}m{2em} ||>{\centering\arraybackslash}m{2em} | >{\centering\arraybackslash}m{2em}| >{\centering\arraybackslash}m{2em}| >{\centering\arraybackslash}m{2em}|}
            \hline
            & \textbf{$Alt_1$} & \textbf{$Alt_2$}& \textbf{$Alt_3$}& \textbf{$Alt_4$}\\
            \hline\hline
            \textbf{$Alt_1$} & \cellcolor{gr_l}{1}&         0.78         &      0.88            &   -   \\
            \textbf{$Alt_2$} &          1.29      &  \cellcolor{gr_l}{1} &      1.13            &   -   \\
            \textbf{$Alt_3$} &          1.14      &         0.89         &  \cellcolor{gr_l}{1} &   -   \\
            \textbf{$Alt_4$} &          -         &          -           &       -              &   \cellcolor{gr_l}{-}  \\ 
            \hline
        \end{tabular}
        \caption{Matriz de comparación de $C_{16}$}
        \label{tab:MComC16}
    \end{minipage}
    \begin{minipage}[b]{0.5\linewidth}
        \scriptsize
        \centering
            \begin{tabular}{|>{\centering\arraybackslash}m{2em} ||>{\centering\arraybackslash}m{2em} | >{\centering\arraybackslash}m{2em}| >{\centering\arraybackslash}m{2em}| >{\centering\arraybackslash}m{2em}|>{\centering\arraybackslash}m{2em}|}
            \hline
            & \textbf{$Alt_1$} & \textbf{$Alt_2$}& \textbf{$Alt_3$}& \textbf{$Alt_4$}& \textbf{$V_{C_{16}}$}\\
            \hline\hline
            \textbf{$Alt_1$} & 0.29 &  0.29  &   0.29   &    -   &  0.29    \\
            \textbf{$Alt_2$} & 0.38 &  0.38  &   0.38   &    -   &  \cellcolor{gr_l}{0.38} \\
            \textbf{$Alt_3$} & 0.33 &  0.33  &   0.33   &    -   &  0.33   \\
            \textbf{$Alt_4$} &   -  &   -    &    -     &    -   &    -   \\ 
            \hline
        \end{tabular}
        \caption{Matriz normalizada de $C_{16}$ y $V_{C_{16}}$}
        \label{tab:MNorm_C16}
    \end{minipage}
\end{table}
\FloatBarrier
%-------------------C16 Método de comunicación----------------------
Una vez que se cuenta con los valores del cumplimiento de cada alternativa de solución con respecto a cada criterio, se utiliza en conjunto con el vector de prioridad para obtener la puntuación de cada alternativa de solución. La alternativa elegida sera la de mayor valor.
%------------------------------Selección----------------------------------
\begin{center}
\footnotesize
\centering
    \begin{longtable}[!htb]{|>{\centering\arraybackslash}m{3em} ||>{\centering\arraybackslash}m{3em} | >{\centering\arraybackslash}m{3em}| >{\centering\arraybackslash}m{3em}| >{\centering\arraybackslash}m{3em}|>{\centering\arraybackslash}m{3em}|}
    \hline
    & \textbf{$CS_1$} & \textbf{$CS_2$}& \textbf{$CS_3$}& \textbf{$CS_4$} & \textbf{$V_{Cr}$}\\
    \hline\hline
    $V_{Cr_{1}}$ &  0.56 & 0.06 & 0.11 & 0.28 & 0.133\\
    \hline
    $V_{Cr_{2}}$ &  0.14 & 0.36 & 0.36 & 0.14 & 0.064\\
    \hline
    $V_{Cr_{3}}$ &  0.53 & 0.20 & 0.13 & 0.13 & 0.010\\
    \hline
    $V_{Cr_{4}}$ &  0.35 & 0.30 & 0.25 & 0.10 & 0.038\\
    \hline
    $V_{Cr_{5}}$ & 0.31 & 0.35 & 0.27 & 0.08 & 0.038\\
    \hline
    $V_{Cr_{6}}$ & 0.27 & 0.30 & 0.30 & 0.12 & 0.013\\
    \hline
    $V_{Cr_{7}}$ & 0.45 & 0.09 & 0.14 & 0.32 & 0.038\\
    \hline
    $V_{Cr_{8}}$ & 0.50 & 0.17 & 0.22 & 0.11 & 0.064\\
    \hline
    $V_{Cr_{9}}$ & 0.22 & 0.22 & 0.25 & 0.31 & 0.128\\
    \hline
    $V_{Cr_{10}}$ & 0.47 & 0.05 & 0.11 & 0.37 & 0.064\\
    \hline
    $V_{Cr_{11}}$ & 0.31 & 0.19 & 0.15 & 0.35 & 0.038\\
    \hline
    $V_{Cr_{12}}$ & 0.56 & 0.11 & 0.11 & 0.22 & 0.026\\
    \hline
    $V_{Cr_{13}}$ & 0.43 & 0.21 & 0.14 & 0.21 & 0.064\\
    \hline
    $V_{Cr_{14}}$ & 0.42 & 0.04 & 0.21 & 0.33 & 0.026\\
    \hline
    $V_{Cr_{21}}$ & 0.33 & 0.17 & 0.17 & 0.33 & 0.102\\
    \hline
    $V_{Cr_{22}}$ & 0.44 & 0.06 & 0.06 & 0.44 & 0.026\\
    \hline
    $V_{Cr_{24}}$ & 0.48 & 0.05 & 0.10 & 0.38 & 0.013\\
    \hline
    $V_{Cr_{25}}$ & 0.42 & 0.11 & 0.16 & 0.32 & 0.064\\
    \hline
    $V_{Cr_{27}}$ & 0.18 & 0.32 & 0.32 & 0.18 & 0.051\\
    \hline
    \textbf{Total} & \cellcolor{gr_l}{0.376} & 0.174 & 0.191 & 0.259 & 1.000\\
    \hline
    
    \caption{Matriz de decisión}
    \label{tab:MatrizDesicion}
    \end{longtable}
\end{center}
\FloatBarrier
%------------------------------Selección----------------------------------
Por ultimo, la siguiente tabla hace una recopilación de la alternativa de solución elegida en conjunto con las alternativas independientes que fueron elegidas. 
%------------------------------Elegido----------------------------------
\begin{center}
\footnotesize
\centering
    \begin{longtable}[!htbp]{|>{\centering\arraybackslash}m{20em} |>{\centering\arraybackslash}m{20em} |}
    \hline
    \textbf{Características} & \textbf{Concepto solución}\\
    \hline\hline
    Forma de movimiento de los paneles & Rotacional en base prismática \\
    \hline
    Tipo de actuación & Con motor y acopladores magnéticos \\
    \hline
    Método para la reflexión & Paneles reflectivos  \\
    \hline
    Sensado de las posiciones de los paneles & Encoders, sensores de contacto o visión artificial  \\
    \hline
    Posicionamiento de paneles & Anclados a la base prismática \\
    \hline
    Sistema de alimentación de paneles & No \\
    \hline
    Grabación en varias posiciones & No \\
    \hline
    Generación de ondas de sonido & Bocina externa \\
    \hline
    Dispositivo de grabación & Micrófono externo \\
    \hline
    Plataforma para el cálculo de las posiciones deseadas & Microcontrolador embebido \\
    \hline
    Plataforma de procesado del control &  Microcontrolador embebido\\
    \hline
    Plataforma de procesado de la acústica & Microcontrolador embebido \\
    \hline
    Sensado de las posiciones de los paneles & Encoders \\
    \hline
    Plataforma de visualización de información &  Equipo de cómputo\\
    \hline
    Control de interfaz & Ratón y teclado \\
    \hline
    Encapsulado de sistemas embebidos &  Si\\
    \hline
    Inclusión del sistema energético en sistema embebido & Si \\
    \hline
    Anclaje al suelo & Si \\
    \hline
    Método de anclaje &  Tornillos \\
    \hline
    Método de comunicación & Wi-fi \\
    \hline    
    \caption{Características del concepto solución elegido}
    \label{tab:ConceptoElegido}
    \end{longtable}
\end{center}
\FloatBarrier

Se desarrollo el diseño del concepto solución elegido en un software CAD, es a partir de esta propuesta inicial que se comenzaran a desarrollar los cálculos y validaciones para el diseño final.
\begin{figure}[!htb]
    \centering
    \includegraphics[width=1\textwidth]{imagenes/ConceptoSolucion.jpg}
    \caption{\footnotesize Concepto solución elegido}
    \label{fig:ConceptoSolucionElegido}
\end{figure}
\FloatBarrier
%------------------------------Elegido----------------------------------

\subsection{Diseño detallado}
\subsubsection{Software}
\subsubsection{MF2. Modulo de generación y medición de la acústica}

Para este módulo, se hizo uso del entorno de MATLAB. El programa consiste en un dispositivo de reproducción y grabación simultanea, que envían una señal de excitación hacia el cuarto por medio de un altavoz, mientras un micrófono recoge tanto el sonido del altavoz como las reflexiones de las superficies del recinto. Posteriormente, el programa se encarga de utilizar la señal de excitación y la capturada, para estimar la respuesta al impulso del estudio.\hfill \break
Dentro del \textit{Audio Toolbox} de MATLAB (\href{https://www.mathworks.com/products/audio.html}{Liga al \textit{Toolbox}}) existe  el objeto \textit{AudioPlayerRecorder}, que nos permite escribir y leer paquetes de audio al dispositivo de la computadora. \hfill \break
Las primeras lineas del código están referidas a la inicialización y configuración del dispositivo de lectura y escritura.\hfill \break
\begin{lstlisting}[frame=single,numbers=left, style=Matlab-editor, basicstyle=\tiny]
clc; clear; close all;

fs = 44100;
room_dim = [6, 5, 3.5];
device = "ASIO4ALL v2";
aPR = audioPlayerRecorder("SampleRate",fs,"Device",device);
\end{lstlisting}
El dispositivo ``ASIO4ALL v2" \href{https://asio4all.org/}{(Liga a la pagina)} es un \textit{driver} para Windows, que nos permite utilizar los dispositivos de audio conectados a la computadora como tarjetas ASIO (\textit{Audio Stream Input-Output}) con la posibilidad de combinar dispositivos de entrada y salida. \hfill\break
Existen distintos métodos para medir la respuesta al impulso de un recinto, sin embargo, por la poca calibración que necesita para obtener resultados óptimos, además de que permite obtener un excelente ratio entre señal y sonido, el mejor método es el barrido senoidal. \href{https://people.montefiore.uliege.be/stan/ArticleJAES.pdf}{cita} \hfill\break
El barrido senoidal consiste en una señal senoidal, que aumenta de frecuencia conforme avanza el tiempo. El dispositivo ASIO emite la señal mientras va leyendo la respuesta del recinto. Se puede estimar la respuesta al impulso al realizar una deconvolucion entre la señal emitida y la capturada. El siguiente paso en el código es definir los parámetros del barrido senoidal, como la amplitud, rango y duración.
\begin{lstlisting}[frame=single,numbers=left, style=Matlab-editor, basicstyle=\tiny]
sweep_dur = 5;
duration_per_run = 8;
start_silence = 1;
silence_dur = duration_per_run - sweep_dur - start_silence;
sweep_range = [10 8000];
percentage = 100;
outputLevel = 20*log10(percentage/100);
sweepsine = sweeptone(sweep_dur,silence_dur,fs,"SweepFrequencyRange", ...
    sweep_range,"ExcitationLevel",outputLevel);
\end{lstlisting}
Para asegurar la simultaneidad de las señales de entrada es salida, es necesario utilizar \textit{buffers} que se encargaran de gestionar el flujo de datos. Además, se incluyen tiempos de silencio al antes y después de la señal, que permita compensar por pequeños desfases.
\begin{lstlisting}[frame=single,numbers=left, style=Matlab-editor, basicstyle=\tiny]
excitation = [zeros(start_silence*fs,1); sweepsine];
sampling = 1024;
excitation_length = length(excitation);
bufOut = dsp.AsyncBuffer(excitation_length);
bufIn = dsp.AsyncBuffer(excitation_length);
write(bufOut,excitation);
\end{lstlisting}
Se leerán y escribirán datos a la tarjeta de audio mediante paquetes de 1024 muestras, hasta que se acaben los datos del \textit{buffer}.
\begin{lstlisting}[frame=single,numbers=left, style=Matlab-editor, basicstyle=\tiny]
while bufOut.NumUnreadSamples > 0
    exci = read(bufOut,sampling);
    [rec,num_under,num_over] = aPR(exci);
    write(bufIn,rec);
    if num_under>0 || num_over>0
        fprintf("Underrun by %d frames, overrun by %d frames.\n",num_under,num_over)
    end
end
release(aPR);
read(bufIn,start_silence*fs);
audioFromDevice = read(bufIn);
\end{lstlisting}

Por ultimo, se hace el proceso de deconvolucion mediante la funcion de matlab \textit{impzest}, se normaliza para tener amplitud de 1 y se crea el vector de tiempo. Los datos de la respuesta al impulse se guardan en una estructura junto con el tiempo de muestreo $f_{s}$.

\begin{lstlisting}[frame=single,numbers=left, style=Matlab-editor, basicstyle=\tiny]
time = (1:length(audioFromDevice))/fs;
RIR = impzest(sweepsine,audioFromDevice);
[max_RIR,Idx] = max(abs(RIR));
n_RIR = RIR(Idx-round(fs/100):end)/max_RIR;
time_RIR = (1:length(n_RIR))/fs;
ImpulseResponse.fs = fs;
ImpulseResponse.y = n_RIR;

plot(time,audioFromDevice)
title('Audio recorded')
xlabel('Time(s)')
ylabel('Amplitude')

plot(time_RIR, n_RIR)
title('Estimated impulse response')
xlabel('Time(s)')
ylabel('Amplitude')
\end{lstlisting}

Las gráficas resultantes, muestran la señal de excitación, la grabación y la estimación de la respuesta al impulso.
\begin{figure}[!htb]
    \centering
     \begin{subfigure}{0.3\textwidth}
        \centering
        \includegraphics[width=\linewidth]{imagenes/ExcitationSignal_RIR_Measurement.jpg}
        \caption{\footnotesize Señal de excitación (Barrido senoidal)}
        \label{fig:sub1}
    \end{subfigure}
    \hfill
    \begin{subfigure}{0.3\textwidth}
        \centering
        \includegraphics[width=\linewidth]{imagenes/AudioFromDevice_RIR_Measurement.png}
        \caption{\footnotesize Audio grabado}
        \label{fig:sub2}
    \end{subfigure}
    \hfill
    \begin{subfigure}{0.3\textwidth}
        \centering
        \includegraphics[width=\linewidth]{imagenes/EstimatedImpulseResponse_RIR_Measurement.png}
        \caption{\footnotesize Respuesta al impulso estimada}
        \label{fig:sub2}
    \end{subfigure}
    \caption{Gráficas resultantes de generación y medición de la acústica}
\end{figure}
\FloatBarrier

El programa completo se encuentra como () en el Anexo ().

\subsubsection{MF1.1. Procesamiento de la respuesta y calculo de la acústica}
A partir de la respuesta al impulso de un recinto, se pueden derivar distintos parámetros acústicos, como los establecidos en la \href{https://www.iso.org/standard/36201.html}{ISO-3382}. Siendo que la respuesta al impulso la podemos obtener de mediciones físicas, de simulaciones del recinto, e incluso de bancos de datos con respuestas de múltiples recintos para fines de validación, fue necesario tener una metodología general que nos permitiera analizar datos de diferentes fuentes. \hfill\break
El programa de generación y medición de la acústica, tiene como salida una estructura con los datos de la respuesta al impulso y el tiempo de muestreo; para las simulaciones y los bancos de datos, las respuestas al impulso se guardan como un archivo de tipo \textit{wav}. Se iniciara con una linea de lectura de archivos de este tipo, que se desactivara en caso de usarse consecutivo al programa de generación y medición de la acústica. Dicho código también se encargara de combinar las señales en caso de leer archivos \textit{wav} con múltiples canales.
\begin{lstlisting}[frame=single,numbers=left, style=Matlab-editor, basicstyle=\tiny]
[temp_y,ImpulseResponse.fs] = audioread(['C:\Users\User\Desktop\RIR_Database\' ...
    '382908__uzbazur__little-basement-ir-impulse-response.wav']);
if size(temp_y,2) > 1
    ImpulseResponse.y = mean(temp_y,2);
end
\end{lstlisting}
De acuerdo con la norma ISO-3382, los parámetros acústicos que caracterizan un cuarto son:
\begin{itemize}
    \item $EDT$, tiempo de decaimiento temprano
    \item $T_{20}$ tiempo de reverberación referido a -20 Db
    \item $T_{30}$ tiempo de reverberación referido a -30 Db
    \item $C_{50}$ claridad a 50 ms
    \item $C_{80}$ claridad a 80 ms
    \item $D_{50}$ ratio de energía útil
    \item $G$ fuerza del sonido 
\end{itemize}
Todos los parámetros anteriores, deben calcularse en las frecuencias centrales de las bandas de frecuencia pertenecientes octavas de 125 Hz a 4000 Hz. Por esto, primero se deben pasar por filtros pasa-bandas, distribuidos a lo largo de las octavas.
Se paso la respuesta al impulso por 6 filtros, con frecuencias centrales y rangos como se muestra a continuación \href{https://www.doctorproaudio.com/content.php?2402-octaves-and-third-octaves#:~:text=As%20a%20reference%2C%20the%20central,exactly%20to%20double%20or%20half.}{Fuente}:
\begin{center}
    \begin{longtable}[!htb]{ |c|c|c| }
    \hline
    Frecuencia central & Limite inferior & Limite superior \\ 
    \hline
    125 Hz & 88.39 Hz & 176.8 Hz   \\  
    250 Hz & 176.8 Hz & 353,6 Hz   \\  
    500 Hz & 353.6 Hz & 707.1 Hz   \\  
    1000 Hz & 707.1 Hz & 1414 Hz   \\  
    2000 Hz & 1414 Hz & 2828 Hz     \\  
    4000 Hz & 2828 Hz & 5657 HZ   \\
    \hline
    \caption{Rangos de frecuencias para filtros}
    \label{tab:Rangos De Frecuencias}
    \end{longtable} 
\end{center}
\begin{lstlisting}[frame=single,numbers=left, style=Matlab-editor, basicstyle=\tiny]
RIR = struct('General',ImpulseResponse); 
RIR.f125.y = bandpass(ImpulseResponse.y,[88.39 176.8],ImpulseResponse.fs);
RIR.f125.fs = ImpulseResponse.fs;
RIR.f250.y = bandpass(ImpulseResponse.y,[176.8 353.6],ImpulseResponse.fs);
RIR.f250.fs = ImpulseResponse.fs;
RIR.f500.y = bandpass(ImpulseResponse.y,[353.6 707.1],ImpulseResponse.fs);
RIR.f500.fs = ImpulseResponse.fs;
RIR.f1k.y = bandpass(ImpulseResponse.y,[707.1 1414],ImpulseResponse.fs);
RIR.f1k.fs = ImpulseResponse.fs;
RIR.f2k.y = bandpass(ImpulseResponse.y,[1414 2828],ImpulseResponse.fs);
RIR.f2k.fs = ImpulseResponse.fs;
RIR.f4k.y = bandpass(ImpulseResponse.y,[2828 5657],ImpulseResponse.fs);
RIR.f4k.fs = ImpulseResponse.fs;
\end{lstlisting}
Se pueden observar algunas la respuesta del recinto para ciertas frecuencias.
\begin{figure}[!htb]
    \centering
     \begin{subfigure}{0.3\textwidth}
        \centering
        \includegraphics[width=\linewidth]{imagenes/RIR_500Hz_RIR_Measurement.jpg}
        \caption{\footnotesize Respuesta al impulso a 500 Hz}
        \label{fig:sub1}
    \end{subfigure}
    \hfill
    \begin{subfigure}{0.3\textwidth}
        \centering
        \includegraphics[width=\linewidth]{imagenes/RIR_1000Hz_RIR_Measurement.jpg}
        \caption{\footnotesize Respuesta al impulso a 1000 Hz}
        \label{fig:sub2}
    \end{subfigure}
    \hfill
    \begin{subfigure}{0.3\textwidth}
        \centering
        \includegraphics[width=\linewidth]{imagenes/RIR_2000Hz_RIR_Measurement.jpg}
        \caption{\footnotesize Respuesta al impulso a 2000 Hz}
        \label{fig:sub2}
    \end{subfigure}
    \caption{Respuesta al impulso para diferentes frecuencias}
\end{figure}
\FloatBarrier

A continuación podemos hacer el análisis de la respuesta al impulso en sus diferentes frecuencias, además de la general, lo que nos dará los parámetros acústicos actuales del recinto. Este proceso se hace mediante otro \textit{script} llamado \textbf{$RIR\_Analysis$}.\hfill\break
La función toma como primera entrada el método de empaquetado de la señal, se debe ingresar un 1 en caso de que se quiera analizar un archivo de tipo \textit{wav}, y cualquier otro numero en caso de que la señal ya venga empaquetada en una estructura con el tiempo de muestreo y la amplitud. A continuación se introduce la señal correspondiente, además de las dimensiones del cuarto en el que se grabo la respuesta al impulso y una variable booleana que indica al programa si debe mostrar la gráfica de decaimiento de la energía junto con los ajustes lineales y los parámetros.
La función retorna la respuesta al impulso cuadrada, la gráfica de decaimiento, el vector de tiempo correspondiente y una estructura con los parámetros acústicos.

\begin{lstlisting}[frame=single,numbers=left, style=Matlab-editor, basicstyle=\tiny]
function [RIRsq,DC,t,AcousticParams] = RIR_Analisys(method, input, room_dimensions,printFlag)
\end{lstlisting}
La gráfica del decaimiento se obtiene mediante la integral de Schroeder, la cual es una integración hacia atrás, que se hace desde un punto donde la señal ya solo contiene ruido de fondo. \href{https://www.roomeqwizard.com/help/help_en-GB/html/graph_filteredir.html}{Fuentes}
\begin{lstlisting}[frame=single,numbers=left, style=Matlab-editor, basicstyle=\tiny]
    RIRsq = y.^2;
    schroeder_cumsum = cumsum(flipud(RIRsq));
    schroeder_normalized = schroeder_cumsum / max(schroeder_cumsum);
    DC = flipud(10*log10(schroeder_normalized));
\end{lstlisting}
La gráfica se muestra en decibelios y va de 0 a $-\infty$. \hline\break
La curva de decaimiento de la energía nos permite ver el decaimiento exponencial del sonido como un decaimiento lineal, y así, poder ajustar rectas que se apeguen mas al decaimiento de la señal sin verse interferida por el ruido de fondo. Adicionalmente, podemos usar esta curva para calcular los otros parámetros acústicos como $D_{50}$ o $C_{50}$.\hfill\break
\begin{figure}[!htb]
    \centering
    \includegraphics[width=\linewidth]{imagenes/DecayCurve_RIR_Analysis.jpg}
    \caption{\footnotesize Curva de decaimiento}
    \label{fig:DecayCurve}
\end{figure}
\FloatBarrier
El comportamiento de la curva deja de ser lineal cuando la señal comienza a verse afectada por el ruido de fondo. Idealmente, el ruido de fondo debería presentarse como un valle horizontal, y la señal con un decaimiento lo mas recto posible, hasta curvarse al final para convertirse en el valle del ruido de fondo. \hfill\break
El análisis de esta curva revela la importancia de tener una excitación lo bastante fuerte como para superar al ruido de fondo y ver el decaimiento de la curva por mas tiempo antes de comenzar a perder información. Es por esto también que, a pesar de que el tiempo de reverberación esta referido a -60 Db, no suele medirse el decaimiento de esta manera, si no que se recurre a medir un decaimiento de -20 Db o -30 Db y se hace una extrapolación hasta los -60 Db. La norma ISO-3382 recomienda comenzar a medir el decaimiento hasta los -5 Db, para evitar interferencias de armónicos y entonces disminuir 20 o 30 Db. Intentar observar el decaimiento hasta -65 Db, implica que para un ruido de fondo de unos 30 Db, requeriríamos excitar el cuarto con una señal de al menos 95 Db, lo cual escapa de las posibilidades dentro de un recinto relativamente pequeño.\hfill\break
Posteriormente, se ajustan rectas tomando en cuenta los intervalos de -20 Db, -30 Db y del EDT.
\begin{lstlisting}[frame=single,numbers=left, style=Matlab-editor, basicstyle=\tiny]
    [~, EDT_Idx1] = min(abs(LF_EDT+10));
    [~, EDT_Idx2] = min(abs(LF_EDT));
    T60delEDT = 6*(t(EDT_Idx1)-t(EDT_Idx2));
    
    [~, EDT_Idx1] = min(abs(LF_T20+20));
    [~, EDT_Idx2] = min(abs(LF_T20));
    T60delT20 = 3*(t(EDT_Idx1)-t(EDT_Idx2));
    
    [~, EDT_Idx1] = min(abs(LF_T30+30));
    [~, EDT_Idx2] = min(abs(LF_T30));
    T60delT30 = 2*(t(EDT_Idx1)-t(EDT_Idx2));
\end{lstlisting}
La relación de energía útil ($D$) se define como la relación entre la energía total y la energía de la señal con arribo temprano. Se utilizan $D_{50}$ y $D_{80}$ para definir este parámetro usando 50 y 80 ms respectivamente como el limite para el arribo temprano. Cabe resaltar que este tiempo se toma a partir del momento que llega la señal directa (en linea recta de la fuente al receptor).
\begin{lstlisting}[frame=single,numbers=left, style=Matlab-editor, basicstyle=\tiny]
    [~,maxIdx] = max(RIRsq);
    RIRsq_trunc = RIRsq(maxIdx:end);
    Energy0_50 = trapz(RIRsq_trunc(1:round(0.05*fs)));
    Energy0_80 = trapz(RIRsq_trunc(1:round(0.08*fs)));
    Energy0_end = trapz(RIRsq_trunc(1:end));
    
    D50 = Energy0_50/Energy0_end;
    D80 = Energy0_80/Energy0_end;
\end{lstlisting}
Los índices de claridad $C_{50}$ y $C_{80}$ siguen la misma definición que el índice $D$, pero, en lugar de representar un ratio o porcentaje, estos se presentan en Db como escala logarítmica. El índice $C_{50}$ se puede entender como la claridad del habla y $C_{80}$ como la claridad de la musica. Ambos se pueden calcular a partir del índice $D$ como:
$$C_{50/80} = 10 log \left( \frac{D_{50/80}}{1-D_{50/80}} \right)$$

\begin{lstlisting}[frame=single,numbers=left, style=Matlab-editor, basicstyle=\tiny]
    C50 = 10*log(D50/(1-D50));
    C80 = 10*log(D80/(1-D80));
\end{lstlisting}

Por ultimo, la fuerza del sonido $G$ se define como la relación entre el nivel de presión del sonido en el recinto y el nivel de presión del mismo sonido en un campo libre, es decir, sin reflexiones, únicamente el sonido directo. \hfill\break
La medición de este parámetro es un poco mas complicada, ya que se requiere aislar de todas las reflexiones, al pico en la amplitud producido por el sonido directo. Por esto, se decidió hacer una estimación validada por \cite{Rossing2007}, que relaciona la fuerza del sonido con el tiempo de reverberación en un recinto y sus dimensiones.
$$G = 10 log_{10}\left(\frac{RT_{60}}{V}\right)$$

El ultimo análisis que se hace a la respuesta al impulso es respecto a los modos de vibración. Un recinto con paredes ortogonales, presenta distancias entre paredes que es un múltiplo de la longitud de onda de distintas frecuencias. Lo anterior ocasiona que las ondas oscilen entre paredes como si estuvieran ancladas a las paredes. Cuando el sonido es persistente o el tiempo de reverberación alto, un oyente puede encontrar que existen zonas del cuarto donde el sonido es muy fuerte y otras donde el sonido es muy bajo y se puede medir el cambio en el volumen si se desplaza por el recinto. \hfill\break
Los modos de vibración pueden ser axiales, tangenciales y oblicuos, dependiendo del numero de superficies en las que rebota. Los modos de vibración axiales suelen ser los mas significativos y por suerte, son los mas fáciles de predecir. \hfill\break
Podemos calcular los modos de vibración calculando las frecuencias en donde las dimensiones del cuarto son múltiplos de sus respectivas longitudes de onda. El programa cuenta con un \textit{script} externo llamado \textbf{CalcRoomModes} que realiza esta operación. El programa retorna los primeros n modos axiales de vibración.
\begin{lstlisting}[frame=single,numbers=left, style=Matlab-editor, basicstyle=\tiny]
    c = 343; % m/s
    f_length = c/room_dim(1);
    f_width = c/room_dim(2);
    f_depth = c/room_dim(3);
    modes_length = zeros(1,num_modes);
    modes_width = zeros(1,num_modes);
    modes_depth = zeros(1,num_modes);

    for i = 1:num_modes
        modes_length(i) = f_length*(0.5*i); 
        modes_width(i) = f_width*(0.5*i);
        modes_depth(i) = f_depth*(0.5*i);
    end
\end{lstlisting}
Adicionalmente, el programa se encarga de calcular la frecuencia de Schroeder, la cual nos indica la frecuencia a la que el sonido dentro de un recinto deja de estar dominado por los modos de vibración y pasa a comportarse como un campo difuso. Se calcula con la siguiente formula:
$$f_s = 2000\sqrt{\frac{RT_{60}}{V}}$$
El calculo de este parámetro es importante, debido a que el resto de parámetros acústicos solo son validos si el sonido presente en el recinto se comporta mayormente como un campo difuso. Como se ve en la formula, la variable libre de la que depende esta frecuencia es el tiempo de reverberación; esto nos ayudara a introducir cotas para el tiempo de reverberación que nos asegure no tener modos de vibración significativos durante la reproducción de musica en el estudio.
La ecuación de Sabine (cita) relaciona el tiempo de reverberación con la absorción de las superficies del cuarto y su geometría. 
\begin{displaymath}
    RT_{60} = \frac{0.161 V}{S \alpha}
\end{displaymath}\
Podemos utilizar esta ecuación para tener una relación entre el tiempo de reverberación del cuarto y la superficie de absorción. Debido a que conocemos las absorciones tanto de los paneles como de las paredes, eso deja a la superficie como parámetro libre. Entonces, el control acústico consistirá en encontrar la cantidad de absorción especifica para alcanzar un tiempo de reverberación deseado, y con ella, obtener la cantidad de superficie de paneles acústicos que debemos añadir en el cuarto para alcanzar dicho tiempo de reverberación.
\begin{lstlisting}[frame=single,numbers=left, style=Matlab-editor, basicstyle=\tiny]
    V = prod(room_dim);
    Current_abs = 0.161*V/AcousticParams.General.T60delT20;
    DesiredT60 = 0.27
    Needed_abs = 0.161*V/DesiredT60;
    Offset_abs = Needed_abs-Current_abs;
    MaterialAbs.Abs_Panels = 0.8;
    MaterialAbs.Wall = 0.02;
    Surface = Offset_abs/(MaterialAbs.Abs_Panels-MaterialAbs.Wall)
\end{lstlisting}

%--------------------------------------------------------------------
\paragraph{RayTracingImpulseResponse}\hfill \break
En este programa se simula la respuesta al impulso en una sala con el objetivo de modelar las propiedades reverberantes de un espacio sin tener que realizar mediciones acústicas. La metodología usada presente en \href{http://publications.rwth-aachen.de/record/50580/files/3875.pdf}{Real-Time Auralization}, consiste en tratar el sonido como rayos que viajan en un recinto, los cuales se emiten de manera aleatoria y uniforme, y chocan contra diferentes superficies, perdiendo energía en el proceso. Además, cuando los rayos chocan con una superficie esta refleja los rayos de manera difusa, por lo que una parte de la energía termina en el receptor. La simulación consiste en seguir el camino de los rayos para observar como pierden energía y calcular cuanta energía llega al receptor en cada reflexión, esto para construir un histograma dependiente de la frecuencia y después, obtener la respuesta al impulso pesando un proceso aleatorio de Poisson con el histograma. \\
El programa esta basado en una implementación derivada de \href{https://www.mathworks.com/help/audio/ug/room-impulse-response-simulation-with-stochastic-ray-tracing.html}{\textit{Room Impulse Response Simulation with Stochastic Ray Tracing}}, la cual no nos permite tener caracteristicas localizadas, sino que las toma uniformes en toda la pared, lo cual entre en conflicto con la colocación de paneles acústicos. \\
Para empezar se colocan las dimensiones de la sala, para posteriormente colocar las posiciones del emisor y del receptor y colocar el radio del micrófono, el cual en este caso es de 8.75cm.\\
\begin{figure}[!htb]
    \centering
    \includegraphics[width=\linewidth]{imagenes/plotRoom.jpg}
    \caption{\footnotesize Geometría del cuarto simulado y posiciones del emisor y receptor}
    \label{fig:plotRoom}
\end{figure}
\FloatBarrier
En la siguiente sección se generan los rayos, los cuales son emanados de la fuente en direcciones aleatorias. Para generar los rayos se utiliza la función RandSampleSphere, dichos rayos son una matriz N por 3 y cada fila de rayos mantiene la dirección del vector de rayos tridimensional.\\
Posteriormente en el código se definen los coeficientes de reflexión y dispersión. Un rayo de sonido se refleja cuando incide sobre una superficie. La reflexión es una combinación de un componente especular y un componente difuso. El coeficiente de absorción es una medida de cuánto sonido se absorbe (en lugar de reflejarse) al golpear una superficie, mientras que el coeficiente de difusión indica que tan especular o difusa es la reflexión.\\
Debido a que los parámetros acústicos se calculan para diferentes bandas de frecuencia, se tiene que hacer un análisis en diferentes frecuencias, dadas por FVect.
\begin{lstlisting}[frame=single,numbers=left, style=Matlab-editor, basicstyle=\tiny]
clear; close all; clc;

%% SetUp
SetUpStruct.room = [10 8 4];
SetUpStruct.src_pos = [2 2 2];
SetUpStruct.mic_pos = [5 5 1.8];
SetUpStruct.mic_radius = 0.0875;
impResTime = 10;

plotRoom(SetUpStruct.room,SetUpStruct.mic_pos,SetUpStruct.src_pos,1)

%% Generate Rays
N = 5000;
rng(0)
rays = RandSampleSphere(N);

%% Reflections and Scattering Coefficients
FVect = [125 250 500 1000 2000 4000];

abs_coeffs = [];
abs_coeffs(:,1) = [0.02,0.02,0.03,0.03,0.04,0.05,0.05]; %Concrete
abs_coeffs(:,2) = [0.70,0.45,0.65,0.60,0.75,0.65,0.65]; %AbsPanels
abs_coeffs(:,3) = [0.14,0.10,0.06,0.08,0.10,0.10,0.10]; %Door

scatt_coeffs = [];
scatt_coeffs(:,1) = [0.30,0.50,0.60,0.60,0.70,0.70,0.70];
scatt_coeffs(:,2) = [0.30,0.50,0.60,0.60,0.70,0.70,0.70];
scatt_coeffs(:,3) =  [0.30,0.50,0.60,0.60,0.70,0.70,0.70];
\end{lstlisting}
A continuación se hace la modificación al programa, creando un mapa de cada una de las paredes, que contiene valores diferentes de absorción y difusión para cada centímetro cuadrado de la pared. 
\begin{lstlisting}[frame=single,numbers=left, style=Matlab-editor, basicstyle=\tiny]
abs_map = {};
abs_map{1} = zeros(round(room_dim(2)*100),round(room_dim(3)*100),num_FBands);
abs_map{2} = zeros(round(room_dim(2)*100),round(room_dim(3)*100),num_FBands);
abs_map{3} = zeros(round(room_dim(1)*100),round(room_dim(3)*100),num_FBands);
abs_map{4} = zeros(round(room_dim(1)*100),round(room_dim(3)*100),num_FBands);
abs_map{5} = zeros(round(room_dim(1)*100),round(room_dim(2)*100),num_FBands);
abs_map{6} = zeros(round(room_dim(1)*100),round(room_dim(2)*100),num_FBands);

scatt_map = {};
scatt_map{1} = zeros(round(room_dim(2)*100),round(room_dim(3)*100),num_FBands);
scatt_map{2} = zeros(round(room_dim(2)*100),round(room_dim(3)*100),num_FBands);
scatt_map{3} = zeros(round(room_dim(1)*100),round(room_dim(3)*100),num_FBands);
scatt_map{4} = zeros(round(room_dim(1)*100),round(room_dim(3)*100),num_FBands);
scatt_map{5} = zeros(round(room_dim(1)*100),round(room_dim(2)*100),num_FBands);
scatt_map{6} = zeros(round(room_dim(1)*100),round(room_dim(2)*100),num_FBands);
\end{lstlisting}
A continuación, la función \textit{updateAbsScattCoefs} nos ayuda a actualizar los mapas de absorción y difusión, colocando los valores de absorción y difusión correspondientes a cada banda de frecuencia, en las índices del mapa correspondientes a sus posiciones en la pared.
\begin{lstlisting}[frame=single,numbers=left, style=Matlab-editor, basicstyle=\tiny]
[abs_map, scatt_map] = updateAbsScattCoeffs(abs_map,scatt_map,1,abs_coeffs(:,1),...
    scatt_coeffs(:,1),[0,0],[room_dim(2),room_dim(3)]); %SmallWall
[abs_map, scatt_map] = updateAbsScattCoeffs(abs_map,scatt_map,2,abs_coeffs(:,1),...
    scatt_coeffs(:,1),[0,0],[room_dim(2),room_dim(3)]); %OpSmallWall
[abs_map, scatt_map] = updateAbsScattCoeffs(abs_map,scatt_map,3,abs_coeffs(:,1),...
    scatt_coeffs(:,1),[0,0],[room_dim(1),room_dim(3)]); %LargeWall
[abs_map, scatt_map] = updateAbsScattCoeffs(abs_map,scatt_map,4,abs_coeffs(:,1),...
    scatt_coeffs(:,1),[0,0],[room_dim(1),room_dim(3)]); %OpLargeWall
[abs_map, scatt_map] = updateAbsScattCoeffs(abs_map,scatt_map,5,abs_coeffs(:,1),...
    scatt_coeffs(:,1),[0,0],[room_dim(1),room_dim(2)]); %Floor
[abs_map, scatt_map] = updateAbsScattCoeffs(abs_map,scatt_map,6,abs_coeffs(:,1),...
    scatt_coeffs(:,1),[0,0],[room_dim(1),room_dim(2)]); %Ceiling
\end{lstlisting}
Como los mapas están vacíos al crearse, primero se definen todas las paredes del cuarto con los respectivos coeficientes de absorción y difusión, y posteriormente ya se pueden actualizar ciertas zonas, con los coeficientes correspondientes a los paneles. \\
El mapa de reflexiones se obtiene a partir del mapa de absorción siguiendo la siguiente \href{https://www.researchgate.net/publication/276288771_Scattering_in_Room_Acoustics_and_Related_Activities_in_ISO_and_AES}{formula}
\begin{displaymath}
    Ref = \sqrt{1-Abs}
\end{displaymath}
\begin{lstlisting}[frame=single,numbers=left, style=Matlab-editor, basicstyle=\tiny]
ref_map = cell(1,num_FBands);
for i = 1:numel(abs_map)
    ref_map{i} = sqrt(1-abs_map{i});
end
\end{lstlisting}
Se definen tambien los parametros del histograma.
\begin{lstlisting}[frame=single,numbers=left, style=Matlab-editor, basicstyle=\tiny]
histTimeStep = 0.0010;
nTBins = round(impResTime/histTimeStep);
nFBins = length(FVect);
TFHist = zeros(nTBins,nFBins);
\end{lstlisting}
El proceso de \textit{Ray-Tracing} comienza tomando un rayo dentro de una banda de frecuencia, tomando su posición, dirección, el tiempo del rayo y su energía, y posteriormente, calcular en donde va a colisionar. Esto se hace observando los signos de la dirección del rayo, los cuales nos dice con cual, de entre dos paredes paralelas, va a colisionar el rayo. A continuación, se calcula el desplazamiento necesario para llegar a la coordenada constante de una pared y resaltando que chocara con la pared para la que necesite el menor desplazamiento. Este calculo se hace dentro de la funcion \textit{GetImpactWall}
\begin{lstlisting}[frame=single,numbers=left, style=Matlab-editor, basicstyle=\tiny]
function [surfaceofimpact,displacement] = getImpactWall(ray_xyz,ray_dxyz,roomDims)
% GETIMPACTWALL Determine which wall the ray encounters
surfaceofimpact = -1;
displacement = 1000;
%  Compute time to intersection with x-surfaces
if (ray_dxyz(1) < 0)
    displacement = -ray_xyz(1) / ray_dxyz(1);
    if displacement==0
        displacement=1000;
    end
    surfaceofimpact = 1; %"SmallWall"
elseif (ray_dxyz(1) > 0)
    displacement = (roomDims(1) - ray_xyz(1)) / ray_dxyz(1);
    if displacement==0
        displacement=1000;c
    end
    surfaceofimpact = 2; %"OpSmallWall"
end
% Compute time to intersection with y-surfaces
if ray_dxyz(2)<0
    t = -ray_xyz(2) / ray_dxyz(2);
    if (t<displacement) && t>0
        surfaceofimpact = 3; %"LargeWall";
        displacement = t;
    end
elseif ray_dxyz(2)>0
    t = (roomDims(2) - ray_xyz(2)) / ray_dxyz(2);
    if (t<displacement) && t>0
        surfaceofimpact =  4; %"OpLargeWall";
        displacement = t;
    end
end
% Compute time to intersection with z-surfaces
if ray_dxyz(3)<0
    t = -ray_xyz(3) / ray_dxyz(3);
    if (t<displacement) && t>0
        surfaceofimpact = 5; %"Floor";
        displacement = t;
    end
elseif ray_dxyz(3)>0
    t = (roomDims(3) - ray_xyz(3)) / ray_dxyz(3);
    if (t<displacement) && t>0
        surfaceofimpact = 6; %"Ceiling";
        displacement = t;
    end
end

displacement = displacement * ray_dxyz;

end
\end{lstlisting}
El desplazamiento y la pared de impacto, nos permite obtener las coordenadas en las que choca el rayo y así, consultar nuestro mapa de reflexión y difusión, para calcular el decaimiento de la energía con base en la reflexión y una nueva dirección para el rayo, la cual es combinación de la reflexión especular y la difusa. \\
Adicionalmente, se usa la difusión en ese punto, para calcular que cantidad de la energía restante del rayo va a capturar el receptor, la cual disminuye conforme mas se aleja del vector normal de la pared. \\
Este proceso se repite para la nueva posición, dirección, tiempo y energía del rayo; hasta que se supere el tiempo de simulación o la energía del rayo disminuya hasta ser despreciable.
\begin{lstlisting}[frame=single,numbers=left, style=Matlab-editor, basicstyle=\tiny]
for iBand = 1:nFBins
    fprintf("Calculating rays for band %d\n",iBand)
    % Perform ray tracing independently for each frequency band.
    for iRay = 1:size(rays,1)
        % Select ray direction
        ray = rays(iRay,:);
        % All rays start at the source/transmitter
        ray_xyz = source_pos;
        % Set initial ray direction. This direction changes as the ray is
        % reflected off surfaces.
        ray_dxyz = ray;
        % Initialize ray travel time. Ray tracing is terminated when the
        % travel time exceeds the impulse response length.
        ray_time = 0;
        % Initialize the ray energy to a normalized value of 1.     Energy
        % decreases when the ray hits a surface.
        ray_energy = 1;

        while (ray_time <= impResTime)

            % Determine the surface that the ray encounters
            [surfaceofimpact,displacement] = getImpactWall(ray_xyz,...
                                             ray_dxyz,room_dim);
            
            % Determine the distance traveled by the ray
            distance = sqrt(sum(displacement.^2));

            % Determine the coordinates of the impact point
            impactCoord = ray_xyz+displacement;

            if surfaceofimpact > 4
                pointOfImpact = [impactCoord(1),impactCoord(2)];
            elseif surfaceofimpact < 3
                pointOfImpact = [impactCoord(2),impactCoord(3)];
            else
                pointOfImpact = [impactCoord(1),impactCoord(3)];
            end

            % Update ray location/source
            ray_xyz = impactCoord;

            % Update cumulative ray travel time
            c = 343; % speed of light (m/s)
            ray_time = ray_time+distance/c;

            % Apply surface reflection to ray's energy
            % This is the amount of energy that is not lost through
            % absorption.

            ReflectionAtPoint = ref_map{surfaceofimpact}(ceil(100*pointOfImpact(1)),ceil(100*pointOfImpact(2)),iBand);
            ray_energy = ray_energy*ReflectionAtPoint;

            % Apply diffuse reflection to ray energy
            % This is the fraction of energy used to determine what is
            % detected at the receiver
            DifussionAtPoint = scatt_map{surfaceofimpact}(ceil(100*pointOfImpact(1)),ceil(100*pointOfImpact(2)),iBand);
            rayrecv_energy = ray_energy*DifussionAtPoint;

            % Determine impact point-to-receiver direction.
            rayrecvvector = mic_pos-impactCoord;

            % Determine the ray's time of arrival at receiver.
            distance = sqrt(sum(rayrecvvector.*rayrecvvector));
            recv_timeofarrival = ray_time+distance/c;

            if recv_timeofarrival>impResTime
                break
            end

            if ray_energy < 0.000001
                break
            end

            % Determine amount of diffuse energy that reaches the receiver.
            % See (5.20) in [2].

            % Compute received energy
            N = getWallNormalVector(surfaceofimpact);
            cosTheta = sum(rayrecvvector.*N)/(sqrt(sum(rayrecvvector.^2)));
            cosAlpha = sqrt(sum(rayrecvvector.^2)-mic_radius^2)/sum(rayrecvvector.^2);
            E = (1-cosAlpha)*2*cosTheta*rayrecv_energy;

            % Update energy histogram
            tbin = floor(recv_timeofarrival/histTimeStep + 0.5);
            TFHist(tbin,iBand) = TFHist(tbin,iBand) + E;

            % Compute a new direction for the ray.
            % Pick a random direction that is in the hemisphere of the
            % normal to the impact surface.
            d = rand(1,3);
            d = d/norm(d);
            if sum(d.*N)<0
                d = -d;
            end

            % Derive the specular reflection with respect to the incident
            % wall
            ref = ray_dxyz-2*(sum(ray_dxyz.*N))*N;

            % Combine the specular and random components
            d = d/norm(d);
            ref = ref/norm(ref);
            ray_dxyz = DifussionAtPoint*d+(1-DifussionAtPoint)*ref;
            ray_dxyz = ray_dxyz/norm(ray_dxyz);
        end
    end
end
\end{lstlisting}
Este proceso nos deja con un histograma dependiente de la frecuencia, el cual representa la envoltura de la respuesta al impulso.
\begin{figure}[!htb]
    \centering
    \includegraphics[width=\linewidth]{imagenes/Histogram.jpg}
    \caption{\footnotesize Histograma dependiente de la frecuencia}
    \label{fig:Histograma}
\end{figure}
\FloatBarrier
Para la construcción de la respuesta al impulso, es necesario modelar la estructura detallada a partir del histograma. Esto se puede hacer mediante ruido aleatorio con una distribución de Poisson, y se construye tomando la reflexión de un rayo como evento.\\
Es importante pasar el proceso aleatorio de Poisson por filtros pasa bandas que empaten con las frecuencias del histograma, para asi, poder multiplicarlos. 
\begin{figure}[!htb]
    \centering
    \includegraphics[width=\linewidth]{imagenes/BandPassFilters.png}
    \caption{\footnotesize Filtros pasa-bandas para el proceso de Poisson}
    \label{fig:PoissonFilters}
\end{figure}
\FloatBarrier
\begin{lstlisting}[frame=single,numbers=left, style=Matlab-editor, basicstyle=\tiny]
fs = 44100;
V = prod(room_dim);
t0 = ((2*V*log(2))/(4*pi*c^3))^(1/3); % eq 5.45 in [2]
poissonProcess = [];
timeValues = [];
t = t0;
while (t<impResTime)
    timeValues = [timeValues t]; %#ok
    % Determine polarity.
    if (round(t*fs)-t*fs) < 0 
        poissonProcess = [poissonProcess 1]; %#ok
    else
        poissonProcess = [poissonProcess -1];%#ok
    end
    % Determine the mean event occurence (eq 5.44 in [2])
    mu = min(1e4,4*pi*c^3*t^2/V); 
    % Determine the interval size (eq. 5.44 in [2])
    deltaTA = (1/mu)*log(1/rand); % eq. 5.43 in [2])
    t = t+deltaTA;
end
randSeq = zeros(ceil(impResTime*fs),1);
for index=1:length(timeValues)
    randSeq(round(timeValues(index)*fs)) = poissonProcess(index);
end
flow = [115 225 450 900 1800 3600 7200];
fhigh = [135 275 550 1100 2200 4400 8800];
NFFT = 8192;
win = hann(882,"symmetric");
sfft = dsp.STFT(Window = win,OverlapLength=441,FFTLength=NFFT,FrequencyRange="onesided");
isfft = dsp.ISTFT(Window=win,OverlapLength=441,FrequencyRange="onesided");
F = sfft.getFrequencyVector(fs);
RCF = zeros(length(FVect),length(F));
for index0 = 1:length(FVect)
    for index=1:length(F)
        f = F(index);
        if f<FVect(index0) && f>=flow(index0)
            RCF(index0,index) = .5*(1+cos(2*pi*f/FVect(index0)));
        end
        if f<fhigh(index0) && f>=FVect(index0)
            RCF(index0,index) = .5*(1-cos(2*pi*f/(FVect(index0)+1)));
        end
    end
end
frameLength = 441;
numFrames = length(randSeq)/frameLength;
y = zeros(length(randSeq),numel(FVect));
for index=1:numFrames
    x = randSeq((index-1)*frameLength+1:index*frameLength);
    X = sfft(x);    
    X = X.*RCF.';
    y((index-1)*frameLength+1:index*frameLength,:) = isfft(X);
end
impTimes = (1/fs)*(0:size(y,1)-1);
hisTimes = histTimeStep/2 + histTimeStep*(0:nTBins);
W = zeros(size(impTimes,2),numel(FVect));
BW = fhigh-flow;
for k=1:size(TFHist,1)
    gk0 = floor((k-1)*fs*histTimeStep)+1;
    gk1 = floor(k*fs*histTimeStep);
    yy = y(gk0:gk1,:).^2;
    val = sqrt(TFHist(k,:)./sum(yy,1)).*sqrt(BW/(fs/2));
    for iRay=gk0:gk1
        W(iRay,:)= val;
    end
end
\end{lstlisting}
Por ultimo, podemos crear la respuesta al impulso con los pesos y el histograma. Es importante notar que se agrego ruido de fondo, que nos permita observar el decaimiento en la integral de Schroeder.
\begin{lstlisting}[frame=single,numbers=left, style=Matlab-editor, basicstyle=\tiny]
y_2 = y.*W;
ip = sum(y_2,2);
ip = ip./max(abs(ip));
ip = ip + rand(length(ip),1)/1000;
vectorTiempo = (1/fs)*(0:numel(ip)-1);
figure
plot(vectorTiempo,ip.^2)
grid on
xlabel("Time (s)")
ylabel("Impulse Response")
\end{lstlisting}
\begin{figure}[!htb]
    \centering
    \includegraphics[width=\linewidth]{imagenes/RIRsquared_Simulated.jpg}
    \caption{\footnotesize RIR cuadrada obtenida de simulación}
    \label{fig:RIRsqSimulated}
\end{figure}
\FloatBarrier
A esta respuesta al impulso podemos aplicarle los mismos análisis que hacíamos con una respuesta al impulso obtenida mediante una medición.
\begin{figure}[!htb]
    \centering
    \includegraphics[width=\linewidth]{imagenes/RIRSimulated_Analysis.jpg}
    \caption{\footnotesize Análisis de la respuesta al impulso simulada}
    \label{fig:RIRSimulated_Analysis}
\end{figure}
\FloatBarrier
%------------------------------------------------------------------------------------------
Para comprobar el funcionamiento de nuestra simulación, recurrimos al uso de una paquetería de Python llamada PyRoomAcoustics, la cual permite hacer la simulación tal como la hicimos en MATLAB, pero de manera mas eficiente y con métodos mas completos. Sin profundizar en el funcionamiento, se creo un cuarto igual al de la implementación de MATLAB y se simulo la respuesta al impulso (Se adjunta el codigo utilizado en Anexo ?).
\begin{figure}[!htb]
    \centering
     \begin{subfigure}{0.3\textwidth}
        \centering
        \includegraphics[width=\linewidth]{imagenes/PyRoom_Room.png}
        \caption{\footnotesize Cuarto simulado en PyRoomAcoustics}
        \label{fig:sub2_1}
    \end{subfigure}
    \hfill
    \begin{subfigure}{0.3\textwidth}
        \centering
        \includegraphics[width=\linewidth]{imagenes/PyRoom_RIR.png}
        \caption{\footnotesize Respuesta al impulso en PyRoomAcoustics}
        \label{fig:sub2_2}
    \end{subfigure}
    \hfill
    \begin{subfigure}{0.3\textwidth}
        \centering
        \includegraphics[width=\linewidth]{imagenes/PyRoom_Decay.png}
        \caption{\footnotesize Análisis de la respuesta al impulso en PyRoomAcoustics}
        \label{fig:sub2_3}
    \end{subfigure}
    \caption{Simulación con PyRoomAcoustics}
\end{figure}
\FloatBarrier
Como se puede observar, la simulación en Python arroja resultados muy similares a los obtenidos con la implementación en MATLAB. En el caso de MATLAB, el tiempo de reverberación calculado fue de $6.19 segundos$ mientras que en Python fue de $6.024 segundos$, un tiempo muy cercado considerando que se pierde resolución conforme aumenta el tiempo de reverberación debido a la naturaleza exponencial del decaimiento. \\
Una vez comprobada la implementación en MATLAB, podemos utilizarla con características localizadas de los materiales en las paredes. La simulación aun no es capaz de generar la respuesta al impulso de un recinto lleno (de instrumentos, muebles, decoración, etc.) debido a la complejidad de las interacciones. Para este propósito se buscara la utilización de un software especializado llamado CADNAR.
%------------------------------------------------------------------------

%---------------MF4-----------------------------
\subsubsection{MF3. Módulo modificador de la acústica}
Para realizar la modificación de la acústica en el estudio se debe de cambiar la superficie que interactúa con las ondas de sonido generadas, para poder realizar el cambio de las superficies debemos de girar los prismas triangulares de nuestro trabajo terminal, para lo cual debemos de trasmitir dicho movimiento desde un motor hasta dichos primas. Existen varias formas de transmitir dichos movimiento, en la Tabla \ref{tab:ComTransmision} se pueden 
%-----------------------------tabla-----------------------------
\begin{center}
\footnotesize
    \begin{longtable}[!htb]{| m{5em} | m{12em} | m{12em}| m{12em}|}
    \hline
    \textbf{Tipo}& \textbf{Descripción} & \textbf{Ventajas} & \textbf{Desventajas}\\
    \hline\hline
%---------------------------
    Bandas y poleas& Utilizar poleas y bandas dentadas para poder realizar el movimiento giratorio de los primas triangulares &
    \begin{itemize}
        \item Ruido mínimo
        \item Vibraciones mínimas
        \item Movimiento preciso con una sincronización exacta
        \item Resistencia a la abrasión, al óxido, productos químicos y contaminantes.
    \end{itemize}
    & 
    \begin{itemize}
        \item Se necesitan bandas de largas dimensiones
        \item Mayor costo
        \item Ideal para transferir a una potencia relativamente baja
        \item La potencia de transferencia está a una distancia relativamente menor en comparación con otras bandas de transmisión
    \end{itemize}\\
    \hline
%------------------------
    Tornillo sin fin y corona & Utilizar un tornillo sin fin que mueva coronas, las cuales están acopladas a los primas triangulares &
    \begin{itemize}
        \item Elevada capacidad de carga
        \item Ruido mínimo
        \item Movimiento preciso con una sincronización exacta
        \item Compacto
    \end{itemize}
    & 
    \begin{itemize}
        \item Bajo rendimiento en las etapas de reducción
        \item Requiere mantenimiento periódico debido a que sufren desgaste por fricción
        \item Costo de mantenimiento elevado si se llegara a requerir reparaciones.
    \end{itemize}\\
    \hline
%------------------------    
    Movimiento individual& Colocar un motor por cada uno de los prismas triangulares para controlarlos individualmente& 
    \begin{itemize}
        \item Control personalizado a cada prisma sin la necesidad de ningún otro mecanismo 
        \item Movimiento preciso con una sincronización exacta
    \end{itemize}
    & 
    \begin{itemize}
        \item Elevado costo debido a que se necesitan comprar una gran cantidad de motores
        \item Mayor consumo energético comparado con las otras opciones.
    \end{itemize}\\
    \hline
    \caption{Ventajas y desventajas de los distintos tipos de transmisi\'on de movimiento}
    \label{tab:ComTransmision}
    \end{longtable}
\end{center}

Después de observar las ventajas y desventajas de cada uno de los métodos de transmisión de movimiento nos decidimos por la trasmisión por medio de un tornillo sin fin y corona debido a que se tiene un buen control de la posición de los prismas que portan los paneles, tiene una gran capacidad de carga, además de que el ruido producido es bajo. Al elegir este tipo de transmisión surgieron problemas en cuanto las dimensiones del tornillo sin fin, ya que si optabamos por utilizar un solo tornillo sin fin para mover todos los primas este debia de ser de un largo bastante considerable, y realizar la fabricación de este sería una tarea difícil, por lo que en la Tabla \ref{tab:ComTornilloSF} se comparan las opciones que se tienen para realizar la transmisión de tornillo sin fin y corona.
%-----------------------------tabla-----------------------------
\begin{center}
\footnotesize
    \begin{longtable}[!htb]{| m{5em} | m{12em} | m{12em}| m{12em}|}
    \hline
    \textbf{Tipo}& \textbf{Descripción} & \textbf{Ventajas} & \textbf{Desventajas}\\
    \hline\hline
%---------------------------
    Bandas y poleas& Utilizar poleas y bandas dentadas para poder realizar el movimiento giratorio de los primas triangulares &
    \begin{itemize}
        \item Ruido mínimo
        \item Vibraciones mínimas
        \item Movimiento preciso con una sincronización exacta
        \item Resistencia a la abrasión, al óxido, productos químicos y contaminantes.
    \end{itemize}
    & 
    \begin{itemize}
        \item Se necesitan bandas de largas dimensiones
        \item Mayor costo
        \item Ideal para transferir a una potencia relativamente baja
        \item La potencia de transferencia está a una distancia relativamente menor en comparación con otras bandas de transmisión
    \end{itemize}\\
    \hline
%------------------------
    Tornillo sin fin y corona & Utilizar un tornillo sin fin que mueva coronas, las cuales están acopladas a los primas triangulares &
    \begin{itemize}
        \item Elevada capacidad de carga
        \item Ruido mínimo
        \item Movimiento preciso con una sincronización exacta
        \item Compacto
    \end{itemize}
    & 
    \begin{itemize}
        \item Bajo rendimiento en las etapas de reducción
        \item Requiere mantenimiento periódico debido a que sufren desgaste por fricción
        \item Costo de mantenimiento elevado si se llegara a requerir reparaciones.
    \end{itemize}\\
    \hline
%------------------------    
    Movimiento individual& Colocar un motor por cada uno de los prismas triangulares para controlarlos individualmente& 
    \begin{itemize}
        \item Control personalizado a cada prima sin la necesidad de ningún otro mecanismo 
        \item Movimiento preciso con una sincronización exacta
    \end{itemize}
    & 
    \begin{itemize}
        \item Elevado costo debido a que se necesitan comprar una gran cantidad de motores
        \item Mayor consumo energético comparado con las otras opciones.
    \end{itemize}\\
    \hline
    \caption{Ventajas y desventajas relacionadas a la utilización de un tornillo sin fin único o varios segmentos de tornillo sin fin}
    \label{tab:ComTornilloSF}
    \end{longtable}
\end{center}
\subsubsection{MF4. Módulo de interfaz de usuario}

\paragraph{MF4.1}


\begin{figure}[!htb]
    \centering
     \begin{subfigure}{0.3\textwidth}
        \centering
        \includegraphics[width=\linewidth]{imagenes/Landing.png}
        \caption{\footnotesize Señal de excitación (Barrido senoidal)}
        \label{fig:sub1}
    \end{subfigure}
    \hfill
    \begin{subfigure}{0.3\textwidth}
        \centering
        \includegraphics[width=\linewidth]{imagenes/Guitarra.png}
        \caption{\footnotesize Audio grabado}
        \label{fig:sub2}
    \end{subfigure}
    \hfill
    \begin{subfigure}{0.3\textwidth}
        \centering
        \includegraphics[width=\linewidth]{imagenes/Control Manual.png}
        \caption{\footnotesize Respuesta al impulso estimada}
        \label{fig:sub2}
    \begin{subfigure}{0.3\textwidth}
        \centering
        \includegraphics[width=\linewidth]{imagenes/Historial.png}
        \caption{\footnotesize Audio grabado}
        \label{fig:sub2}
    \end{subfigure}
    \hfill
    \end{subfigure}
    \caption{Gráficas resultantes de generación y medición de la acústica}
\end{figure}
\FloatBarrier


\hfill\break
\begin{figure}[!htb]
    \centering
    \includegraphics[width=0.8\linewidth]{imagenes/Landing.png}
    \caption{\footnotesize Interfaz de usuario. Página de la acústica actual}
    \label{fig:DecayCurve}
\end{figure}
\FloatBarrier

\hfill\break
\begin{figure}[!htb]
    \centering
    \includegraphics[width=0.8\linewidth]{imagenes/Guitarra.png}
    \caption{\footnotesize Interfaz de usuario. Página generación de una nueva acústica}
    \label{fig:DecayCurve}
\end{figure}
\FloatBarrier

\hfill\break
\begin{figure}[!htb]
    \centering
    \includegraphics[width=\linewidth]{imagenes/Control Manual.png}
    \caption{\footnotesize Interfaz de usuario. Página de control manual}
    \label{fig:DecayCurve}
\end{figure}
\FloatBarrier

\hfill\break
\begin{figure}[!htb]
    \centering
    \includegraphics[width=\linewidth]{imagenes/Historial.png}
    \caption{\footnotesize Interfaz de usuario. Página del historial de configuraciones}
    \label{fig:DecayCurve}
\end{figure}
\FloatBarrier



%---------------Media 3---------------
% %\section*{Administración del proyecto}
%\subsection*{Presupuesto estimado e infraestructura}
%tablas 5, 6 y 7 
%-----------------------------tabla 5-----------------------------
\begin{center}
\footnotesize
    \begin{longtable}[!htb]{| m{15em} | m{6em} | m{6em}| m{6em}|}
    \hline
    \textbf{Personal}& \textbf{\'Area de aporte} & \textbf{Instituci\'on} & \textbf{Tiempo m\'inimo}\\
    \hline\hline
    Estudiante: Barbosa Mercado José Aarón & STEM & UPIITA-IPN & 600 hrs\\
    \hline
    Estudiante: Camarena Rodríguez Alberto & STEM & UPIITA-IPN & 600 hrs\\
    \hline
    Estudiante: Muñoz Ceballos Teddy Xavier & STEM & UPIITA-IPN & 600 hrs\\
    \hline
    Estudiante: Sánchez Trujillo Daniel & STEM & UPIITA-IPN & 600 hrs\\
    \hline
    Asesor: Dr. Rafael Trovamala Landa & STEM & UPIITA-IPN/Ik'Atl & 60 hrs\\
    \hline
    Asesor: Dr. Alberto Luviano Juárez & STEM & UPIITA-IPN & 60 hrs\\
    \hline
    Especialista Externo: Rogelio Israel Quintero Tiscareño & Ingeniero en audio & Espiral Est\'ereo & 50 hrs\\
    \hline

    \caption{Tabla de recursos humanos.}
    \label{tab:RH}
    \end{longtable}
\end{center}
%-----------------------------tabla 5-----------------------------

%-----------------------------tabla 6-----------------------------
\begin{center}
\footnotesize
    \begin{longtable}[!htb]{| m{10em} | m{12em} | m{12em}|}
    \hline
    \textbf{Infraestructura}& \textbf{Desciripci\'on} & \textbf{Uso} \\
    \hline\hline
    Estudio de grabaci\'on & Recinto sobre el cual se desea hacer el tratamiento acústico. & Es el cuarto destinado sobre el que se trabajará el trabajo terminal.\\
    \hline
    Laboratorio de pesados & Taller donde se encuentran las máquinas e instrumentos. & En él, se manufacturarán las piezas.\\
    \hline

    \caption{Infraestructura para el desarrollo del proyecto.}
    \label{tab:Infraestructura}
    \end{longtable}
\end{center}
%-----------------------------tabla 6-----------------------------
%-----------------------------tabla 7-----------------------------
\begin{center}
\scriptsize
    \begin{longtable}[!htb]{| m{4em} | m{8em} | m{4.5em}| m{4em}| m{3em}| m{8em}| m{4em}|}
    \hline
    \textbf{M\'odulo}& \textbf{Material} & \textbf{Cantidad} & \textbf{Costo unitario} & \textbf{Costo total} & \textbf{Financiamiento} & \textbf{Fuente}\\
    \hline\hline
    \multirow{7}{*}{MMRP} 
    & Monitor de audio con amplificador & 1 & $\$$3,514 & $\$$3,514 & Privado & [20]\\
    \cline{2-7}
    & Micr\'ofono de condensador & 1 & $\$$3,599 & $\$$3,599 & Privado & [21]\\
    \cline{2-7}
    & Interfaz de audio & 1 & $\$$2,199 & $\$$2,199 & Privado & [22]\\
    \cline{2-7}
    & Cable XLR balanceado & 1 & $\$$167 & $\$$167 & Privado & [23]\\
    \cline{2-7}
    & Pedestal & 1 & $\$$188 & $\$$188 & Privado & [24]\\
    \cline{2-7}
    & Software generador de onda arbitraria & 1 & $\$$0 & $\$$0 & Privado & Estimado\\
    \cline{2-7}
    & Software para adquisici\'on de datos & 1 & $\$$0 & $\$$0 & Privado & Estimado\\
    \hline

    \multirow{5}{*}{MMP} 
    & Tarjeta de desarrollo & 1 & $\$$752 & $\$$752 & Propio & [25]\\
    \cline{2-7}
    & Motor & 4 & $\$$256 & $\$$1,024 & Propio & [26]\\
    \cline{2-7}
    & Panel ac\'ustico profesional de absorci\'on & 5 & $\$$673 & $\$$3,375 & Privado & [27]\\
    \cline{2-7}
    & Cople de uni\'on  & 4 & $\$$200 & $\$$800 & Propio & Estimado\\
    \cline{2-7}
    & Material de construci\'on de paneles de difusi\'on  y reflexi\'on (Madera) & 5 & $\$$385 & $\$$1,935 & Propio & [28]\\
    \hline

    & Esponja aislante & 5 & $\$$623 & $\$$3,115 & Propio & [29]\\
    \hline
    
    & Espuma absorbente & 5 & $\$$1,892 & $\$$9,460 & Propio & [30]\\
    \hline

    MCI & Software de control  & 1 & $\$$0 & $\$$0 & Privado & Estimado\\
    \hline
    MPC & Software de interfaz & 1 & $\$$0 & $\$$0 & Privado & Estimado\\
    \hline
    
    \caption{Presupuesto estimado}
    \label{tab:Presupuesto}
    \end{longtable}
\end{center}


%-----------------------------tabla 7-----------------------------
El costo total aproximado del proyecto sería $\$30,108$

\subsection*{Planeaci\'on de actividades}
\begin{figure}[!htb]
    \centering
    \includegraphics[width=1\textwidth]{imagenes/Protocolo21.jpg}
    \caption{\footnotesize Cronograma de actividades para TT1.}
\end{figure}
\FloatBarrier

\begin{figure}[!htb]
    \centering
    \includegraphics[width=1\textwidth]{imagenes/Protocolo22.png}
    \caption{\footnotesize Cronograma de actividades para TT2.}
\end{figure}
\FloatBarrier

\section{Diseño del sistema}
\subsection{Diseño conceptual}
\subsubsection{Necesidades y requerimientos}
\subsubsection{Arquitectura funcional}
\subsubsection{Propuestas solución}

$S_{1}$: Sistema de medición de la acústica \\
\tab    $M_{1}$: Modulo de generación del tren de pulsos \\
\tab    $M_{2}$: Modulo de medición de la respuesta \\
\tab    $M_{3}$: Modulo de procesado y calculo \\
$S_{2}$: Sistema de control y procesado \\
$S_{3}$: Sistema de movimiento de paneles acústicos \\
$S_{4}$: Sistema de interfaz humano-maquina \\
\tab    $M_{4}$: Modulo de despliegue de información
$S_{5}$: Sistema de gestión de la energía \\

Se realizo una matriz de trazabilidad para verificar que los sistemas y módulos propuestos para la arquitectura física cumplieran con las funciones a ser desempeñadas.
%----------------------------------------------------

%---------------Media 4---------------
\paragraph{Calculo de Posiciones para los paneles}

Del calculo de la acústica obtuvimos el tiempo de reverberación, de cual se puede calcular la absorción estimada el cuarto. La absorción calculada nos permite calcular, de acuerdo a la absorción de los paneles, la cantidad de superficie que se tiene que añadir al cuarto para alcanzar un tiempo de reverberación deseado. \\
Entonces, resta obtener un programa que a partir de las posiciones de los paneles y la superficie necesaria, nos entregue las nuevas posiciones de los paneles que resulten en la acústica deseada. 

La implementación se hizo en Python y la lógica es la siguiente:
\begin{itemize}
    \item Encontrar el numero de paneles que se deben de cambiar de la posición de reflexión a la posición de absorción, con base en la absorción efectiva del panel y la diferencia entre tiempos de reveberación (actual y deseado)
    \item Si el numero de paneles supera al numero de paneles disponibles, solo mover los disponibles
    \item Preguntar al usuario cuantos de los paneles que no se cambiaran a la posición de absorción quiere cambiar a la posición de difusión. Se deja este proceso al usuario ya que la especialidad no es un parámetro ideal, si no que varia en función al deseo artístico del usuario.
    \item Dada una jerarquía de paneles (primero mover los de en medio, e ir moviéndose hacia los extremos para después terminar los huecos) cambiar el indicador de tipo de panel a absorción y los deseados a difracción.
\end{itemize}

Para encontrar el numero de paneles faltantes, utilizamos la formula de la reverberación: 
\begin{equation}
    RT = \frac{0.161 V}{S \alpha}
\end{equation}
Ya que la frecuencia central utilizada por la mayoría de instrumentos y tratamientos acústicos es a los 500 Hz, se tomara la absorción de los paneles a esta frecuencia. 
\begin{lstlisting}[frame=single,numbers=left, style=matlab-editor, basicstyle=\tiny]
def PanelsToChange(Current_RT60, Needed_RT60, room_dim):
    V = np.prod(room_dim)
    Panel_Surface = np.prod(PanelSize)
    Current_Abs = 0.161*V/Current_RT60
    Needed_Abs = 0.161*V/Needed_RT60
    AbsFromSinglePanel = Panel_Surface*(AbsCoeffs['AbsPanels'][6]-AbsCoeffs['Concrete'][6])
    PanelsToChange = (Needed_Abs-Current_Abs)//AbsFromSinglePanel
    return PanelsToChange

x = PanelsToChange(0.9,0.4,room_dim)
print(f'Changing {int(x)} panels to Absorption')
\end{lstlisting}
Se puede observar que se colocaron los coeficientes de absorción en diccionarios, que se pueden acceder por medio del indicador \textit{string}.
\begin{lstlisting}[frame=single,numbers=left, style=matlab-editor, basicstyle=\tiny]
if x > 18:
    print('Only 18 panels available')
    x = 18

DPanels = int(input(f"Enter the number of diffusion panels to put ({18-x} Available): "))
if DPanels > 18-x :
    print("Not a valid number, using 0")
    DPanels = 0
\end{lstlisting}
Después, se hace la validación de la cantidad de paneles disponibles y, en función de los restantes, la cantidad de paneles de difusión a cambiar.\\
Para la modificación del tipo de panel en función a la jerarquía se decidió utilizar un tipo de estructura llamada \textit{linked list}, ya que nos permitirá guardar los datos del panel (Que tipo de superficie se pondrá y que posición ocupa en el arreglo de paneles) en un nodo, además de poder indicar, cual es el siguiente nodo a visitar. De esta manera, podemos crear una programa que va visitando los paneles en función de la lista jerárquica, la cual sera, el orden de los paneles en la lista. Para el movimiento en si, se puede visitar la lista y obtener los datos del nodo, lo que nos permitirá asociar una posición en arreglo con cierto tipo de panel.
\begin{lstlisting}[frame=single,numbers=left, style=matlab-editor, basicstyle=\tiny]
class Node:
    def __init__(self, data, position):
        self.Type = data
        self.Position = position
        self.next = None
\end{lstlisting}
El nodo solo contiene el tipo de panel (1:'Reflexion',2:'Absorcion',3:'Difraccion'), la posición que ocupa en el arreglo, existen 18 posiciones que indican los 9 paneles que hay por pared, y la referencia al siguiente nodo, el cual nos da la lista jerárquica. \\
El constructor de la lista solo indica cual es la \textit{cabeza} o inicio de la lista y después, se van añadiendo nodos conforme a la lista jerárquica.
\begin{lstlisting}[frame=single,numbers=left, style=matlab-editor, basicstyle=\tiny]
PanelOrder = LinkedList() 
PanelOrder.insertAtBegin(NameToKey['Refle'], 14)
PanelOrder.insertAtEnd(NameToKey['Refle'], 5)
PanelOrder.insertAtEnd(NameToKey['Refle'], 12)
PanelOrder.insertAtEnd(NameToKey['Refle'], 16)
PanelOrder.insertAtEnd(NameToKey['Refle'], 3)
PanelOrder.insertAtEnd(NameToKey['Refle'], 7)
PanelOrder.insertAtEnd(NameToKey['Refle'], 10)
PanelOrder.insertAtEnd(NameToKey['Refle'], 18)
PanelOrder.insertAtEnd(NameToKey['Refle'], 1)
PanelOrder.insertAtEnd(NameToKey['Refle'], 9)
PanelOrder.insertAtEnd(NameToKey['Refle'], 13)
PanelOrder.insertAtEnd(NameToKey['Refle'], 15)
PanelOrder.insertAtEnd(NameToKey['Refle'], 4)
PanelOrder.insertAtEnd(NameToKey['Refle'], 6)
PanelOrder.insertAtEnd(NameToKey['Refle'], 11)
PanelOrder.insertAtEnd(NameToKey['Refle'], 17)
PanelOrder.insertAtEnd(NameToKey['Refle'], 2)
PanelOrder.insertAtEnd(NameToKey['Refle'], 8)
\end{lstlisting}
En la lista jerárquica la prioridad es la siguiente:
\begin{itemize}
    \item (1) Panel central (Posición 5)
    \item (2) Panel izquierdo con uno de separación del central (Posición 3)
    \item (2) Panel derecho con uno de separación del central (Posición 7)
    \item (3) Panel del extremo izquierdo (Posición 1)
    \item (3) Panel del extremo derecho (Posición 9)
    \item (4) Panel izquierdo pegado al central (Posición 4)
    \item (4) Panel derecho pegado al central (Posición 6)
    \item (5) Panel izquierdo pegado al del extremo (Posición 2)
    \item (5) Panel derecho pegado al del extremo (Posición 8)
\end{itemize}
Esta lista jerárquica nos permite mantener el balance de paneles en el cuarto y los paneles con el mismo numero se consideran un conjunto. Cabe resaltar que esta lista es solo para una pared, los de las demás paredes siguen la misma lista jerárquica, con la diferencia que al terminar un conjunto, se modifica ese mismo conjunto en la siguiente pared y posteriormente se vuelve a modificar el siguiente conjunto de la primer pared.
El cambio de tipo de los paneles se hace con el siguiente código:
\begin{lstlisting}[frame=single,numbers=left, style=matlab-editor, basicstyle=\tiny]
for i in range(18):
    if i < x:
        PanelOrder.updateNode(NameToKey['Abs'],i)
    elif i < x + DPanels:
        PanelOrder.updateNode(NameToKey['Scatt'],i) 
    else:
        PanelOrder.updateNode(NameToKey['Refle'],i) 
\end{lstlisting}
Se puede ver como va por la lista jerárquica y cambia los absorbentes, luego los de difracción indicados y los restantes los mantiene en reflexivos. \\
Por ultimo, para acceder a los paneles en orden, se guardan en un diccionario, donde la \textit{key} es la posición en el arreglo (No la lista), y el valor es el tipo de panel en esa posición.
\begin{lstlisting}[frame=single,numbers=left, style=matlab-editor, basicstyle=\tiny]
def GetPanels(PanelOrder):
    PanelArray = {}
    for i in range(18):
        currentPanel_Type, currentPanel_Pos = PanelOrder.returnAtIndex(i)
        PanelArray[currentPanel_Pos] = currentPanel_Type
    return PanelArray

PanelArray = GetPanels(PanelOrder)
for i in range(1,19):
    print(PanelArray[i])
\end{lstlisting}
Iterar a lo largo del diccionario nos da el tipo de panel de cada posición y es el utilizado para el control. \\
Para validarlo, se realizaron varios ejemplos. En el primer caso, se cambiara el tiempo de reverberación de 0.9 s a 0.85 s. El programa calcula que es necesario cambiar dos paneles a absorción y en este caso, se indico que agregara cinco paneles de difusión. La lista de paneles entregada es: $(1,1,3,1,2,1,3,1,1) (3,1,3,1,2,1,3,1,1)$. Se puede observar que solo los paneles centrales se cambiaron a absorción, que los de difracción están balanceados en el arreglo derecho y que el panel de difusión extra en la derecha esta en el extremo izquierdo. \\
\begin{figure}[!htb]
    \centering
    \includegraphics[width=0.8\textwidth]{imagenes/PanelsExample1.jpg}
    \caption{Prueba uno de calculo de posiciones para los paneles}
    \label{fig:PanelsExample1}
\end{figure}
\FloatBarrier
En el segundo caso, se cambiara el tiempo de reverberación de 0.9 s a 0.7 s. El programa calcula que es necesario cambiar doce paneles a absorción y en este caso, se indico que agregara 2 paneles de difusión. La lista de paneles entregada es: $(2,1,2,3,2,3,2,1,2) (2,1,2,2,2,2,2,1,2)$. Se puede observar que en ambos arreglos, los paneles impares se cambiaron a absorción, y los dos sobrantes se colocaron junto al central y se puede ver también que los de difracción están balanceados en el arreglo izquierdo. \\
\begin{figure}[!htb]
    \centering
    \includegraphics[width=0.8\textwidth]{imagenes/PanelsExample2.jpg}
    \caption{Prueba dos de calculo de posiciones para los paneles}
    \label{fig:PanelsExample2}
\end{figure}
\FloatBarrier

\paragraph{Paneles de Absorción}

\paragraph{Paneles de difracción}
Los paneles de difracción están caracterizados por su capacidad de reflejar las ondas de sonido en muchas direcciones, lo que permite una mejor distribución de la energía acústica en el cuarto. Los difusores permiten que los oyentes reciban muchas mas reflexiones de menor energía y que por tanto, tengan una mayor sensación de profundidad y de volumen. \cite{PilchDiffusers}\\
Existen diversos tipos de difusores, a continuación se listan los mas importantes \cite{TypesDiffusers}:
\begin{itemize}
    \item Difusores de tipo \textit{Skyline}. Consta de agrupaciones de bloques a diferentes alturas que recuerdan a los rascacielos de una ciudad (De ahí su nombre). Las diferentes alturas permiten que las ondas reboten en diferentes instantes del tiempo, además, la diferencia en la dirección de las superficies permite que se reflejen hacia diversas direcciones.
    \item Difusores de residuo cuadráticos. Consta de diferentes canales que corren paralelos en el panel, a diferentes profundidades. Las profundidades no son aleatorias, sino que están diseñadas para ser eficientes a ciertas frecuencias.
    \item Difusores de residuo cuadráticos en 2D. Son muy similares a los de tipo \textit{Skyline} con la diferencia que están cuidadosamente planificados para ser eficientes a ciertas frecuencias, además, los bloques se separan entre ellos.
    \item Difusores de barril. Se asemejan a un cilindro cortado a la mitad, y se colocan convexos al cuarto. Su geometría permite reflejos especulares que varían de acuerdo al punto de impacto. Se diferencian del resto por su capacidad de albergar materiales absorbentes en el interior, lo que permite absorber frecuencias que no se difractan. 
    \item Difusores piramidales. Constan de pirámides cuadrangulares de diferentes alturas, lo que modifica los angulos de incidencia.
\end{itemize}
Para nuestra aplicación, se decidió utilizar los Difusores de residuo cuadrático (también llamados \textit{Schoroeder Diffusers} o QRD), ya que, además de permitirnos atacar ciertas frecuencias importantes para nosotros, también son fáciles de manufacturar. 
La sección transversal de un panel QRD es la siguiente \cite{PilchDiffusers}:
\begin{figure}[!htb]
    \centering
    \includegraphics[width=0.8\textwidth]{imagenes/QRD-150-diffusers-cross-section.png}
    \caption{Sección transversal de un panel QRD}
    \label{fig:QRDCross}
\end{figure}
\FloatBarrier
La manufactura constará de una operación con la fresadora para realizar cada uno de los canales a partir de un bloque solido, la altura de la fresadora se ajustara para las diferentes profundidades. \\
Los paneles QRD pueden colocarse paralelos o rotados 90 grados uno respecto del otro. Si se colocan todos paralelos, la difraccion de las ondas ocurrirá solamente en el eje perpendicular a la dirección de los canales. La rotación de los difusores QRD permite que las ondas sea difractados en ambos ejes.

% \section{Capitulo III}
\addcontentsline{toc}{section}{Apéndices}
\section*{Apéndices}
\appendix
% \subsection{Anexos}
\section{Códigos y scripts}


\subsection{Codigo del programa Absorptions}
\lstinputlisting[frame=single,numbers=left, style=Matlab-editor, basicstyle=\footnotesize]{Codigos/Absorptions.m}


\subsection{Código para realizar el trazado de los rayos y calcular la espuesta al impulso}
\lstinputlisting[frame=single,numbers=left, style=Matlab-editor, basicstyle=\footnotesize]{Codigos/RayTracingImpulseResponse.m}

\subsection{Código para medir la respuesta al impulso}
\lstinputlisting[frame=single,numbers=left, style=Matlab-editor, basicstyle=\footnotesize]{Codigos/ImpulseResponseMeasurer.m}

\subsection{Código de la función para calcular los parámetros acústicos}
\lstinputlisting[frame=single,numbers=left, style=Matlab-editor, basicstyle=\footnotesize]{Codigos/CalcAcousticParams.m}

\subsection{Código de la función para obtener la absorción del cuarto}
\lstinputlisting[frame=single,numbers=left, style=Matlab-editor, basicstyle=\footnotesize]{Codigos/GetAbsorption.m}

\subsection{Código de la función para obtener la coordenada en donde choco el rayo de sonido}
\lstinputlisting[frame=single,numbers=left, style=Matlab-editor, basicstyle=\footnotesize]{Codigos/getImpactWall.m}

\subsection{Código de la función que realiza la interpolación entre EDT, T20, T30, y T60, y un vector de tiempo}
\lstinputlisting[frame=single,numbers=left, style=Matlab-editor, basicstyle=\footnotesize]{Codigos/GetLinearFits.m}

\subsection{Código de la función que nos proporciona el vector normal a la pared en la que el rayo chocó}
\lstinputlisting[frame=single,numbers=left, style=Matlab-editor, basicstyle=\footnotesize]{Codigos/getWallNormalVector.m}

\subsection{Código para realizar el trazado de los rayos y calcular la espuesta al impulso}
\lstinputlisting[frame=single,numbers=left, style=Matlab-editor, basicstyle=\footnotesize]{Codigos/RayTracingImpulseResponse.m}

\subsection{Código de la función que genera un sonido que simula el disparo de una pistola}
\lstinputlisting[frame=single,numbers=left, style=Matlab-editor, basicstyle=\footnotesize]{Codigos/Gunshot.m}

\subsection{Código de la clase Panel}
\lstinputlisting[frame=single,numbers=left, style=Matlab-editor, basicstyle=\footnotesize]{Codigos/Panel.m}

\subsection{Código de la función que gráfica el cuarto de trabajo}
\lstinputlisting[frame=single,numbers=left, style=Matlab-editor, basicstyle=\footnotesize]{Codigos/plotRoom.m}

\subsection{Código de la función que gráfica las superficies del cuarto}
\lstinputlisting[frame=single,numbers=left, style=Matlab-editor, basicstyle=\footnotesize]{Codigos/plotSurfaces.m}

\subsection{Código de RandSampleSphere}
\lstinputlisting[frame=single,numbers=left, style=Matlab-editor, basicstyle=\footnotesize]{Codigos/RandSampleSphere.m}

\subsection{Código de SimuleteRIR}
\lstinputlisting[frame=single,numbers=left, style=Matlab-editor, basicstyle=\footnotesize]{Codigos/SimulateRIR.m}

\subsection{Código Sweep}
\lstinputlisting[frame=single,numbers=left, style=Matlab-editor, basicstyle=\footnotesize]{Codigos/Sweep.m}

%\begin{lstlisting}[language=Python, caption=Estres residual anillos 01-MatLab, label={cod:MatLab}]
%\end{lstlisting}

% \section{Capitulo III}
\addcontentsline{toc}{section}{Anexos}
\section*{Anexos}
\appendix
% \subsection{Anexos}
\section{Especificaciones del motor monofásico}
\begin{figure}[!htb]
    \centering
    \includegraphics[width=\linewidth]{imagenes/MotoresComercialesTruper1.jpg}
    \label{fig:DecayCurve}
\end{figure}

\begin{figure}[!htb]
    \centering
    \includegraphics[width=\linewidth]{imagenes/MotoresComercialesTruper2.jpg}
    \label{fig:DecayCurve}
\end{figure}
\FloatBarrier
%\bibliographystyle{ieeetr}
\selectlanguage{english}
\printbibliography
\selectlanguage{spanish}

\end{document}
